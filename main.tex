\documentclass[a4paper,12pt,ARIAL]{book}
\usepackage{graphicx} % Required for inserting images
\usepackage{eumat}
\linespread{1,5}
\pagestyle{plain}
\usepackage{titlesec}
\titleformat{\chapter}[display]{\huge\brseries\centering}{\chaptertitlename\thechapter}{20pt}{\huge}
\renewcommand{\chaptername}{BAB}

\begin{document}
\begin{center}
    \huge{TUGAS AKHIR APLIKASI KOMPUTER}
\end{center}
\vspace{2,5 cm}
\begin{minipage}{\linewidth}
    \begin{center}
        \includegraphics[width=5cm,height=5cm]{images/logo uny.png}
    \end{center}
\end{minipage}
\vspace{1,5cm}

\begin{center}
    \large{Naela Rizqy Arofah\\22305144042\\Matematika B}
\end{center}
\vspace{1,5cm}

\begin{center}
    \Large{\textwidth{DEPARTEMEN PENDIDIKAN MATEMATIKA\\FAKULTAS MATEMATIKA DAN ILMU PENGETAHUAN ALAM\\UNIVERSITAS NEGERI YOGYAKARTA\\2023}}
\end{center}

\maketitle

\chapter{Pendahuluan dan Pengenalan Cara Kerja EMT}
\begin{eulercomment}
Selamat datang! Ini adalah pengantar pertama ke Euler Math Toolbox (disingkat EMT
atau Euler). EMT adalah sistem terintegrasi yang merupakan perpaduan kernel
numerik Euler dan program komputer aljabar Maxima.

- Bagian numerik, GUI, dan komunikasi dengan Maxima telah dikembangkan oleh R.
Grothmann, seorang profesor matematika di Universitas Eichstätt, Jerman. Banyak
algoritma numerik dan pustaka software open source yang digunakan di dalamnya.

- Maxima adalah program open source yang matang dan sangat kaya untuk perhitungan
simbolik dan aritmatika tak terbatas. Software ini dikelola oleh sekelompok
pengembang di internet.

- Beberapa program lain (LaTeX, Povray, Tiny C Compiler, Python) dapat digunakan
di Euler untuk memungkinkan perhitungan yang lebih cepat maupun tampilan atau
grafik yang lebih baik.

Yang sedang Anda baca (jika dibaca di EMT) ini adalah berkas notebook di EMT.
Notebook aslinya bawaan EMT (dalam bahasa Inggris) dapat dibuka melalui menu
File, kemudian pilih "Open Tutorias and Example", lalu pilih file "00 First
Steps.en". Perhatikan, file notebook EMT memiliki ekstensi ".en". Melalui
notebook ini Anda akan belajar menggunakan software Euler untuk menyelesaikan
berbagai masalah matematika.
\end{eulercomment}
\begin{eulercomment}
Panduan ini ditulis dengan Euler dalam bentuk notebook Euler, yang berisi teks
(deskriptif), baris-baris perintah, tampilan hasil perintah (numerik, ekspresi
matematika, atau gambar/plot), dan gambar yang disisipkan dari file gambar.

Untuk menambah jendela EMT, Anda dapat menekan [F11]. EMT akan menampilkan
jendela grafik di layar desktop Anda. Tekan [F11] lagi untuk kembali ke tata
letak favorit Anda. Tata letak disimpan untuk sesi berikutnya.

Anda juga dapat menggunakan [Ctrl]+[G] untuk menyembunyikan jendela grafik.
Selanjutnya Anda dapat beralih antara grafik dan teks dengan tombol [TAB].

Seperti yang Anda baca, notebook ini berisi tulisan (teks) berwarna hijau, yang
dapat Anda edit dengan mengklik kanan teks atau tekan menu Edit -\textgreater{} Edit Comment
atau tekan [F5], dan juga baris perintah EMT yang ditandai dengan "\textgreater{}" dan
berwarna merah. Anda dapat menyisipkan baris perintah baru dengan cara menekan
tiga tombol bersamaan: [Shift]+[Ctrl]+[Enter].

\end{eulercomment}
\eulersubheading{Komentar (Teks Uraian)}
\begin{eulercomment}
Komentar atau teks penjelasan dapat berisi beberapa "markup" dengan sintaks
sebagai berikut.

\end{eulercomment}
\begin{eulerttcomment}
   - * Judul
   - ** Sub-Judul
   - latex: F (x) = \(\backslash\)int_a^x f (t) \(\backslash\), dt
   - mathjax: \(\backslash\)frac\{x^2-1\}\{x-1\} = x + 1
   - maxima: 'integrate(x^3,x) = integrate(x^3,x) + C
   - http://www.euler-math-toolbox.de
   - See: http://www.google.de | Google
   - image: hati.png
   - ---
\end{eulerttcomment}
\begin{eulercomment}

Hasil sintaks-sintaks di atas (tanpa diawali tanda strip) adalah sebagai berikut.

\begin{eulercomment}
\eulerheading{Judul}
\begin{eulercomment}
\end{eulercomment}
\eulersubheading{Sub-Judul}
\begin{eulercomment}
\end{eulercomment}
\begin{eulerformula}
\[
F(x) = \int_a^x f(t) \, dt
\]
\end{eulerformula}
\begin{eulerformula}
\[
\frac{x^2-1}{x-1} = x + 1
\]
\end{eulerformula}
\begin{eulercomment}
maxima: 'integrate(x\textasciicircum{}3,x) = integrate(x\textasciicircum{}3,x) + C\\
http://www.euler-math-toolbox.de\\
See: http://www.google.de \textbar{} Google\\
image: hati.png\\
\end{eulercomment}
\eulersubheading{}
\begin{eulercomment}
Gambar diambil dari folder images di tempat file notebook berada dan tidak dapat
dibaca dari Web. Untuk "See:", tautan (URL)web lokal dapat digunakan.

Paragraf terdiri atas satu baris panjang di editor. Pergantian baris akan memulai
baris baru. Paragraf harus dipisahkan dengan baris kosong.
\end{eulercomment}
\begin{eulerprompt}
>// baris perintah diawali dengan >, komentar (keterangan) diawali dengan //
\end{eulerprompt}
\eulerheading{Baris Perintah}
\begin{eulercomment}
Mari kita tunjukkan cara menggunakan EMT sebagai kalkulator yang sangat
canggih.

EMT berorientasi pada baris perintah. Anda dapat menuliskan satu atau lebih
perintah dalam satu baris perintah. Setiap perintah harus diakhiri dengan koma
atau titik koma.

- Titik koma menyembunyikan output (hasil) dari perintah.\\
- Sebuah koma mencetak hasilnya.\\
- Setelah perintah terakhir, koma diasumsikan secara otomatis (boleh tidak
ditulis).

Dalam contoh berikut, kita mendefinisikan variabel r yang diberi nilai 1,25.
Output dari definisi ini adalah nilai variabel. Tetapi karena tanda titik koma,
nilai ini tidak ditampilkan. Pada kedua perintah di belakangnya, hasil kedua
perhitungan tersebut ditampilkan.
\end{eulercomment}
\begin{eulerprompt}
>r=1.25; pi*r^2, 2*pi*r
\end{eulerprompt}
\begin{euleroutput}
  4.90873852123
  7.85398163397
\end{euleroutput}
\eulersubheading{Latihan untuk Anda}
\begin{eulercomment}
- Sisipkan beberapa baris perintah baru\\
- Tulis perintah-perintah baru untuk melakukan suatu perhitungan yang
Anda inginkan, boleh menggunakan variabel, boleh tanpa variabel.\\
\end{eulercomment}
\eulersubheading{}
\begin{eulerprompt}
> 25^2
\end{eulerprompt}
\begin{euleroutput}
  625
\end{euleroutput}
\begin{eulerprompt}
> (5^2)
\end{eulerprompt}
\begin{euleroutput}
  25
\end{euleroutput}
\begin{eulerprompt}
> (5^3)*(6^3)
\end{eulerprompt}
\begin{euleroutput}
  27000
\end{euleroutput}
\begin{eulercomment}
Penjelasan :\\
Cara agar dapat menambah komentar adalah dengan cara menekan
F5,setelah itu klik 'OK'. Dan cara untuk menambah prtintah adalah
dengan menekan (Shift+Ctrl+Enter).\\
\end{eulercomment}
\eulersubheading{}
\begin{eulercomment}
Beberapa catatan yang harus Anda perhatikan tentang penulisan sintaks
perintah EMT.

- Pastikan untuk menggunakan titik desimal, bukan koma desimal untuk
bilangan!\\
- Gunakan * untuk perkalian dan \textasciicircum{} untuk eksponen (pangkat).\\
- Seperti biasa, * dan / bersifat lebih kuat daripada + atau -.\\
- \textasciicircum{} mengikat lebih kuat dari *, sehingga pi * r \textasciicircum{} 2 merupakan rumus
luas lingkaran.\\
- Jika perlu, Anda harus menambahkan tanda kurung, seperti pada 2 \textasciicircum{} (2
\textasciicircum{} 3).

Perintah r = 1.25 adalah menyimpan nilai ke variabel di EMT. Anda juga
dapat menulis r: = 1.25 jika mau. Anda dapat menggunakan spasi sesuka
Anda.

Anda juga dapat mengakhiri baris perintah dengan komentar yang diawali
dengan dua garis miring (//).
\end{eulercomment}
\begin{eulerprompt}
> r := 1.25 // Komentar: Menggunakan  := sebagai ganti =
\end{eulerprompt}
\begin{euleroutput}
  1.25
\end{euleroutput}
\begin{eulercomment}
Argumen atau input untuk fungsi ditulis di dalam tanda kurung.
\end{eulercomment}
\begin{eulerprompt}
>sin(45°), cos(pi), log(sqrt(E))
\end{eulerprompt}
\begin{euleroutput}
  0.707106781187
  -1
  0.5
\end{euleroutput}
\begin{eulercomment}
Seperti yang Anda lihat, fungsi trigonometri bekerja dengan radian, dan derajat
dapat diubah dengan °. Jika keyboard Anda tidak memiliki karakter derajat tekan
[F7], atau gunakan fungsi deg() untuk mengonversi.

EMT menyediakan banyak sekali fungsi dan operator matematika.Hampir semua fungsi
matematika sudah tersedia di EMT. Anda dapat melihat daftar lengkap fungsi-fungsi
matematika di EMT pada berkas Referensi (klik menu Help -\textgreater{} Reference)

Untuk membuat rangkaian komputasi lebih mudah, Anda dapat merujuk ke hasil
sebelumnya dengan "\%". Cara ini sebaiknya hanya digunakan untuk merujuk hasil
perhitungan dalam baris perintah yang sama.
\end{eulercomment}
\begin{eulerprompt}
>(sqrt(5)+1)/2, %^2-%+1 // Memeriksa solusi x^2-x+1=0
\end{eulerprompt}
\begin{euleroutput}
  1.61803398875
  2
\end{euleroutput}
\begin{eulerprompt}
>p=23423429; isprime(p)
\end{eulerprompt}
\begin{euleroutput}
  1
\end{euleroutput}
\begin{eulerprompt}
>x=3423; \{g,a,b\}=gcdext(x,p); g, a*x+b*p,
\end{eulerprompt}
\begin{euleroutput}
  1
  1
\end{euleroutput}
\begin{eulerprompt}
>mod(a*x,p)
\end{eulerprompt}
\begin{euleroutput}
  1
\end{euleroutput}
\eulersubheading{Latihan untuk Anda}
\begin{eulercomment}
- Buka berkas Reference dan baca fungsi-fungsi matematika yang
tersedia di EMT.\\
- Sisipkan beberapa baris perintah baru.\\
- Lakukan contoh-contoh perhitungan menggunakan fungsi-fungsi
matematika di EMT.\\
\end{eulercomment}
\eulersubheading{}
\begin{eulerprompt}
> nonzeros(isprime(1:200))
\end{eulerprompt}
\begin{euleroutput}
  [2,  3,  5,  7,  11,  13,  17,  19,  23,  29,  31,  37,  41,  43,  47,
  53,  59,  61,  67,  71,  73,  79,  83,  89,  97,  101,  103,  107,
  109,  113,  127,  131,  137,  139,  149,  151,  157,  163,  167,  173,
  179,  181,  191,  193,  197,  199]
\end{euleroutput}
\eulersubheading{}
\begin{eulerprompt}
>1miles  // 1 mil = 1609,344 m
\end{eulerprompt}
\begin{euleroutput}
  1609.344
\end{euleroutput}
\begin{eulercomment}
Beberapa satuan yang sudah dikenal di dalam EMT adalah sebagai
berikut. Semua unit diakhiri dengan tanda dolar (\textdollar{}), namun boleh tidak
perlu ditulis dengan mengaktifkan easyunits. 

kilometer\textdollar{}:=1000;\\
km\textdollar{}:=kilometer\textdollar{};\\
cm\textdollar{}:=0.01;\\
mm\textdollar{}:=0.001;\\
minute\textdollar{}:=60;\\
min\textdollar{}:=minute\textdollar{};\\
minutes\textdollar{}:=minute\textdollar{};\\
hour\textdollar{}:=60*minute\textdollar{};\\
h\textdollar{}:=hour\textdollar{};\\
hours\textdollar{}:=hour\textdollar{};\\
day\textdollar{}:=24*hour\textdollar{};\\
days\textdollar{}:=day\textdollar{};\\
d\textdollar{}:=day\textdollar{};\\
year\textdollar{}:=365.2425*day\textdollar{};\\
years\textdollar{}:=year\textdollar{};\\
y\textdollar{}:=year\textdollar{};\\
inch\textdollar{}:=0.0254;\\
in\textdollar{}:=inch\textdollar{};\\
feet\textdollar{}:=12*inch\textdollar{};\\
foot\textdollar{}:=feet\textdollar{};\\
ft\textdollar{}:=feet\textdollar{};\\
yard\textdollar{}:=3*feet\textdollar{};\\
yards\textdollar{}:=yard\textdollar{};\\
yd\textdollar{}:=yard\textdollar{};\\
mile\textdollar{}:=1760*yard\textdollar{};\\
miles\textdollar{}:=mile\textdollar{};\\
kg\textdollar{}:=1;\\
sec\textdollar{}:=1;\\
ha\textdollar{}:=10000;\\
Ar\textdollar{}:=100;\\
Tagwerk\textdollar{}:=3408;\\
Acre\textdollar{}:=4046.8564224;\\
pt\textdollar{}:=0.376mm;

Untuk konversi ke dan antar unit, EMT menggunakan operator khusus,
yakni -\textgreater{}.
\end{eulercomment}
\begin{eulerprompt}
>4km -> miles, 4inch -> " mm"
\end{eulerprompt}
\begin{euleroutput}
  2.48548476895
  101.6 mm
\end{euleroutput}
\eulerheading{Format Tampilan Nilai}
\begin{eulercomment}
Akurasi internal untuk nilai bilangan di EMT adalah standar IEEE,
sekitar 16 digit desimal. Aslinya, EMT tidak mencetak semua digit
suatu bilangan. Ini untuk menghemat tempat dan agar terlihat lebih
baik. Untuk mengatrtamilan satu bilangan, operator berikut dapat
digunakan.

\end{eulercomment}
\begin{eulerprompt}
>pi
\end{eulerprompt}
\begin{euleroutput}
  3.14159265359
\end{euleroutput}
\begin{eulerprompt}
>longest pi
\end{eulerprompt}
\begin{euleroutput}
        3.141592653589793 
\end{euleroutput}
\begin{eulerprompt}
>long pi
\end{eulerprompt}
\begin{euleroutput}
  3.14159265359
\end{euleroutput}
\begin{eulerprompt}
>short pi
\end{eulerprompt}
\begin{euleroutput}
  3.1416
\end{euleroutput}
\begin{eulerprompt}
>shortest pi
\end{eulerprompt}
\begin{euleroutput}
     3.1 
\end{euleroutput}
\begin{eulerprompt}
>fraction pi
\end{eulerprompt}
\begin{euleroutput}
  312689/99532
\end{euleroutput}
\begin{eulerprompt}
>short 1200*1.03^10, long E, longest pi
\end{eulerprompt}
\begin{euleroutput}
  1612.7
  2.71828182846
        3.141592653589793 
\end{euleroutput}
\begin{eulercomment}
Format aslinya untuk menampilkan nilai menggunakan sekitar 10 digit.
Format tampilan nilai dapat diatur secara global atau hanya untuk satu
nilai.

Anda dapat mengganti format tampilan bilangan untuk semua perintah
selanjutnya. Untuk mengembalikan ke format aslinya dapat digunakan
perintah "defformat" atau "reset".
\end{eulercomment}
\begin{eulerprompt}
>longestformat; pi, defformat; pi
\end{eulerprompt}
\begin{euleroutput}
  3.141592653589793
  3.14159265359
\end{euleroutput}
\begin{eulercomment}
Kernel numerik EMT bekerja dengan bilangan titik mengambang (floating point)
dalam presisi ganda IEEE (berbeda dengan bagian simbolik EMT). Hasil numerik
dapat ditampilkan dalam bentuk pecahan.
\end{eulercomment}
\begin{eulerprompt}
>1/7+1/4, fraction %
\end{eulerprompt}
\begin{euleroutput}
  0.392857142857
  11/28
\end{euleroutput}
\eulerheading{Perintah Multibaris}
\begin{eulercomment}
Perintah multi-baris membentang di beberapa baris yang terhubung
dengan "..." di setiap akhir baris, kecuali baris terakhir. Untuk
menghasilkan tanda pindah baris tersebut, gunakan tombol
[Ctrl]+[Enter]. Ini akan menyambung perintah ke baris berikutnya dan
menambahkan "..." di akhir baris sebelumnya. Untuk menggabungkan suatu
baris ke baris sebelumnya, gunakan [Ctrl]+[Backspace].

Contoh perintah multi-baris berikut dapat dijalankan setiap kali
kursor berada di salah satu barisnya. Ini juga menunjukkan bahwa ...
harus berada di akhir suatu baris meskipun baris tersebut memuat
komentar.
\end{eulercomment}
\begin{eulerprompt}
>a=4; b=15; c=2; // menyelesaikan a*x^2+b*x+c=0 secara manual ...
>D=sqrt(b^2/(a^2*4)-c/a); ...
>-b/(2*a) + D, ...
>-b/(2*a) - D
\end{eulerprompt}
\begin{euleroutput}
  -0.138444501319
  -3.61155549868
\end{euleroutput}
\eulerheading{Menampilkan Daftar Variabe}
\begin{eulercomment}
Untuk menampilkan semua variabel yang sudah pernah Anda definisikan
sebelumnya (dan dapat dilihat kembali nilainya), gunakan perintah
"listvar".
\end{eulercomment}
\begin{eulerprompt}
>listvar
\end{eulerprompt}
\begin{euleroutput}
  r                   1.25
  p                   23423429
  x                   3423
  g                   1
  a                   4
  b                   15
  c                   2
  D                   1.73655549868123
\end{euleroutput}
\begin{eulercomment}
Perintah listvar hanya menampilkan variabel buatan pengguna.
Dimungkinkan untuk menampilkan variabel lain, dengan menambahkan
string  termuat di dalam nama variabel yang diinginkan.

Perlu Anda perhatikan, bahwa EMT membedakan huruf besar dan huruf
kecil. Jadi variabel "d" berbeda dengan variabel "D".

Contoh berikut ini menampilkan semua unit yang diakhiri dengan "m"
dengan mencari semua variabel yang berisi "m\textdollar{}".
\end{eulercomment}
\begin{eulerprompt}
>listvar m$
\end{eulerprompt}
\begin{euleroutput}
  km$                 1000
  cm$                 0.01
  mm$                 0.001
  nm$                 1853.24496
  gram$               0.001
  m$                  1
  hquantum$           6.62606957e-34
  atm$                101325
\end{euleroutput}
\begin{eulercomment}
Untuk menghapus variabel tanpa harus memulai ulang EMT gunakan
perintah "remvalue".
\end{eulercomment}
\begin{eulerprompt}
>remvalue a,b,c,D
>D
\end{eulerprompt}
\begin{euleroutput}
  Variable D not found!
  Error in:
  D ...
   ^
\end{euleroutput}
\eulerheading{Menampilkan Panduan}
\begin{eulercomment}
Untuk mendapatkan panduan tentang penggunaan perintah atau fungsi di EMT, buka
jendela panduan dengan menekan [F1] dan cari fungsinya. Anda juga dapat
mengklik dua kali pada fungsi yang tertulis di baris perintah atau di teks
untuk membuka jendela panduan.

Coba klik dua kali pada perintah "intrandom" berikut ini!
\end{eulercomment}
\begin{eulerprompt}
>intrandom(10,6)
\end{eulerprompt}
\begin{euleroutput}
  [4,  2,  6,  2,  4,  2,  3,  2,  2,  6]
\end{euleroutput}
\begin{eulercomment}
Di jendela panduan, Anda dapat mengklik kata apa saja untuk menemukan
referensi atau fungsi.

Misalnya, coba klik kata "random" di jendela panduan. Kata tersebut
boleh ada dalam teks atau di bagian "See:" pada panduan. Anda akan
menemukan penjelasan fungsi "random", untuk menghasilkan bilangan acak
berdistribusi uniform antara 0,0 dan 1,0. Dari panduan untuk "random"
Anda dapat menampilkan panduan untuk fungsi "normal", dll.
\end{eulercomment}
\begin{eulerprompt}
>random(10)
\end{eulerprompt}
\begin{euleroutput}
  [0.270906,  0.704419,  0.217693,  0.445363,  0.308411,  0.914541,
  0.193585,  0.463387,  0.095153,  0.595017]
\end{euleroutput}
\begin{eulerprompt}
>normal(10)
\end{eulerprompt}
\begin{euleroutput}
  [-0.495418,  1.6463,  -0.390056,  -1.98151,  3.44132,  0.308178,
  -0.733427,  -0.526167,  1.10018,  0.108453]
\end{euleroutput}
\eulerheading{Matriks dan Vektor}
\begin{eulercomment}
EMT merupakan suatu aplikasi matematika yang mengerti "bahasa matriks". Artinya,
EMT menggunakan vektor dan matriks untuk perhitungan-perhitungan tingkat lanjut.
Suatu vektor atau matriks dapat didefinisikan dengan tanda kurung siku.
Elemen-elemennya dituliskan di dalam tanda kurung siku, antar elemen dalam satu
baris dipisahkan oleh koma(,), antar baris dipisahkan oleh titik koma (;).

Vektor dan matriks dapat diberi nama seperti variabel biasa.
\end{eulercomment}
\begin{eulerprompt}
>v=[4,5,6,3,2,1]
\end{eulerprompt}
\begin{euleroutput}
  [4,  5,  6,  3,  2,  1]
\end{euleroutput}
\begin{eulerprompt}
>A=[1,2,3;4,5,6;7,8,9]
\end{eulerprompt}
\begin{euleroutput}
              1             2             3 
              4             5             6 
              7             8             9 
\end{euleroutput}
\begin{eulercomment}
Karena EMT mengerti bahasa matriks, EMT memiliki kemampuan yang sangat canggih
untuk melakukan perhitungan matematis untuk masalah-masalah aljabar linier,
statistika, dan optimisasi.

Vektor juga dapat didefinisikan dengan menggunakan rentang nilai dengan interval
tertentu menggunakan tanda titik dua (:),seperti contoh berikut ini.
\end{eulercomment}
\begin{eulerprompt}
>c=1:5
\end{eulerprompt}
\begin{euleroutput}
  [1,  2,  3,  4,  5]
\end{euleroutput}
\begin{eulerprompt}
>w=0:0.1:1
\end{eulerprompt}
\begin{euleroutput}
  [0,  0.1,  0.2,  0.3,  0.4,  0.5,  0.6,  0.7,  0.8,  0.9,  1]
\end{euleroutput}
\begin{eulerprompt}
>mean(w^2)
\end{eulerprompt}
\begin{euleroutput}
  0.35
\end{euleroutput}
\eulerheading{Bilangan Kompleks}
\begin{eulercomment}
EMT juga dapat menggunakan bilangan kompleks. Tersedia banyak fungsi untuk
bilangan kompleks di EMT. Bilangan imaginer

\end{eulercomment}
\begin{eulerformula}
\[
i = \sqrt{-1}
\]
\end{eulerformula}
\begin{eulercomment}
dituliskan dengan huruf I (huruf besar I), namun akan ditampilkan dengan huruf i
(i kecil).

\end{eulercomment}
\begin{eulerttcomment}
  re(x) : bagian riil pada bilangan kompleks x.
  im(x) : bagian imaginer pada bilangan kompleks x.
  complex(x) : mengubah bilangan riil x menjadi bilangan kompleks.
  conj(x) : Konjugat untuk bilangan bilangan komplkes x.
  arg(x) : argumen (sudut dalam radian) bilangan kompleks x.
  real(x) : mengubah x menjadi bilangan riil.
\end{eulerttcomment}
\begin{eulercomment}

Apabila bagian imaginer x terlalu besar, hasilnya akan menampilkan pesan
kesalahan.

\end{eulercomment}
\begin{eulerttcomment}
  >sqrt(-1) // Error!
  >sqrt(complex(-1))
\end{eulerttcomment}
\begin{eulerprompt}
>z=2+3*I, re(z), im(z), conj(z), arg(z), deg(arg(z)), deg(arctan(3/2))
\end{eulerprompt}
\begin{euleroutput}
  2+3i
  2
  3
  2-3i
  0.982793723247
  56.309932474
  56.309932474
\end{euleroutput}
\begin{eulerprompt}
>deg(arg(I)) // 90°
\end{eulerprompt}
\begin{euleroutput}
  90
\end{euleroutput}
\begin{eulerprompt}
>sqrt(-1)
\end{eulerprompt}
\begin{euleroutput}
  Floating point error!
  Error in sqrt
  Error in:
  sqrt(-1) ...
          ^
\end{euleroutput}
\begin{eulerprompt}
>sqrt(complex(-1))
\end{eulerprompt}
\begin{euleroutput}
  0+1i
\end{euleroutput}
\begin{eulercomment}
EMT selalu menganggap semua hasil perhitungan berupa bilangan riil dan tidak
akan secara otomatis mengubah ke bilangan kompleks.

Jadi akar kuadrat -1 akan menghasilkan kesalahan, tetapi akar kuadrat kompleks
didefinisikan untuk bidang koordinat dengan cara seperti biasa. Untuk mengubah
bilangan riil menjadi kompleks, Anda dapat menambahkan 0i atau menggunakan
fungsi "complex".
\end{eulercomment}
\begin{eulerprompt}
>complex(-1), sqrt(%)
\end{eulerprompt}
\begin{euleroutput}
  -1+0i 
  0+1i
\end{euleroutput}
\eulerheading{Matematika Simbolik}
\begin{eulercomment}
EMT dapat melakukan perhitungan matematika simbolis (eksak) dengan bantuan
software Maxima. Software Maxima otomatis sudah terpasang di komputer Anda ketika
Anda memasang EMT. Meskipun demikian, Anda dapat juga memasang software Maxima
tersendiri (yang terpisah dengan instalasi Maxima di EMT).

Pengguna Maxima yang sudah mahir harus memperhatikan bahwa terdapat sedikit
perbedaan dalam sintaks antara sintaks asli Maxima dan sintaks ekspresi simbolik
di EMT.

Untuk melakukan perhitungan matematika simbolis di EMT, awali perintah Maxima
dengan tanda "\&". Setiap ekspresi yang dimulai dengan "\&" adalah ekspresi
simbolis dan dikerjakan oleh Maxima.
\end{eulercomment}
\begin{eulerprompt}
>&(a+b)^2
\end{eulerprompt}
\begin{euleroutput}
  
                                        2
                                 (b + a)
  
\end{euleroutput}
\begin{eulerprompt}
>&expand((a+b)^2), &factor(x^2+5*x+6)
\end{eulerprompt}
\begin{euleroutput}
  
                              2            2
                             b  + 2 a b + a
  
  
                             (x + 2) (x + 3)
  
\end{euleroutput}
\begin{eulerprompt}
>&solve(a*x^2+b*x+c,x) // rumus abc
\end{eulerprompt}
\begin{euleroutput}
  
                       2                         2
               - sqrt(b  - 4 a c) - b      sqrt(b  - 4 a c) - b
          [x = ----------------------, x = --------------------]
                        2 a                        2 a
  
\end{euleroutput}
\begin{eulerprompt}
>&(a^2-b^2)/(a+b), &ratsimp(%) // ratsimp menyederhanakan bentuk pecahan
\end{eulerprompt}
\begin{euleroutput}
  
                                  2    2
                                 a  - b
                                 -------
                                  b + a
  
  
                                  a - b
  
\end{euleroutput}
\begin{eulerprompt}
>10! // nilai faktorial (modus EMT)
\end{eulerprompt}
\begin{euleroutput}
  3628800
\end{euleroutput}
\begin{eulerprompt}
>&10! //nilai faktorial (simbolik dengan Maxima)
\end{eulerprompt}
\begin{euleroutput}
  
                                 3628800
  
\end{euleroutput}
\begin{eulercomment}
Untuk menggunakan perintah Maxima secara langsung (seperti perintah pada layar
Maxima) awali perintahnya dengan tanda "::" pada baris perintah EMT. Sintaks
Maxima disesuaikan dengan sintaks EMT (disebut "modus kompatibilitas").
\end{eulercomment}
\begin{eulerprompt}
>factor(1000) // mencari semua faktor 1000 (EMT)
\end{eulerprompt}
\begin{euleroutput}
  [2,  2,  2,  5,  5,  5]
\end{euleroutput}
\begin{eulerprompt}
>:: factor(1000) // faktorisasi prima 1000 (dengan Maxima) 
\end{eulerprompt}
\begin{euleroutput}
  
                                   3  3
                                  2  5
  
\end{euleroutput}
\begin{eulerprompt}
>:: factor(20!)
\end{eulerprompt}
\begin{euleroutput}
  
                          18  8  4  2
                         2   3  5  7  11 13 17 19
  
\end{euleroutput}
\begin{eulercomment}
Jika Anda sudah mahir menggunakan Maxima, Anda dapat menggunakan sintaks asli
perintah Maxima dengan menggunakan tanda ":::" untuk mengawali setiap perintah
Maxima di EMT. Perhatikan, harus ada spasi antara ":::" dan perintahnya.
\end{eulercomment}
\begin{eulerprompt}
>::: binomial(5,2); // nilai C(5,2)
\end{eulerprompt}
\begin{euleroutput}
  
                                    10
  
\end{euleroutput}
\begin{eulerprompt}
>::: binomial(m,4); // C(m,4)=m!/(4!(m-4)!)
\end{eulerprompt}
\begin{euleroutput}
  
                        (m - 3) (m - 2) (m - 1) m
                        -------------------------
                                   24
  
\end{euleroutput}
\begin{eulerprompt}
>::: trigexpand(cos(x+y)); // rumus cos(x+y)=cos(x) cos(y)-sin(x)sin(y) 
\end{eulerprompt}
\begin{euleroutput}
  
                      cos(x) cos(y) - sin(x) sin(y)
  
\end{euleroutput}
\begin{eulerprompt}
>::: trigexpand(sin(x+y));
\end{eulerprompt}
\begin{euleroutput}
  
                      cos(x) sin(y) + sin(x) cos(y)
  
\end{euleroutput}
\begin{eulerprompt}
>::: trigsimp(((1-sin(x)^2)*cos(x))/cos(x)^2+tan(x)*sec(x)^2) //menyederhanakan fungsi trigonometri
\end{eulerprompt}
\begin{euleroutput}
  
                                         4
                             sin(x) + cos (x)
                             ----------------
                                    3
                                 cos (x)
  
\end{euleroutput}
\begin{eulercomment}
Untuk menyimpan ekspresi simbolik ke dalam suatu variabel digunakan tanda "\&=".
\end{eulercomment}
\begin{eulerprompt}
>p1 &= (x^3+1)/(x+1)
\end{eulerprompt}
\begin{euleroutput}
  
                                   3
                                  x  + 1
                                  ------
                                  x + 1
  
\end{euleroutput}
\begin{eulerprompt}
>&ratsimp(p1)
\end{eulerprompt}
\begin{euleroutput}
  
                                 2
                                x  - x + 1
  
\end{euleroutput}
\begin{eulercomment}
Untuk mensubstitusikan suatu nilai ke dalam variabel dapat digunakan perintah
"with".
\end{eulercomment}
\begin{eulerprompt}
>&p1 with x=3 // (3^3+1)/(3+1)
\end{eulerprompt}
\begin{euleroutput}
  
                                    7
  
\end{euleroutput}
\begin{eulerprompt}
>&p1 with x=a+b, &ratsimp(%) //substitusi dengan variabel baru
\end{eulerprompt}
\begin{euleroutput}
  
                                      3
                               (b + a)  + 1
                               ------------
                                b + a + 1
  
  
                       2                  2
                      b  + (2 a - 1) b + a  - a + 1
  
\end{euleroutput}
\begin{eulerprompt}
>&diff(p1,x) //turunan p1 terhadap x
\end{eulerprompt}
\begin{euleroutput}
  
                                2      3
                             3 x      x  + 1
                             ----- - --------
                             x + 1          2
                                     (x + 1)
  
\end{euleroutput}
\begin{eulerprompt}
>&integrate(p1,x) // integral p1 terhadap x
\end{eulerprompt}
\begin{euleroutput}
  
                               3      2
                            2 x  - 3 x  + 6 x
                            -----------------
                                    6
  
\end{euleroutput}
\eulerheading{Tampilan Matematika Simbolik dengan LaTeX}
\begin{eulercomment}
Anda dapat menampilkan hasil perhitunagn simbolik secara lebih bagus
menggunakan LaTeX. Untuk melakukan hal ini, tambahkan tanda dolar (\textdollar{}) di depan
tanda \& pada setiap perintah Maxima.\\
Perhatikan, hal ini hanya dapat menghasilkan tampilan yang diinginkan apabila
komputer Anda sudah terpasang software LaTeX.
\end{eulercomment}
\begin{eulerprompt}
>$&(a+b)^2
\end{eulerprompt}
\begin{eulerformula}
\[
\left(b+a\right)^2
\]
\end{eulerformula}
\begin{eulerprompt}
>$&expand((a+b)^2), $&factor(x^2+5*x+6)
\end{eulerprompt}
\begin{eulerformula}
\[
b^2+2\,a\,b+a^2
\]
\end{eulerformula}
\begin{eulerformula}
\[
\left(x+2\right)\,\left(x+3\right)
\]
\end{eulerformula}
\begin{eulerprompt}
>$&solve(a*x^2+b*x+c,x) // rumus abc
\end{eulerprompt}
\begin{eulerformula}
\[
\left[ x=\frac{-\sqrt{b^2-4\,a\,c}-b}{2\,a} , x=\frac{\sqrt{b^2-4\,
 a\,c}-b}{2\,a} \right] 
\]
\end{eulerformula}
\begin{eulerprompt}
>$&(a^2-b^2)/(a+b), $&ratsimp(%)
\end{eulerprompt}
\begin{eulerformula}
\[
\frac{a^2-b^2}{b+a}
\]
\end{eulerformula}
\begin{eulerformula}
\[
a-b
\]
\end{eulerformula}
\eulerheading{Selamat Belajar dan Berlatih!}
\begin{eulercomment}
Baik, itulah sekilas pengantar penggunaan software EMT. Masih banyak
kemampuan EMT yang akan Anda pelajari dan praktikkan.

Sebagai latihan untuk memperlancar penggunaan perintah-perintah EMT
yang sudah dijelaskan di atas, silakan Anda lakukan hal-hal sebagai
berikut.

- Carilah soal-soal matematika dari buku-buku Matematika.\\
- Tambahkan beberapa baris perintah EMT pada notebook ini.\\
- Selesaikan soal-soal matematika tersebut dengan menggunakan EMT.\\
Pilih soal-soal yang sesuai dengan perintah-perintah yang sudah
dijelaskan dan dicontohkan di atas.\\
\end{eulercomment}
\eulersubheading{}
\begin{eulercomment}
\begin{eulercomment}
\eulerheading{contoh soal}
\begin{eulercomment}
Suatu kelas terdiri atas 22 siswa akan mengadakan studi banding.
Tersedia 3 kendaraan dengan daya tampung masing-masing 4,7,11. Ada
berapa cra mendistribusikan pasra siswa-siswa tersebut ke dalam
kendaraan yang tersedia?
\end{eulercomment}
\begin{eulerprompt}
>22!/(4!*7!*11!)
\end{eulerprompt}
\begin{euleroutput}
  232792560
\end{euleroutput}

\chapter{EMT untuk Perhitungan Aljabar}
\begin{eulercomment}
Pada notebook ini Anda belajar menggunakan EMT untuk melakukan
berbagai perhitungan terkait dengan materi atau topik dalam Aljabar.
Kegiatan yang harus Anda lakukan adalah sebagai berikut:

- Membaca secara cermat dan teliti notebook ini;\\
- Menerjemahkan teks bahasa Inggris ke bahasa Indonesia;\\
- Mencoba contoh-contoh perhitungan (perintah EMT) dengan cara
meng-ENTER setiap perintah EMT yang ada (pindahkan kursor ke baris
perintah)\\
- Jika perlu Anda dapat memodifikasi perintah yang ada dan memberikan
keterangan/penjelasan tambahan terkait hasilnya.\\
- Menyisipkan baris-baris perintah baru untuk mengerjakan soal-soal
Aljabar dari file PDF yang saya berikan;\\
- Memberi catatan hasilnya.\\
- Jika perlu tuliskan soalnya pada teks notebook (menggunakan format
LaTeX).\\
- Gunakan tampilan hasil semua perhitungan yang eksak atau simbolik
dengan format LaTeX. (Seperti contoh-contoh pada notebook ini.)

\end{eulercomment}
\eulersubheading{Contoh pertama}
\begin{eulercomment}
Menyederhanakan bentuk aljabar:

\end{eulercomment}
\begin{eulerformula}
\[
6x^{-3}y^5\times -7x^2y^{-9}
\]
\end{eulerformula}
\begin{eulercomment}
\end{eulercomment}
\begin{eulerprompt}
>$&6*x^(-3)*y^5*-7*x^2*y^(-9)
\end{eulerprompt}
\begin{eulerformula}
\[
-\frac{42}{x\,y^4}
\]
\end{eulerformula}
\begin{eulercomment}
Menjabarkan:

\end{eulercomment}
\begin{eulerformula}
\[
(6x^{-3}+y^5)(-7x^2-y^{-9})
\]
\end{eulerformula}
\begin{eulerprompt}
>$&showev('expand((6*x^(-3)+y^5)*(-7*x^2-y^(-9))))
\end{eulerprompt}
\begin{eulerformula}
\[
{\it expand}\left(\left(-\frac{1}{y^9}-7\,x^2\right)\,\left(y^5+  \frac{6}{x^3}\right)\right)=-7\,x^2\,y^5-\frac{1}{y^4}-\frac{6}{x^3  \,y^9}-\frac{42}{x}
\]
\end{eulerformula}
\begin{eulercomment}
\end{eulercomment}
\eulersubheading{Baris Perintah}
\begin{eulercomment}
Baris perintah Euler terdiri dari satu atau beberapa perintah Euler
diikuti dengan titik koma ";" atau koma ",". Titik koma mencegah
pencetakan hasilnya. Koma setelah perintah terakhir dapat dihilangkan.

Baris perintah berikut hanya akan mencetak hasil ekspresi, bukan tugas
atau perintah format.
\end{eulercomment}
\begin{eulerprompt}
>r:=2; h:=4; pi*r^2*h/3
\end{eulerprompt}
\begin{euleroutput}
  16.7551608191
\end{euleroutput}
\begin{eulercomment}
Perintah harus dipisahkan dengan yang kosong. Baris perintah berikut
mencetak kedua hasilnya.
\end{eulercomment}
\begin{eulerprompt}
>pi*2*r*h, %+2*pi*r*h // Ingat tanda % menyatakan hasil perhitungan terakhir sebelumnya
\end{eulerprompt}
\begin{euleroutput}
  50.2654824574
  100.530964915
\end{euleroutput}
\begin{eulercomment}
Baris perintah dijalankan sesuai urutan yang ditekan pengguna kembali.
Jadi, Anda mendapatkan nilai baru setiap kali Anda menjalankan baris
kedua.
\end{eulercomment}
\begin{eulerprompt}
>x := 1;
>x := cos(x) // nilai cosinus (x dalam radian)
\end{eulerprompt}
\begin{euleroutput}
  0.540302305868
\end{euleroutput}
\begin{eulerprompt}
>x := cos(x)
\end{eulerprompt}
\begin{euleroutput}
  0.857553215846
\end{euleroutput}
\begin{eulercomment}
Jika dua jalur dihubungkan dengan "..." kedua jalur akan selalu
dijalankan secara bersamaan.
\end{eulercomment}
\begin{eulerprompt}
>x := 1.5; ...
>x := (x+2/x)/2, x := (x+2/x)/2, x := (x+2/x)/2, 
\end{eulerprompt}
\begin{euleroutput}
  1.41666666667
  1.41421568627
  1.41421356237
\end{euleroutput}
\begin{eulercomment}
Ini juga merupakan cara yang baik untuk menyebarkan perintah panjang
ke dua baris atau lebih. Anda dapat menekan Ctrl+Return untuk\\
membagi baris menjadi dua pada posisi kursor saat ini, atau Ctlr+Back
untuk menggabungkan baris.

Untuk melipat semua multi-garis tekan Ctrl+L. Maka garis-garis
berikutnya hanya akan terlihat, jika salah satunya mendapat fokus.
Untuk melipat satu multi-baris, mulailah baris pertama dengan "\%+".
\end{eulercomment}
\begin{eulerprompt}
>%+ x=4+5; ...
\end{eulerprompt}
\begin{eulercomment}
Baris yang dimulai dengan \%\% tidak akan terlihat sama sekali.
\end{eulercomment}
\begin{euleroutput}
  81
\end{euleroutput}
\begin{eulercomment}
Euler mendukung loop di baris perintah, asalkan cocok ke dalam satu
baris atau multi-baris. Tentu saja, pembatasan ini tidak berlaku dalam\\
program. Untuk informasi lebih lanjut lihat pendahuluan berikut.

\end{eulercomment}
\begin{eulerprompt}
>x=1; for i=1 to 5; x := (x+2/x)/2, end; // menghitung akar 2
\end{eulerprompt}
\begin{euleroutput}
  1.5
  1.41666666667
  1.41421568627
  1.41421356237
  1.41421356237
\end{euleroutput}
\begin{eulercomment}
Tidak apa-apa menggunakan multi-baris. Pastikan baris diakhiri dengan
"...".

\end{eulercomment}
\begin{eulerprompt}
>x := 1.5; // comments go here before the ...
>repeat xnew:=(x+2/x)/2; until xnew~=x; ...
>   x := xnew; ...
>end; ...
>x,
\end{eulerprompt}
\begin{euleroutput}
  1.41421356237
\end{euleroutput}
\begin{eulercomment}
Struktur bersyarat juga berfungsi.
\end{eulercomment}
\begin{eulerprompt}
>if E^pi>pi^E; then "Thought so!", endif;
\end{eulerprompt}
\begin{euleroutput}
  Thought so!
\end{euleroutput}
\begin{eulercomment}
Saat Anda menjalankan perintah, kursor dapat berada di posisi mana pun
di baris perintah. Anda dapat kembali ke perintah sebelumnya\\
atau melompat ke perintah berikutnya dengan tombol panah. Atau Anda
dapat mengklik bagian komentar di atas perintah untuk membuka
perintah.

Saat Anda menggerakkan kursor di sepanjang garis, pasangan tanda
kurung atau tanda kurung buka dan tutup akan disorot. Juga, perhatikan
baris status. Setelah tanda kurung buka dari fungsi sqrt(), baris
status akan menampilkan teks bantuan untuk fungsi tersebut.\\
Jalankan perintah dengan kunci kembali.
\end{eulercomment}
\begin{eulerprompt}
>sqrt(sin(10°)/cos(20°))
\end{eulerprompt}
\begin{euleroutput}
  0.429875017772
\end{euleroutput}
\begin{eulercomment}
Untuk melihat bantuan untuk perintah terbaru, buka jendela bantuan
dengan F1. Di sana, Anda dapat memasukkan teks untuk\\
dicari. Pada baris kosong, bantuan untuk jendela bantuan akan
ditampilkan. Anda dapat menekan escape untuk menghapus garis,\\
atau untuk menutup jendela bantuan.

Anda dapat mengklik dua kali pada perintah apa pun untuk membuka
bantuan untuk perintah ini. Coba klik dua kali pada exp command di
bawah pada baris perintah.
\end{eulercomment}
\begin{eulerprompt}
>exp(log(2.5))
\end{eulerprompt}
\begin{euleroutput}
  2.5
\end{euleroutput}
\begin{eulercomment}
Anda juga dapat menyalin dan menempel di Euler. Gunakan Ctrl-C dan
Ctrl-V untuk ini. Untuk menandai teks, seret mouse atau gunakan shift\\
bersamaan dengan tombol kursor apa pun. Selain itu, Anda dapat
menyalin tanda kurung yang disorot.
\end{eulercomment}
\begin{eulercomment}

\end{eulercomment}
\eulersubheading{Sintaks Dasar}
\begin{eulercomment}
Euler mengetahui fungsi matematika biasa. Seperti yang Anda lihat di
atas, fungsi trigonometri bekerja dalam radian atau derajat. Untuk\\
mengonversi ke derajat, tambahkan simbol derajat (dengan tombol F7) ke
nilainya, atau gunakan fungsi rad(x). Fungsi akar kuadrat disebut\\
sqrt di Euler. Tentu saja, x\textasciicircum{}(1/2) juga dimungkinkan.

Untuk menyetel variabel, gunakan "=" atau ":=". Demi kejelasan,
pendahuluan ini menggunakan bentuk yang terakhir. Spasi tidak penting.\\
Tapi jarak antar perintah diharapkan.

Beberapa perintah dalam satu baris dipisahkan dengan "," atau ";".
Titik koma menekan keluaran perintah. Di akhir baris perintah, ","\\
diasumsikan, jika ";" hilang.
\end{eulercomment}
\begin{eulerprompt}
>g:=9.81; t:=2.5; 1/2*g*t^2
\end{eulerprompt}
\begin{euleroutput}
  30.65625
\end{euleroutput}
\begin{eulercomment}
EMT uses a programming syntax for expressions. To enter

\end{eulercomment}
\begin{eulerformula}
\[
e^2 \cdot \left( \frac{1}{3+4 \log(0.6)}+\frac{1}{7} \right)
\]
\end{eulerformula}
\begin{eulercomment}
you have to set the correct brackets and use / for fractions. Watch the highlighted
brackets for assistance. Note that the Euler constant e is named E in EMT.
\end{eulercomment}
\begin{eulerprompt}
>E^2*(1/(3+4*log(0.6))+1/7)
\end{eulerprompt}
\begin{euleroutput}
  8.77908249441
\end{euleroutput}
\begin{eulercomment}
To compute a complicate expression like

\end{eulercomment}
\begin{eulerformula}
\[
\left(\frac{\frac17 + \frac18 + 2}{\frac13 + \frac12}\right)^2 \pi
\]
\end{eulerformula}
\begin{eulercomment}
you need to enter it in line form.
\end{eulercomment}
\begin{eulerprompt}
>((1/7 + 1/8 + 2) / (1/3 + 1/2))^2 * pi
\end{eulerprompt}
\begin{euleroutput}
  23.2671801626
\end{euleroutput}
\begin{eulercomment}
Carefully put brackets around sub-expressions that need to be computed first. 
EMT assists you by highlighting the expression that the closing bracket finishes. 
You will also have to enter the name "pi" for the Greek letter pi.

The result of this computation is a floating point number. It is by
default printed with about 12 digits accuracy.
In the following command line, we also learn how we can refer to the previous
result within the same line.
\end{eulercomment}
\begin{eulerprompt}
>1/3+1/7, fraction %
\end{eulerprompt}
\begin{euleroutput}
  0.47619047619
  10/21
\end{euleroutput}
\begin{eulercomment}
An Euler command can be an expression or a primitive command. An expression
is made of operators and functions. If necessary, it must contain brackets to
force the correct order of execution. In doubt, setting a bracket is a good
idea. Note that EMT shows opening and closing brackets while editing the
command line.
\end{eulercomment}
\begin{eulerprompt}
>(cos(pi/4)+1)^3*(sin(pi/4)+1)^2
\end{eulerprompt}
\begin{euleroutput}
  14.4978445072
\end{euleroutput}
\begin{eulercomment}
The numerical operators of Euler include

\end{eulercomment}
\begin{eulerttcomment}
 + unary or operator plus
 - unary or operator minus
 *, /
 . the matrix product
 a^b power for positive a or integer b (a**b works too)
 n! the factorial operator
\end{eulerttcomment}
\begin{eulercomment}

and many more.

Here are some of the functions you might need. There are many more.

\end{eulercomment}
\begin{eulerttcomment}
 sin,cos,tan,atan,asin,acos,rad,deg
 log,exp,log10,sqrt,logbase
 bin,logbin,logfac,mod,floor,ceil,round,abs,sign
 conj,re,im,arg,conj,real,complex
 beta,betai,gamma,complexgamma,ellrf,ellf,ellrd,elle
 bitand,bitor,bitxor,bitnot
\end{eulerttcomment}
\begin{eulercomment}

Some commands have aliases, e.g. ln for log.
\end{eulercomment}
\begin{eulerprompt}
>ln(E^2), arctan(tan(0.5))
\end{eulerprompt}
\begin{euleroutput}
  2
  0.5
\end{euleroutput}
\begin{eulerprompt}
>sin(30°)
\end{eulerprompt}
\begin{euleroutput}
  0.5
\end{euleroutput}
\begin{eulercomment}
Make sure to use parentheses (round brackets), whenever there is doubt about
the order of execution! The following is not the same as (2\textasciicircum{}3)\textasciicircum{}4, which is
the default for 2\textasciicircum{}3\textasciicircum{}4 in EMT (some numerical systems do it the other way).
\end{eulercomment}
\begin{eulerprompt}
>2^3^4, (2^3)^4, 2^(3^4)
\end{eulerprompt}
\begin{euleroutput}
  2.41785163923e+24
  4096
  2.41785163923e+24
\end{euleroutput}
\eulersubheading{Real Numbers}
\begin{eulercomment}
The primary data type in Euler is the real number. Reals are
represented in IEEE format with about 16 decimal digits of accuracy.
\end{eulercomment}
\begin{eulerprompt}
>longest 1/3
\end{eulerprompt}
\begin{euleroutput}
       0.3333333333333333 
\end{euleroutput}
\begin{eulercomment}
The internal dual representation takes 8 bytes.
\end{eulercomment}
\begin{eulerprompt}
>printdual(1/3)
\end{eulerprompt}
\begin{euleroutput}
  1.0101010101010101010101010101010101010101010101010101*2^-2
\end{euleroutput}
\begin{eulerprompt}
>printhex(1/3)
\end{eulerprompt}
\begin{euleroutput}
  5.5555555555554*16^-1
\end{euleroutput}
\eulersubheading{Strings}
\begin{eulercomment}
A string in Euler is defined with "...".
\end{eulercomment}
\begin{eulerprompt}
>"A string can contain anything."
\end{eulerprompt}
\begin{euleroutput}
  A string can contain anything.
\end{euleroutput}
\begin{eulercomment}
Strings can be concatenated with \textbar{} or with +. This also works with numbers,
which are converted to strings in that case.
\end{eulercomment}
\begin{eulerprompt}
>"The area of the circle with radius " + 2 + " cm is " + pi*4 + " cm^2."
\end{eulerprompt}
\begin{euleroutput}
  The area of the circle with radius 2 cm is 12.5663706144 cm^2.
\end{euleroutput}
\begin{eulercomment}
The print function does also convert a number to a string. It can take a
number of digits and a number of places (0 for dense output), and optimally a
unit.
\end{eulercomment}
\begin{eulerprompt}
>"Golden Ratio : " + print((1+sqrt(5))/2,5,0)
\end{eulerprompt}
\begin{euleroutput}
  Golden Ratio : 1.61803
\end{euleroutput}
\begin{eulercomment}
There is a special string none, which does not print. It is returned by some
functions, when the result does not matter. (It is returned automatically, if
the function does not have a return statement.)
\end{eulercomment}
\begin{eulerprompt}
>none
\end{eulerprompt}
\begin{eulercomment}
To convert a string to a number simply evaluate it. This works for
expressions too (see below).
\end{eulercomment}
\begin{eulerprompt}
>"1234.5"()
\end{eulerprompt}
\begin{euleroutput}
  1234.5
\end{euleroutput}
\begin{eulercomment}
To define a string vector, use the vector [...] notation.
\end{eulercomment}
\begin{eulerprompt}
>v:=["affe","charlie","bravo"]
\end{eulerprompt}
\begin{euleroutput}
  affe
  charlie
  bravo
\end{euleroutput}
\begin{eulercomment}
The empty string vector is denoted by [none]. String vectors can be
concatenated.
\end{eulercomment}
\begin{eulerprompt}
>w:=[none]; w|v|v
\end{eulerprompt}
\begin{euleroutput}
  affe
  charlie
  bravo
  affe
  charlie
  bravo
\end{euleroutput}
\begin{eulercomment}
Strings can contain Unicode characters. Internally, these strings contain
UTF-8 code. To generate such a string, use u"..." and one of the HTML
entities.

Unicode strings can be concatenated like other strings.
\end{eulercomment}
\begin{eulerprompt}
>u"&alpha; = " + 45 + u"&deg;" // pdfLaTeX mungkin gagal menampilkan secara benar
\end{eulerprompt}
\begin{euleroutput}
  α = 45°
\end{euleroutput}
\begin{eulercomment}
I
\end{eulercomment}
\begin{eulercomment}
In comments, the same entities like α, β etc. can be used. This may be
a quick alternative to Latex. (More details on comments below).
\end{eulercomment}
\begin{eulercomment}
There are some functions to create or analyze unicode strings. The function
strtochar() will recognize Unicode strings, and translate them correctly.
\end{eulercomment}
\begin{eulerprompt}
>v=strtochar(u"&Auml; is a German letter")
\end{eulerprompt}
\begin{euleroutput}
  [196,  32,  105,  115,  32,  97,  32,  71,  101,  114,  109,  97,  110,
  32,  108,  101,  116,  116,  101,  114]
\end{euleroutput}
\begin{eulercomment}
The result is a vector of Unicode numbers. The converse function is
chartoutf().
\end{eulercomment}
\begin{eulerprompt}
>v[1]=strtochar(u"&Uuml;")[1]; chartoutf(v)
\end{eulerprompt}
\begin{euleroutput}
  Ü is a German letter
\end{euleroutput}
\begin{eulercomment}
The function utf() can translate a string with entities in a variable into a
Unicode string.
\end{eulercomment}
\begin{eulerprompt}
>s="We have &alpha;=&beta;."; utf(s) // pdfLaTeX mungkin gagal menampilkan secara benar
\end{eulerprompt}
\begin{euleroutput}
  We have α=β.
\end{euleroutput}
\begin{eulercomment}
It is also possible to use numerical entities.
\end{eulercomment}
\begin{eulerprompt}
>u"&#196;hnliches"
\end{eulerprompt}
\begin{euleroutput}
  Ähnliches
\end{euleroutput}
\eulersubheading{Boolean Values}
\begin{eulercomment}
Boolean values are represented with 1=true or 0=false in Euler.
Strings can be compared, just like numbers.
\end{eulercomment}
\begin{eulerprompt}
>2<1, "apel"<"banana"
\end{eulerprompt}
\begin{euleroutput}
  0
  1
\end{euleroutput}
\begin{eulercomment}
"and" is the operator "\&\&" and "or" is the operator "\textbar{}\textbar{}", as in the C
language. (The words "and" and "or" can only be used in conditions for "if".)
\end{eulercomment}
\begin{eulerprompt}
>2<E && E<3
\end{eulerprompt}
\begin{euleroutput}
  1
\end{euleroutput}
\begin{eulercomment}
Boolean operators obey the rules of the matrix language.
\end{eulercomment}
\begin{eulerprompt}
>(1:10)>5, nonzeros(%)
\end{eulerprompt}
\begin{euleroutput}
  [0,  0,  0,  0,  0,  1,  1,  1,  1,  1]
  [6,  7,  8,  9,  10]
\end{euleroutput}
\begin{eulercomment}
You can use the function nonzeros() to extract specific elements form a
vector. In the example, we use the conditional isprime(n).
\end{eulercomment}
\begin{eulerprompt}
>N=2|3:2:99 // N berisi elemen 2 dan bilangan2 ganjil dari 3 s.d. 99
\end{eulerprompt}
\begin{euleroutput}
  [2,  3,  5,  7,  9,  11,  13,  15,  17,  19,  21,  23,  25,  27,  29,
  31,  33,  35,  37,  39,  41,  43,  45,  47,  49,  51,  53,  55,  57,
  59,  61,  63,  65,  67,  69,  71,  73,  75,  77,  79,  81,  83,  85,
  87,  89,  91,  93,  95,  97,  99]
\end{euleroutput}
\begin{eulerprompt}
>N[nonzeros(isprime(N))] //pilih anggota2 N yang prima
\end{eulerprompt}
\begin{euleroutput}
  [2,  3,  5,  7,  11,  13,  17,  19,  23,  29,  31,  37,  41,  43,  47,
  53,  59,  61,  67,  71,  73,  79,  83,  89,  97]
\end{euleroutput}
\eulersubheading{Output Formats}
\begin{eulercomment}
The default output format of EMT prints 12 digits. To make sure that
we see the default, we reset the format.
\end{eulercomment}
\begin{eulerprompt}
>defformat; pi
\end{eulerprompt}
\begin{euleroutput}
  3.14159265359
\end{euleroutput}
\begin{eulercomment}
Internally, EMT uses the IEEE standard for double numbers with about 16
decimal digits. To see the full number of digits, use the command
"longestformat", or we use the operator "longest" to display the result in
the longest format.
\end{eulercomment}
\begin{eulerprompt}
>longest pi
\end{eulerprompt}
\begin{euleroutput}
        3.141592653589793 
\end{euleroutput}
\begin{eulercomment}
Here is the internal hexadecimal representation of a double number.
\end{eulercomment}
\begin{eulerprompt}
>printhex(pi)
\end{eulerprompt}
\begin{euleroutput}
  3.243F6A8885A30*16^0
\end{euleroutput}
\begin{eulercomment}
The output format can be changed permanently with a format command.
\end{eulercomment}
\begin{eulerprompt}
>format(12,5); 1/3, pi, sin(1)
\end{eulerprompt}
\begin{euleroutput}
      0.33333 
      3.14159 
      0.84147 
\end{euleroutput}
\begin{eulercomment}
The default is format(12).
\end{eulercomment}
\begin{eulerprompt}
>format(12); 1/3
\end{eulerprompt}
\begin{euleroutput}
  0.333333333333
\end{euleroutput}
\begin{eulercomment}
Functions like "shortestformat", "shortformat", "longformat" work for vectors
in the following way.
\end{eulercomment}
\begin{eulerprompt}
>shortestformat; random(3,8)
\end{eulerprompt}
\begin{euleroutput}
    0.66    0.2   0.89   0.28   0.53   0.31   0.44    0.3 
    0.28   0.88   0.27    0.7   0.22   0.45   0.31   0.91 
    0.19   0.46  0.095    0.6   0.43   0.73   0.47   0.32 
\end{euleroutput}
\begin{eulercomment}
The default format for scalars is format(12). But this can be changed.
\end{eulercomment}
\begin{eulerprompt}
>setscalarformat(5); pi
\end{eulerprompt}
\begin{euleroutput}
  3.1416
\end{euleroutput}
\begin{eulercomment}
The function "longestformat" set the scalar format too.
\end{eulercomment}
\begin{eulerprompt}
>longestformat; pi
\end{eulerprompt}
\begin{euleroutput}
  3.141592653589793
\end{euleroutput}
\begin{eulercomment}
For reference, here is a list of the most important output formats.

\end{eulercomment}
\begin{eulerttcomment}
 shortestformat shortformat longformat, longestformat
 format(length,digits) goodformat(length)
 fracformat(length)
 defformat
\end{eulerttcomment}
\begin{eulercomment}

The internal accuracy of EMT is about 16 decimal places, which is the IEEE
standard. Numbers are stored in this internal format.

But the output format of EMT can be set in a flexible way.
\end{eulercomment}
\begin{eulerprompt}
>longestformat; pi,
\end{eulerprompt}
\begin{euleroutput}
  3.141592653589793
\end{euleroutput}
\begin{eulerprompt}
>format(10,5); pi
\end{eulerprompt}
\begin{euleroutput}
    3.14159 
\end{euleroutput}
\begin{eulercomment}
The default is defformat().
\end{eulercomment}
\begin{eulerprompt}
>defformat; // default
\end{eulerprompt}
\begin{eulercomment}
There are short operators which print only one value. The operator "longest"
will print all valid digits of a number.
\end{eulercomment}
\begin{eulerprompt}
>longest pi^2/2
\end{eulerprompt}
\begin{euleroutput}
        4.934802200544679 
\end{euleroutput}
\begin{eulercomment}
There is also a short operator for printing a result in fractional format. We
have already used it above.
\end{eulercomment}
\begin{eulerprompt}
>fraction 1+1/2+1/3+1/4
\end{eulerprompt}
\begin{euleroutput}
  25/12
\end{euleroutput}
\begin{eulercomment}
Since the internal format uses a binary way to store numbers, the value 0.1
will not be represented exactly. The error adds up a bit, as you see in the
following computation.
\end{eulercomment}
\begin{eulerprompt}
>longest 0.1+0.1+0.1+0.1+0.1+0.1+0.1+0.1+0.1+0.1-1
\end{eulerprompt}
\begin{euleroutput}
   -1.110223024625157e-16 
\end{euleroutput}
\begin{eulercomment}
But with the default "longformat" you will not notice this. For convenience,
the output of very small numbers is 0.
\end{eulercomment}
\begin{eulerprompt}
>0.1+0.1+0.1+0.1+0.1+0.1+0.1+0.1+0.1+0.1-1
\end{eulerprompt}
\begin{euleroutput}
  0
\end{euleroutput}
\eulerheading{Expressions}
\begin{eulercomment}
Strings or names can be used to store mathematical expressions, which can be evaluated
by EMT. For this, use parentheses after the expression. If you intend to use a string
as an expression, use the convention to name it "fx" or "fxy" etc. Expressions take
precedence over functions.

Global variables can be used in the evaluation.
\end{eulercomment}
\begin{eulerprompt}
>r:=2; fx:="pi*r^2"; longest fx()
\end{eulerprompt}
\begin{euleroutput}
        12.56637061435917 
\end{euleroutput}
\begin{eulercomment}
Parameters are assigned to x, y, and z in that order. Additional parameters
can be added using assigned parameters.
\end{eulercomment}
\begin{eulerprompt}
>fx:="a*sin(x)^2"; fx(5,a=-1)
\end{eulerprompt}
\begin{euleroutput}
  -0.919535764538
\end{euleroutput}
\begin{eulercomment}
Note that expression will always use global variables, even if there is a
variable in a function with the same name. (Otherwise the evaluation of
expressions in functions could have very confusing results for the user that
called the function.)
\end{eulercomment}
\begin{eulerprompt}
>at:=4; function f(expr,x,at) := expr(x); ...
>f("at*x^2",3,5) // computes 4*3^2 not 5*3^2
\end{eulerprompt}
\begin{euleroutput}
  36
\end{euleroutput}
\begin{eulercomment}
If you want to use another value for "at" than the global value you need to
add "at=value".
\end{eulercomment}
\begin{eulerprompt}
>at:=4; function f(expr,x,a) := expr(x,at=a); ...
>f("at*x^2",3,5)
\end{eulerprompt}
\begin{euleroutput}
  45
\end{euleroutput}
\begin{eulercomment}
For reference, we remark that call collections (discussed elsewhere) can
contain expressions. So we can make the above example as follows.
\end{eulercomment}
\begin{eulerprompt}
>at:=4; function f(expr,x) := expr(x); ...
>f(\{\{"at*x^2",at=5\}\},3)
\end{eulerprompt}
\begin{euleroutput}
  45
\end{euleroutput}
\begin{eulercomment}
Expressions in x are often used just like functions.\\
Note that defining a function with the same name like a global symbolic
expression deletes this variable to avoid confusion between symbolic
expressions and functions.
\end{eulercomment}
\begin{eulerprompt}
>f &= 5*x;
>function f(x) := 6*x;
>f(2)
\end{eulerprompt}
\begin{euleroutput}
  12
\end{euleroutput}
\begin{eulercomment}
By way of convention, symbolic or numerical expressions should be named fx,
fxy etc. This naming scheme should not be used for functions.
\end{eulercomment}
\begin{eulerprompt}
>fx &= diff(x^x,x); $&fx
\end{eulerprompt}
\begin{eulerformula}
\[
x^{x}\,\left(\log x+1\right)
\]
\end{eulerformula}
\begin{eulercomment}
A special form of an expression allows any variable as an unnamed parameter
to the evaluation of the expression, not just "x", "y" etc. For this, start
the expression with "@(variables) ...".
\end{eulercomment}
\begin{eulerprompt}
>"@(a,b) a^2+b^2", %(4,5)
\end{eulerprompt}
\begin{euleroutput}
  @(a,b) a^2+b^2
  41
\end{euleroutput}
\begin{eulercomment}
This allows to manipulate expressions in other variables for functions of EMT
which need an expression in "x".

The most elementary way to define a simple function is to store its formula
in a symbolic or numerical expression. If the main variable is x, the
expression can be evaluated just like a function.

As you see in the following example, global variables are visible during the
evaluation.
\end{eulercomment}
\begin{eulerprompt}
>fx &= x^3-a*x;  ...
>a=1.2; fx(0.5)
\end{eulerprompt}
\begin{euleroutput}
  -0.475
\end{euleroutput}
\begin{eulercomment}
All other variables in the expression can be specified in the evaluation
using an assigned parameter.
\end{eulercomment}
\begin{eulerprompt}
>fx(0.5,a=1.1)
\end{eulerprompt}
\begin{euleroutput}
  -0.425
\end{euleroutput}
\begin{eulercomment}
An expression needs not be symbolic. This is necessary, if the expression
contains functions, which are only known in the numerical kernel, not in
Maxima.

\begin{eulercomment}
\eulerheading{Symbolic Mathematics}
\begin{eulercomment}
EMT does symbolic math with the help of Maxima. For details, start with the
following tutorial, or browse the reference for Maxima. Experts in Maxima
should note that there are differences in the syntax between the original
syntax of Maxima and the default syntax of symbolic expressions in EMT.

Symbolic math is integrated seamlessly into Euler with \&. Any expression
starting with \& is a symbolic expression. It is evaluated and printed by
Maxima.

First of all, Maxima has an "infinite" arithmetic which can handle
very large numbers.
\end{eulercomment}
\begin{eulerprompt}
>$&44!
\end{eulerprompt}
\begin{eulerformula}
\[
2658271574788448768043625811014615890319638528000000000
\]
\end{eulerformula}
\begin{eulercomment}
This way, you can compute large results exactly. Let us compute

\end{eulercomment}
\begin{eulerformula}
\[
C(44,10) = \frac{44!}{34! \cdot 10!}
\]
\end{eulerformula}
\begin{eulerprompt}
>$& 44!/(34!*10!) // nilai C(44,10)
\end{eulerprompt}
\begin{eulerformula}
\[
2481256778
\]
\end{eulerformula}
\begin{eulercomment}
Of course, Maxima has a more efficient function for this (as does the
numerical part of EMT).
\end{eulercomment}
\begin{eulerprompt}
>$binomial(44,10) //menghitung C(44,10) menggunakan fungsi binomial()
\end{eulerprompt}
\begin{eulerformula}
\[
2481256778
\]
\end{eulerformula}
\begin{eulercomment}
To learn more about a specific function double click on it. E.g., try double clicking
on "\&binomial" in the previous command line. This opens the documentation of Maxima as
provided by the authors of that program.

You will learn that the following works too.

\end{eulercomment}
\begin{eulerformula}
\[
C(x,3)=\frac{x!}{(x-3)!3!}=\frac{(x-2)(x-1)x}{6}
\]
\end{eulerformula}
\begin{eulerprompt}
>$binomial(x,3) // C(x,3)
\end{eulerprompt}
\begin{eulerformula}
\[
\frac{\left(x-2\right)\,\left(x-1\right)\,x}{6}
\]
\end{eulerformula}
\begin{eulercomment}
If you want to replace x with any specific value use "with".
\end{eulercomment}
\begin{eulerprompt}
>$&binomial(x,3) with x=10 // substitusi x=10 ke C(x,3)
\end{eulerprompt}
\begin{eulerformula}
\[
120
\]
\end{eulerformula}
\begin{eulercomment}
That way you can use a solution of an equation in another equation.

Symbolic expressions are printed by Maxima in 2D form. The reason for this is a special
symbolic flag in the string.

As you will have seen in previous and following examples, if you have LaTeX installed,
you can print a symbolic expression with Latex. If not, the following command will
issue an error message.

To print a symbolic expression with LaTeX, use \textdollar{} infront of \& (or you may ommit \&)
before the command. Do not run the Maxima commands with \textdollar{}, if you don't have LaTeX
installed.
\end{eulercomment}
\begin{eulerprompt}
>$(3+x)/(x^2+1)
\end{eulerprompt}
\begin{eulerformula}
\[
\frac{x+3}{x^2+1}
\]
\end{eulerformula}
\begin{eulercomment}
Symbolic expressions are parsed by Euler. If you need a complex syntax in one
expression, you can enclose the expression in "...". To use more than a
simple expression is possible, but strongly discouraged.
\end{eulercomment}
\begin{eulerprompt}
>&"v := 5; v^2"
\end{eulerprompt}
\begin{euleroutput}
  
                                    25
  
\end{euleroutput}
\begin{eulercomment}
For completeness, we remark that symbolic expressions can be used in
programs, but need to be enclosed in quotes. Moreover, it is much more
effective to call Maxima at compile time if possible.
\end{eulercomment}
\begin{eulerprompt}
>$&expand((1+x)^4), $&factor(diff(%,x)) // diff: turunan, factor: faktor
\end{eulerprompt}
\begin{eulerformula}
\[
4\,\left(x+1\right)^3
\]
\end{eulerformula}
\eulerimg{0}{images/EMT4Aljabar_Naela Rizqy Arofah_22305144042-013-large.png}
\begin{eulercomment}
Again, \% refers to the previous result.

To make things easier we save the solution to a symbolic variable.
Symbolic variables are defined with "\&=".
\end{eulercomment}
\begin{eulerprompt}
>fx &= (x+1)/(x^4+1); $&fx
\end{eulerprompt}
\begin{eulerformula}
\[
\frac{x+1}{x^4+1}
\]
\end{eulerformula}
\begin{eulercomment}
Symbolic expressions can be used in other symbolic expressions.
\end{eulercomment}
\begin{eulerprompt}
>$&factor(diff(fx,x))
\end{eulerprompt}
\begin{eulerformula}
\[
\frac{-3\,x^4-4\,x^3+1}{\left(x^4+1\right)^2}
\]
\end{eulerformula}
\begin{eulercomment}
A direct input of Maxima commands is available too. Start the command line
with "::". The syntax of Maxima is adapted to the syntax of EMT (called the
"compatibility mode").
\end{eulercomment}
\begin{eulerprompt}
>&factor(20!)
\end{eulerprompt}
\begin{euleroutput}
  
                           2432902008176640000
  
\end{euleroutput}
\begin{eulerprompt}
>::: factor(10!)
\end{eulerprompt}
\begin{euleroutput}
  
                                 8  4  2
                                2  3  5  7
  
\end{euleroutput}
\begin{eulerprompt}
>:: factor(20!)
\end{eulerprompt}
\begin{euleroutput}
  
                          18  8  4  2
                         2   3  5  7  11 13 17 19
  
\end{euleroutput}
\begin{eulercomment}
If you are an expert in Maxima, you may wish to use the original syntax of
Maxima. You can do this with ":::".
\end{eulercomment}
\begin{eulerprompt}
>::: av:g$ av^2;
\end{eulerprompt}
\begin{euleroutput}
  
                                     2
                                    g
  
\end{euleroutput}
\begin{eulerprompt}
>fx &= x^3*exp(x), $fx
\end{eulerprompt}
\begin{euleroutput}
  
                                   3  x
                                  x  E
  
\end{euleroutput}
\begin{eulerformula}
\[
x^3\,e^{x}
\]
\end{eulerformula}
\begin{eulercomment}
Such variables can be used in other symbolic expressions. Note, that in the
following command the right hand side of \&= is evaluated before the
assignment to Fx.
\end{eulercomment}
\begin{eulerprompt}
>&(fx with x=5), $%, &float(%)
\end{eulerprompt}
\begin{euleroutput}
  
                                       5
                                  125 E
  
\end{euleroutput}
\begin{eulerformula}
\[
125\,e^5
\]
\end{eulerformula}
\begin{euleroutput}
  
                            18551.64488782208
  
\end{euleroutput}
\begin{eulerprompt}
>fx(5)
\end{eulerprompt}
\begin{euleroutput}
  18551.6448878
\end{euleroutput}
\begin{eulercomment}
For the evaluation of an expression with specific values of the variables,
you can use the "with" operator.

The following command line also demonstrates that Maxima can evaluate an
expression numerically with float().
\end{eulercomment}
\begin{eulerprompt}
>&(fx with x=10)-(fx with x=5), &float(%)
\end{eulerprompt}
\begin{euleroutput}
  
                                  10        5
                            1000 E   - 125 E
  
  
                           2.20079141499189e+7
  
\end{euleroutput}
\begin{eulerprompt}
>$factor(diff(fx,x,2))
\end{eulerprompt}
\begin{eulerformula}
\[
x\,\left(x^2+6\,x+6\right)\,e^{x}
\]
\end{eulerformula}
\begin{eulercomment}
To get the Latex code for an expression, you can use the tex command.
\end{eulercomment}
\begin{eulerprompt}
>tex(fx)
\end{eulerprompt}
\begin{euleroutput}
  x^3\(\backslash\),e^\{x\}
\end{euleroutput}
\begin{eulercomment}
Symbolic expressions can be evaluated just like numerical expressions.
\end{eulercomment}
\begin{eulerprompt}
>fx(0.5)
\end{eulerprompt}
\begin{euleroutput}
  0.206090158838
\end{euleroutput}
\begin{eulercomment}
In symbolic expressions, this does not work, since Maxima does not support
it. Instead, use the "with" syntax (a nicer form of the at(...) command of
Maxima).
\end{eulercomment}
\begin{eulerprompt}
>$&fx with x=1/2
\end{eulerprompt}
\begin{eulerformula}
\[
\frac{\sqrt{e}}{8}
\]
\end{eulerformula}
\begin{eulercomment}
The assignment can also be symbolic.
\end{eulercomment}
\begin{eulerprompt}
>$&fx with x=1+t
\end{eulerprompt}
\begin{eulerformula}
\[
\left(t+1\right)^3\,e^{t+1}
\]
\end{eulerformula}
\begin{eulercomment}
The command solve solves symbolic expressions for a variable in Maxima. The
result is a vector of solutions.
\end{eulercomment}
\begin{eulerprompt}
>$&solve(x^2+x=4,x)
\end{eulerprompt}
\begin{eulerformula}
\[
\left[ x=\frac{-\sqrt{17}-1}{2} , x=\frac{\sqrt{17}-1}{2} \right] 
\]
\end{eulerformula}
\begin{eulercomment}
Compare with the numerical "solve" command in Euler, which needs a start
value, and optionally a target value.
\end{eulercomment}
\begin{eulerprompt}
>solve("x^2+x",1,y=4)
\end{eulerprompt}
\begin{euleroutput}
  1.56155281281
\end{euleroutput}
\begin{eulercomment}
The numerical values of the symbolic solution can be computed by evaluation
of the symbolic result. Euler will read over the assignments x= etc. If you
do not need the numerical results for further computations you can also let
Maxima find the numerical values.
\end{eulercomment}
\begin{eulerprompt}
>sol &= solve(x^2+2*x=4,x); $&sol, sol(), $&float(sol)
\end{eulerprompt}
\begin{eulerformula}
\[
\left[ x=-\sqrt{5}-1 , x=\sqrt{5}-1 \right] 
\]
\end{eulerformula}
\begin{euleroutput}
  [-3.23607,  1.23607]
\end{euleroutput}
\begin{eulerformula}
\[
\left[ x=-3.23606797749979 , x=1.23606797749979 \right] 
\]
\end{eulerformula}
\begin{eulercomment}
To get a specific symbolic solution, one can use "with" and an index.
\end{eulercomment}
\begin{eulerprompt}
>$&solve(x^2+x=1,x), x2 &= x with %[2]; $&x2
\end{eulerprompt}
\begin{eulerformula}
\[
\frac{\sqrt{5}-1}{2}
\]
\end{eulerformula}
\eulerimg{1}{images/EMT4Aljabar_Naela Rizqy Arofah_22305144042-025-large.png}
\begin{eulercomment}
To solve a system of equations, use a vector of equations. The result is a
vector of solutions.
\end{eulercomment}
\begin{eulerprompt}
>sol &= solve([x+y=3,x^2+y^2=5],[x,y]); $&sol, $&x*y with sol[1]
\end{eulerprompt}
\begin{eulerformula}
\[
2
\]
\end{eulerformula}
\eulerimg{0}{images/EMT4Aljabar_Naela Rizqy Arofah_22305144042-027-large.png}
\begin{eulercomment}
Symbolic expressions can have flags, which indicate a special treatment in
Maxima. Some flags can be used as commands too, others can't. Flags are
appended with "\textbar{}" (a nicer form of "ev(...,flags)")
\end{eulercomment}
\begin{eulerprompt}
>$& diff((x^3-1)/(x+1),x) //turunan bentuk pecahan
\end{eulerprompt}
\begin{eulerformula}
\[
\frac{3\,x^2}{x+1}-\frac{x^3-1}{\left(x+1\right)^2}
\]
\end{eulerformula}
\begin{eulerprompt}
>$& diff((x^3-1)/(x+1),x) | ratsimp //menyederhanakan pecahan
\end{eulerprompt}
\begin{eulerformula}
\[
\frac{2\,x^3+3\,x^2+1}{x^2+2\,x+1}
\]
\end{eulerformula}
\begin{eulerprompt}
>$&factor(%)
\end{eulerprompt}
\begin{eulerformula}
\[
\frac{2\,x^3+3\,x^2+1}{\left(x+1\right)^2}
\]
\end{eulerformula}
\eulerheading{Functions}
\begin{eulercomment}
In EMT, functions are programs defined with the command "function". It can be a
one-line function or multiline function.\\
A one-line function can be numerical or symbolic. A numerical one-line function is
defined by ":=".
\end{eulercomment}
\begin{eulerprompt}
>function f(x) := x*sqrt(x^2+1)
\end{eulerprompt}
\begin{eulercomment}
For an overview, we show all possible definitions for one-line functions. A
function can be evaluated just like any built-in Euler function.
\end{eulercomment}
\begin{eulerprompt}
>f(2)
\end{eulerprompt}
\begin{euleroutput}
  4.472135955
\end{euleroutput}
\begin{eulercomment}
This function will work for vectors too, obeying the matrix language of
Euler, since the expressions used in the function are vectorized.
\end{eulercomment}
\begin{eulerprompt}
>f(0:0.1:1)
\end{eulerprompt}
\begin{euleroutput}
  [0,  0.100499,  0.203961,  0.313209,  0.430813,  0.559017,  0.699714,
  0.854459,  1.0245,  1.21083,  1.41421]
\end{euleroutput}
\begin{eulercomment}
Functions can be plotted. Instead of expressions, we need only provide the
function name.

In contrast to symbolic or numerical expressions, the function name must be
provided in a string.
\end{eulercomment}
\begin{eulerprompt}
>solve("f",1,y=1)
\end{eulerprompt}
\begin{euleroutput}
  0.786151377757
\end{euleroutput}
\begin{eulercomment}
By default, if you need to overwrite a built-in function, you must add the
keyword "overwrite". Overwriting built-in functions is dangerous and can
cause problems for other functions depending on them.

You can still call the built-in function as "\_...", if it is function in the
Euler core.
\end{eulercomment}
\begin{eulerprompt}
>function overwrite sin (x) := _sin(x°) // redine sine in degrees
>sin(45)
\end{eulerprompt}
\begin{euleroutput}
  0.707106781187
\end{euleroutput}
\begin{eulercomment}
We better remove this redefinition of sin.
\end{eulercomment}
\begin{eulerprompt}
>forget sin; sin(pi/4)
\end{eulerprompt}
\begin{euleroutput}
  0.707106781187
\end{euleroutput}
\eulersubheading{Default Parameters}
\begin{eulercomment}
Numerical function can have default parameters.
\end{eulercomment}
\begin{eulerprompt}
>function f(x,a=1) := a*x^2
\end{eulerprompt}
\begin{eulercomment}
Omitting this parameter uses the default value.
\end{eulercomment}
\begin{eulerprompt}
>f(4)
\end{eulerprompt}
\begin{euleroutput}
  16
\end{euleroutput}
\begin{eulercomment}
Setting it overwrites the default value.
\end{eulercomment}
\begin{eulerprompt}
>f(4,5)
\end{eulerprompt}
\begin{euleroutput}
  80
\end{euleroutput}
\begin{eulercomment}
An assigned parameter overwrite it too. This is used by many Euler functions
like plot2d, plot3d.
\end{eulercomment}
\begin{eulerprompt}
>f(4,a=1)
\end{eulerprompt}
\begin{euleroutput}
  16
\end{euleroutput}
\begin{eulercomment}
If a variable is not a parameter, it must be global. One-line functions can
see global variables.
\end{eulercomment}
\begin{eulerprompt}
>function f(x) := a*x^2
>a=6; f(2)
\end{eulerprompt}
\begin{euleroutput}
  24
\end{euleroutput}
\begin{eulercomment}
But an assigned parameter overrides the global value.

If the argument is not in the list of pre-defined parameters, it must be
declared with ":="!
\end{eulercomment}
\begin{eulerprompt}
>f(2,a:=5)
\end{eulerprompt}
\begin{euleroutput}
  20
\end{euleroutput}
\begin{eulercomment}
Symbolic functions are defined with "\&=". They are defined in Euler and
Maxima, and work in both worlds. The defining expression is run through
Maxima before the definition.
\end{eulercomment}
\begin{eulerprompt}
>function g(x) &= x^3-x*exp(-x); $&g(x)
\end{eulerprompt}
\begin{eulerformula}
\[
x^3-x\,e^ {- x }
\]
\end{eulerformula}
\begin{eulercomment}
Symbolic functions can be used in symbolic expressions.
\end{eulercomment}
\begin{eulerprompt}
>$&diff(g(x),x), $&% with x=4/3
\end{eulerprompt}
\begin{eulerformula}
\[
\frac{e^ {- \frac{4}{3} }}{3}+\frac{16}{3}
\]
\end{eulerformula}
\eulerimg{1}{images/EMT4Aljabar_Naela Rizqy Arofah_22305144042-033-large.png}
\begin{eulercomment}
They can also be used in numerical expressions. Of course, this will only
work if EMT can interpret everything inside the function.
\end{eulercomment}
\begin{eulerprompt}
>g(5+g(1))
\end{eulerprompt}
\begin{euleroutput}
  178.635099908
\end{euleroutput}
\begin{eulercomment}
They can be used to define other symbolic functions or expressions.
\end{eulercomment}
\begin{eulerprompt}
>function G(x) &= factor(integrate(g(x),x)); $&G(c) // integrate: mengintegralkan
\end{eulerprompt}
\begin{eulerformula}
\[
\frac{e^ {- c }\,\left(c^4\,e^{c}+4\,c+4\right)}{4}
\]
\end{eulerformula}
\begin{eulerprompt}
>solve(&g(x),0.5)
\end{eulerprompt}
\begin{euleroutput}
  0.703467422498
\end{euleroutput}
\begin{eulercomment}
The following works too, since Euler uses the symbolic expression in the
function g, if it does not find a symbolic variable g, and if there is a
symbolic function g.
\end{eulercomment}
\begin{eulerprompt}
>solve(&g,0.5)
\end{eulerprompt}
\begin{euleroutput}
  0.703467422498
\end{euleroutput}
\begin{eulerprompt}
>function P(x,n) &= (2*x-1)^n; $&P(x,n)
\end{eulerprompt}
\begin{eulerformula}
\[
\left(2\,x-1\right)^{n}
\]
\end{eulerformula}
\begin{eulerprompt}
>function Q(x,n) &= (x+2)^n; $&Q(x,n)
\end{eulerprompt}
\begin{eulerformula}
\[
\left(x+2\right)^{n}
\]
\end{eulerformula}
\begin{eulerprompt}
>$&P(x,4), $&expand(%)
\end{eulerprompt}
\begin{eulerformula}
\[
16\,x^4-32\,x^3+24\,x^2-8\,x+1
\]
\end{eulerformula}
\eulerimg{0}{images/EMT4Aljabar_Naela Rizqy Arofah_22305144042-038-large.png}
\begin{eulerprompt}
>P(3,4)
\end{eulerprompt}
\begin{euleroutput}
  625
\end{euleroutput}
\begin{eulerprompt}
>$&P(x,4)+ Q(x,3), $&expand(%)
\end{eulerprompt}
\begin{eulerformula}
\[
16\,x^4-31\,x^3+30\,x^2+4\,x+9
\]
\end{eulerformula}
\eulerimg{0}{images/EMT4Aljabar_Naela Rizqy Arofah_22305144042-040-large.png}
\begin{eulerprompt}
>$&P(x,4)-Q(x,3), $&expand(%), $&factor(%)
\end{eulerprompt}
\begin{eulerformula}
\[
16\,x^4-33\,x^3+18\,x^2-20\,x-7
\]
\end{eulerformula}
\eulerimg{0}{images/EMT4Aljabar_Naela Rizqy Arofah_22305144042-042-large.png}
\eulerimg{0}{images/EMT4Aljabar_Naela Rizqy Arofah_22305144042-043-large.png}
\begin{eulerprompt}
>$&P(x,4)*Q(x,3), $&expand(%), $&factor(%)
\end{eulerprompt}
\begin{eulerformula}
\[
\left(x+2\right)^3\,\left(2\,x-1\right)^4
\]
\end{eulerformula}
\eulerimg{0}{images/EMT4Aljabar_Naela Rizqy Arofah_22305144042-045-large.png}
\eulerimg{0}{images/EMT4Aljabar_Naela Rizqy Arofah_22305144042-046-large.png}
\begin{eulerprompt}
>$&P(x,4)/Q(x,1), $&expand(%), $&factor(%)
\end{eulerprompt}
\begin{eulerformula}
\[
\frac{\left(2\,x-1\right)^4}{x+2}
\]
\end{eulerformula}
\eulerimg{1}{images/EMT4Aljabar_Naela Rizqy Arofah_22305144042-048-large.png}
\eulerimg{1}{images/EMT4Aljabar_Naela Rizqy Arofah_22305144042-049-large.png}
\begin{eulerprompt}
>function f(x) &= x^3-x; $&f(x)
\end{eulerprompt}
\begin{eulerformula}
\[
x^3-x
\]
\end{eulerformula}
\begin{eulercomment}
With \&= the function is symbolic, and can be used in other symbolic
expressions.
\end{eulercomment}
\begin{eulerprompt}
>$&integrate(f(x),x)
\end{eulerprompt}
\begin{eulerformula}
\[
\frac{x^4}{4}-\frac{x^2}{2}
\]
\end{eulerformula}
\begin{eulercomment}
With := the function is numerical. A good example is a definite integral like

\end{eulercomment}
\begin{eulerformula}
\[
f(x) = \int_1^x t^t \, dt,
\]
\end{eulerformula}
\begin{eulercomment}
which can not be evaluated symbolically.

If we redefine the function with the keyword "map" it can be used for vectors
x. Internally, the function is called for all values of x once, and the
results are stored in a vector.
\end{eulercomment}
\begin{eulerprompt}
>function map f(x) := integrate("x^x",1,x)
>f(0:0.5:2)
\end{eulerprompt}
\begin{euleroutput}
  [-0.783431,  -0.410816,  0,  0.676863,  2.05045]
\end{euleroutput}
\begin{eulercomment}
Functions can have default values for parameters.
\end{eulercomment}
\begin{eulerprompt}
>function mylog (x,base=10) := ln(x)/ln(base);
\end{eulerprompt}
\begin{eulercomment}
Now the function can be called with or without a parameter "base".
\end{eulercomment}
\begin{eulerprompt}
>mylog(100), mylog(2^6.7,2)
\end{eulerprompt}
\begin{euleroutput}
  2
  6.7
\end{euleroutput}
\begin{eulercomment}
Moreover, it is possible to use assigned parameters.
\end{eulercomment}
\begin{eulerprompt}
>mylog(E^2,base=E)
\end{eulerprompt}
\begin{euleroutput}
  2
\end{euleroutput}
\begin{eulercomment}
Often, we want to use functions for vectors at one place, and for individual
elements at other places. This is possible with vector parameters.
\end{eulercomment}
\begin{eulerprompt}
>function f([a,b]) &= a^2+b^2-a*b+b; $&f(a,b), $&f(x,y)
\end{eulerprompt}
\begin{eulerformula}
\[
y^2-x\,y+y+x^2
\]
\end{eulerformula}
\eulerimg{0}{images/EMT4Aljabar_Naela Rizqy Arofah_22305144042-053-large.png}
\begin{eulercomment}
Such a symbolic function can be used for symbolic variables.

But the function can also be used for a numerical vector.
\end{eulercomment}
\begin{eulerprompt}
>v=[3,4]; f(v)
\end{eulerprompt}
\begin{euleroutput}
  17
\end{euleroutput}
\begin{eulercomment}
There are also purely symbolic functions, which cannot be used numerically.
\end{eulercomment}
\begin{eulerprompt}
>function lapl(expr,x,y) &&= diff(expr,x,2)+diff(expr,y,2)//turunan parsial kedua
\end{eulerprompt}
\begin{euleroutput}
  
                   diff(expr, y, 2) + diff(expr, x, 2)
  
\end{euleroutput}
\begin{eulerprompt}
>$&realpart((x+I*y)^4), $&lapl(%,x,y)
\end{eulerprompt}
\begin{eulerformula}
\[
0
\]
\end{eulerformula}
\eulerimg{0}{images/EMT4Aljabar_Naela Rizqy Arofah_22305144042-055-large.png}
\begin{eulercomment}
But of course, they can be used in symbolic expressions or in the definition
of symbolic functions.
\end{eulercomment}
\begin{eulerprompt}
>function f(x,y) &= factor(lapl((x+y^2)^5,x,y)); $&f(x,y)
\end{eulerprompt}
\begin{eulerformula}
\[
10\,\left(y^2+x\right)^3\,\left(9\,y^2+x+2\right)
\]
\end{eulerformula}
\begin{eulercomment}
To summarize

- \&= defines symbolic functions,\\
- := defines numerical functions,\\
- \&\&= defines purely symbolic functions.

\begin{eulercomment}
\eulerheading{Solving Expressions}
\begin{eulercomment}
Expressions can be solved numerically and symbolically.

To solve a simple expression of one variable, we can use the solve()
function. It needs a start value to start the search. Internally,
solve() uses the secant method.
\end{eulercomment}
\begin{eulerprompt}
>solve("x^2-2",1)
\end{eulerprompt}
\begin{euleroutput}
  1.41421356237
\end{euleroutput}
\begin{eulercomment}
This works for symbolic expression too. Take the following function.
\end{eulercomment}
\begin{eulerprompt}
>$&solve(x^2=2,x)
\end{eulerprompt}
\begin{eulerformula}
\[
\left[ x=-\sqrt{2} , x=\sqrt{2} \right] 
\]
\end{eulerformula}
\begin{eulerprompt}
>$&solve(x^2-2,x)
\end{eulerprompt}
\begin{eulerformula}
\[
\left[ x=-\sqrt{2} , x=\sqrt{2} \right] 
\]
\end{eulerformula}
\begin{eulerprompt}
>$&solve(a*x^2+b*x+c=0,x)
\end{eulerprompt}
\begin{eulerformula}
\[
\left[ x=\frac{-\sqrt{b^2-4\,a\,c}-b}{2\,a} , x=\frac{\sqrt{b^2-4\,  a\,c}-b}{2\,a} \right] 
\]
\end{eulerformula}
\begin{eulerprompt}
>$&solve([a*x+b*y=c,d*x+e*y=f],[x,y])
\end{eulerprompt}
\begin{eulerformula}
\[
\left[ \left[ x=-\frac{c\,e}{b\,\left(d-5\right)-a\,e} , y=\frac{c  \,\left(d-5\right)}{b\,\left(d-5\right)-a\,e} \right]  \right] 
\]
\end{eulerformula}
\begin{eulerprompt}
>px &= 4*x^8+x^7-x^4-x; $&px
\end{eulerprompt}
\begin{eulerformula}
\[
4\,x^8+x^7-x^4-x
\]
\end{eulerformula}
\begin{eulercomment}
Now we search the point, where the polynomial is 2. In solve(), the default
target value y=0 can be changed with an assigned variable.\\
We use y=2 and check by evaluating the polynomial at the previous result.
\end{eulercomment}
\begin{eulerprompt}
>solve(px,1,y=2), px(%)
\end{eulerprompt}
\begin{euleroutput}
  0.966715594851
  2
\end{euleroutput}
\begin{eulercomment}
Solving a symbolic expression in symbolic form returns a list of solutions.
We use the symbolic solver solve() provided by Maxima.
\end{eulercomment}
\begin{eulerprompt}
>sol &= solve(x^2-x-1,x); $&sol
\end{eulerprompt}
\begin{eulerformula}
\[
\left[ x=\frac{1-\sqrt{5}}{2} , x=\frac{\sqrt{5}+1}{2} \right] 
\]
\end{eulerformula}
\begin{eulercomment}
The easiest way to get the numerical values is to evaluate the solution
numerically just like an expression.
\end{eulercomment}
\begin{eulerprompt}
>longest sol()
\end{eulerprompt}
\begin{euleroutput}
      -0.6180339887498949       1.618033988749895 
\end{euleroutput}
\begin{eulercomment}
To use the solutions symbolically in other expressions, the easiest way is
"with".
\end{eulercomment}
\begin{eulerprompt}
>$&x^2 with sol[1], $&expand(x^2-x-1 with sol[2])
\end{eulerprompt}
\begin{eulerformula}
\[
0
\]
\end{eulerformula}
\eulerimg{0}{images/EMT4Aljabar_Naela Rizqy Arofah_22305144042-064-large.png}
\begin{eulercomment}
Solving systems of equations symbolically can be done with vectors of
equations and the symbolic solver solve(). The answer is a list of lists of
equations.
\end{eulercomment}
\begin{eulerprompt}
>$&solve([x+y=2,x^3+2*y+x=4],[x,y])
\end{eulerprompt}
\begin{eulerformula}
\[
\left[ \left[ x=-1 , y=3 \right]  , \left[ x=1 , y=1 \right]  ,   \left[ x=0 , y=2 \right]  \right] 
\]
\end{eulerformula}
\begin{eulercomment}
The function f() can see global variables. But often we want to use
local parameters.

\end{eulercomment}
\begin{eulerformula}
\[
a^x-x^a = 0.1
\]
\end{eulerformula}
\begin{eulercomment}
with a=3.
\end{eulercomment}
\begin{eulerprompt}
>function f(x,a) := x^a-a^x;
\end{eulerprompt}
\begin{eulercomment}
One way to pass the additional parameter to f() is to use a list with the
function name and the parameters (the other way are semicolon parameters).
\end{eulercomment}
\begin{eulerprompt}
>solve(\{\{"f",3\}\},2,y=0.1)
\end{eulerprompt}
\begin{euleroutput}
  2.54116291558
\end{euleroutput}
\begin{eulercomment}
This does also work with expressions. But then, a named list element has to
be used. (More on lists in the tutorial about the syntax of EMT).
\end{eulercomment}
\begin{eulerprompt}
>solve(\{\{"x^a-a^x",a=3\}\},2,y=0.1)
\end{eulerprompt}
\begin{euleroutput}
  2.54116291558
\end{euleroutput}
\eulerheading{Menyelesaikan Pertidaksamaan}
\begin{eulercomment}
Untuk menyelesaikan pertidaksamaan, EMT tidak akan dapat melakukannya,
melainkan dengan bantuan Maxima, artinya secara eksak (simbolik).
Perintah Maxima yang digunakan adalah fourier\_elim(), yang harus
dipanggil dengan perintah "load(fourier\_elim)" terlebih dahulu.
\end{eulercomment}
\begin{eulerprompt}
>&load(fourier_elim)
\end{eulerprompt}
\begin{euleroutput}
  
          C:/Program Files/Euler x64/maxima/share/maxima/5.35.1/share/f\(\backslash\)
  ourier_elim/fourier_elim.lisp
  
\end{euleroutput}
\begin{eulerprompt}
>$&fourier_elim([x^2 - 1>0],[x]) // x^2-1 > 0
\end{eulerprompt}
\begin{eulerformula}
\[
\left[ 1<x \right] \lor \left[ x<-1 \right] 
\]
\end{eulerformula}
\begin{eulerprompt}
>$&fourier_elim([x^2 - 1<0],[x]) // x^2-1 < 0
\end{eulerprompt}
\begin{eulerformula}
\[
\left[ -1<x , x<1 \right] 
\]
\end{eulerformula}
\begin{eulerprompt}
>$&fourier_elim([x^2 - 1 # 0],[x]) // x^-1 <> 0
\end{eulerprompt}
\begin{eulerformula}
\[
\left[ -1<x , x<1 \right] \lor \left[ 1<x \right] \lor \left[ x<-1   \right] 
\]
\end{eulerformula}
\begin{eulerprompt}
>$&fourier_elim([x # 6],[x])
\end{eulerprompt}
\begin{eulerformula}
\[
\left[ x<6 \right] \lor \left[ 6<x \right] 
\]
\end{eulerformula}
\begin{eulerprompt}
>$&fourier_elim([x < 1, x > 1],[x]) // tidak memiliki penyelesaian
\end{eulerprompt}
\begin{eulerformula}
\[
{\it emptyset}
\]
\end{eulerformula}
\begin{eulerprompt}
>$&fourier_elim([minf < x, x < inf],[x]) // solusinya R
\end{eulerprompt}
\begin{eulerformula}
\[
{\it universalset}
\]
\end{eulerformula}
\begin{eulerprompt}
>$&fourier_elim([x^3 - 1 > 0],[x])
\end{eulerprompt}
\begin{eulerformula}
\[
\left[ 1<x , x^2+x+1>0 \right] \lor \left[ x<1 , -x^2-x-1>0   \right] 
\]
\end{eulerformula}
\begin{eulerprompt}
>$&fourier_elim([cos(x) < 1/2],[x]) // ??? gagal
\end{eulerprompt}
\begin{eulerformula}
\[
\left[ 1-2\,\cos x>0 \right] 
\]
\end{eulerformula}
\begin{eulerprompt}
>$&fourier_elim([y-x < 5, x - y < 7, 10 < y],[x,y]) // sistem pertidaksamaan
\end{eulerprompt}
\begin{eulerformula}
\[
\left[ y-5<x , x<y+7 , 10<y \right] 
\]
\end{eulerformula}
\begin{eulerprompt}
>$&fourier_elim([y-x < 5, x - y < 7, 10 < y],[y,x])
\end{eulerprompt}
\begin{eulerformula}
\[
\left[ {\it max}\left(10 , x-7\right)<y , y<x+5 , 5<x \right] 
\]
\end{eulerformula}
\begin{eulerprompt}
>$&fourier_elim((x + y < 5) and (x - y >8),[x,y])
\end{eulerprompt}
\begin{eulerformula}
\[
\left[ y+8<x , x<5-y , y<-\frac{3}{2} \right] 
\]
\end{eulerformula}
\begin{eulerprompt}
>$&fourier_elim(((x + y < 5) and x < 1) or  (x - y >8),[x,y])
\end{eulerprompt}
\begin{eulerformula}
\[
\left[ y+8<x \right] \lor \left[ x<{\it min}\left(1 , 5-y\right)   \right] 
\]
\end{eulerformula}
\begin{eulerprompt}
>&fourier_elim([max(x,y) > 6, x # 8, abs(y-1) > 12],[x,y])
\end{eulerprompt}
\begin{euleroutput}
  
          [6 < x, x < 8, y < - 11] or [8 < x, y < - 11]
   or [x < 8, 13 < y] or [x = y, 13 < y] or [8 < x, x < y, 13 < y]
   or [y < x, 13 < y]
  
\end{euleroutput}
\begin{eulerprompt}
>$&fourier_elim([(x+6)/(x-9) <= 6],[x])
\end{eulerprompt}
\begin{eulerformula}
\[
\left[ x=12 \right] \lor \left[ 12<x \right] \lor \left[ x<9   \right] 
\]
\end{eulerformula}
\eulerheading{The Matrix Language}
\begin{eulercomment}
The documentation of the EMT core contains a detailed discussion on the
matrix language of Euler.

Vectors and matrices are entered with square brackets, elements separated by
commas, rows separated by semicolons.
\end{eulercomment}
\begin{eulerprompt}
>A=[1,2;3,4]
\end{eulerprompt}
\begin{euleroutput}
              1             2 
              3             4 
\end{euleroutput}
\begin{eulercomment}
The matrix product is denoted by a dot.
\end{eulercomment}
\begin{eulerprompt}
>b=[3;4]
\end{eulerprompt}
\begin{euleroutput}
              3 
              4 
\end{euleroutput}
\begin{eulerprompt}
>b' // transpose b
\end{eulerprompt}
\begin{euleroutput}
  [3,  4]
\end{euleroutput}
\begin{eulerprompt}
>inv(A) //inverse A
\end{eulerprompt}
\begin{euleroutput}
             -2             1 
            1.5          -0.5 
\end{euleroutput}
\begin{eulerprompt}
>A.b //perkalian matriks
\end{eulerprompt}
\begin{euleroutput}
             11 
             25 
\end{euleroutput}
\begin{eulerprompt}
>A.inv(A)
\end{eulerprompt}
\begin{euleroutput}
              1             0 
              0             1 
\end{euleroutput}
\begin{eulercomment}
The main point of a matrix language is that all functions and operators work
element for element.
\end{eulercomment}
\begin{eulerprompt}
>A.A
\end{eulerprompt}
\begin{euleroutput}
              7            10 
             15            22 
\end{euleroutput}
\begin{eulerprompt}
>A^2 //perpangkatan elemen2 A
\end{eulerprompt}
\begin{euleroutput}
              1             4 
              9            16 
\end{euleroutput}
\begin{eulerprompt}
>A.A.A
\end{eulerprompt}
\begin{euleroutput}
             37            54 
             81           118 
\end{euleroutput}
\begin{eulerprompt}
>power(A,3) //perpangkatan matriks
\end{eulerprompt}
\begin{euleroutput}
             37            54 
             81           118 
\end{euleroutput}
\begin{eulerprompt}
>A/A //pembagian elemen-elemen matriks yang seletak
\end{eulerprompt}
\begin{euleroutput}
              1             1 
              1             1 
\end{euleroutput}
\begin{eulerprompt}
>A/b //pembagian elemen2 A oleh elemen2 b kolom demi kolom (karena b vektor kolom)
\end{eulerprompt}
\begin{euleroutput}
       0.333333      0.666667 
           0.75             1 
\end{euleroutput}
\begin{eulerprompt}
>A\(\backslash\)b // hasilkali invers A dan b, A^(-1)b 
\end{eulerprompt}
\begin{euleroutput}
             -2 
            2.5 
\end{euleroutput}
\begin{eulerprompt}
>inv(A).b
\end{eulerprompt}
\begin{euleroutput}
             -2 
            2.5 
\end{euleroutput}
\begin{eulerprompt}
>A\(\backslash\)A   //A^(-1)A
\end{eulerprompt}
\begin{euleroutput}
              1             0 
              0             1 
\end{euleroutput}
\begin{eulerprompt}
>inv(A).A
\end{eulerprompt}
\begin{euleroutput}
              1             0 
              0             1 
\end{euleroutput}
\begin{eulerprompt}
>A*A //perkalin elemen-elemen matriks seletak
\end{eulerprompt}
\begin{euleroutput}
              1             4 
              9            16 
\end{euleroutput}
\begin{eulercomment}
This is not the matrix product, but a multiplication element by element. The
same works for vectors.
\end{eulercomment}
\begin{eulerprompt}
>b^2 // perpangkatan elemen-elemen matriks/vektor
\end{eulerprompt}
\begin{euleroutput}
              9 
             16 
\end{euleroutput}
\begin{eulercomment}
If one of the operands is a vector or a scalar it is expanded in the
natural way.
\end{eulercomment}
\begin{eulerprompt}
>2*A
\end{eulerprompt}
\begin{euleroutput}
              2             4 
              6             8 
\end{euleroutput}
\begin{eulercomment}
E.g., if the operand is a column vector its elements are applied to
all rows of A.
\end{eulercomment}
\begin{eulerprompt}
>[1,2]*A
\end{eulerprompt}
\begin{euleroutput}
              1             4 
              3             8 
\end{euleroutput}
\begin{eulercomment}
If it is a row vector it is applied to all columns of A.
\end{eulercomment}
\begin{eulerprompt}
>A*[2,3]
\end{eulerprompt}
\begin{euleroutput}
              2             6 
              6            12 
\end{euleroutput}
\begin{eulercomment}
One can imagine this multiplication as if the row vector v had been
duplicated to form a matrix of the same size as A.
\end{eulercomment}
\begin{eulerprompt}
>dup([1,2],2) // dup: menduplikasi/menggandakan vektor [1,2] sebanyak 2 kali (baris)
\end{eulerprompt}
\begin{euleroutput}
              1             2 
              1             2 
\end{euleroutput}
\begin{eulerprompt}
>A*dup([1,2],2) 
\end{eulerprompt}
\begin{euleroutput}
              1             4 
              3             8 
\end{euleroutput}
\begin{eulercomment}
This does also apply for two vectors where one is a row vector and the
other is a column vector. We compute i*j for i,j from 1 to 5. The trick is to multiply 1:5
with its transpose. The matrix language of Euler automatically
generates a table of values.
\end{eulercomment}
\begin{eulerprompt}
>(1:5)*(1:5)' // hasilkali elemen-elemen vektor baris dan vektor kolom
\end{eulerprompt}
\begin{euleroutput}
              1             2             3             4             5 
              2             4             6             8            10 
              3             6             9            12            15 
              4             8            12            16            20 
              5            10            15            20            25 
\end{euleroutput}
\begin{eulercomment}
Again, remember that this is not the matrix product!
\end{eulercomment}
\begin{eulerprompt}
>(1:5).(1:5)' // hasilkali vektor baris dan vektor kolom
\end{eulerprompt}
\begin{euleroutput}
  55
\end{euleroutput}
\begin{eulerprompt}
>sum((1:5)*(1:5)) // sama hasilnya
\end{eulerprompt}
\begin{euleroutput}
  55
\end{euleroutput}
\begin{eulercomment}
Even operators like \textless{} or == work in the same way.
\end{eulercomment}
\begin{eulerprompt}
>(1:10)<6 // menguji elemen-elemen yang kurang dari 6
\end{eulerprompt}
\begin{euleroutput}
  [1,  1,  1,  1,  1,  0,  0,  0,  0,  0]
\end{euleroutput}
\begin{eulercomment}
E.g., we can count the number of elements satisfying a certain
condition with the function sum().
\end{eulercomment}
\begin{eulerprompt}
>sum((1:10)<6) // banyak elemen yang kurang dari 6
\end{eulerprompt}
\begin{euleroutput}
  5
\end{euleroutput}
\begin{eulercomment}
Euler has comparison operators, like "==", which checks for equality.

We get a vector of 0 and 1, where 1 stands for true.
\end{eulercomment}
\begin{eulerprompt}
>t=(1:10)^2; t==25 //menguji elemen2 t yang sama dengan 25 (hanya ada 1)
\end{eulerprompt}
\begin{euleroutput}
  [0,  0,  0,  0,  1,  0,  0,  0,  0,  0]
\end{euleroutput}
\begin{eulercomment}
From such a vector, "nonzeros" selects the non-zero elements.

In this case, we get the indices of all elements greater than 50.
\end{eulercomment}
\begin{eulerprompt}
>nonzeros(t>50) //indeks elemen2 t yang lebih besar daripada 50
\end{eulerprompt}
\begin{euleroutput}
  [8,  9,  10]
\end{euleroutput}
\begin{eulercomment}
Of course, we can use this vector of indices to get the corresponding
values in t.
\end{eulercomment}
\begin{eulerprompt}
>t[nonzeros(t>50)] //elemen2 t yang lebih besar daripada 50
\end{eulerprompt}
\begin{euleroutput}
  [64,  81,  100]
\end{euleroutput}
\begin{eulercomment}
As an example, let us find all squares of the numbers 1 to 1000, which
are 5 modulo 11 and 3 modulo 13.
\end{eulercomment}
\begin{eulerprompt}
>t=1:1000; nonzeros(mod(t^2,11)==5 && mod(t^2,13)==3)
\end{eulerprompt}
\begin{euleroutput}
  [4,  48,  95,  139,  147,  191,  238,  282,  290,  334,  381,  425,
  433,  477,  524,  568,  576,  620,  667,  711,  719,  763,  810,  854,
  862,  906,  953,  997]
\end{euleroutput}
\begin{eulercomment}
EMT is not completely effective for integer computations. It uses
double precision floating point internally. However, it is often very
useful.

We can check for primality. Let us find out, how many squares plus 1
are primes.
\end{eulercomment}
\begin{eulerprompt}
>t=1:1000; length(nonzeros(isprime(t^2+1)))
\end{eulerprompt}
\begin{euleroutput}
  112
\end{euleroutput}
\begin{eulercomment}
The function nonzeros() works only for vectors. For matrices, there is
mnonzeros().
\end{eulercomment}
\begin{eulerprompt}
>seed(2); A=random(3,4)
\end{eulerprompt}
\begin{euleroutput}
       0.765761      0.401188      0.406347      0.267829 
        0.13673      0.390567      0.495975      0.952814 
       0.548138      0.006085      0.444255      0.539246 
\end{euleroutput}
\begin{eulercomment}
It returns the indices of the elements, which are not zeros.
\end{eulercomment}
\begin{eulerprompt}
>k=mnonzeros(A<0.4) //indeks elemen2 A yang kurang dari 0,4
\end{eulerprompt}
\begin{euleroutput}
              1             4 
              2             1 
              2             2 
              3             2 
\end{euleroutput}
\begin{eulercomment}
These indices can be used to set the elements to some value.
\end{eulercomment}
\begin{eulerprompt}
>mset(A,k,0) //mengganti elemen2 suatu matriks pada indeks tertentu
\end{eulerprompt}
\begin{euleroutput}
       0.765761      0.401188      0.406347             0 
              0             0      0.495975      0.952814 
       0.548138             0      0.444255      0.539246 
\end{euleroutput}
\begin{eulercomment}
The function mset() can also set the elements at the indices to the
entries of some other matrix.
\end{eulercomment}
\begin{eulerprompt}
>mset(A,k,-random(size(A)))
\end{eulerprompt}
\begin{euleroutput}
       0.765761      0.401188      0.406347     -0.126917 
      -0.122404     -0.691673      0.495975      0.952814 
       0.548138     -0.483902      0.444255      0.539246 
\end{euleroutput}
\begin{eulercomment}
And it is possible to get the elements in a vector.
\end{eulercomment}
\begin{eulerprompt}
>mget(A,k)
\end{eulerprompt}
\begin{euleroutput}
  [0.267829,  0.13673,  0.390567,  0.006085]
\end{euleroutput}
\begin{eulercomment}
Another useful function is extrema, which returns the minimal and
maximal values in each row of the matrix and their positions.
\end{eulercomment}
\begin{eulerprompt}
>ex=extrema(A)
\end{eulerprompt}
\begin{euleroutput}
       0.267829             4      0.765761             1 
        0.13673             1      0.952814             4 
       0.006085             2      0.548138             1 
\end{euleroutput}
\begin{eulercomment}
We can use this to extract the maximal values in each row.
\end{eulercomment}
\begin{eulerprompt}
>ex[,3]'
\end{eulerprompt}
\begin{euleroutput}
  [0.765761,  0.952814,  0.548138]
\end{euleroutput}
\begin{eulercomment}
This, of course, is the same as the function max().
\end{eulercomment}
\begin{eulerprompt}
>max(A)'
\end{eulerprompt}
\begin{euleroutput}
  [0.765761,  0.952814,  0.548138]
\end{euleroutput}
\begin{eulercomment}
But with mget(), we can extract the indices and use this information
to extract the elements at the same positions from another matrix.
\end{eulercomment}
\begin{eulerprompt}
>j=(1:rows(A))'|ex[,4], mget(-A,j)
\end{eulerprompt}
\begin{euleroutput}
              1             1 
              2             4 
              3             1 
  [-0.765761,  -0.952814,  -0.548138]
\end{euleroutput}
\begin{eulercomment}
\begin{eulercomment}
\eulerheading{Other Matrix Functions (Building Matrix)}
\begin{eulercomment}
To build a matrix, we can stack one matrix on top of another. If both
do not have the same number of columns, the shorter one will be filled
with 0.
\end{eulercomment}
\begin{eulerprompt}
>v=1:3; v_v
\end{eulerprompt}
\begin{euleroutput}
              1             2             3 
              1             2             3 
\end{euleroutput}
\begin{eulercomment}
Likewise, we can attach a matrix to another side by side, if both have
the same number of rows.
\end{eulercomment}
\begin{eulerprompt}
>A=random(3,4); A|v'
\end{eulerprompt}
\begin{euleroutput}
       0.032444     0.0534171      0.595713      0.564454             1 
        0.83916      0.175552      0.396988       0.83514             2 
      0.0257573      0.658585      0.629832      0.770895             3 
\end{euleroutput}
\begin{eulercomment}
If they do not have the same number of rows the shorter matrix is
filled with 0.

There is an exception to this rule. A real number attached to a matrix
will be used as a column filled with that real number.
\end{eulercomment}
\begin{eulerprompt}
>A|1
\end{eulerprompt}
\begin{euleroutput}
       0.032444     0.0534171      0.595713      0.564454             1 
        0.83916      0.175552      0.396988       0.83514             1 
      0.0257573      0.658585      0.629832      0.770895             1 
\end{euleroutput}
\begin{eulercomment}
It is possible to make a matrix of row and column vectors.
\end{eulercomment}
\begin{eulerprompt}
>[v;v]
\end{eulerprompt}
\begin{euleroutput}
              1             2             3 
              1             2             3 
\end{euleroutput}
\begin{eulerprompt}
>[v',v']
\end{eulerprompt}
\begin{euleroutput}
              1             1 
              2             2 
              3             3 
\end{euleroutput}
\begin{eulercomment}
The main purpose of this is to interpret a vector of expressions for
column vectors.
\end{eulercomment}
\begin{eulerprompt}
>"[x,x^2]"(v')
\end{eulerprompt}
\begin{euleroutput}
              1             1 
              2             4 
              3             9 
\end{euleroutput}
\begin{eulercomment}
To get the size of A, we can use the following functions.
\end{eulercomment}
\begin{eulerprompt}
>C=zeros(2,4); rows(C), cols(C), size(C), length(C)
\end{eulerprompt}
\begin{euleroutput}
  2
  4
  [2,  4]
  4
\end{euleroutput}
\begin{eulercomment}
For vectors, there is length().
\end{eulercomment}
\begin{eulerprompt}
>length(2:10)
\end{eulerprompt}
\begin{euleroutput}
  9
\end{euleroutput}
\begin{eulercomment}
There are many other functions, which generate matrices.
\end{eulercomment}
\begin{eulerprompt}
>ones(2,2)
\end{eulerprompt}
\begin{euleroutput}
              1             1 
              1             1 
\end{euleroutput}
\begin{eulercomment}
This can also be used with one parameter. To get a vector with another
number than 1, use the following.
\end{eulercomment}
\begin{eulerprompt}
>ones(5)*6
\end{eulerprompt}
\begin{euleroutput}
  [6,  6,  6,  6,  6]
\end{euleroutput}
\begin{eulercomment}
Also a matrix of random numbers can be generated with random (uniform
distribution) or normal (Gauß distribution).
\end{eulercomment}
\begin{eulerprompt}
>random(2,2)
\end{eulerprompt}
\begin{euleroutput}
        0.66566      0.831835 
          0.977      0.544258 
\end{euleroutput}
\begin{eulercomment}
Here is another useful function, which restructures the elements of a
matrix into another matrix.
\end{eulercomment}
\begin{eulerprompt}
>redim(1:9,3,3) // menyusun elemen2 1, 2, 3, ..., 9 ke bentuk matriks 3x3
\end{eulerprompt}
\begin{euleroutput}
              1             2             3 
              4             5             6 
              7             8             9 
\end{euleroutput}
\begin{eulercomment}
With the following function, we can use this and the dup function to
write a rep() function, which repeats a vector n times.
\end{eulercomment}
\begin{eulerprompt}
>function rep(v,n) := redim(dup(v,n),1,n*cols(v))
\end{eulerprompt}
\begin{eulercomment}
Let us test.
\end{eulercomment}
\begin{eulerprompt}
>rep(1:3,5)
\end{eulerprompt}
\begin{euleroutput}
  [1,  2,  3,  1,  2,  3,  1,  2,  3,  1,  2,  3,  1,  2,  3]
\end{euleroutput}
\begin{eulercomment}
The function multdup() duplicates elements of a vector.
\end{eulercomment}
\begin{eulerprompt}
>multdup(1:3,5), multdup(1:3,[2,3,2])
\end{eulerprompt}
\begin{euleroutput}
  [1,  1,  1,  1,  1,  2,  2,  2,  2,  2,  3,  3,  3,  3,  3]
  [1,  1,  2,  2,  2,  3,  3]
\end{euleroutput}
\begin{eulercomment}
The functions flipx() and flipy() revert the order of the rows or
columns of a matrix. I.e., the function flipx() flips horizontally.
\end{eulercomment}
\begin{eulerprompt}
>flipx(1:5) //membalik elemen2 vektor baris
\end{eulerprompt}
\begin{euleroutput}
  [5,  4,  3,  2,  1]
\end{euleroutput}
\begin{eulercomment}
For rotations, Euler has rotleft() and rotright().
\end{eulercomment}
\begin{eulerprompt}
>rotleft(1:5) // memutar elemen2 vektor baris
\end{eulerprompt}
\begin{euleroutput}
  [2,  3,  4,  5,  1]
\end{euleroutput}
\begin{eulercomment}
A special function is drop(v,i), which removes the elements with the
indices in i from the vector v.
\end{eulercomment}
\begin{eulerprompt}
>drop(10:20,3)
\end{eulerprompt}
\begin{euleroutput}
  [10,  11,  13,  14,  15,  16,  17,  18,  19,  20]
\end{euleroutput}
\begin{eulercomment}
Note that the vector i in drop(v,i) refers to indices of elements in
v, not the values of the elements. If you want to remove elements, you
need to find the elements first. The function indexof(v,x) can be used
to find elements x in a sorted vector v.
\end{eulercomment}
\begin{eulerprompt}
>v=primes(50), i=indexof(v,10:20), drop(v,i)
\end{eulerprompt}
\begin{euleroutput}
  [2,  3,  5,  7,  11,  13,  17,  19,  23,  29,  31,  37,  41,  43,  47]
  [0,  5,  0,  6,  0,  0,  0,  7,  0,  8,  0]
  [2,  3,  5,  7,  23,  29,  31,  37,  41,  43,  47]
\end{euleroutput}
\begin{eulercomment}
As you see, it does not harm to include indices out of range (like 0),
double indices, or unsorted indices.
\end{eulercomment}
\begin{eulerprompt}
>drop(1:10,shuffle([0,0,5,5,7,12,12]))
\end{eulerprompt}
\begin{euleroutput}
  [1,  2,  3,  4,  6,  8,  9,  10]
\end{euleroutput}
\begin{eulercomment}
There are some special functions to set diagonals or to generate a
diagonal matrix.

We start with the identity matrix.
\end{eulercomment}
\begin{eulerprompt}
>A=id(5) // matriks identitas 5x5
\end{eulerprompt}
\begin{euleroutput}
              1             0             0             0             0 
              0             1             0             0             0 
              0             0             1             0             0 
              0             0             0             1             0 
              0             0             0             0             1 
\end{euleroutput}
\begin{eulercomment}
Then we set the lower diagonal (-1) to 1:4.
\end{eulercomment}
\begin{eulerprompt}
>setdiag(A,-1,1:4) //mengganti diagonal di bawah diagonal utama
\end{eulerprompt}
\begin{euleroutput}
              1             0             0             0             0 
              1             1             0             0             0 
              0             2             1             0             0 
              0             0             3             1             0 
              0             0             0             4             1 
\end{euleroutput}
\begin{eulercomment}
Note that we did not change the matrix A. We get a new matrix as
result of setdiag().

Here is a function, which returns a tri-diagonal matrix.
\end{eulercomment}
\begin{eulerprompt}
>function tridiag (n,a,b,c) := setdiag(setdiag(b*id(n),1,c),-1,a); ...
>tridiag(5,1,2,3)
\end{eulerprompt}
\begin{euleroutput}
              2             3             0             0             0 
              1             2             3             0             0 
              0             1             2             3             0 
              0             0             1             2             3 
              0             0             0             1             2 
\end{euleroutput}
\begin{eulercomment}
The diagonal of a matrix can also be extracted from the matrix. To
demonstrate this, we restructure the vector 1:9 to a 3x3 matrix.
\end{eulercomment}
\begin{eulerprompt}
>A=redim(1:9,3,3)
\end{eulerprompt}
\begin{euleroutput}
              1             2             3 
              4             5             6 
              7             8             9 
\end{euleroutput}
\begin{eulercomment}
Now we can extract the diagonal.
\end{eulercomment}
\begin{eulerprompt}
>d=getdiag(A,0)
\end{eulerprompt}
\begin{euleroutput}
  [1,  5,  9]
\end{euleroutput}
\begin{eulercomment}
E.g. We can divide the matrix by its diagonal. The matrix language
takes care that the column vector d is applied to the matrix row by
row.
\end{eulercomment}
\begin{eulerprompt}
>fraction A/d'
\end{eulerprompt}
\begin{euleroutput}
          1         2         3 
        4/5         1       6/5 
        7/9       8/9         1 
\end{euleroutput}
\eulerheading{Vectorization}
\begin{eulercomment}
Almost all functions in Euler work for matrix and vector input too,
whenever this makes sense.

E.g., the sqrt() function computes the square root of all elements of
the vector or matrix.
\end{eulercomment}
\begin{eulerprompt}
>sqrt(1:3)
\end{eulerprompt}
\begin{euleroutput}
  [1,  1.41421,  1.73205]
\end{euleroutput}
\begin{eulercomment}
So you can easily create a table of values. This is one way to plot a
function (the alternative uses an expression).
\end{eulercomment}
\begin{eulerprompt}
>x=1:0.01:5; y=log(x)/x^2; // terlalu panjang untuk ditampikan
\end{eulerprompt}
\begin{eulercomment}
With this and the colon operator a:delta:b, vectors of values of functions
can be generated easily.

In the following example, we generate a vector of values t[i] with spacing
0.1 from -1 to 1. Then we generate a vector of values of the function

\end{eulercomment}
\begin{eulerformula}
\[
s = t^3-t
\]
\end{eulerformula}
\begin{eulerprompt}
>t=-1:0.1:1; s=t^3-t
\end{eulerprompt}
\begin{euleroutput}
  [0,  0.171,  0.288,  0.357,  0.384,  0.375,  0.336,  0.273,  0.192,
  0.099,  0,  -0.099,  -0.192,  -0.273,  -0.336,  -0.375,  -0.384,
  -0.357,  -0.288,  -0.171,  0]
\end{euleroutput}
\begin{eulercomment}
EMT expands operators for scalars, vectors, and matrices in the obvious way.

E.g., a column vector times a row vector expands to matrix, if an an operator
is applied. In the following, v' is the transposed vector (a column vector).
\end{eulercomment}
\begin{eulerprompt}
>shortest (1:5)*(1:5)'
\end{eulerprompt}
\begin{euleroutput}
       1      2      3      4      5 
       2      4      6      8     10 
       3      6      9     12     15 
       4      8     12     16     20 
       5     10     15     20     25 
\end{euleroutput}
\begin{eulercomment}
Note, that this is quite different from the matrix product. The matrix
product is denoted with a dot "." in EMT.
\end{eulercomment}
\begin{eulerprompt}
>(1:5).(1:5)'
\end{eulerprompt}
\begin{euleroutput}
  55
\end{euleroutput}
\begin{eulercomment}
By default, row vectors are printed in a compact format.
\end{eulercomment}
\begin{eulerprompt}
>[1,2,3,4]
\end{eulerprompt}
\begin{euleroutput}
  [1,  2,  3,  4]
\end{euleroutput}
\begin{eulercomment}
For matrices the special operator . denotes matrix multiplication, and A'
denotes transposing. A 1x1 matrix can be used just like a real number.
\end{eulercomment}
\begin{eulerprompt}
>v:=[1,2]; v.v', %^2
\end{eulerprompt}
\begin{euleroutput}
  5
  25
\end{euleroutput}
\begin{eulercomment}
To transpose a matrix we use the apostrophe.
\end{eulercomment}
\begin{eulerprompt}
>v=1:4; v'
\end{eulerprompt}
\begin{euleroutput}
              1 
              2 
              3 
              4 
\end{euleroutput}
\begin{eulercomment}
So we can compute matrix A times vector b.
\end{eulercomment}
\begin{eulerprompt}
>A=[1,2,3,4;5,6,7,8]; A.v'
\end{eulerprompt}
\begin{euleroutput}
             30 
             70 
\end{euleroutput}
\begin{eulercomment}
Note that v is still a row vector. So v'.v is different from v.v'.
\end{eulercomment}
\begin{eulerprompt}
>v'.v
\end{eulerprompt}
\begin{euleroutput}
              1             2             3             4 
              2             4             6             8 
              3             6             9            12 
              4             8            12            16 
\end{euleroutput}
\begin{eulercomment}
v.v' computes the norm of v squared for row vectors v. The result is a
1x1 vector, which works just like a real number.
\end{eulercomment}
\begin{eulerprompt}
>v.v'
\end{eulerprompt}
\begin{euleroutput}
  30
\end{euleroutput}
\begin{eulercomment}
There is also the function norm (along with many other function of
Linear Algebra).
\end{eulercomment}
\begin{eulerprompt}
>norm(v)^2
\end{eulerprompt}
\begin{euleroutput}
  30
\end{euleroutput}
\begin{eulercomment}
Operators and functions obey the matrix language of Euler.

Here is a summary of the rules.

- A function applied to a vector or matrix is applied to each element.

- An operator operating on two matrices of same size is applied pairwise to
the elements of the matrices.

- If the two matrices have different dimensions, both are expanded in a
sensible way, so that they have the same size.

E.g., a scalar value times a vector multiplies the value with each element of
the vector. Or a matrix times a vector (with *, not .) expands the vector to
the size of the matrix by duplicating it.

The following is a simple case with the operator \textasciicircum{}.
\end{eulercomment}
\begin{eulerprompt}
>[1,2,3]^2
\end{eulerprompt}
\begin{euleroutput}
  [1,  4,  9]
\end{euleroutput}
\begin{eulercomment}
Here is a more complicated case. A row vector times a column vector expands
both by duplicating.
\end{eulercomment}
\begin{eulerprompt}
>v:=[1,2,3]; v*v'
\end{eulerprompt}
\begin{euleroutput}
              1             2             3 
              2             4             6 
              3             6             9 
\end{euleroutput}
\begin{eulercomment}
Note that the scalar product uses the matrix product, not the *!
\end{eulercomment}
\begin{eulerprompt}
>v.v'
\end{eulerprompt}
\begin{euleroutput}
  14
\end{euleroutput}
\begin{eulercomment}
There are numerous functions for matrices. We give a short list. You should to consult
the documentation for more information on these commands.

\end{eulercomment}
\begin{eulerttcomment}
  sum,prod computes the sum and products of the rows
  cumsum,cumprod does the same cumulatively
  computes the extremal values of each row
  extrema returns a vector with the extremal information
  diag(A,i) returns the i-th diagonal
  setdiag(A,i,v) sets the i-th diagonal
  id(n) the identity matrix
  det(A) the determinant
  charpoly(A) the characteristic polynomial
  eigenvalues(A) the eigenvalues
\end{eulerttcomment}
\begin{eulerprompt}
>v*v, sum(v*v), cumsum(v*v)
\end{eulerprompt}
\begin{euleroutput}
  [1,  4,  9]
  14
  [1,  5,  14]
\end{euleroutput}
\begin{eulercomment}
The : operator generates an equally spaces row vector, optionally with a step
size.
\end{eulercomment}
\begin{eulerprompt}
>1:4, 1:2:10
\end{eulerprompt}
\begin{euleroutput}
  [1,  2,  3,  4]
  [1,  3,  5,  7,  9]
\end{euleroutput}
\begin{eulercomment}
To concatenate matrices and vectors there are the operators "\textbar{}" and "\_".
\end{eulercomment}
\begin{eulerprompt}
>[1,2,3]|[4,5], [1,2,3]_1
\end{eulerprompt}
\begin{euleroutput}
  [1,  2,  3,  4,  5]
              1             2             3 
              1             1             1 
\end{euleroutput}
\begin{eulercomment}
The elements of a matrix are referred with "A[i,j]".
\end{eulercomment}
\begin{eulerprompt}
>A:=[1,2,3;4,5,6;7,8,9]; A[2,3]
\end{eulerprompt}
\begin{euleroutput}
  6
\end{euleroutput}
\begin{eulercomment}
For row or column vectors, v[i] is the i-th element of the vector. For
matrices, this returns the complete i-th row of the matrix.
\end{eulercomment}
\begin{eulerprompt}
>v:=[2,4,6,8]; v[3], A[3]
\end{eulerprompt}
\begin{euleroutput}
  6
  [7,  8,  9]
\end{euleroutput}
\begin{eulercomment}
The indices can also be row vectors of indices. : denotes all indices.
\end{eulercomment}
\begin{eulerprompt}
>v[1:2], A[:,2]
\end{eulerprompt}
\begin{euleroutput}
  [2,  4]
              2 
              5 
              8 
\end{euleroutput}
\begin{eulercomment}
A short form for : is omitting the index completely.
\end{eulercomment}
\begin{eulerprompt}
>A[,2:3]
\end{eulerprompt}
\begin{euleroutput}
              2             3 
              5             6 
              8             9 
\end{euleroutput}
\begin{eulercomment}
For purposes of vectorization, the elements of a matrix can be accessed as if
they were vectors.
\end{eulercomment}
\begin{eulerprompt}
>A\{4\}
\end{eulerprompt}
\begin{euleroutput}
  4
\end{euleroutput}
\begin{eulercomment}
A matrix can also be flattened, using the redim() function. This is
implemented in the function flatten().
\end{eulercomment}
\begin{eulerprompt}
>redim(A,1,prod(size(A))), flatten(A)
\end{eulerprompt}
\begin{euleroutput}
  [1,  2,  3,  4,  5,  6,  7,  8,  9]
  [1,  2,  3,  4,  5,  6,  7,  8,  9]
\end{euleroutput}
\begin{eulercomment}
To use matrices for tables, let us reset to the default format, and
compute a table of sine and cosine values. Note that angles are in
radians by default.
\end{eulercomment}
\begin{eulerprompt}
>defformat; w=0°:45°:360°; w=w'; deg(w)
\end{eulerprompt}
\begin{euleroutput}
              0 
             45 
             90 
            135 
            180 
            225 
            270 
            315 
            360 
\end{euleroutput}
\begin{eulercomment}
Now we append columns to a matrix.
\end{eulercomment}
\begin{eulerprompt}
>M = deg(w)|w|cos(w)|sin(w)
\end{eulerprompt}
\begin{euleroutput}
              0             0             1             0 
             45      0.785398      0.707107      0.707107 
             90        1.5708             0             1 
            135       2.35619     -0.707107      0.707107 
            180       3.14159            -1             0 
            225       3.92699     -0.707107     -0.707107 
            270       4.71239             0            -1 
            315       5.49779      0.707107     -0.707107 
            360       6.28319             1             0 
\end{euleroutput}
\begin{eulercomment}
Using the matrix language, we can generate several tables of several
functions at once.

In the following example, we compute t[j]\textasciicircum{}i for i from 1 to n. We get a matrix,
where each row is a table of t\textasciicircum{}i for one i. I.e., the matrix has the
elements
\end{eulercomment}
\begin{eulerformula}
\[
a_{i,j} = t_j^i, \quad 1 \le j \le 101, \quad 1 \le i \le n
\]
\end{eulerformula}
\begin{eulercomment}
A function which does not work for vector input should be "vectorized". This
can be achieved by the "map" keyword in the function definition. Then the
function will be evaluated for each element of a vector parameter.

The numerical integration integrate() works only for scalar interval bounds.
So we need to vectorize it.
\end{eulercomment}
\begin{eulerprompt}
>function map f(x) := integrate("x^x",1,x)
\end{eulerprompt}
\begin{eulercomment}
The "map" keyword vectorizes the function. The function will now work\\
for vectors of numbers.
\end{eulercomment}
\begin{eulerprompt}
>f([1:5])
\end{eulerprompt}
\begin{euleroutput}
  [0,  2.05045,  13.7251,  113.336,  1241.03]
\end{euleroutput}
\eulerheading{Sub-Matrices and Matrix-Elements}
\begin{eulercomment}
To access a matrix element, use the bracket notation.
\end{eulercomment}
\begin{eulerprompt}
>A=[1,2,3;4,5,6;7,8,9], A[2,2]
\end{eulerprompt}
\begin{euleroutput}
              1             2             3 
              4             5             6 
              7             8             9 
  5
\end{euleroutput}
\begin{eulercomment}
We can access a complete line of a matrix.
\end{eulercomment}
\begin{eulerprompt}
>A[2]
\end{eulerprompt}
\begin{euleroutput}
  [4,  5,  6]
\end{euleroutput}
\begin{eulercomment}
In case of row or column vectors, this returns an element of the
vector.
\end{eulercomment}
\begin{eulerprompt}
>v=1:3; v[2]
\end{eulerprompt}
\begin{euleroutput}
  2
\end{euleroutput}
\begin{eulercomment}
To make sure, you get the first row for a 1xn and a mxn matrix,
specify all columns using an empty second index.
\end{eulercomment}
\begin{eulerprompt}
>A[2,]
\end{eulerprompt}
\begin{euleroutput}
  [4,  5,  6]
\end{euleroutput}
\begin{eulercomment}
If the index is a vector of indices, Euler will return the
corresponding rows of the matrix.

Here we want the first and second row of A.
\end{eulercomment}
\begin{eulerprompt}
>A[[1,2]]
\end{eulerprompt}
\begin{euleroutput}
              1             2             3 
              4             5             6 
\end{euleroutput}
\begin{eulercomment}
We can even reorder A using vectors of indices. To be precise, we do
not change A here, but compute a reordered version of A.
\end{eulercomment}
\begin{eulerprompt}
>A[[3,2,1]]
\end{eulerprompt}
\begin{euleroutput}
              7             8             9 
              4             5             6 
              1             2             3 
\end{euleroutput}
\begin{eulercomment}
The index trick works with columns too.

This example selects all rows of A and the second and third column.
\end{eulercomment}
\begin{eulerprompt}
>A[1:3,2:3]
\end{eulerprompt}
\begin{euleroutput}
              2             3 
              5             6 
              8             9 
\end{euleroutput}
\begin{eulercomment}
For abbreviation ":" denotes all row or column indices.
\end{eulercomment}
\begin{eulerprompt}
>A[:,3]
\end{eulerprompt}
\begin{euleroutput}
              3 
              6 
              9 
\end{euleroutput}
\begin{eulercomment}
Alternatively, leave the first index empty.
\end{eulercomment}
\begin{eulerprompt}
>A[,2:3]
\end{eulerprompt}
\begin{euleroutput}
              2             3 
              5             6 
              8             9 
\end{euleroutput}
\begin{eulercomment}
We can also get the last line of A.
\end{eulercomment}
\begin{eulerprompt}
>A[-1]
\end{eulerprompt}
\begin{euleroutput}
  [7,  8,  9]
\end{euleroutput}
\begin{eulercomment}
Now let us change elements of A by assigning a submatrix of A to some
value. This does in fact change the stored matrix A.
\end{eulercomment}
\begin{eulerprompt}
>A[1,1]=4
\end{eulerprompt}
\begin{euleroutput}
              4             2             3 
              4             5             6 
              7             8             9 
\end{euleroutput}
\begin{eulercomment}
We can also assign a value to a row of A.
\end{eulercomment}
\begin{eulerprompt}
>A[1]=[-1,-1,-1]
\end{eulerprompt}
\begin{euleroutput}
             -1            -1            -1 
              4             5             6 
              7             8             9 
\end{euleroutput}
\begin{eulercomment}
We can even assign to a sub-matrix if it has the proper size.
\end{eulercomment}
\begin{eulerprompt}
>A[1:2,1:2]=[5,6;7,8]
\end{eulerprompt}
\begin{euleroutput}
              5             6            -1 
              7             8             6 
              7             8             9 
\end{euleroutput}
\begin{eulercomment}
Moreover, some shortcuts are allowed.
\end{eulercomment}
\begin{eulerprompt}
>A[1:2,1:2]=0
\end{eulerprompt}
\begin{euleroutput}
              0             0            -1 
              0             0             6 
              7             8             9 
\end{euleroutput}
\begin{eulercomment}
A warning: Indices out of bounds return empty matrices, or an error
message, depending on a system setting. The default is an error
message. Remember, however, that negative indices may be used to
access the elements of a matrix counting from the end.
\end{eulercomment}
\begin{eulerprompt}
>A[4]
\end{eulerprompt}
\begin{euleroutput}
  Row index 4 out of bounds!
  Error in:
  A[4] ...
      ^
\end{euleroutput}
\eulerheading{Sorting and Shuffling}
\begin{eulercomment}
The function sort() sorts a row vector.
\end{eulercomment}
\begin{eulerprompt}
>sort([5,6,4,8,1,9])
\end{eulerprompt}
\begin{euleroutput}
  [1,  4,  5,  6,  8,  9]
\end{euleroutput}
\begin{eulercomment}
It is often necessary to know the indices of the sorted vector in the
original vector. This can be used to reorder another vector in the
same way.

Let us shuffle a vector.
\end{eulercomment}
\begin{eulerprompt}
>v=shuffle(1:10)
\end{eulerprompt}
\begin{euleroutput}
  [4,  5,  10,  6,  8,  9,  1,  7,  2,  3]
\end{euleroutput}
\begin{eulercomment}
The indices contain the proper order of v.
\end{eulercomment}
\begin{eulerprompt}
>\{vs,ind\}=sort(v); v[ind]
\end{eulerprompt}
\begin{euleroutput}
  [1,  2,  3,  4,  5,  6,  7,  8,  9,  10]
\end{euleroutput}
\begin{eulercomment}
This works for string vectors too.
\end{eulercomment}
\begin{eulerprompt}
>s=["a","d","e","a","aa","e"]
\end{eulerprompt}
\begin{euleroutput}
  a
  d
  e
  a
  aa
  e
\end{euleroutput}
\begin{eulerprompt}
>\{ss,ind\}=sort(s); ss
\end{eulerprompt}
\begin{euleroutput}
  a
  a
  aa
  d
  e
  e
\end{euleroutput}
\begin{eulercomment}
As you see, the position of double entries is somewhat random.
\end{eulercomment}
\begin{eulerprompt}
>ind
\end{eulerprompt}
\begin{euleroutput}
  [4,  1,  5,  2,  6,  3]
\end{euleroutput}
\begin{eulercomment}
The function unique returns a sorted list of unique elements of a
vector.
\end{eulercomment}
\begin{eulerprompt}
>intrandom(1,10,10), unique(%)
\end{eulerprompt}
\begin{euleroutput}
  [4,  4,  9,  2,  6,  5,  10,  6,  5,  1]
  [1,  2,  4,  5,  6,  9,  10]
\end{euleroutput}
\begin{eulercomment}
This works for string vectors too.
\end{eulercomment}
\begin{eulerprompt}
>unique(s)
\end{eulerprompt}
\begin{euleroutput}
  a
  aa
  d
  e
\end{euleroutput}
\eulerheading{Linear Algebra}
\begin{eulercomment}
EMT has lots of functions to solve linear systems, sparse systems, or
regression problems.

For linear systems Ax=b, you can use the Gauss algorithm, the inverse matrix
or a linear fit. The operator A\textbackslash{}b uses a version of the Gauss algorithm.
\end{eulercomment}
\begin{eulerprompt}
>A=[1,2;3,4]; b=[5;6]; A\(\backslash\)b
\end{eulerprompt}
\begin{euleroutput}
             -4 
            4.5 
\end{euleroutput}
\begin{eulercomment}
For another example, we generate a 200x200 matrix and the sum of its rows.
Then we solve Ax=b using the inverse matrix. We measure the error as the
maximal deviation of all elements from 1, which of course is the correct
solution.
\end{eulercomment}
\begin{eulerprompt}
>A=normal(200,200); b=sum(A); longest totalmax(abs(inv(A).b-1))
\end{eulerprompt}
\begin{euleroutput}
    8.790745908981989e-13 
\end{euleroutput}
\begin{eulercomment}
If the system does not have a solution, a linear fit minimizes the norm of
the error Ax-b.
\end{eulercomment}
\begin{eulerprompt}
>A=[1,2,3;4,5,6;7,8,9]
\end{eulerprompt}
\begin{euleroutput}
              1             2             3 
              4             5             6 
              7             8             9 
\end{euleroutput}
\begin{eulercomment}
The determinant of this matrix is 0.
\end{eulercomment}
\begin{eulerprompt}
>det(A)
\end{eulerprompt}
\begin{euleroutput}
  0
\end{euleroutput}
\eulerheading{Symbolic Matrices}
\begin{eulercomment}
Maxima has symbolic matrices. Of course, Maxima can be used for such simple linear algebra problems.
We can define the matrix for Euler and Maxima with \&:=, and then use
it in symbolic expressions.
The usual [...] form to define matrices can be used in Euler to define symbolic
matrices.
\end{eulercomment}
\begin{eulerprompt}
>A &= [a,1,1;1,a,1;1,1,a]; $A
\end{eulerprompt}
\begin{eulerformula}
\[
\begin{pmatrix}a & 1 & 1 \\ 1 & a & 1 \\ 1 & 1 & a \\ \end{pmatrix}
\]
\end{eulerformula}
\begin{eulerprompt}
>$&det(A), $&factor(%)
\end{eulerprompt}
\begin{eulerformula}
\[
\left(a-1\right)^2\,\left(a+2\right)
\]
\end{eulerformula}
\eulerimg{0}{images/EMT4Aljabar_Naela Rizqy Arofah_22305144042-081-large.png}
\begin{eulerprompt}
>$&invert(A) with a=0
\end{eulerprompt}
\begin{eulerformula}
\[
\begin{pmatrix}-\frac{1}{2} & \frac{1}{2} & \frac{1}{2} \\ \frac{1  }{2} & -\frac{1}{2} & \frac{1}{2} \\ \frac{1}{2} & \frac{1}{2} & -  \frac{1}{2} \\ \end{pmatrix}
\]
\end{eulerformula}
\begin{eulerprompt}
>A &= [1,a;b,2]; $A
\end{eulerprompt}
\begin{eulerformula}
\[
\begin{pmatrix}1 & a \\ b & 2 \\ \end{pmatrix}
\]
\end{eulerformula}
\begin{eulercomment}
Like all symbolic variables, these matrices can be used in other
symbolic expressions.
\end{eulercomment}
\begin{eulerprompt}
>$&det(A-x*ident(2)), $&solve(%,x)
\end{eulerprompt}
\begin{eulerformula}
\[
\left[ x=\frac{3-\sqrt{4\,a\,b+1}}{2} , x=\frac{\sqrt{4\,a\,b+1}+3  }{2} \right] 
\]
\end{eulerformula}
\eulerimg{1}{images/EMT4Aljabar_Naela Rizqy Arofah_22305144042-085-large.png}
\begin{eulercomment}
The eigenvalues can also be computed automatically. The result is a
vector with two vectors of eigenvalues and multiplicities.
\end{eulercomment}
\begin{eulerprompt}
>$&eigenvalues([a,1;1,a])
\end{eulerprompt}
\begin{eulerformula}
\[
\left[ \left[ a-1 , a+1 \right]  , \left[ 1 , 1 \right]  \right] 
\]
\end{eulerformula}
\begin{eulercomment}
To extract a specific eigenvector needs careful indexing.
\end{eulercomment}
\begin{eulerprompt}
>$&eigenvectors([a,1;1,a]), &%[2][1][1]
\end{eulerprompt}
\begin{eulerformula}
\[
\left[ \left[ \left[ a-1 , a+1 \right]  , \left[ 1 , 1 \right]    \right]  , \left[ \left[ \left[ 1 , -1 \right]  \right]  , \left[   \left[ 1 , 1 \right]  \right]  \right]  \right] 
\]
\end{eulerformula}
\begin{euleroutput}
  
                                 [1, - 1]
  
\end{euleroutput}
\begin{eulercomment}
Symbolic matrices can be evaluated in Euler numerically just like
other symbolic expressions.
\end{eulercomment}
\begin{eulerprompt}
>A(a=4,b=5)
\end{eulerprompt}
\begin{euleroutput}
              1             4 
              5             2 
\end{euleroutput}
\begin{eulercomment}
In symbolic expressions, use with.
\end{eulercomment}
\begin{eulerprompt}
>$&A with [a=4,b=5]
\end{eulerprompt}
\begin{eulerformula}
\[
\begin{pmatrix}1 & 4 \\ 5 & 2 \\ \end{pmatrix}
\]
\end{eulerformula}
\begin{eulercomment}
Access to rows of symbolic matrices work just like with numerical
matrices.
\end{eulercomment}
\begin{eulerprompt}
>$&A[1]
\end{eulerprompt}
\begin{eulerformula}
\[
\left[ 1 , a \right] 
\]
\end{eulerformula}
\begin{eulercomment}
A symbolic expression can contain an assignment. And that changes the
matrix A.
\end{eulercomment}
\begin{eulerprompt}
>&A[1,1]:=t+1; $&A
\end{eulerprompt}
\begin{eulerformula}
\[
\begin{pmatrix}t+1 & a \\ b & 2 \\ \end{pmatrix}
\]
\end{eulerformula}
\begin{eulercomment}
There are symbolic functions in Maxima to create vectors and matrices.
For this, refer to the documentation of Maxima or to the tutorial
about Maxima in EMT.
\end{eulercomment}
\begin{eulerprompt}
>v &= makelist(1/(i+j),i,1,3); $v
\end{eulerprompt}
\begin{eulerformula}
\[
\left[ \frac{1}{j+1} , \frac{1}{j+2} , \frac{1}{j+3} \right] 
\]
\end{eulerformula}
\begin{eulerttcomment}
 
\end{eulerttcomment}
\begin{eulerprompt}
>B &:= [1,2;3,4]; $B, $&invert(B)
\end{eulerprompt}
\begin{eulerformula}
\[
\begin{pmatrix}-2 & 1 \\ \frac{3}{2} & -\frac{1}{2} \\   \end{pmatrix}
\]
\end{eulerformula}
\eulerimg{1}{images/EMT4Aljabar_Naela Rizqy Arofah_22305144042-093-large.png}
\begin{eulercomment}
The result can be evaluated numerically in Euler. For more information
about Maxima, see the introduction to Maxima.
\end{eulercomment}
\begin{eulerprompt}
>$&invert(B)()
\end{eulerprompt}
\begin{euleroutput}
             -2             1 
            1.5          -0.5 
\end{euleroutput}
\begin{eulercomment}
Euler has also a powerful function xinv(), which makes a bigger effort
and gets more exact results.

Note, that with \&:= the matrix B has been defined as symbolic in
symbolic expressions and as numerical in numerical expressions. So we
can use it here.
\end{eulercomment}
\begin{eulerprompt}
>longest B.xinv(B)
\end{eulerprompt}
\begin{euleroutput}
                        1                       0 
                        0                       1 
\end{euleroutput}
\begin{eulercomment}
E.g. the eigenvalues of A can be computed numerically.
\end{eulercomment}
\begin{eulerprompt}
>A=[1,2,3;4,5,6;7,8,9]; real(eigenvalues(A))
\end{eulerprompt}
\begin{euleroutput}
  [16.1168,  -1.11684,  0]
\end{euleroutput}
\begin{eulercomment}
Or symbolically. See the tutorial about Maxima for details on this.
\end{eulercomment}
\begin{eulerprompt}
>$&eigenvalues(@A)
\end{eulerprompt}
\begin{eulerformula}
\[
\left[ \left[ \frac{15-3\,\sqrt{33}}{2} , \frac{3\,\sqrt{33}+15}{2}   , 0 \right]  , \left[ 1 , 1 , 1 \right]  \right] 
\]
\end{eulerformula}
\eulerheading{Numerical Values in symbolic Expressions}
\begin{eulercomment}
A symbolic expression is just a string containing an expression. If we
want to define a value both for symbolic expressions and for numerical
expressions, we must use "\&:=".
\end{eulercomment}
\begin{eulerprompt}
>A &:= [1,pi;4,5]
\end{eulerprompt}
\begin{euleroutput}
              1       3.14159 
              4             5 
\end{euleroutput}
\begin{eulercomment}
There is still a difference between the numerical and the symbolic
form. When transferring the matrix to the symbolic form, fractional
approximations for reals will be used.
\end{eulercomment}
\begin{eulerprompt}
>$&A
\end{eulerprompt}
\begin{eulerformula}
\[
\begin{pmatrix}1 & \frac{1146408}{364913} \\ 4 & 5 \\ \end{pmatrix}
\]
\end{eulerformula}
\begin{eulercomment}
To avoid this, there is the function "mxmset(variable)".
\end{eulercomment}
\begin{eulerprompt}
>mxmset(A); $&A
\end{eulerprompt}
\begin{eulerformula}
\[
\begin{pmatrix}1 & 3.141592653589793 \\ 4 & 5 \\ \end{pmatrix}
\]
\end{eulerformula}
\begin{eulercomment}
Maxima can also compute with floating point numbers, and even with big
floating numbers with 32 digits. The evaluation is much slower,
however.
\end{eulercomment}
\begin{eulerprompt}
>$&bfloat(sqrt(2)), $&float(sqrt(2))
\end{eulerprompt}
\begin{eulerformula}
\[
1.414213562373095
\]
\end{eulerformula}
\eulerimg{0}{images/EMT4Aljabar_Naela Rizqy Arofah_22305144042-098-large.png}
\begin{eulercomment}
The precision of the big floating point numbers can be changed.
\end{eulercomment}
\begin{eulerprompt}
>&fpprec:=100; &bfloat(pi)
\end{eulerprompt}
\begin{euleroutput}
  
          3.14159265358979323846264338327950288419716939937510582097494\(\backslash\)
  4592307816406286208998628034825342117068b0
  
\end{euleroutput}
\begin{eulercomment}
A numerical variable can be used in any symbolic expressions using
"@var".

Note that this is only necessary, if the variable has been defined
with ":=" or "=" as a numerical variable.
\end{eulercomment}
\begin{eulerprompt}
>B:=[1,pi;3,4]; $&det(@B)
\end{eulerprompt}
\begin{eulerformula}
\[
-5.424777960769379
\]
\end{eulerformula}
\begin{eulercomment}
\begin{eulercomment}
\eulerheading{Demo - Interest Rates}
\begin{eulercomment}
Below, we use Euler Math Toolbox (EMT) for the calculation of interest rates.
We do that numerically and symbolically to show you how Euler can be used to
solve real life problems.

Assume you have a seed capital of 5000 (say in dollars).
\end{eulercomment}
\begin{eulerprompt}
>K=5000
\end{eulerprompt}
\begin{euleroutput}
  5000
\end{euleroutput}
\begin{eulercomment}
Now we assume an interest rate of 3\% per year. Let us add one simple rate and
compute the result.
\end{eulercomment}
\begin{eulerprompt}
>K*1.03
\end{eulerprompt}
\begin{euleroutput}
  5150
\end{euleroutput}
\begin{eulercomment}
Euler would understand the following syntax too.
\end{eulercomment}
\begin{eulerprompt}
>K+K*3%
\end{eulerprompt}
\begin{euleroutput}
  5150
\end{euleroutput}
\begin{eulercomment}
But it is easier to use the factor
\end{eulercomment}
\begin{eulerprompt}
>q=1+3%, K*q
\end{eulerprompt}
\begin{euleroutput}
  1.03
  5150
\end{euleroutput}
\begin{eulercomment}
For 10 years, we can simply multiply the factors and get the final value with
compound interest rates.
\end{eulercomment}
\begin{eulerprompt}
>K*q^10
\end{eulerprompt}
\begin{euleroutput}
  6719.58189672
\end{euleroutput}
\begin{eulercomment}
For our purposes, we can set the format to 2 digits after the decimal dot.
\end{eulercomment}
\begin{eulerprompt}
>format(12,2); K*q^10
\end{eulerprompt}
\begin{euleroutput}
      6719.58 
\end{euleroutput}
\begin{eulercomment}
Let us print that rounded to 2 digits in a complete sentence.
\end{eulercomment}
\begin{eulerprompt}
>"Starting from " + K + "$ you get " + round(K*q^10,2) + "$."
\end{eulerprompt}
\begin{euleroutput}
  Starting from 5000$ you get 6719.58$.
\end{euleroutput}
\begin{eulercomment}
What if we want to know the intermediate results from year 1 to year 9? For
this, Euler's matrix language is a big help. You do not have to write a loop,
but simply enter
\end{eulercomment}
\begin{eulerprompt}
>K*q^(0:10)
\end{eulerprompt}
\begin{euleroutput}
  Real 1 x 11 matrix
  
      5000.00     5150.00     5304.50     5463.64     ...
\end{euleroutput}
\begin{eulercomment}
How does this miracle work? First the expression 0:10 returns a vector of
integers.
\end{eulercomment}
\begin{eulerprompt}
>short 0:10
\end{eulerprompt}
\begin{euleroutput}
  [0,  1,  2,  3,  4,  5,  6,  7,  8,  9,  10]
\end{euleroutput}
\begin{eulercomment}
Then all operators and functions in Euler can be applied to vectors element
for element. So
\end{eulercomment}
\begin{eulerprompt}
>short q^(0:10)
\end{eulerprompt}
\begin{euleroutput}
  [1,  1.03,  1.0609,  1.0927,  1.1255,  1.1593,  1.1941,  1.2299,
  1.2668,  1.3048,  1.3439]
\end{euleroutput}
\begin{eulercomment}
is a vector of factors q\textasciicircum{}0 to q\textasciicircum{}10. This is multiplied by K, and we get the
vector of values.
\end{eulercomment}
\begin{eulerprompt}
>VK=K*q^(0:10);
\end{eulerprompt}
\begin{eulercomment}
Of course, the realistic way to compute these interest rates would be to
round to the nearest cent after each year. Let us add a function for this.
\end{eulercomment}
\begin{eulerprompt}
>function oneyear (K) := round(K*q,2)
\end{eulerprompt}
\begin{eulercomment}
Let us compare the two results, with and without rounding.
\end{eulercomment}
\begin{eulerprompt}
>longest oneyear(1234.57), longest 1234.57*q
\end{eulerprompt}
\begin{euleroutput}
                  1271.61 
                1271.6071 
\end{euleroutput}
\begin{eulercomment}
Now there is no simple formula for the n-th year, and we must loop over the
years. Euler provides many solutions for this.

The easiest way is the function iterate, which iterates a given function a
number of times.
\end{eulercomment}
\begin{eulerprompt}
>VKr=iterate("oneyear",5000,10)
\end{eulerprompt}
\begin{euleroutput}
  Real 1 x 11 matrix
  
      5000.00     5150.00     5304.50     5463.64     ...
\end{euleroutput}
\begin{eulercomment}
We can print that in a friendly way, using our format with fixed decimal
places.
\end{eulercomment}
\begin{eulerprompt}
>VKr'
\end{eulerprompt}
\begin{euleroutput}
      5000.00 
      5150.00 
      5304.50 
      5463.64 
      5627.55 
      5796.38 
      5970.27 
      6149.38 
      6333.86 
      6523.88 
      6719.60 
\end{euleroutput}
\begin{eulercomment}
To get a specific element of the vector, we use indices in square brackets.
\end{eulercomment}
\begin{eulerprompt}
>VKr[2], VKr[1:3]
\end{eulerprompt}
\begin{euleroutput}
      5150.00 
      5000.00     5150.00     5304.50 
\end{euleroutput}
\begin{eulercomment}
Surprisingly, we can also use a vector of indices. Remember that 1:3 produced
the vector [1,2,3].

Let us compare the last element of the rounded values with the full values.
\end{eulercomment}
\begin{eulerprompt}
>VKr[-1], VK[-1]
\end{eulerprompt}
\begin{euleroutput}
      6719.60 
      6719.58 
\end{euleroutput}
\begin{eulercomment}
The difference is very small.

\begin{eulercomment}
\eulerheading{Solving Equations}
\begin{eulercomment}
Now we take a more advanced function, which adds a certain rate of money each
year.
\end{eulercomment}
\begin{eulerprompt}
>function onepay (K) := K*q+R
\end{eulerprompt}
\begin{eulercomment}
We do not have to specify q or R for the definition of the function. Only if
we run the command, we have to define these values. We select R=200.
\end{eulercomment}
\begin{eulerprompt}
>R=200; iterate("onepay",5000,10)
\end{eulerprompt}
\begin{euleroutput}
  Real 1 x 11 matrix
  
      5000.00     5350.00     5710.50     6081.82     ...
\end{euleroutput}
\begin{eulercomment}
What if we remove the same amount each year?
\end{eulercomment}
\begin{eulerprompt}
>R=-200; iterate("onepay",5000,10)
\end{eulerprompt}
\begin{euleroutput}
  Real 1 x 11 matrix
  
      5000.00     4950.00     4898.50     4845.45     ...
\end{euleroutput}
\begin{eulercomment}
We see that the money decreases. Obviously, if we get only 150 of interest in
the first year, but remove 200, we lose money each year.

How can we determine the number of years the money will last? We would have
to write a loop for this. The easiest way is to iterate long enough.
\end{eulercomment}
\begin{eulerprompt}
>VKR=iterate("onepay",5000,50)
\end{eulerprompt}
\begin{euleroutput}
  Real 1 x 51 matrix
  
      5000.00     4950.00     4898.50     4845.45     ...
\end{euleroutput}
\begin{eulercomment}
Using the matrix language, we can determine the first negative value in the
following way.
\end{eulercomment}
\begin{eulerprompt}
>min(nonzeros(VKR<0))
\end{eulerprompt}
\begin{euleroutput}
        48.00 
\end{euleroutput}
\begin{eulercomment}
The reason for this is that nonzeros(VKR\textless{}0) returns a vector of indices i,
where VKR[i]\textless{}0, and min computes the minimal index.

Since vectors always start with index 1, the answer is 47 years.

The function iterate() has one more trick. It can take an end condition as an
argument. Then it will return the value and the number of iterations.
\end{eulercomment}
\begin{eulerprompt}
>\{x,n\}=iterate("onepay",5000,till="x<0"); x, n,
\end{eulerprompt}
\begin{euleroutput}
       -19.83 
        47.00 
\end{euleroutput}
\begin{eulercomment}
Let us try to answer a more ambiguous question. Assume we know that the value
is 0 after 50 years. What would be the interest rate?

This is a question, which can only be answered numerically. Below, we will
derive the necessary formulas. Then you will see that there is no easy
formula for the interest rate. But for now, we aim for a numerical solution.

The first step is to define a function which does the iteration n times. We
add all parameters to this function.
\end{eulercomment}
\begin{eulerprompt}
>function f(K,R,P,n) := iterate("x*(1+P/100)+R",K,n;P,R)[-1]
\end{eulerprompt}
\begin{eulercomment}
The iteration is just as above

\end{eulercomment}
\begin{eulerformula}
\[
x_{n+1} = x_n \cdot \left(1+ \frac{P}{100}\right) + R
\]
\end{eulerformula}
\begin{eulercomment}
But we do longer use the global value of R in our expression. Functions like
iterate() have a special trick in Euler. You can pass the values of variables
in the expression as semicolon parameters. In this case P and R.

Moreover, we are only interested in the last value. So we take the index
[-1].

Let us try a test.
\end{eulercomment}
\begin{eulerprompt}
>f(5000,-200,3,47)
\end{eulerprompt}
\begin{euleroutput}
       -19.83 
\end{euleroutput}
\begin{eulercomment}
Now we can solve our problem.
\end{eulercomment}
\begin{eulerprompt}
>solve("f(5000,-200,x,50)",3)
\end{eulerprompt}
\begin{euleroutput}
         3.15 
\end{euleroutput}
\begin{eulercomment}
The solve routine solves expression=0 for the variable x. The answer is 3.15\%
per year. We take the start value of 3\% for the algorithm. The solve()
function always needs a start value.

We can use the same function to solve the following question: How much can we
remove per year so that the seed capital is exhausted after 20 years assuming
an interest rate of 3\% per year.
\end{eulercomment}
\begin{eulerprompt}
>solve("f(5000,x,3,20)",-200)
\end{eulerprompt}
\begin{euleroutput}
      -336.08 
\end{euleroutput}
\begin{eulercomment}
Note that you cannot solve for the number of years, since our function
assumes n to be an integer value.

\end{eulercomment}
\eulersubheading{Solusi Simbolis Masalah Suku Bunga}
\begin{eulercomment}
Kita dapat menggunakan bagian simbolis dari Euler untuk mempelajari
masalahnya. Pertama kita mendefinisikan fungsi onepay() kita secara
simbolis.
\end{eulercomment}
\begin{eulerprompt}
>function op(K) &= K*q+R; $&op(K)
\end{eulerprompt}
\begin{eulerformula}
\[
R+q\,K
\]
\end{eulerformula}
\begin{eulercomment}
Sekarang kita dapat mengulanginya.
\end{eulercomment}
\begin{eulerprompt}
>$&op(op(op(op(K)))), $&expand
\end{eulerprompt}
\begin{eulerformula}
\[
{\it expand}
\]
\end{eulerformula}
\eulerimg{0}{images/EMT4Aljabar_Naela Rizqy Arofah_22305144042-102-large.png}
\begin{eulercomment}
Kita melihat sebuah pola. Setelah n periode yang kita miliki

\end{eulercomment}
\begin{eulerformula}
\[
Kn = q^n K + R (1+q+\ldots+q^{n-1}) = q^n K + \frac{q^n-1}{q-1} R
\]
\end{eulerformula}
\begin{eulercomment}
Rumusnya adalah rumus jumlah geometri yang diketahui Maxima.\\
uk mempelajari masalahnya. Pertama kita mendefinisikan fungsi onepay()
kita secara simbolis.
\end{eulercomment}
\begin{eulerprompt}
>&sum(q^k,k,0,n-1); $& % = ev(%,simpsum)
\end{eulerprompt}
\begin{eulerformula}
\[
\sum_{k=0}^{n-1}{q^{k}}=\frac{q^{n}-1}{q-1}
\]
\end{eulerformula}
\begin{eulercomment}
Ini agak rumit. Jumlahnya dievaluasi dengan tanda "simpsum" untuk
menguranginya menjadi hasil bagi. Mari kita membuat fungsi untuk ini.
\end{eulercomment}
\begin{eulerprompt}
>function fs(K,R,P,n) &= (1+P/100)^n*K + ((1+P/100)^n-1)/(P/100)*R; $&fs(K,R,P,n)
\end{eulerprompt}
\begin{eulerformula}
\[
\frac{100\,\left(\left(\frac{P}{100}+1\right)^{n}-1\right)\,R}{P}+K  \,\left(\frac{P}{100}+1\right)^{n}
\]
\end{eulerformula}
\begin{eulercomment}
Fungsinya sama dengan fungsi f kita sebelumnya. Tapi ini lebih
efektif.
\end{eulercomment}
\begin{eulerprompt}
>longest f(5000,-200,3,47), longest fs(5000,-200,3,47)
\end{eulerprompt}
\begin{euleroutput}
       -19.82504734650985 
       -19.82504734652684 
\end{euleroutput}
\begin{eulercomment}
Sekarang kita dapat menggunakannya untuk menanyakan waktu n. Kapan
modal kita habis? Perkiraan awal kami adalah 30 tahun.
\end{eulercomment}
\begin{eulerprompt}
>solve("fs(5000,-330,3,x)",30)
\end{eulerprompt}
\begin{euleroutput}
        20.51 
\end{euleroutput}
\begin{eulercomment}
Jawaban ini mengatakan akan menjadi negatif setelah 21 tahun.

Kita juga dapat menggunakan sisi simbolis Euler untuk menghitung rumus
pembayaran.

Asumsikan kita mendapatkan pinjaman sebesar K, dan membayar n
pembayaran sebesar R (dimulai setelah tahun pertama)  meninggalkan
sisa hutang sebesar Kn (pada saat pembayaran terakhir). Rumusnya jelas
\end{eulercomment}
\begin{eulerprompt}
>equ &= fs(K,R,P,n)=Kn; $&equ
\end{eulerprompt}
\begin{eulerformula}
\[
\frac{100\,\left(\left(\frac{P}{100}+1\right)^{n}-1\right)\,R}{P}+K  \,\left(\frac{P}{100}+1\right)^{n}={\it Kn}
\]
\end{eulerformula}
\begin{eulercomment}
Biasanya rumus ini diberikan dalam bentuk:

\end{eulercomment}
\begin{eulerformula}
\[
i = \frac{P}{100}
\]
\end{eulerformula}
\begin{eulerprompt}
>equ &= (equ with P=100*i); $&equ
\end{eulerprompt}
\begin{eulerformula}
\[
\frac{\left(\left(i+1\right)^{n}-1\right)\,R}{i}+\left(i+1\right)^{  n}\,K={\it Kn}
\]
\end{eulerformula}
\begin{eulercomment}
Kita dapat menyelesaikan nilai R secara simbolis.
\end{eulercomment}
\begin{eulerprompt}
>$&solve(equ,R)
\end{eulerprompt}
\begin{eulerformula}
\[
\left[ R=\frac{i\,{\it Kn}-i\,\left(i+1\right)^{n}\,K}{\left(i+1  \right)^{n}-1} \right] 
\]
\end{eulerformula}
\begin{eulercomment}
Seperti yang Anda lihat dari rumusnya, fungsi ini mengembalikan
kesalahan floating point untuk i=0. Euler tetap merencanakannya.


entu saja kita memiliki limit berikut
\end{eulercomment}
\begin{eulerprompt}
>$&limit(R(5000,0,x,10),x,0)
\end{eulerprompt}
\begin{eulerformula}
\[
\lim_{x\rightarrow 0}{R\left(5000 , 0 , x , 10\right)}
\]
\end{eulerformula}
\begin{eulercomment}
Yang jelas tanpa bunga kita harus membayar kembali 10 tarif 500.

Persamaan tersebut juga dapat diselesaikan untuk n. Akan terlihat
lebih bagus jika kita menerapkan beberapa penyederhanaan padanya.
\end{eulercomment}
\begin{eulerprompt}
>fn &= solve(equ,n) | ratsimp; $&fn
\end{eulerprompt}
\begin{eulerformula}
\[
\left[ n=\frac{\log \left(\frac{R+i\,{\it Kn}}{R+i\,K}\right)}{  \log \left(i+1\right)} \right] 
\]
\end{eulerformula}
\eulersubheading{Latihan Soal R2}
\begin{eulercomment}
Soal No. 49\\
\end{eulercomment}
\begin{eulerformula}
\[
\left(\frac{24a^{10}b^{-8}c^7}{12a^6b^{-3}c^5}\right)^{-5}
\]
\end{eulerformula}
\begin{eulerprompt}
>$&((24*a^10*b^(-8)*c^7)/(12*a^6*b^(-3)*c^5))^(-5)
\end{eulerprompt}
\begin{eulerformula}
\[
\frac{b^{25}}{32\,a^{20}\,c^{10}}
\]
\end{eulerformula}
\begin{eulercomment}
Soal No. 50
\end{eulercomment}
\begin{eulerprompt}
>$&((125*p^12*q^(-14)*r^22)/(25*p^8*q^6*r^(-15)))^(-4)
\end{eulerprompt}
\begin{eulerformula}
\[
\frac{q^{80}}{625\,p^{16}\,r^{148}}
\]
\end{eulerformula}
\begin{eulerprompt}
>$&(2^6*2^(-3)/(2^10)/(2^(-8)))
\end{eulerprompt}
\begin{eulerformula}
\[
2
\]
\end{eulerformula}
\begin{eulercomment}
Soal No. 91
\end{eulercomment}
\begin{eulerprompt}
>$&(4*(8-6)^2-4*3+2*8)/(3^1+19^0)
\end{eulerprompt}
\begin{eulerformula}
\[
5
\]
\end{eulerformula}
\begin{eulercomment}
Soal No. 92
\end{eulercomment}
\begin{eulerprompt}
>$&((4*(8-6)^2+4)*(3-2*8))/(2^2*(2^3+5))
\end{eulerprompt}
\begin{eulerformula}
\[
-5
\]
\end{eulerformula}
\begin{eulercomment}
Soal No. 104
\end{eulercomment}
\begin{eulerprompt}
>$&((m^(x-b)*n(x+b)^x)*(m^b*n^(-b))^x)
\end{eulerprompt}
\begin{eulerformula}
\[
m^{x-b}\,\left(\frac{m^{b}}{n^{b}}\right)^{x}\,n^{x}\left(x+b  \right)
\]
\end{eulerformula}
\eulersubheading{Latihan Soal R3}
\begin{eulercomment}
Soal No. 9
\end{eulercomment}
\begin{eulerprompt}
>$&((3*x^2-2*x-x^3+2)-(5*x^2-8*x-x^3+4))
\end{eulerprompt}
\begin{eulerformula}
\[
-2\,x^2+6\,x-2
\]
\end{eulerformula}
\begin{eulercomment}
Soal No. 13
\end{eulercomment}
\begin{eulerprompt}
>$&((3*a^2)*((-7)*a^4))
\end{eulerprompt}
\begin{eulerformula}
\[
-21\,a^6
\]
\end{eulerformula}
\begin{eulercomment}
Soal No. 14
\end{eulercomment}
\begin{eulerprompt}
>$&((8*y^5)*(9*y))
\end{eulerprompt}
\begin{eulerformula}
\[
72\,y^6
\]
\end{eulerformula}
\begin{eulercomment}
Soal No. 15
\end{eulercomment}
\begin{eulerprompt}
>$&((6*x*y^3)*(9*x^4*y^2))
\end{eulerprompt}
\begin{eulerformula}
\[
54\,x^5\,y^5
\]
\end{eulerformula}
\begin{eulercomment}
Soal. No 36
\end{eulercomment}
\begin{eulerprompt}
>$%((4*x^2-5*y)^2)
\end{eulerprompt}
\begin{eulerformula}
\[
\left(54\,x^5\,y^5\right)(\left(4\,x^2-5\,y\right)^2)
\]
\end{eulerformula}
\eulersubheading{Latihan Soal R4}
\begin{eulercomment}
Soal No. 23
\end{eulercomment}
\begin{eulerprompt}
>$& factor(t^2+8*t+15)
\end{eulerprompt}
\begin{eulerformula}
\[
\left(t+3\right)\,\left(t+5\right)
\]
\end{eulerformula}
\begin{eulercomment}
Soal No. 24
\end{eulercomment}
\begin{eulerprompt}
>$& factor(y^2+12*y+27)
\end{eulerprompt}
\begin{eulerformula}
\[
\left(y+3\right)\,\left(y+9\right)
\]
\end{eulerformula}
\begin{eulercomment}
Soal No. 121
\end{eulercomment}
\begin{eulerprompt}
>$& factor(y^4-84+5*y^2)
\end{eulerprompt}
\begin{eulerformula}
\[
\left(y^2-7\right)\,\left(y^2+12\right)
\]
\end{eulerformula}
\begin{eulercomment}
Soal No. 77
\end{eulercomment}
\begin{eulerprompt}
>$& factor((18*a^2*b)-(15*a*b^2))
\end{eulerprompt}
\begin{eulerformula}
\[
-3\,a\,b\,\left(5\,b-6\,a\right)
\]
\end{eulerformula}
\begin{eulercomment}
Soal No. 78
\end{eulercomment}
\begin{eulerprompt}
>$& factor(4*x^2*y+12*x*y^2)
\end{eulerprompt}
\begin{eulerformula}
\[
4\,x\,y\,\left(3\,y+x\right)
\]
\end{eulerformula}
\eulersubheading{Latihan Soal R5}
\begin{eulercomment}
Soal No. 31
\end{eulercomment}
\begin{eulerprompt}
>$& solve(7*(3*x+6)= 11-(x+2))
\end{eulerprompt}
\begin{eulerformula}
\[
\left[ x=-\frac{3}{2} \right] 
\]
\end{eulerformula}
\begin{eulercomment}
Soal No. 32
\end{eulercomment}
\begin{eulerprompt}
>$& solve(9*(2*x+8)= 20-(x-5))
\end{eulerprompt}
\begin{eulerformula}
\[
\left[ x=-\frac{47}{19} \right] 
\]
\end{eulerformula}
\begin{eulercomment}
Soal No. 35
\end{eulercomment}
\begin{eulerprompt}
>$& solve(x^2+3*x-28=0)
\end{eulerprompt}
\begin{eulerformula}
\[
\left[ x=4 , x=-7 \right] 
\]
\end{eulerformula}
\begin{eulercomment}
Soal No. 36
\end{eulercomment}
\begin{eulerprompt}
>$& solve(y^2-4*y-45=0)
\end{eulerprompt}
\begin{eulerformula}
\[
\left[ y=9 , y=-5 \right] 
\]
\end{eulerformula}
\begin{eulercomment}
Soal No. 37
\end{eulercomment}
\begin{eulerprompt}
>$& solve(x^2+5*x=0)
\end{eulerprompt}
\begin{eulerformula}
\[
\left[ x=-5 , x=0 \right] 
\]
\end{eulerformula}
\eulersubheading{Latihan Soal R6}
\begin{eulercomment}
Soal No. 9
\end{eulercomment}
\begin{eulerprompt}
>$&ratsimp((x^2-4)/(x^2-4*x+4))
\end{eulerprompt}
\begin{eulerformula}
\[
\frac{x+2}{x-2}
\]
\end{eulerformula}
\begin{eulercomment}
Soal No. 11
\end{eulercomment}
\begin{eulerprompt}
>$&ratsimp((x^3-6*x^2+9*x)/(x^3-3*x^2))
\end{eulerprompt}
\begin{eulerformula}
\[
\frac{x-3}{x}
\]
\end{eulerformula}
\begin{eulercomment}
Soal No. 14
\end{eulercomment}
\begin{eulerprompt}
>$&ratsimp((2*x^2-20*x+50)/(10*x^2-30*x-100))
\end{eulerprompt}
\begin{eulerformula}
\[
\frac{x-5}{5\,x+10}
\]
\end{eulerformula}
\begin{eulercomment}
Soal No. 15
\end{eulercomment}
\begin{eulerprompt}
>$&ratsimp((6-x)/(x^2-36))
\end{eulerprompt}
\begin{eulerformula}
\[
-\frac{1}{x+6}
\]
\end{eulerformula}
\begin{eulercomment}
Soal No. 20
\end{eulercomment}
\begin{eulerprompt}
>$&ratsimp(((x^2-2*x-35)/(2*x^3-3*x^2))*((4*x^3-9*x)/(7*x-49)))
\end{eulerprompt}
\begin{eulerformula}
\[
\frac{2\,x^2+13\,x+15}{7\,x}
\]
\end{eulerformula}
\eulersubheading{Latihan soal 3.1}
\begin{eulercomment}
Soal No. 11
\end{eulercomment}
\begin{eulerprompt}
>$ powerdisp: true
\end{eulerprompt}
\begin{eulerformula}
\[
\mathbf{true}
\]
\end{eulerformula}
\begin{eulerprompt}
> $&((-5+3*I)+(7+8*I))
\end{eulerprompt}
\begin{eulerformula}
\[
2+11\,i
\]
\end{eulerformula}
\begin{eulercomment}
Soal No. 31
\end{eulercomment}
\begin{eulerprompt}
> $& (sqrt(-4)*(sqrt(-36)))
\end{eulerprompt}
\begin{eulerformula}
\[
-12
\]
\end{eulerformula}
\begin{eulercomment}
Soal No. 53
\end{eulercomment}
\begin{eulerprompt}
>$& solve(2*x^2+1=5*x)
\end{eulerprompt}
\begin{eulerformula}
\[
\left[ x=\frac{5-\sqrt{17}}{4} , x=\frac{5+\sqrt{17}}{4} \right] 
\]
\end{eulerformula}
\begin{eulercomment}
Soal No. 84
\end{eulercomment}
\begin{eulerprompt}
>$& solve(y^4-15*y^2-16=0)
\end{eulerprompt}
\begin{eulerformula}
\[
\left[ y=-i , y=i , y=-4 , y=4 \right] 
\]
\end{eulerformula}
\begin{eulercomment}
Soal No. 79
\end{eulercomment}
\begin{eulerprompt}
>$& solve(x^4-3*x^2+2=0)
\end{eulerprompt}
\begin{eulerformula}
\[
\left[ x=-\sqrt{2} , x=\sqrt{2} , x=-1 , x=1 \right] 
\]
\end{eulerformula}
\eulersubheading{Latihan Soal 3.4}
\begin{eulercomment}
Soal No. 01
\end{eulercomment}
\begin{eulerprompt}
>$& solve((1/4)+(1/5)=(1/t))
\end{eulerprompt}
\begin{eulerformula}
\[
\left[ t=\frac{20}{9} \right] 
\]
\end{eulerformula}
\begin{eulercomment}
Soal No. 9
\end{eulercomment}
\begin{eulerprompt}
>$& solve(x+(6/x)=5)
\end{eulerprompt}
\begin{eulerformula}
\[
\left[ x=3 , x=2 \right] 
\]
\end{eulerformula}
\begin{eulercomment}
Soal No. 15
\end{eulercomment}
\begin{eulerprompt}
>$& solve((2/(x+5))+(1/(x-5))=(16/(x^2-25)))
\end{eulerprompt}
\begin{eulerformula}
\[
\left[ x=7 \right] 
\]
\end{eulerformula}
\begin{eulercomment}
Soal No. 29
\end{eulercomment}
\begin{eulerprompt}
>$& solve(sqrt(3*x-4)=1)
\end{eulerprompt}
\begin{eulerformula}
\[
\left[ x=\frac{5}{3} \right] 
\]
\end{eulerformula}
\begin{eulercomment}
Soal No. 33
\end{eulercomment}
\begin{eulerprompt}
>$& solve(sqrt(2*x-5)=2)
\end{eulerprompt}
\begin{eulerformula}
\[
\left[ x=\frac{9}{2} \right] 
\]
\end{eulerformula}
\eulersubheading{Latihan Soal 3.5}
\begin{eulercomment}
Soal NO. 23
\end{eulercomment}
\begin{eulerprompt}
>$& solve(abs(x+3)-2=8,[x])
\end{eulerprompt}
\begin{eulerformula}
\[
\left[ \left| 3+x\right| =10 \right] 
\]
\end{eulerformula}
\begin{eulercomment}
Soal No. 24
\end{eulercomment}
\begin{eulerprompt}
>$& fourier_elim(abs(x-4)+3=9,[x])
\end{eulerprompt}
\begin{eulerformula}
\[
\left[ x=10 \right] \lor \left[ x=-2 \right] 
\]
\end{eulerformula}
\begin{eulercomment}
Soal No. 25
\end{eulercomment}
\begin{eulerprompt}
>$& fourier_elim(abs(3*x+1)-4=(-1),[x])
\end{eulerprompt}
\begin{eulerformula}
\[
\left[ x=\frac{2}{3} \right] \lor \left[ x=-\frac{4}{3} \right] 
\]
\end{eulerformula}
\begin{eulercomment}
Soal No. 32
\end{eulercomment}
\begin{eulerprompt}
>$& solve(5-abs(4*x+3)=2)
\end{eulerprompt}
\begin{eulerformula}
\[
\left[ \left| 3+4\,x\right| =3 \right] 
\]
\end{eulerformula}
\begin{eulercomment}
Soal No. 30
\end{eulercomment}
\begin{eulerprompt}
>$& fourier_elim(9-abs(x-2)=7,[x])
\end{eulerprompt}
\begin{eulerformula}
\[
\left[ x=4 \right] \lor \left[ x=0 \right] 
\]
\end{eulerformula}
\eulersubheading{Latihan Soal 4.3}
\begin{eulercomment}
Soal No. 70
\end{eulercomment}
\begin{eulerprompt}
> 
>$ powerdisp: true
\end{eulerprompt}
\begin{eulerformula}
\[
\mathbf{true}
\]
\end{eulerformula}
\begin{eulerprompt}
> $&( ratsimp((x^4-y^4)/(x-y)))
\end{eulerprompt}
\begin{eulerformula}
\[
x^3+x^2\,y+x\,y^2+y^3
\]
\end{eulerformula}
\begin{eulercomment}
Soal No. 19
\end{eulercomment}
\begin{eulerprompt}
>$& ratsimp((x^4-1)/(x-1))
\end{eulerprompt}
\begin{eulerformula}
\[
1+x+x^2+x^3
\]
\end{eulerformula}
\begin{eulercomment}
Soal No. 17
\end{eulercomment}
\begin{eulerprompt}
>$ powerdisp:true
\end{eulerprompt}
\begin{eulerformula}
\[
\mathbf{true}
\]
\end{eulerformula}
\begin{eulerprompt}
>$& ratsimp((x^5+x^3-x)/(x-3))
\end{eulerprompt}
\begin{eulerformula}
\[
\frac{-x+x^3+x^5}{-3+x}
\]
\end{eulerformula}
\begin{eulercomment}
Soal No. 22
\end{eulercomment}
\begin{eulerprompt}
>$& solve(ratsimp((3*x^2-2*x^2+2)/(x-(1/4))))
\end{eulerprompt}
\begin{eulerformula}
\[
\left[ x=-\sqrt{2}\,i , x=\sqrt{2}\,i \right] 
\]
\end{eulerformula}
\begin{eulercomment}
Soal No. 11
\end{eulercomment}
\begin{eulerprompt}
>$& (ratsimp((2*x^4+7*x^3+x-12)/(x+3)))
\end{eulerprompt}
\begin{eulerformula}
\[
\frac{-12+x+7\,x^3+2\,x^4}{3+x}
\]
\end{eulerformula}
\eulersubheading{Latihan Soal R7}
\begin{eulercomment}
Soal No. 76
\end{eulercomment}
\begin{eulerprompt}
>$& factor(m^(6*n)-m^(3*n))
\end{eulerprompt}
\begin{eulerformula}
\[
m^{3\,n}\,\left(-1+m^{n}\right)\,\left(1+m^{n}+m^{2\,n}\right)
\]
\end{eulerformula}
\begin{eulercomment}
Soal No. 70
\end{eulercomment}
\begin{eulerprompt}
>$& expand(((x^n)+10)*((x^n)-4))
\end{eulerprompt}
\begin{eulerformula}
\[
-40+6\,x^{n}+x^{2\,n}
\]
\end{eulerformula}
\begin{eulercomment}
Soal No. 75
\end{eulercomment}
\begin{eulerprompt}
>$& factor(x^(2*t)-3*x^t-28)
\end{eulerprompt}
\begin{eulerformula}
\[
\left(-7+x^{t}\right)\,\left(4+x^{t}\right)
\]
\end{eulerformula}
\begin{eulercomment}
Soal No. 74
\end{eulercomment}
\begin{eulerprompt}
>$& factor(y^(2*n)+16*y^n+64)
\end{eulerprompt}
\begin{eulerformula}
\[
\left(8+y^{n}\right)^2
\]
\end{eulerformula}
\begin{eulercomment}
Soal No. 73
\end{eulercomment}
\begin{eulerprompt}
>$& expand((a^n-b^n)^3)
\end{eulerprompt}
\begin{eulerformula}
\[
a^{3\,n}-3\,a^{2\,n}\,b^{n}+3\,a^{n}\,b^{2\,n}-b^{3\,n}
\]
\end{eulerformula}
\end{eulercomment}
\chapter{EMT untuk plot 2D}
\begin{eulercomment}

Notebook ini menjelaskan tentang cara menggambar berbagaikurva dan
grafik 2D dengan software EMT. EMT menyediakan fungsi plot2d() untuk
menggambar berbagai kurva dan grafik dua dimensi (2D).
\end{eulercomment}
\eulersubheading{Plot Dasar}
Ada fungsi plot yang sangat mendasar. Terdapat koordinat layar yang
selalu berkisar antara 0 hingga 1024 di setiap sumbu,\\
tidak peduli apakah layarnya berbentuk persegi atau tidak.  Semut
terdapat koordinat plot, yang dapat diatur dengan setplot(). Pemetaan
antar koordinat bergantung pada jendela plot saat ini. Misalnya,
shrinkwindow() default memberikan ruang untuk\\
label sumbu dan judul plot.

Dalam contoh ini, kita hanya menggambar beberapa garis acak dengan
berbagai warna. Untuk rincian tentang fungsi-fungsi ini, pelajari
fungsi inti EMT.
\end{eulercomment}
\begin{eulerprompt}
>clg; // clear screen
>window(0,0,1024,1024); // use all of the window
>setplot(0,1,0,1); // set plot coordinates
>hold on; // start overwrite mode
>n=100; X=random(n,2); Y=random(n,2);  // get random points
>colors=rgb(random(n),random(n),random(n)); // get random colors
>loop 1 to n; color(colors[#]); plot(X[#],Y[#]); end; // plot
>hold off; // end overwrite mode
>insimg; // insert to notebook
\end{eulerprompt}
\eulerimg{27}{images/EMT4Plot2D-Naela Rizqy Arofah-22305144042-001.png}
\begin{eulerprompt}
>reset;
\end{eulerprompt}
\begin{eulercomment}
Grafik perlu ditahan, karena perintah plot() akan menghapus jendela
plot.


ntuk menghapus semua yang kami lakukan, kami menggunakan reset().


ntuk menampilkan gambar hasil plot di layar notebook, perintah
plot2d() dapat diakhiri dengan titik dua (:). Cara lain adalah
perintah plot2d() diakhiri dengan titik koma (;), kemudian menggunakan
perintah insimg() untuk menampilkan gambar hasil plot.

Untuk contoh lain, kita menggambar plot sebagai sisipan di plot lain.
Hal ini dilakukan dengan mendefinisikan jendela plot yang lebih kecil.\\
Perhatikan bahwa jendela ini tidak memberikan ruang untuk label sumbu
di luar jendela plot. Kita harus menambahkan beberapa\\
margin untuk ini sesuai kebutuhan. Perhatikan bahwa kita menyimpan dan
memulihkan jendela penuh, dan menahan plot saat ini\\
sementara kita memplot inset.
\end{eulercomment}
\begin{eulerprompt}
>plot2d("x^3-x");
>xw=200; yw=100; ww=300; hw=300;
>ow=window();
>window(xw,yw,xw+ww,yw+hw);
>hold on;
>barclear(xw-50,yw-10,ww+60,ww+60);
>plot2d("x^4-x",grid=6):
\end{eulerprompt}
\eulerimg{27}{images/EMT4Plot2D-Naela Rizqy Arofah-22305144042-002.png}
\begin{eulerprompt}
>hold off;
>window(ow);
\end{eulerprompt}
\begin{eulercomment}
Plot dengan banyak gambar dicapai dengan cara yang sama. Ada fungsi
utilitas figure() untuk ini.

\end{eulercomment}
\eulersubheading{Aspek Plot}
\begin{eulercomment}
Plot default menggunakan jendela plot persegi. Anda dapat mengubahnya
dengan fungsi aspek(). Jangan lupa untuk mengatur\\
ulang aspeknya nanti. Anda juga dapat mengubah default ini di menu
dengan "Set Aspect" ke rasio aspek tertentu atau ke ukuran\\
jendela grafik saat ini

Tapi Anda juga bisa mengubahnya untuk satu plot. Untuk ini, ukuran
area plot saat ini diubah, dan jendela diatur sehingga label\\
memiliki cukup ruang.
\end{eulercomment}
\begin{eulerprompt}
>aspect(2); // rasio panjang dan lebar 2:1
>plot2d(["sin(x)","cos(x)"],0,2pi):
\end{eulerprompt}
\eulerimg{13}{images/EMT4Plot2D-Naela Rizqy Arofah-22305144042-003.png}
\begin{eulerprompt}
>aspect();
>reset;
\end{eulerprompt}
\begin{eulercomment}
Fungsi reset() mengembalikan default plot termasuk rasio aspek.

\begin{eulercomment}
\eulerheading{Plot 2D di Euler}
\begin{eulercomment}
EMT Math Toolbox memiliki plot dalam 2D, baik untuk data maupun
fungsi. EMT menggunakan fungsi plot2d. Fungsi ini dapat memplot fungsi
dan data.

Dimungkinkan untuk membuat plot di Maxima menggunakan Gnuplot atau
dengan Python menggunakan Math Plot Lib.

Euler dapat memplot plot 2D dari :

- ekspresi\\
- fungsi, variabel, atau kurva berparameter,\\
- vektor nilai x-y,\\
- clouds of points in the plane,\\
- kurva implisit dengan level atau wilayah level.\\
- fungsi kompleks

Gaya plot mencakup berbagai gaya untuk garis dan titik, plot batang,
dan plot berbayang.

\begin{eulercomment}
\eulerheading{Plot Ekspresi atau Variabel}
\begin{eulercomment}
Ekspresi tunggal dalam "x" (misalnya "4*x\textasciicircum{}2") atau nama suatu fungsi
(misalnya "f") menghasilkan grafik fungsi tersebut.

Berikut adalah contoh paling dasar, yang menggunakan rentang default
dan menetapkan rentang y yang tepat agar sesuai dengan plot\\
fungsinya.

atatan: Jika Anda mengakhiri baris perintah dengan titik dua ":", plot
akan dimasukkan ke dalam jendela teks. Jika tidak, tekan\\
TAB untuk melihat plot jika jendela plot tertutup.
\end{eulercomment}
\begin{eulerprompt}
>plot2d("x^2"):
\end{eulerprompt}
\eulerimg{27}{images/EMT4Plot2D-Naela Rizqy Arofah-22305144042-004.png}
\begin{eulerprompt}
>aspect(1.5); plot2d("x^3-x"):
\end{eulerprompt}
\eulerimg{17}{images/EMT4Plot2D-Naela Rizqy Arofah-22305144042-005.png}
\begin{eulerprompt}
>a:=5.6; plot2d("exp(-a*x^2)/a"); insimg(30); // menampilkan gambar hasil plot setinggi 25 baris
\end{eulerprompt}
\eulerimg{17}{images/EMT4Plot2D-Naela Rizqy Arofah-22305144042-006.png}
\begin{eulercomment}
Dari beberapa contoh sebelumnya Anda dapat melihat bahwa aslinya
gambar plot menggunakan sumbu X dengan rentang nilai dari -2 sampai
dengan 2. Untuk mengubah rentang nilai X dan Y, Anda dapat menambahkan
nilai-nilai batas X (dan Y) di belakang ekspresi yang digambar.

Rentang plot diatur dengan parameter yang ditetapkan sebagai berikut :

- a,b: rentang x(default -2,2)\\
- c,d: rentang y (default: scale with values)\\
- r: alternatifnya radius di sekitar pusat plot\\
- cx,cy: koordinat pusat plot (default 0,0)
\end{eulercomment}
\begin{eulerprompt}
>plot2d("x^3-x",-1,2):
\end{eulerprompt}
\eulerimg{17}{images/EMT4Plot2D-Naela Rizqy Arofah-22305144042-007.png}
\begin{eulerprompt}
>plot2d("sin(x)",-2*pi,2*pi): // plot sin(x) pada interval [-2pi, 2pi]
\end{eulerprompt}
\eulerimg{17}{images/EMT4Plot2D-Naela Rizqy Arofah-22305144042-008.png}
\begin{eulerprompt}
>plot2d("cos(x)","sin(3*x)",xmin=0,xmax=2pi):
\end{eulerprompt}
\eulerimg{17}{images/EMT4Plot2D-Naela Rizqy Arofah-22305144042-009.png}
\begin{eulercomment}
Alternatif untuk titik dua adalah perintah insimg(baris), yang
menyisipkan plot yang menempati sejumlah baris teks tertentu.


alam opsi, plot dapat diatur agar muncul

- in a separate resizable window,\\
- di jendela buku catatan

Lebih banyak gaya dapat dicapai dengan perintah plot tertentu.


agaimanapun, tekan tombol tabulator untuk melihat plotnya, jika
tersembunyi.

Untuk membagi jendela menjadi beberapa plot, gunakan perintah
figure(). Dalam contoh, kita memplot x\textasciicircum{}1 hingga x\textasciicircum{}4 menjadi 4 bagian
jendela. gambar(0) mengatur ulang jendela default.

Gaya plot mencakup berbagai gaya untuk garis dan titik, plot batang,
dan plot berbayang.

\begin{eulercomment}
\eulerheading{Plot Ekspresi atau Variabel}
\begin{eulercomment}
Ekspresi tunggal dalam "x" (misalnya "4*x\textasciicircum{}2") atau nama suatu fungsi
(misalnya "f") menghasilkan grafik fungsi tersebut.

Berikut adalah contoh paling dasar, yang menggunakan rentang default
dan menetapkan rentang y yang tepat agar sesuai dengan plot\\
fungsinya.

atatan: Jika Anda mengakhiri baris perintah dengan titik dua ":", plot
akan dimasukkan ke dalam jendela teks. Jika tidak, tekan\\
TAB untuk melihat plot jika jendela plot tertutup.
\end{eulercomment}
\begin{eulerprompt}
>reset;
>figure(2,2); ...
>for n=1 to 4; figure(n); plot2d("x^"+n); end; ...
>figure(0):
\end{eulerprompt}
\eulerimg{27}{images/EMT4Plot2D-Naela Rizqy Arofah-22305144042-010.png}
\begin{eulercomment}
Di plot2d(), ada gaya alternatif yang tersedia dengan grid=x. Untuk
gambaran umum, kami menampilkan berbagai gaya kisi dalam satu gambar
(lihat di bawah untuk perintah figure()). Gaya grid=0 tidak
disertakan. Ini tidak menunjukkan kisi dan bingkai.

gaya dapat dicapai dengan perintah plot tertentu.


Bagaimanapun, tekan tombol tabulator untuk melihat plotnya, jika
tersembunyi.

Untuk membagi jendela menjadi beberapa plot, gunakan perintah
figure(). Dalam contoh, kita memplot x\textasciicircum{}1 hingga x\textasciicircum{}4 menjadi 4 bagian
jendela. gambar(0) mengatur ulang jendela default.

Gaya plot mencakup berbagai gaya untuk garis dan titik, plot batang,
dan plot berbayang.

\begin{eulercomment}
\eulerheading{Plot Ekspresi atau Variabel}
\begin{eulercomment}
Ekspresi tunggal dalam "x" (misalnya "4*x\textasciicircum{}2") atau nama suatu fungsi
(misalnya "f") menghasilkan grafik fungsi tersebut.

Berikut adalah contoh paling dasar, yang menggunakan rentang default
dan menetapkan rentang y yang tepat agar sesuai dengan plot\\
fungsinya.

Catatan: Jika Anda mengakhiri baris perintah dengan titik dua ":",
plot akan dimasukkan ke dalam jendela teks. Jika tidak, tekan\\
TAB untuk melihat plot jika jendela plot tertutup.
\end{eulercomment}
\begin{eulerprompt}
>figure(3,3); ...
>for k=1:9; figure(k); plot2d("x^3-x",-2,1,grid=k); end; ...
>figure(0):
\end{eulerprompt}
\eulerimg{27}{images/EMT4Plot2D-Naela Rizqy Arofah-22305144042-011.png}
\begin{eulercomment}
Jika argumen pada plot2d() adalah ekspresi yang diikuti oleh empat
angka, angka-angka tersebut adalah rentang x dan y untuk plot
tersebut.

Alternatifnya, a, b, c, d dapat ditentukan sebagai parameter yang
ditetapkan sebagai a=... dll.

Pada contoh berikut, kita mengubah gaya kisi, menambahkan label, dan
menggunakan label vertikal untuk sumbu y.\\
ihat plotnya, jika tersembunyi.

Untuk membagi jendela menjadi beberapa plot, gunakan perintah
figure(). Dalam contoh, kita memplot x\textasciicircum{}1 hingga x\textasciicircum{}4 menjadi 4 bagian
jendela. gambar(0) mengatur ulang jendela default.

Gaya plot mencakup berbagai gaya untuk garis dan titik, plot batang,
dan plot berbayang.

\begin{eulercomment}
\eulerheading{Plot Ekspresi atau Variabel}
\begin{eulercomment}
Ekspresi tunggal dalam "x" (misalnya "4*x\textasciicircum{}2") atau nama suatu fungsi
(misalnya "f") menghasilkan grafik fungsi tersebut.

Berikut adalah contoh paling dasar, yang menggunakan rentang default
dan menetapkan rentang y yang tepat agar sesuai dengan plot\\
fungsinya.

Catatan: Jika Anda mengakhiri baris perintah dengan titik dua ":",
plot akan dimasukkan ke dalam jendela teks. Jika tidak, tekan\\
TAB untuk melihat plot jika jendela plot tertutup.
\end{eulercomment}
\begin{eulerprompt}
>aspect(1.5); plot2d("sin(x)",0,2pi,-1.2,1.2,grid=3,xl="x",yl="sin(x)"):
\end{eulerprompt}
\eulerimg{17}{images/EMT4Plot2D-Naela Rizqy Arofah-22305144042-012.png}
\begin{eulerprompt}
>plot2d("sin(x)+cos(2*x)",0,4pi):
\end{eulerprompt}
\eulerimg{17}{images/EMT4Plot2D-Naela Rizqy Arofah-22305144042-013.png}
\begin{eulercomment}
Gambar yang dihasilkan dengan memasukkan plot ke dalam jendela teks
disimpan dalam direktori yang sama dengan buku catatan, secara default
dalam subdirektori bernama "gambar". Mereka juga digunakan oleh ekspor
HTML.

Anda cukup menandai gambar apa saja dan menyalinnya ke clipboard
dengan Ctrl-C. Tentu saja, Anda juga dapat mengekspor grafik saat ini
dengan fungsi di menu File.

Fungsi atau ekspresi di plot2d dievaluasi secara adaptif. Agar lebih
cepat, nonaktifkan plot adaptif dengan \textless{}adaptive dan tentukan jumlah
subinterval dengan n=... Hal ini hanya diperlukan dalam kasus yang
jarang terjadi.
\end{eulercomment}
\begin{eulerprompt}
>plot2d("sign(x)*exp(-x^2)",-1,1,<adaptive,n=10000):
\end{eulerprompt}
\eulerimg{17}{images/EMT4Plot2D-Naela Rizqy Arofah-22305144042-014.png}
\begin{eulerprompt}
>plot2d("x^x",r=1.2,cx=1,cy=1):
\end{eulerprompt}
\eulerimg{17}{images/EMT4Plot2D-Naela Rizqy Arofah-22305144042-015.png}
\begin{eulercomment}
Fungsi atau ekspresi di plot2d dievaluasi secara adaptif. Agar lebih
cepat, nonaktifkan plot adaptif dengan \textless{}adaptive dan tentukan jumlah
subinterval dengan n=... Hal ini hanya diperlukan dalam kasus yang
jarang terjadi.
\end{eulercomment}
\begin{eulerprompt}
>plot2d("log(x)",-0.1,2):
\end{eulerprompt}
\eulerimg{17}{images/EMT4Plot2D-Naela Rizqy Arofah-22305144042-016.png}
\begin{eulercomment}
Parameter square=true (atau \textgreater{}square) memilih rentang y secara otomatis
sehingga hasilnya adalah jendela plot persegi.\\
Perhatikan bahwa secara default, Euler menggunakan spasi persegi di
dalam jendela plot.
\end{eulercomment}
\begin{eulerprompt}
>plot2d("x^3-x",>square):
\end{eulerprompt}
\eulerimg{17}{images/EMT4Plot2D-Naela Rizqy Arofah-22305144042-017.png}
\begin{eulerprompt}
>plot2d(''integrate("sin(x)*exp(-x^2)",0,x)'',0,2): // plot integral
\end{eulerprompt}
\eulerimg{17}{images/EMT4Plot2D-Naela Rizqy Arofah-22305144042-018.png}
\begin{eulercomment}
Jika Anda memerlukan lebih banyak ruang untuk label y, panggil
shrinkwindow() dengan parameter lebih kecil, atau tetapkan nilai
positif untuk "lebih kecil" di plot2d().
\end{eulercomment}
\begin{eulerprompt}
>plot2d("gamma(x)",1,10,yl="y-values",smaller=6,<vertical):
\end{eulerprompt}
\eulerimg{17}{images/EMT4Plot2D-Naela Rizqy Arofah-22305144042-019.png}
\begin{eulercomment}
Ekspresi simbolik juga dapat digunakan karena disimpan sebagai
ekspresi string sederhana.\\
ter lebih kecil, atau tetapkan nilai positif untuk "lebih kecil" di
plot2d().
\end{eulercomment}
\begin{eulerprompt}
>x=linspace(0,2pi,1000); plot2d(sin(5x),cos(7x)):
\end{eulerprompt}
\eulerimg{17}{images/EMT4Plot2D-Naela Rizqy Arofah-22305144042-020.png}
\begin{eulerprompt}
>a:=5.6; expr &= exp(-a*x^2)/a; // define expression
>plot2d(expr,-2,2): // plot from -2 to 2
\end{eulerprompt}
\eulerimg{17}{images/EMT4Plot2D-Naela Rizqy Arofah-22305144042-021.png}
\begin{eulerprompt}
>plot2d(expr,r=1,thickness=2): // plot in a square around (0,0)
\end{eulerprompt}
\eulerimg{17}{images/EMT4Plot2D-Naela Rizqy Arofah-22305144042-022.png}
\begin{eulerprompt}
>plot2d(&diff(expr,x),>add,style="--",color=red): // add another plot
\end{eulerprompt}
\eulerimg{17}{images/EMT4Plot2D-Naela Rizqy Arofah-22305144042-023.png}
\begin{eulerprompt}
>plot2d(&diff(expr,x,2),a=-2,b=2,c=-2,d=1): // plot in rectangle
\end{eulerprompt}
\eulerimg{17}{images/EMT4Plot2D-Naela Rizqy Arofah-22305144042-024.png}
\begin{eulerprompt}
>plot2d(&diff(expr,x),a=-2,b=2,>square): // keep plot square
\end{eulerprompt}
\eulerimg{17}{images/EMT4Plot2D-Naela Rizqy Arofah-22305144042-025.png}
\begin{eulerprompt}
>plot2d("x^2",0,1,steps=1,color=red,n=10):
\end{eulerprompt}
\eulerimg{17}{images/EMT4Plot2D-Naela Rizqy Arofah-22305144042-026.png}
\begin{eulerprompt}
>plot2d("x^2",>add,steps=2,color=blue,n=10):
\end{eulerprompt}
\eulerimg{17}{images/EMT4Plot2D-Naela Rizqy Arofah-22305144042-027.png}
\eulerheading{Fungsi dalam satu Parameter}
\begin{eulercomment}
Fungsi plot yang paling penting untuk plot planar adalah plot2d().
Fungsi ini diimplementasikan dalam bahasa Euler di file\\
"plot.e", yang dimuat di awal program.

Berikut beberapa contoh penggunaan suatu fungsi. Seperti biasa di EMT,
fungsi yang berfungsi untuk fungsi atau ekspresi lain, Anda dapat
meneruskan parameter tambahan (selain x) yang bukan merupakan variabel
global ke fungsi dengan parameter titik koma atau dengan kumpulan
panggilan.
\end{eulercomment}
\begin{eulerprompt}
>function f(x,a) := x^2/a+a*x^2-x; // define a function
>a=0.3; plot2d("f",0,1;a): // plot with a=0.3
\end{eulerprompt}
\eulerimg{17}{images/EMT4Plot2D-Naela Rizqy Arofah-22305144042-028.png}
\begin{eulerprompt}
>plot2d("f",0,1;0.4): // plot with a=0.4
\end{eulerprompt}
\eulerimg{17}{images/EMT4Plot2D-Naela Rizqy Arofah-22305144042-029.png}
\begin{eulerprompt}
>plot2d(\{\{"f",0.2\}\},0,1): // plot with a=0.2
\end{eulerprompt}
\eulerimg{17}{images/EMT4Plot2D-Naela Rizqy Arofah-22305144042-030.png}
\begin{eulerprompt}
>plot2d(\{\{"f(x,b)",b=0.1\}\},0,1): // plot with 0.1
\end{eulerprompt}
\eulerimg{17}{images/EMT4Plot2D-Naela Rizqy Arofah-22305144042-031.png}
\begin{eulerprompt}
>function f(x) := x^3-x; ...
>plot2d("f",r=1):
\end{eulerprompt}
\eulerimg{17}{images/EMT4Plot2D-Naela Rizqy Arofah-22305144042-032.png}
\begin{eulercomment}
Here is a summary of the accepted functions

- expressions or symbolic expressions in x\\
- functions or symbolic functions by name as "f"\\
- symbolic functions just by the name f

dengan nama f Fungsi plot2d() juga menerima\\
fungsi simbolik. Untuk fungsi simbolik, namanya saja yang berfungsi.
\end{eulercomment}
\begin{eulerprompt}
>function f(x) &= diff(x^x,x)
\end{eulerprompt}
\begin{euleroutput}
  
                              x
                             x  (log(x) + 1)
  
\end{euleroutput}
\begin{eulerprompt}
>plot2d(f,0,2):
\end{eulerprompt}
\eulerimg{17}{images/EMT4Plot2D-Naela Rizqy Arofah-22305144042-033.png}
\begin{eulercomment}
Of course, for expressions or symbolic expressions the name of the variable is enough to
plot them.
\end{eulercomment}
\begin{eulerprompt}
>expr &= sin(x)*exp(-x)
\end{eulerprompt}
\begin{euleroutput}
  
                                - x
                               E    sin(x)
  
\end{euleroutput}
\begin{eulerprompt}
>plot2d(expr,0,3pi):
\end{eulerprompt}
\eulerimg{17}{images/EMT4Plot2D-Naela Rizqy Arofah-22305144042-034.png}
\begin{eulerprompt}
>function f(x) &= x^x;
>plot2d(f,r=1,cx=1,cy=1,color=blue,thickness=2);
>plot2d(&diff(f(x),x),>add,color=red,style="-.-"):
\end{eulerprompt}
\eulerimg{17}{images/EMT4Plot2D-Naela Rizqy Arofah-22305144042-035.png}
\begin{eulercomment}
Untuk gaya garis ada berbagai pilihan. 

- style="...". Select from "-", "--", "-.", ".", ".-.", "-.-".\\
- color: See below for colors.\\
- thickness: Default is 1.

Warna dapat dipilih sebagai salah satu warna default, atau sebagai
warna RGB. 


\end{eulercomment}
\begin{eulerttcomment}
 0..15: the default color indices.
\end{eulerttcomment}
\begin{eulercomment}
- color constants: white, black, red, green, blue, cyan, olive,
lightgray, gray, darkgray, orange, lightgreen, turquoise, lightblue,
lightorange, yellow\\
- rgb(red,green,blue): parameters are reals in [0,1].
\end{eulercomment}
\begin{eulerprompt}
>plot2d("exp(-x^2)",r=2,color=red,thickness=3,style="--"):
\end{eulerprompt}
\eulerimg{17}{images/EMT4Plot2D-Naela Rizqy Arofah-22305144042-036.png}
\begin{eulercomment}
Here is a view of the predefined colors of EMT.
\end{eulercomment}
\begin{eulerprompt}
>aspect(2); columnsplot(ones(1,16),lab=0:15,grid=0,color=0:15):
\end{eulerprompt}
\eulerimg{13}{images/EMT4Plot2D-Naela Rizqy Arofah-22305144042-037.png}
\begin{eulercomment}
But you can use any color.
\end{eulercomment}
\begin{eulerprompt}
>columnsplot(ones(1,16),grid=0,color=rgb(0,0,linspace(0,1,15))):
\end{eulerprompt}
\eulerimg{13}{images/EMT4Plot2D-Naela Rizqy Arofah-22305144042-038.png}
\eulerheading{Menggambar Beberapa Kurva pada bidang koordinat yang sama}
\begin{eulercomment}
Plot lebih dari satu fungsi (multiple function) ke dalam satu jendela
dapat dilakukan dengan berbagai cara. Salah satu metodenya adalah
menggunakan \textgreater{}add untuk beberapa panggilan ke plot2d secara
keseluruhan, kecuali panggilan pertama. Kami telah menggunakan fitur
ini pada contoh di atas.
\end{eulercomment}
\begin{eulerprompt}
>aspect(); plot2d("cos(x)",r=2,grid=6); plot2d("x",style=".",>add):
\end{eulerprompt}
\eulerimg{27}{images/EMT4Plot2D-Naela Rizqy Arofah-22305144042-039.png}
\begin{eulerprompt}
>aspect(1.5); plot2d("sin(x)",0,2pi); plot2d("cos(x)",color=blue,style="--",>add):
\end{eulerprompt}
\eulerimg{17}{images/EMT4Plot2D-Naela Rizqy Arofah-22305144042-040.png}
\begin{eulercomment}
Salah satu kegunaan \textgreater{}add adalah untuk menambahkan titik pada kurva.
\end{eulercomment}
\begin{eulerprompt}
>plot2d("sin(x)",0,pi); plot2d(2,sin(2),>points,>add):
\end{eulerprompt}
\eulerimg{17}{images/EMT4Plot2D-Naela Rizqy Arofah-22305144042-041.png}
\begin{eulercomment}
Kita tambahkan titik perpotongan dengan label (pada posisi "cl" untuk
kiri tengah), dan masukkan hasilnya ke dalam buku catatan.\\
Kami juga menambahkan judul pada plot.
\end{eulercomment}
\begin{eulerprompt}
>plot2d(["cos(x)","x"],r=1.1,cx=0.5,cy=0.5, ...
>  color=[black,blue],style=["-","."], ...
>  grid=1);
>x0=solve("cos(x)-x",1);  ...
>  plot2d(x0,x0,>points,>add,title="Intersection Demo");  ...
>  label("cos(x) = x",x0,x0,pos="cl",offset=20):
\end{eulerprompt}
\eulerimg{17}{images/EMT4Plot2D-Naela Rizqy Arofah-22305144042-042.png}
\begin{eulercomment}
Dalam demo berikut, kita memplot fungsi sin(x)=sin(x)/x dan ekspansi
Taylor ke-8 dan ke-16. Kami menghitung perluasan ini menggunakan
Maxima melalui ekspresi simbolik.

Plot ini dilakukan dalam perintah multi-baris berikut dengan tiga
panggilan ke plot2d(). Yang kedua dan ketiga memiliki kumpulan\\
tanda \textgreater{}add, yang membuat plot menggunakan rentang sebelumnya.


kami menambahkan kotak label yang menjelaskan fungsinya.
\end{eulercomment}
\begin{eulerprompt}
>$taylor(sin(x)/x,x,0,4)
\end{eulerprompt}
\begin{eulerformula}
\[
\frac{x^4}{120}-\frac{x^2}{6}+1
\]
\end{eulerformula}
\begin{eulerprompt}
>plot2d("sinc(x)",0,4pi,color=green,thickness=2); ...
>  plot2d(&taylor(sin(x)/x,x,0,8),>add,color=blue,style="--"); ...
>  plot2d(&taylor(sin(x)/x,x,0,16),>add,color=red,style="-.-"); ...
>  labelbox(["sinc","T8","T16"],styles=["-","--","-.-"], ...
>    colors=[black,blue,red]):
\end{eulerprompt}
\eulerimg{17}{images/EMT4Plot2D-Naela Rizqy Arofah-22305144042-044.png}
\begin{eulercomment}
In the following example, we generate the Bernstein-Polynomials.

\end{eulercomment}
\begin{eulerformula}
\[
B_i(x) = \binom{n}{i} x^i (1-x)^{n-i}
\]
\end{eulerformula}
\begin{eulerprompt}
>plot2d("(1-x)^10",0,1); // plot first function
>for i=1 to 10; plot2d("bin(10,i)*x^i*(1-x)^(10-i)",>add); end;
>insimg;
\end{eulerprompt}
\eulerimg{17}{images/EMT4Plot2D-Naela Rizqy Arofah-22305144042-045.png}
\begin{eulercomment}
Cara kedua adalah dengan menggunakan pasangan matriks bernilai x dan
matriks bernilai y yang berukuran sama.


Kami menghasilkan matriks nilai dengan satu Polinomial Bernstein di
setiap baris. Untuk ini, kita cukup menggunakan vektor kolom i.\\
Lihat pendahuluan tentang bahasa matriks untuk mempelajari lebih
detail.
\end{eulercomment}
\begin{eulerprompt}
>x=linspace(0,1,500);
>n=10; k=(0:n)'; // n is row vector, k is column vector
>y=bin(n,k)*x^k*(1-x)^(n-k); // y is a matrix then
>plot2d(x,y):
\end{eulerprompt}
\eulerimg{17}{images/EMT4Plot2D-Naela Rizqy Arofah-22305144042-046.png}
\begin{eulercomment}
Note that the color parameter can be a vector. Then each color is used for each row of
the matrix.
\end{eulercomment}
\begin{eulerprompt}
>x=linspace(0,1,200); y=x^(1:10)'; plot2d(x,y,color=1:10):
\end{eulerprompt}
\eulerimg{17}{images/EMT4Plot2D-Naela Rizqy Arofah-22305144042-047.png}
\begin{eulercomment}
Another method is using a vector of expressions (strings). You can then use a
color array, an array of styles, and an array of thicknesses of the same length.
\end{eulercomment}
\begin{eulerprompt}
>plot2d(["sin(x)","cos(x)"],0,2pi,color=4:5): 
\end{eulerprompt}
\eulerimg{17}{images/EMT4Plot2D-Naela Rizqy Arofah-22305144042-048.png}
\begin{eulerprompt}
>plot2d(["sin(x)","cos(x)"],0,2pi): // plot vector of expressions
\end{eulerprompt}
\eulerimg{17}{images/EMT4Plot2D-Naela Rizqy Arofah-22305144042-049.png}
\begin{eulercomment}
We can get such a vector from Maxima using makelist() and mxm2str().  
\end{eulercomment}
\begin{eulerprompt}
>v &= makelist(binomial(10,i)*x^i*(1-x)^(10-i),i,0,10) // make list
\end{eulerprompt}
\begin{euleroutput}
  
                 10            9              8  2             7  3
         [(1 - x)  , 10 (1 - x)  x, 45 (1 - x)  x , 120 (1 - x)  x , 
             6  4             5  5             4  6             3  7
  210 (1 - x)  x , 252 (1 - x)  x , 210 (1 - x)  x , 120 (1 - x)  x , 
            2  8              9   10
  45 (1 - x)  x , 10 (1 - x) x , x  ]
  
\end{euleroutput}
\begin{eulerprompt}
>mxm2str(v) // get a vector of strings from the symbolic vector
\end{eulerprompt}
\begin{euleroutput}
  (1-x)^10
  10*(1-x)^9*x
  45*(1-x)^8*x^2
  120*(1-x)^7*x^3
  210*(1-x)^6*x^4
  252*(1-x)^5*x^5
  210*(1-x)^4*x^6
  120*(1-x)^3*x^7
  45*(1-x)^2*x^8
  10*(1-x)*x^9
  x^10
\end{euleroutput}
\begin{eulerprompt}
>plot2d(mxm2str(v),0,1): // plot functions
\end{eulerprompt}
\eulerimg{17}{images/EMT4Plot2D-Naela Rizqy Arofah-22305144042-050.png}
\begin{eulercomment}
Alternatif lain adalah dengan menggunakan bahasa matriks Euler.

Jika suatu ekspresi menghasilkan matriks fungsi, dengan satu fungsi di
setiap baris, semua fungsi tersebut akan diplot ke dalam satu plot.
\end{eulercomment}
\begin{eulerprompt}
>n=(1:10)'; plot2d("x^n",0,1,color=1:10):
\end{eulerprompt}
\eulerimg{17}{images/EMT4Plot2D-Naela Rizqy Arofah-22305144042-051.png}
\begin{eulercomment}
Expressions and one-line functions can see global variables.

If you cannot use a global variable, you need to use a function with an extra parameter,
and pass this parameter as a semicolon parameter.

Take care, to put all assigned parameters to the end of the plot2d command. In the
example we pass a=5 to the function f, which we plot from -10 to 10.
\end{eulercomment}
\begin{eulerprompt}
>function f(x,a) := 1/a*exp(-x^2/a); ...
>plot2d("f",-10,10;5,thickness=2,title="a=5"):
\end{eulerprompt}
\eulerimg{17}{images/EMT4Plot2D-Naela Rizqy Arofah-22305144042-052.png}
\begin{eulercomment}
Alternatively, use a collection with the function name and all extra parameters. These
special lists are called call collections, and that is the preferred way to pass
arguments to a function which is itself passed as an argument to another function.

In the following example, we use a loop to plot several functions (see the tutorial
about programming for loops).
\end{eulercomment}
\begin{eulerprompt}
>plot2d(\{\{"f",1\}\},-10,10); ...
>for a=2:10; plot2d(\{\{"f",a\}\},>add); end:
\end{eulerprompt}
\eulerimg{17}{images/EMT4Plot2D-Naela Rizqy Arofah-22305144042-053.png}
\begin{eulercomment}
We could achieve the same result in the following way using the matrix language of EMT.
Each row of the matrix f(x,a) is one function. Moreover, we can set colors for each row
of the matrix. Double click on the function getspectral() for an explanation.
\end{eulercomment}
\begin{eulerprompt}
>x=-10:0.01:10; a=(1:10)'; plot2d(x,f(x,a),color=getspectral(a/10)):
\end{eulerprompt}
\eulerimg{17}{images/EMT4Plot2D-Naela Rizqy Arofah-22305144042-054.png}
\eulersubheading{Label Teks}
\begin{eulercomment}
Dekorasi sederhana dapat : 


\end{eulercomment}
\begin{eulerttcomment}
 a title with title="..."
\end{eulerttcomment}
\begin{eulercomment}
- x- and y-labels with xl="...", yl="..."\\
- another text label with label("...",x,y)

Perintah label akan memplot ke plot saat ini pada koordinat plot
(x,y). Hal ini memerlukan argumen posisional.
\end{eulercomment}
\begin{eulerprompt}
>plot2d("x^3-x",-1,2,title="y=x^3-x",yl="y",xl="x"):
\end{eulerprompt}
\eulerimg{17}{images/EMT4Plot2D-Naela Rizqy Arofah-22305144042-055.png}
\begin{eulerprompt}
>expr := "log(x)/x"; ...
>  plot2d(expr,0.5,5,title="y="+expr,xl="x",yl="y"); ...
>  label("(1,0)",1,0); label("Max",E,expr(E),pos="lc"):
\end{eulerprompt}
\eulerimg{17}{images/EMT4Plot2D-Naela Rizqy Arofah-22305144042-056.png}
\begin{eulercomment}
Ada juga fungsi labelbox(), yang dapat menampilkan fungsi dan teks.
Dibutuhkan vektor string dan warna, satu item untuk setiap\\
fungsi.
\end{eulercomment}
\begin{eulerprompt}
>function f(x) &= x^2*exp(-x^2);  ...
>plot2d(&f(x),a=-3,b=3,c=-1,d=1);  ...
>plot2d(&diff(f(x),x),>add,color=blue,style="--"); ...
>labelbox(["function","derivative"],styles=["-","--"], ...
>   colors=[black,blue],w=0.4):
\end{eulerprompt}
\eulerimg{17}{images/EMT4Plot2D-Naela Rizqy Arofah-22305144042-057.png}
\begin{eulercomment}
Kotak ini berlabuh di kanan atas secara default, tetapi \textgreater{}kiri berlabuh
di kiri atas. Anda dapat memindahkannya ke tempat mana pun yang Anda
suka. Posisi jangkar berada di pojok kanan atas kotak, dan angkanya
merupakan pecahan dari ukuran jendela grafis. Lebarnya otomatis.

Untuk plot titik, kotak label juga berfungsi. Tambahkan parameter \textgreater{}
points, atau vektor bendera, satu untuk setiap label.


ada contoh berikut, hanya ada satu fungsi. Jadi kita bisa menggunakan
string sebagai pengganti vektor string. Kami mengatur warna teks
menjadi hitam untuk contoh ini.
\end{eulercomment}
\begin{eulerprompt}
>n=10; plot2d(0:n,bin(n,0:n),>addpoints); ...
>labelbox("Binomials",styles="[]",>points,x=0.1,y=0.1, ...
>tcolor=black,>left):
\end{eulerprompt}
\eulerimg{17}{images/EMT4Plot2D-Naela Rizqy Arofah-22305144042-058.png}
\begin{eulercomment}
Gaya plot ini juga tersedia di statplot(). Seperti di plot2d() warna
dapat diatur untuk setiap baris plot. Masih banyak lagi plot khusus
untuk keperluan statistik (lihat tutorial tentang statistik).
\end{eulercomment}
\begin{eulerprompt}
>statplot(1:10,random(2,10),color=[red,blue]):
\end{eulerprompt}
\eulerimg{17}{images/EMT4Plot2D-Naela Rizqy Arofah-22305144042-059.png}
\begin{eulercomment}
A similar feature is the function textbox().

Lebarnya secara default adalah lebar maksimal baris teks. Tapi itu
bisa diatur oleh pengguna juga.
\end{eulercomment}
\begin{eulerprompt}
>function f(x) &= exp(-x)*sin(2*pi*x); ...
>plot2d("f(x)",0,2pi); ...
>textbox(latex("\(\backslash\)text\{Example of a damped oscillation\}\(\backslash\) f(x)=e^\{-x\}sin(2\(\backslash\)pi x)"),w=0.85):
\end{eulerprompt}
\eulerimg{17}{images/EMT4Plot2D-Naela Rizqy Arofah-22305144042-060.png}
\begin{eulercomment}
Label teks, judul, kotak label, dan teks lainnya dapat berisi string
Unicode (lihat sintaks EMT untuk mengetahui lebih lanjut tentang
string Unicode).
\end{eulercomment}
\begin{eulerprompt}
>plot2d("x^3-x",title=u"x &rarr; x&sup3; - x"):
\end{eulerprompt}
\eulerimg{17}{images/EMT4Plot2D-Naela Rizqy Arofah-22305144042-061.png}
\begin{eulercomment}
The labels on the x- and y-axis can be vertical, as well as the axis.
\end{eulercomment}
\begin{eulerprompt}
>plot2d("sinc(x)",0,2pi,xl="x",yl=u"x &rarr; sinc(x)",>vertical):
\end{eulerprompt}
\eulerimg{17}{images/EMT4Plot2D-Naela Rizqy Arofah-22305144042-062.png}
\eulersubheading{LaTeX}
\begin{eulercomment}
Anda juga dapat memplot rumus LaTeX jika Anda telah menginstal sistem
LaTeX. Saya merekomendasikan MiKTeX. Jalur ke biner "lateks" dan
"dvipng" harus berada di jalur sistem, atau Anda harus mengatur LaTeX
di opsi menu.

Perhatikan, penguraian LaTeX lambat. Jika Anda ingin menggunakan LaTeX
dalam plot animasi, Anda harus memanggil latex() sebelum loop satu
kali dan menggunakan hasilnya (gambar dalam matriks RGB).

Pada plot berikut, kami menggunakan LaTeX untuk label x dan y, label,
kotak label, dan judul plot.
\end{eulercomment}
\begin{eulerprompt}
>plot2d("exp(-x)*sin(x)/x",a=0,b=2pi,c=0,d=1,grid=6,color=blue, ...
>  title=latex("\(\backslash\)text\{Function $\(\backslash\)Phi$\}"), ...
>  xl=latex("\(\backslash\)phi"),yl=latex("\(\backslash\)Phi(\(\backslash\)phi)")); ...
>textbox( ...
>  latex("\(\backslash\)Phi(\(\backslash\)phi) = e^\{-\(\backslash\)phi\} \(\backslash\)frac\{\(\backslash\)sin(\(\backslash\)phi)\}\{\(\backslash\)phi\}"),x=0.8,y=0.5); ...
>label(latex("\(\backslash\)Phi",color=blue),1,0.4):
\end{eulerprompt}
\eulerimg{17}{images/EMT4Plot2D-Naela Rizqy Arofah-22305144042-063.png}
\begin{eulercomment}
Seringkali, kita menginginkan spasi dan label teks yang tidak
konformal pada sumbu x. Kita bisa menggunakan xaxis() dan yaxis()
seperti yang akan kita tunjukkan nanti.

Cara termudah adalah membuat plot kosong dengan bingkai menggunakan
grid=4, lalu menambahkan grid dengan ygrid() dan xgrid(). Pada contoh
berikut, kami menggunakan tiga string LaTeX untuk label pada sumbu x
dengan xtick().
\end{eulercomment}
\begin{eulerprompt}
>plot2d("sinc(x)",0,2pi,grid=4,<ticks); ...
>ygrid(-2:0.5:2,grid=6); ...
>xgrid([0:2]*pi,<ticks,grid=6);  ...
>xtick([0,pi,2pi],["0","\(\backslash\)pi","2\(\backslash\)pi"],>latex):
\end{eulerprompt}
\eulerimg{17}{images/EMT4Plot2D-Naela Rizqy Arofah-22305144042-064.png}
\begin{eulercomment}
Of course, functions can also be used.
\end{eulercomment}
\begin{eulerprompt}
>function map f(x) ...
\end{eulerprompt}
\begin{eulerudf}
  if x>0 then return x^4
  else return x^2
  endif
  endfunction
\end{eulerudf}
\begin{eulercomment}
Parameter "peta" membantu menggunakan fungsi untuk vektor. Untuk
plotnya, itu tidak diperlukan. Namun untuk menunjukkan bahwa
vektorisasi bermanfaat, kami menambahkan beberapa poin penting ke plot
di x=-1, x=0, dan x=1.

Pada plot berikut, kami juga memasukkan beberapa kode LaTeX. Kami
menggunakannya untuk dua label dan kotak teks. Tentu saja, Anda hanya
dapat menggunakan LaTeX jika Anda telah menginstal LaTeX dengan benar.

\end{eulercomment}
\begin{eulerprompt}
>plot2d("f",-1,1,xl="x",yl="f(x)",grid=6);  ...
>plot2d([-1,0,1],f([-1,0,1]),>points,>add); ...
>label(latex("x^3"),0.72,f(0.72)); ...
>label(latex("x^2"),-0.52,f(-0.52),pos="ll"); ...
>textbox( ...
>  latex("f(x)=\(\backslash\)begin\{cases\} x^3 & x>0 \(\backslash\)\(\backslash\) x^2 & x \(\backslash\)le 0\(\backslash\)end\{cases\}"), ...
>  x=0.7,y=0.2):
\end{eulerprompt}
\eulerimg{17}{images/EMT4Plot2D-Naela Rizqy Arofah-22305144042-065.png}
\begin{eulercomment}
\end{eulercomment}
\eulersubheading{Interaksi Pengguna}
\begin{eulercomment}
Saat memplot suatu fungsi atau ekspresi, parameter \textgreater{}pengguna
memungkinkan pengguna untuk memperbesar dan menggeser plot dengan
tombol kursor atau mouse. Pengguna dapat :

- memperbesar dengan + or -\\
- memindahkan plot dengan tombol kursor\\
- select a plot window with the mouse\\
- reset the view with space\\
- exit with return

spasi - keluar dengan kembali Tombol spasi akan mengatur ulang plot ke
jendela plot asli.

Saat memplot data, flag \textgreater{}user hanya akan menunggu penekanan tombol.
\end{eulercomment}
\begin{eulerprompt}
>plot2d(\{\{"x^3-a*x",a=1\}\},>user,title="Press any key!"):
\end{eulerprompt}
\eulerimg{17}{images/EMT4Plot2D-Naela Rizqy Arofah-22305144042-066.png}
\begin{eulerprompt}
>plot2d("exp(x)*sin(x)",user=true, ...
>  title="+/- or cursor keys (return to exit)"):
\end{eulerprompt}
\eulerimg{17}{images/EMT4Plot2D-Naela Rizqy Arofah-22305144042-067.png}
\begin{eulercomment}
The following demonstrates an advanced way of user interaction (see
the tutorial about programming for details).

Fungsi bawaan mousedrag() menunggu aktivitas mouse atau keyboard. Ini
melaporkan mouse ke bawah, gerakan mouse atau mouse ke atas, dan
penekanan tombol. Fungsi dragpoints() memanfaatkan ini, dan
memungkinkan pengguna menyeret titik manapun dalam plot.


Kita membutuhkan fungsi plot terlebih dahulu. Misalnya, kita melakukan
interpolasi pada 5 titik dengan polinomial. Fungsi tersebut harus
diplot ke dalam area plot yang tetap.
\end{eulercomment}
\begin{eulerprompt}
>function plotf(xp,yp,select) ...
\end{eulerprompt}
\begin{eulerudf}
    d=interp(xp,yp);
    plot2d("interpval(xp,d,x)";d,xp,r=2);
    plot2d(xp,yp,>points,>add);
    if select>0 then
      plot2d(xp[select],yp[select],color=red,>points,>add);
    endif;
    title("Drag one point, or press space or return!");
  endfunction
\end{eulerudf}
\begin{eulercomment}
Note the semicolon parameters in plot2d (d and xp), which are passed to the evaluation
of the interp() function. Without this, we must write a function plotinterp() first,
accessing the values globally.

Now we generate some random values, and let the user drag the points.
\end{eulercomment}
\begin{eulerprompt}
>t=-1:0.5:1; dragpoints("plotf",t,random(size(t))-0.5):
\end{eulerprompt}
\eulerimg{17}{images/EMT4Plot2D-Naela Rizqy Arofah-22305144042-068.png}
\begin{eulercomment}
There is also a function, which plots another function depending on a vector of
parameters, and lets the user adjust these parameters.

First we need the plot function.
\end{eulercomment}
\begin{eulerprompt}
>function plotf([a,b]) := plot2d("exp(a*x)*cos(2pi*b*x)",0,2pi;a,b);
\end{eulerprompt}
\begin{eulercomment}
Then we need names for the parameters, initial values and a nx2 matrix of ranges,
optionally a heading line.\\
There are interactive sliders, which can set values by the user. The function
dragvalues() provides this.
\end{eulercomment}
\begin{eulerprompt}
>dragvalues("plotf",["a","b"],[-1,2],[[-2,2];[1,10]], ...
>  heading="Drag these values:",hcolor=black):
\end{eulerprompt}
\eulerimg{17}{images/EMT4Plot2D-Naela Rizqy Arofah-22305144042-069.png}
\begin{eulercomment}
It is possible to restrict the dragged values to integers. For an example, we write a
plot function, which plots a Taylor polynomial of degree n to the cosine function.
\end{eulercomment}
\begin{eulerprompt}
>function plotf(n) ...
\end{eulerprompt}
\begin{eulerudf}
  plot2d("cos(x)",0,2pi,>square,grid=6);
  plot2d(&"taylor(cos(x),x,0,@n)",color=blue,>add);
  textbox("Taylor polynomial of degree "+n,0.1,0.02,style="t",>left);
  endfunction
\end{eulerudf}
\begin{eulercomment}
Now we allow the degree n to vary from 0 to 20 in 20 stops. The result of dragvalues()
is used to plot the sketch with this n, and to insert the plot into the notebook.
\end{eulercomment}
\begin{eulerprompt}
>nd=dragvalues("plotf","degree",2,[0,20],20,y=0.8, ...
>   heading="Drag the value:"); ...
>plotf(nd):
\end{eulerprompt}
\eulerimg{17}{images/EMT4Plot2D-Naela Rizqy Arofah-22305144042-070.png}
\begin{eulercomment}
The following is a simple demonstration of the function. The user can draw over the plot
window, leaving a trace of points.
\end{eulercomment}
\begin{eulerprompt}
>function dragtest ...
\end{eulerprompt}
\begin{eulerudf}
    plot2d(none,r=1,title="Drag with the mouse, or press any key!");
    start=0;
    repeat
      \{flag,m,time\}=mousedrag();
      if flag==0 then return; endif;
      if flag==2 then
        hold on; mark(m[1],m[2]); hold off;
      endif;
    end
  endfunction
\end{eulerudf}
\begin{eulerprompt}
>dragtest // lihat hasilnya dan cobalah lakukan!
\end{eulerprompt}
\eulersubheading{Gaya Plot 2D}
\begin{eulercomment}
Secara default, EMT menghitung tick sumbu otomatis dan menambahkan
label ke setiap tick. Ini dapat diubah dengan parameter grid. Gaya
default sumbu dan label dapat diubah. Selain itu, label dan judul
dapat ditambahkan secara manual. Untuk menyetel ulang ke gaya default,
gunakan reset().
\end{eulercomment}
\begin{eulerprompt}
>aspect();
>figure(3,4); ...
> figure(1); plot2d("x^3-x",grid=0); ... // no grid, frame or axis
> figure(2); plot2d("x^3-x",grid=1); ... // x-y-axis
> figure(3); plot2d("x^3-x",grid=2); ... // default ticks
> figure(4); plot2d("x^3-x",grid=3); ... // x-y- axis with labels inside
> figure(5); plot2d("x^3-x",grid=4); ... // no ticks, only labels
> figure(6); plot2d("x^3-x",grid=5); ... // default, but no margin
> figure(7); plot2d("x^3-x",grid=6); ... // axes only
> figure(8); plot2d("x^3-x",grid=7); ... // axes only, ticks at axis
> figure(9); plot2d("x^3-x",grid=8); ... // axes only, finer ticks at axis
> figure(10); plot2d("x^3-x",grid=9); ... // default, small ticks inside
> figure(11); plot2d("x^3-x",grid=10); ...// no ticks, axes only
> figure(0):
\end{eulerprompt}
\eulerimg{27}{images/EMT4Plot2D-Naela Rizqy Arofah-22305144042-071.png}
\begin{eulercomment}
The parameter \textless{}frame switches off the frame, and framecolor=blue sets the frame to a
blue color.

If you want your own ticks, you can use style=0, and add everything later.
\end{eulercomment}
\begin{eulerprompt}
>aspect(1.5); 
>plot2d("x^3-x",grid=0); // plot
>frame; xgrid([-1,0,1]); ygrid(0): // add frame and grid
\end{eulerprompt}
\eulerimg{17}{images/EMT4Plot2D-Naela Rizqy Arofah-22305144042-072.png}
\begin{eulercomment}
For the plot title and labels of the axes, see the following example.
\end{eulercomment}
\begin{eulerprompt}
>plot2d("exp(x)",-1,1);
>textcolor(black); // set the text color to black
>title(latex("y=e^x")); // title above the plot
>xlabel(latex("x")); // "x" for x-axis
>ylabel(latex("y"),>vertical); // vertical "y" for y-axis
>label(latex("(0,1)"),0,1,color=blue): // label a point
\end{eulerprompt}
\eulerimg{17}{images/EMT4Plot2D-Naela Rizqy Arofah-22305144042-073.png}
\begin{eulercomment}
The axis can be drawn separately with xaxis() and yaxis().
\end{eulercomment}
\begin{eulerprompt}
>plot2d("x^3-x",<grid,<frame);
>xaxis(0,xx=-2:1,style="->"); yaxis(0,yy=-5:5,style="->"):
\end{eulerprompt}
\eulerimg{17}{images/EMT4Plot2D-Naela Rizqy Arofah-22305144042-074.png}
\begin{eulercomment}
Text on the plot can be set with label(). In the following example, "lc" means
lower center. It sets the position of the label relative to the plot coordinates.
\end{eulercomment}
\begin{eulerprompt}
>function f(x) &= x^3-x
\end{eulerprompt}
\begin{euleroutput}
  
                                   3
                                  x  - x
  
\end{euleroutput}
\begin{eulerprompt}
>plot2d(f,-1,1,>square);
>x0=fmin(f,0,1); // compute point of minimum
>label("Rel. Min.",x0,f(x0),pos="lc"): // add a label there
\end{eulerprompt}
\eulerimg{17}{images/EMT4Plot2D-Naela Rizqy Arofah-22305144042-075.png}
\begin{eulercomment}
There are also text boxes.
\end{eulercomment}
\begin{eulerprompt}
>plot2d(&f(x),-1,1,-2,2); // function
>plot2d(&diff(f(x),x),>add,style="--",color=red); // derivative
>labelbox(["f","f'"],["-","--"],[black,red]): // label box
\end{eulerprompt}
\eulerimg{17}{images/EMT4Plot2D-Naela Rizqy Arofah-22305144042-076.png}
\begin{eulerprompt}
>plot2d(["exp(x)","1+x"],color=[black,blue],style=["-","-.-"]):
\end{eulerprompt}
\eulerimg{17}{images/EMT4Plot2D-Naela Rizqy Arofah-22305144042-077.png}
\begin{eulerprompt}
>gridstyle("->",color=gray,textcolor=gray,framecolor=gray);  ...
> plot2d("x^3-x",grid=1);   ...
> settitle("y=x^3-x",color=black); ...
> label("x",2,0,pos="bc",color=gray);  ...
> label("y",0,6,pos="cl",color=gray); ...
> reset():
\end{eulerprompt}
\eulerimg{27}{images/EMT4Plot2D-Naela Rizqy Arofah-22305144042-078.png}
\begin{eulercomment}
For even more control, the x-axis and the y-axis can be done manually.

Perintah fullwindow() memperluas jendela plot karena kita tidak lagi
memerlukan tempat untuk label di luar jendela plot. Gunakan
shrinkwindow() atau reset() untuk menyetel ulang ke default.
\end{eulercomment}
\begin{eulerprompt}
>fullwindow; ...
> gridstyle(color=darkgray,textcolor=darkgray); ...
> plot2d(["2^x","1","2^(-x)"],a=-2,b=2,c=0,d=4,<grid,color=4:6,<frame); ...
> xaxis(0,-2:1,style="->"); xaxis(0,2,"x",<axis); ...
> yaxis(0,4,"y",style="->"); ...
> yaxis(-2,1:4,>left); ...
> yaxis(2,2^(-2:2),style=".",<left); ...
> labelbox(["2^x","1","2^-x"],colors=4:6,x=0.8,y=0.2); ...
> reset:
\end{eulerprompt}
\eulerimg{27}{images/EMT4Plot2D-Naela Rizqy Arofah-22305144042-079.png}
\begin{eulercomment}
Here is another example, where Unicode strings are used and axes
outside the plot area.
\end{eulercomment}
\begin{eulerprompt}
>aspect(1.5); 
>plot2d(["sin(x)","cos(x)"],0,2pi,color=[red,green],<grid,<frame); ...
> xaxis(-1.1,(0:2)*pi,xt=["0",u"&pi;",u"2&pi;"],style="-",>ticks,>zero);  ...
> xgrid((0:0.5:2)*pi,<ticks); ...
> yaxis(-0.1*pi,-1:0.2:1,style="-",>zero,>grid); ...
> labelbox(["sin","cos"],colors=[red,green],x=0.5,y=0.2,>left); ...
> xlabel(u"&phi;"); ylabel(u"f(&phi;)"):
\end{eulerprompt}
\eulerimg{17}{images/EMT4Plot2D-Naela Rizqy Arofah-22305144042-080.png}
\eulerheading{Merencanakan Data 2D}
\begin{eulercomment}
Jika x dan y adalah vektor data, maka data tersebut akan digunakan
sebagai koordinat x dan y pada suatu kurva. Dalam hal ini, a, b, c,
dan d, atau radius r dapat ditentukan, atau jendela plot akan
menyesuaikan secara otomatis dengan data. Alternatifnya, \textgreater{}persegi
dapat\\
diatur untuk mempertahankan rasio aspek persegi.

Merencanakan ekspresi hanyalah singkatan dari plot data. Untuk plot
data, Anda memerlukan satu atau beberapa baris nilai x, dan satu atau
beberapa baris nilai y. Dari rentang dan nilai x, fungsi plot2d akan
menghitung data yang akan diplot, secara default dengan evaluasi\\
fungsi yang adaptif. Untuk plot titik gunakan "\textgreater{}titik", untuk garis
dan titik campuran gunakan "\textgreater{}addpoints".

Tapi Anda bisa memasukkan data secara langsung.

- Gunakan vektor baris untuk x dan y untuk satu fungsi.\\
- Matriks untuk x dan y diplot baris demi baris.

Berikut adalah contoh dengan satu baris untuk x dan y.
\end{eulercomment}
\begin{eulerprompt}
>x=-10:0.1:10; y=exp(-x^2)*x; plot2d(x,y):
\end{eulerprompt}
\eulerimg{17}{images/EMT4Plot2D-Naela Rizqy Arofah-22305144042-081.png}
\begin{eulercomment}
Data juga dapat diplot sebagai poin. Gunakan points=true untuk ini.
Plotnya berfungsi seperti poligon, tetapi hanya menggambar.

\textgreater{}p=polyfit(xdata,ydata,1); // dapatkan garis regresi\\
\textgreater{}plot2d("polyval(p,x)",\textgreater{}tambahkan,warna=merah): // tambahkan plot
garis\\
sudut.

\end{eulercomment}
\begin{eulerttcomment}
 style="...": Select from "[]", "<>", "o", ".", "..", "+", "*", "[]#",
\end{eulerttcomment}
\begin{eulercomment}
"\textless{}\textgreater{}#", "o#", "..#", "#", "\textbar{}".

Untuk memplot kumpulan titik, gunakan \textgreater{}titik. Jika warna merupakan
vektor warna, maka setiap titik mendapat warna yang berbeda. Untuk
matriks koordinat dan vektor kolom, warna diterapkan pada baris
matriks.

he parameter \textgreater{}addpoints adds points to line segments for plots of
data.
\end{eulercomment}
\begin{eulerprompt}
>xdata=[1,1.5,2.5,3,4]; ydata=[3,3.1,2.8,2.9,2.7]; // data
>plot2d(xdata,ydata,a=0.5,b=4.5,c=2.5,d=3.5,style="."); // lines
>plot2d(xdata,ydata,>points,>add,style="o"): // add points
\end{eulerprompt}
\eulerimg{17}{images/EMT4Plot2D-Naela Rizqy Arofah-22305144042-082.png}
\begin{eulerprompt}
>p=polyfit(xdata,ydata,1); // get regression line
>plot2d("polyval(p,x)",>add,color=red): // add plot of line
\end{eulerprompt}
\eulerimg{17}{images/EMT4Plot2D-Naela Rizqy Arofah-22305144042-083.png}
\eulerheading{Menggambar Daerah Yang Dibatasi Kurva}
\begin{eulercomment}
Plot data sebenarnya berbentuk poligon. Kita juga dapat memplot kurva
atau kurva

- filled=true fills the plot.\\
- style="...": Select from "#", "/", "\textbackslash{}", "\textbackslash{}/".\\
- fillcolor: See above for available colors.

Warna isian ditentukan oleh argumen "fillcolor", dan pada \textless{}outline
opsional, mencegah menggambar ary terikat untuk semua gaya kecuali
gaya default.
\end{eulercomment}
\begin{eulerprompt}
>t=linspace(0,2pi,1000); // parameter for curve
>x=sin(t)*exp(t/pi); y=cos(t)*exp(t/pi); // x(t) and y(t)
>figure(1,2); aspect(16/9)
>figure(1); plot2d(x,y,r=10); // plot curve
>figure(2); plot2d(x,y,r=10,>filled,style="/",fillcolor=red); // fill curve
>figure(0):
\end{eulerprompt}
\eulerimg{14}{images/EMT4Plot2D-Naela Rizqy Arofah-22305144042-084.png}
\begin{eulercomment}
Dalam contoh berikut kita memplot elips terisi dan dua segi enam
terisi menggunakan kurva tertutup dengan 6 titik dengan gaya isian
berbeda.\\
ills the plot.\\
- style="...": Select from "#", "/", "\textbackslash{}", "\textbackslash{}/".\\
- fillcolor: See above for available colors.
\end{eulercomment}
\begin{eulerprompt}
>x=linspace(0,2pi,1000); plot2d(sin(x),cos(x)*0.5,r=1,>filled,style="/"):
\end{eulerprompt}
\eulerimg{14}{images/EMT4Plot2D-Naela Rizqy Arofah-22305144042-085.png}
\begin{eulerprompt}
>t=linspace(0,2pi,6); ...
>plot2d(cos(t),sin(t),>filled,style="/",fillcolor=red,r=1.2):
\end{eulerprompt}
\eulerimg{14}{images/EMT4Plot2D-Naela Rizqy Arofah-22305144042-086.png}
\begin{eulerprompt}
>t=linspace(0,2pi,6); plot2d(cos(t),sin(t),>filled,style="#"):
\end{eulerprompt}
\eulerimg{14}{images/EMT4Plot2D-Naela Rizqy Arofah-22305144042-087.png}
\begin{eulercomment}
Another example is a septagon, which we create with 7 points on the unit circle.
\end{eulercomment}
\begin{eulerprompt}
>t=linspace(0,2pi,7);  ...
> plot2d(cos(t),sin(t),r=1,>filled,style="/",fillcolor=red):
\end{eulerprompt}
\eulerimg{14}{images/EMT4Plot2D-Naela Rizqy Arofah-22305144042-088.png}
\begin{eulercomment}
The following is the set of the maximal value of four linear conditions less than or
equal 3. This is A[k].v\textless{}=3 for all rows of A. To get nice corners, we use n relatively
large.
\end{eulercomment}
\begin{eulerprompt}
>A=[2,1;1,2;-1,0;0,-1];
>function f(x,y) := max([x,y].A');
>plot2d("f",r=4,level=[0;3],color=green,n=111):
\end{eulerprompt}
\eulerimg{14}{images/EMT4Plot2D-Naela Rizqy Arofah-22305144042-089.png}
\begin{eulercomment}
The main point of the matrix language is that it allows to generate tables of
functions easily.
\end{eulercomment}
\begin{eulerprompt}
>t=linspace(0,2pi,1000); x=cos(3*t); y=sin(4*t);
\end{eulerprompt}
\begin{eulercomment}
Kami sekarang memiliki nilai vektor x dan y. plot2d() dapat memplot
nilai-nilai ini sebagai kurva yang menghubungkan titik-titik. Plotnya
bisa diisi. Dalam hal ini hasil yang bagus diperoleh karena adanya
aturan lilitan yang digunakan untuk pengisian.
\end{eulercomment}
\begin{eulerprompt}
>plot2d(x,y,<grid,<frame,>filled):
\end{eulerprompt}
\eulerimg{14}{images/EMT4Plot2D-Naela Rizqy Arofah-22305144042-090.png}
\begin{eulercomment}
Vektor interval diplot terhadap nilai x sebagai wilayah terisi antara
nilai interval bawah dan atas.

Hal ini dapat berguna untuk memplot kesalahan perhitungan. Namun hal
ini juga dapat digunakan untuk memetakan kesalahan statistik.
\end{eulercomment}
\begin{eulerprompt}
>t=0:0.1:1; ...
> plot2d(t,interval(t-random(size(t)),t+random(size(t))),style="|");  ...
> plot2d(t,t,add=true):
\end{eulerprompt}
\eulerimg{14}{images/EMT4Plot2D-Naela Rizqy Arofah-22305144042-091.png}
\begin{eulercomment}
Jika x adalah vektor yang diurutkan, dan y adalah vektor interval,
maka plot2d akan memplot rentang interval yang terisi pada bidang.
Gaya isiannya sama dengan gaya poligon.
\end{eulercomment}
\begin{eulerprompt}
>t=-1:0.01:1; x=~t-0.01,t+0.01~; y=x^3-x;
>plot2d(t,y):
\end{eulerprompt}
\eulerimg{14}{images/EMT4Plot2D-Naela Rizqy Arofah-22305144042-092.png}
\begin{eulercomment}
Dimungkinkan untuk mengisi wilayah nilai untuk fungsi tertentu. Untuk
ini, level harus berupa matriks 2xn. Baris pertama adalah batas bawah\\
\end{eulercomment}
\begin{eulerttcomment}
 dan baris kedua berisi batas atas.
\end{eulerttcomment}
\begin{eulerprompt}
>expr := "2*x^2+x*y+3*y^4+y"; // define an expression f(x,y)
>plot2d(expr,level=[0;1],style="-",color=blue): // 0 <= f(x,y) <= 1
\end{eulerprompt}
\eulerimg{14}{images/EMT4Plot2D-Naela Rizqy Arofah-22305144042-093.png}
\begin{eulercomment}
We can also fill ranges of values like

\end{eulercomment}
\begin{eulerformula}
\[
-1 \le (x^2+y^2)^2-x^2+y^2 \le 0.
\]
\end{eulerformula}
\begin{eulercomment}
\end{eulercomment}
\begin{eulerprompt}
>plot2d("(x^2+y^2)^2-x^2+y^2",r=1.2,level=[-1;0],style="/"):
\end{eulerprompt}
\eulerimg{14}{images/EMT4Plot2D-Naela Rizqy Arofah-22305144042-094.png}
\begin{eulerprompt}
>plot2d("cos(x)","sin(x)^3",xmin=0,xmax=2pi,>filled,style="/"):
\end{eulerprompt}
\eulerimg{14}{images/EMT4Plot2D-Naela Rizqy Arofah-22305144042-095.png}
\eulerheading{Grafik Fungsi Parametrik}
\begin{eulercomment}
Nilai x tidak perlu diurutkan. (x,y) hanya menggambarkan sebuah kurva.
Jika x diurutkan, kurva tersebut merupakan grafik suatu fungsi.


n the following example, we plot the spiral

\end{eulercomment}
\begin{eulerformula}
\[
\gamma(t) = t \cdot (\cos(2\pi t),\sin(2\pi t))
\]
\end{eulerformula}
\begin{eulercomment}
Kita perlu menggunakan banyak titik untuk tampilan yang halus atau
fungsi adaptif() untuk mengevaluasi ekspresi (lihat fungsi adaptif()
untuk lebih jelasnya).
\end{eulercomment}
\begin{eulerprompt}
>t=linspace(0,1,1000); ...
>plot2d(t*cos(2*pi*t),t*sin(2*pi*t),r=1):
\end{eulerprompt}
\eulerimg{14}{images/EMT4Plot2D-Naela Rizqy Arofah-22305144042-096.png}
\begin{eulercomment}
Sebagai alternatif, dimungkinkan untuk menggunakan dua ekspresi untuk
kurva. Berikut ini plot kurva yang sama seperti di atas.
\end{eulercomment}
\begin{eulerprompt}
>plot2d("x*cos(2*pi*x)","x*sin(2*pi*x)",xmin=0,xmax=1,r=1):
\end{eulerprompt}
\eulerimg{14}{images/EMT4Plot2D-Naela Rizqy Arofah-22305144042-097.png}
\begin{eulerprompt}
>t=linspace(0,1,1000); r=exp(-t); x=r*cos(2pi*t); y=r*sin(2pi*t);
>plot2d(x,y,r=1):
\end{eulerprompt}
\eulerimg{14}{images/EMT4Plot2D-Naela Rizqy Arofah-22305144042-098.png}
\begin{eulercomment}
In the next example, we plot the curve

\end{eulercomment}
\begin{eulerformula}
\[
\gamma(t) = (r(t) \cos(t), r(t) \sin(t))
\]
\end{eulerformula}
\begin{eulercomment}
with

\end{eulercomment}
\begin{eulerformula}
\[
r(t) = 1 + \dfrac{\sin(3t)}{2}.
\]
\end{eulerformula}
\begin{eulerprompt}
>t=linspace(0,2pi,1000); r=1+sin(3*t)/2; x=r*cos(t); y=r*sin(t); ...
>plot2d(x,y,>filled,fillcolor=red,style="/",r=1.5):
\end{eulerprompt}
\eulerimg{14}{images/EMT4Plot2D-Naela Rizqy Arofah-22305144042-099.png}
\eulerheading{Menggambar Grafik Bilangan Kompleks}
\begin{eulercomment}
Serangkaian bilangan kompleks juga dapat diplot. Kemudian titik-titik
grid akan dihubungkan. Jika sejumlah garis kisi ditentukan (atau\\
vektor garis kisi 1x2) dalam argumen cgrid, hanya garis kisi tersebut
yang terlihat.

Matriks bilangan kompleks secara otomatis akan diplot sebagai
kisi-kisi pada bidang kompleks.

ada contoh berikut, kita memplot gambar lingkaran satuan di bawah
fungsi eksponensial. Parameter cgrid menyembunyikan beberapa kurva
grid.
\end{eulercomment}
\begin{eulerprompt}
>aspect(); r=linspace(0,1,50); a=linspace(0,2pi,80)'; z=r*exp(I*a);...
>plot2d(z,a=-1.25,b=1.25,c=-1.25,d=1.25,cgrid=10):
\end{eulerprompt}
\eulerimg{27}{images/EMT4Plot2D-Naela Rizqy Arofah-22305144042-100.png}
\begin{eulerprompt}
>aspect(1.25); r=linspace(0,1,50); a=linspace(0,2pi,200)'; z=r*exp(I*a);
>plot2d(exp(z),cgrid=[40,10]):
\end{eulerprompt}
\eulerimg{21}{images/EMT4Plot2D-Naela Rizqy Arofah-22305144042-101.png}
\begin{eulerprompt}
>r=linspace(0,1,10); a=linspace(0,2pi,40)'; z=r*exp(I*a);
>plot2d(exp(z),>points,>add):
\end{eulerprompt}
\eulerimg{21}{images/EMT4Plot2D-Naela Rizqy Arofah-22305144042-102.png}
\begin{eulercomment}
Vektor bilangan kompleks secara otomatis diplot sebagai kurva pada
bidang kompleks dengan bagian nyata dan bagian imajiner.


n the example, we plot the unit circle with

\end{eulercomment}
\begin{eulerformula}
\[
\gamma(t) = e^{it}
\]
\end{eulerformula}
\begin{eulerprompt}
>t=linspace(0,2pi,1000); ...
>plot2d(exp(I*t)+exp(4*I*t),r=2):
\end{eulerprompt}
\eulerimg{21}{images/EMT4Plot2D-Naela Rizqy Arofah-22305144042-103.png}
\eulerheading{Plot Statistik}
\begin{eulercomment}
Ada banyak fungsi yang dikhususkan pada plot statistik. Salah satu
plot yang sering digunakan adalah plot kolom.

Jumlah kumulatif dari nilai terdistribusi normal 0-1 menghasilkan
jalan acak.
\end{eulercomment}
\begin{eulerprompt}
>plot2d(cumsum(randnormal(1,1000))):
\end{eulerprompt}
\eulerimg{21}{images/EMT4Plot2D-Naela Rizqy Arofah-22305144042-104.png}
\begin{eulercomment}
Using two rows shows a walk in two dimensions.
\end{eulercomment}
\begin{eulerprompt}
>X=cumsum(randnormal(2,1000)); plot2d(X[1],X[2]):
\end{eulerprompt}
\eulerimg{21}{images/EMT4Plot2D-Naela Rizqy Arofah-22305144042-105.png}
\begin{eulerprompt}
>columnsplot(cumsum(random(10)),style="/",color=blue):
\end{eulerprompt}
\eulerimg{21}{images/EMT4Plot2D-Naela Rizqy Arofah-22305144042-106.png}
\begin{eulercomment}
It can also show strings as labels.
\end{eulercomment}
\begin{eulerprompt}
>months=["Jan","Feb","Mar","Apr","May","Jun", ...
>  "Jul","Aug","Sep","Oct","Nov","Dec"];
>values=[10,12,12,18,22,28,30,26,22,18,12,8];
>columnsplot(values,lab=months,color=red,style="-");
>title("Temperature"):
\end{eulerprompt}
\eulerimg{21}{images/EMT4Plot2D-Naela Rizqy Arofah-22305144042-107.png}
\begin{eulerprompt}
>k=0:10;
>plot2d(k,bin(10,k),>bar):
\end{eulerprompt}
\eulerimg{21}{images/EMT4Plot2D-Naela Rizqy Arofah-22305144042-108.png}
\begin{eulerprompt}
>plot2d(k,bin(10,k)); plot2d(k,bin(10,k),>points,>add):
\end{eulerprompt}
\eulerimg{21}{images/EMT4Plot2D-Naela Rizqy Arofah-22305144042-109.png}
\begin{eulerprompt}
>plot2d(normal(1000),normal(1000),>points,grid=6,style=".."):
\end{eulerprompt}
\eulerimg{21}{images/EMT4Plot2D-Naela Rizqy Arofah-22305144042-110.png}
\begin{eulerprompt}
>plot2d(normal(1,1000),>distribution,style="O"):
\end{eulerprompt}
\eulerimg{21}{images/EMT4Plot2D-Naela Rizqy Arofah-22305144042-111.png}
\begin{eulerprompt}
>plot2d("qnormal",0,5;2.5,0.5,>filled):
\end{eulerprompt}
\eulerimg{21}{images/EMT4Plot2D-Naela Rizqy Arofah-22305144042-112.png}
\begin{eulercomment}
To plot an experimental statistical distribution, you can use distribution=n with
plot2d.
\end{eulercomment}
\begin{eulerprompt}
>w=randexponential(1,1000); // exponential distribution
>plot2d(w,>distribution): // or distribution=n with n intervals
\end{eulerprompt}
\eulerimg{21}{images/EMT4Plot2D-Naela Rizqy Arofah-22305144042-113.png}
\begin{eulercomment}
Or you can compute the distribution from the data and plot the result with \textgreater{}bar in
plot3d, or with a column plot.
\end{eulercomment}
\begin{eulerprompt}
>w=normal(1000); // 0-1-normal distribution
>\{x,y\}=histo(w,10,v=[-6,-4,-2,-1,0,1,2,4,6]); // interval bounds v
>plot2d(x,y,>bar):
\end{eulerprompt}
\eulerimg{21}{images/EMT4Plot2D-Naela Rizqy Arofah-22305144042-114.png}
\begin{eulercomment}
The statplot() function sets the style with a simple string.
\end{eulercomment}
\begin{eulerprompt}
>statplot(1:10,cumsum(random(10)),"b"):
\end{eulerprompt}
\eulerimg{21}{images/EMT4Plot2D-Naela Rizqy Arofah-22305144042-115.png}
\begin{eulerprompt}
>n=10; i=0:n; ...
>plot2d(i,bin(n,i)/2^n,a=0,b=10,c=0,d=0.3); ...
>plot2d(i,bin(n,i)/2^n,points=true,style="ow",add=true,color=blue):
\end{eulerprompt}
\eulerimg{21}{images/EMT4Plot2D-Naela Rizqy Arofah-22305144042-116.png}
\begin{eulercomment}
Moreover, data can be plotted as bars. In this case, x should be sorted and one
element longer than y. The bars will extend from x[i] to x[i+1] with values y[i]. If x
has the same size as y, it will be extended by one element with the last spacing.

Fill styles can be used just as above.
\end{eulercomment}
\begin{eulerprompt}
>n=10; k=bin(n,0:n); ...
>plot2d(-0.5:n+0.5,k,bar=true,fillcolor=lightgray):
\end{eulerprompt}
\eulerimg{21}{images/EMT4Plot2D-Naela Rizqy Arofah-22305144042-117.png}
\begin{eulercomment}
The data for bar plots (bar=1) and histograms (histogram=1) can either be explicitly
given in xv and yv, or can be computed from an empirical distribution in xv with
\textgreater{}distribution (or distribution=n). Histograms of xv values will be computed
automatically with \textgreater{}histogram. If \textgreater{}even is specified, the xv values will be counted in
integer intervals.
\end{eulercomment}
\begin{eulerprompt}
>plot2d(normal(10000),distribution=50):
\end{eulerprompt}
\eulerimg{21}{images/EMT4Plot2D-Naela Rizqy Arofah-22305144042-118.png}
\begin{eulerprompt}
>k=0:10; m=bin(10,k); x=(0:11)-0.5; plot2d(x,m,>bar):
\end{eulerprompt}
\eulerimg{21}{images/EMT4Plot2D-Naela Rizqy Arofah-22305144042-119.png}
\begin{eulerprompt}
>columnsplot(m,k):
\end{eulerprompt}
\eulerimg{21}{images/EMT4Plot2D-Naela Rizqy Arofah-22305144042-120.png}
\begin{eulerprompt}
>plot2d(random(600)*6,histogram=6):
\end{eulerprompt}
\eulerimg{21}{images/EMT4Plot2D-Naela Rizqy Arofah-22305144042-121.png}
\begin{eulercomment}
For distributions, there is the parameter distribution=n, which counts values
automatically and prints the relative distribution with n sub-intervals.
\end{eulercomment}
\begin{eulerprompt}
>plot2d(normal(1,1000),distribution=10,style="\(\backslash\)/"):
\end{eulerprompt}
\eulerimg{21}{images/EMT4Plot2D-Naela Rizqy Arofah-22305144042-122.png}
\begin{eulercomment}
With the parameter even=true, this will use integer intervals.
\end{eulercomment}
\begin{eulerprompt}
>plot2d(intrandom(1,1000,10),distribution=10,even=true):
\end{eulerprompt}
\eulerimg{21}{images/EMT4Plot2D-Naela Rizqy Arofah-22305144042-123.png}
\begin{eulercomment}
Note that there are many statistical plots, which might be useful. Have a look at the
tutorial about statistics.
\end{eulercomment}
\begin{eulerprompt}
>columnsplot(getmultiplicities(1:6,intrandom(1,6000,6))):
\end{eulerprompt}
\eulerimg{21}{images/EMT4Plot2D-Naela Rizqy Arofah-22305144042-124.png}
\begin{eulerprompt}
>plot2d(normal(1,1000),>distribution); ...
>  plot2d("qnormal(x)",color=red,thickness=2,>add):
\end{eulerprompt}
\eulerimg{21}{images/EMT4Plot2D-Naela Rizqy Arofah-22305144042-125.png}
\begin{eulercomment}
There are also many special plots for statistics. A boxplot shows the quartiles of
this distribution and lots of outliers. By definition, outliers in a boxplot are data
which exceed 1.5 times the middle 50\% range of the plot.
\end{eulercomment}
\begin{eulerprompt}
>M=normal(5,1000); boxplot(quartiles(M)):
\end{eulerprompt}
\eulerimg{21}{images/EMT4Plot2D-Naela Rizqy Arofah-22305144042-126.png}
\eulerheading{Implicit Functions}
\begin{eulercomment}
Implicit plots show level lines solving f(x,y)=level, where "level" can be a single
value or a vector of values. If level="auto", there will be nc level lines, which will
spread between the minimum and the maximum of the function evenly. Darker or lighter
color can be added with \textgreater{}hue to indicate value of the function. For implicit
functions, xv must be a function or an expression of the parameters x and y, or,
alternatively, xv can be a matrix of values.

Euler can mark the level lines

\end{eulercomment}
\begin{eulerformula}
\[
f(x,y) = c
\]
\end{eulerformula}
\begin{eulercomment}
of any function.

To draw the set f(x,y)=c for one or more constants c you can use plot2d() with its
implicit plots in the plane. The parameter for c is level=c, where c can be vector of
level lines. Additionally, a color scheme can be drawn in the background to indicate
the value of the function for each point in the plot. The parameter "n" determines the
fineness of the plot.
\end{eulercomment}
\begin{eulerprompt}
>aspect(1.5); 
>plot2d("x^2+y^2-x*y-x",r=1.5,level=0,contourcolor=red):
\end{eulerprompt}
\eulerimg{17}{images/EMT4Plot2D-Naela Rizqy Arofah-22305144042-127.png}
\begin{eulerprompt}
>expr := "2*x^2+x*y+3*y^4+y"; // define an expression f(x,y)
>plot2d(expr,level=0): // Solutions of f(x,y)=0
\end{eulerprompt}
\eulerimg{17}{images/EMT4Plot2D-Naela Rizqy Arofah-22305144042-128.png}
\begin{eulerprompt}
>plot2d(expr,level=0:0.5:20,>hue,contourcolor=white,n=200): // nice
\end{eulerprompt}
\eulerimg{17}{images/EMT4Plot2D-Naela Rizqy Arofah-22305144042-129.png}
\begin{eulerprompt}
>plot2d(expr,level=0:0.5:20,>hue,>spectral,n=200,grid=4): // nicer
\end{eulerprompt}
\eulerimg{17}{images/EMT4Plot2D-Naela Rizqy Arofah-22305144042-130.png}
\begin{eulercomment}
This works for data plots too. But you will have to specify the ranges\\
for the axis labels.
\end{eulercomment}
\begin{eulerprompt}
>x=-2:0.05:1; y=x'; z=expr(x,y);
>plot2d(z,level=0,a=-1,b=2,c=-2,d=1,>hue):
\end{eulerprompt}
\eulerimg{17}{images/EMT4Plot2D-Naela Rizqy Arofah-22305144042-131.png}
\begin{eulerprompt}
>plot2d("x^3-y^2",>contour,>hue,>spectral):
\end{eulerprompt}
\eulerimg{17}{images/EMT4Plot2D-Naela Rizqy Arofah-22305144042-132.png}
\begin{eulerprompt}
>plot2d("x^3-y^2",level=0,contourwidth=3,>add,contourcolor=red):
\end{eulerprompt}
\eulerimg{17}{images/EMT4Plot2D-Naela Rizqy Arofah-22305144042-133.png}
\begin{eulerprompt}
>z=z+normal(size(z))*0.2;
>plot2d(z,level=0.5,a=-1,b=2,c=-2,d=1):
\end{eulerprompt}
\eulerimg{17}{images/EMT4Plot2D-Naela Rizqy Arofah-22305144042-134.png}
\begin{eulerprompt}
>plot2d(expr,level=[0:0.2:5;0.05:0.2:5.05],color=lightgray):
\end{eulerprompt}
\eulerimg{17}{images/EMT4Plot2D-Naela Rizqy Arofah-22305144042-135.png}
\begin{eulerprompt}
>plot2d("x^2+y^3+x*y",level=1,r=4,n=100):
\end{eulerprompt}
\eulerimg{17}{images/EMT4Plot2D-Naela Rizqy Arofah-22305144042-136.png}
\begin{eulerprompt}
>plot2d("x^2+2*y^2-x*y",level=0:0.1:10,n=100,contourcolor=white,>hue):
\end{eulerprompt}
\eulerimg{17}{images/EMT4Plot2D-Naela Rizqy Arofah-22305144042-137.png}
\begin{eulercomment}
It is also possible to fill the set

\end{eulercomment}
\begin{eulerformula}
\[
a \le f(x,y) \le b
\]
\end{eulerformula}
\begin{eulercomment}
with a level range.

It is possible to fill regions of values for a specific function. For this, level must
be a 2xn matrix. The first row are the lower bounds and the second row contains the
upper bounds.
\end{eulercomment}
\begin{eulerprompt}
>plot2d(expr,level=[0;1],style="-",color=blue): // 0 <= f(x,y) <= 1
\end{eulerprompt}
\eulerimg{17}{images/EMT4Plot2D-Naela Rizqy Arofah-22305144042-138.png}
\begin{eulercomment}
Implicit plots can also show ranges of levels. Then level must be a 2xn matrix of
level intervals, where the first row contains the start and the second row the end of
each interval. Alternatively, a simple row vector can be used for level, and a
parameter dl extends the level values to intervals.
\end{eulercomment}
\begin{eulerprompt}
>plot2d("x^4+y^4",r=1.5,level=[0;1],color=blue,style="/"):
\end{eulerprompt}
\eulerimg{17}{images/EMT4Plot2D-Naela Rizqy Arofah-22305144042-139.png}
\begin{eulerprompt}
>plot2d("x^2+y^3+x*y",level=[0,2,4;1,3,5],style="/",r=2,n=100):
\end{eulerprompt}
\eulerimg{17}{images/EMT4Plot2D-Naela Rizqy Arofah-22305144042-140.png}
\begin{eulerprompt}
>plot2d("x^2+y^3+x*y",level=-10:20,r=2,style="-",dl=0.1,n=100):
\end{eulerprompt}
\eulerimg{17}{images/EMT4Plot2D-Naela Rizqy Arofah-22305144042-141.png}
\begin{eulerprompt}
>plot2d("sin(x)*cos(y)",r=pi,>hue,>levels,n=100):
\end{eulerprompt}
\eulerimg{17}{images/EMT4Plot2D-Naela Rizqy Arofah-22305144042-142.png}
\begin{eulercomment}
It is also possible to mark a region

\end{eulercomment}
\begin{eulerformula}
\[
a \le f(x,y) \le b.
\]
\end{eulerformula}
\begin{eulercomment}
This is done by adding a level with two rows.
\end{eulercomment}
\begin{eulerprompt}
>plot2d("(x^2+y^2-1)^3-x^2*y^3",r=1.3, ...
>  style="#",color=red,<outline, ...
>  level=[-2;0],n=100):
\end{eulerprompt}
\eulerimg{17}{images/EMT4Plot2D-Naela Rizqy Arofah-22305144042-143.png}
\begin{eulercomment}
It is possible to specify a specific level. E.g., we can plot the solution of an
equation like

\end{eulercomment}
\begin{eulerformula}
\[
x^3-xy+x^2y^2=6
\]
\end{eulerformula}
\begin{eulerprompt}
>plot2d("x^3-x*y+x^2*y^2",r=6,level=1,n=100):
\end{eulerprompt}
\eulerimg{17}{images/EMT4Plot2D-Naela Rizqy Arofah-22305144042-144.png}
\begin{eulerprompt}
>function starplot1 (v, style="/", color=green, lab=none) ...
\end{eulerprompt}
\begin{eulerudf}
    if !holding() then clg; endif;
    w=window(); window(0,0,1024,1024);
    h=holding(1);
    r=max(abs(v))*1.2;
    setplot(-r,r,-r,r);
    n=cols(v); t=linspace(0,2pi,n);
    v=v|v[1]; c=v*cos(t); s=v*sin(t);
    cl=barcolor(color); st=barstyle(style);
    loop 1 to n
      polygon([0,c[#],c[#+1]],[0,s[#],s[#+1]],1);
      if lab!=none then
        rlab=v[#]+r*0.1;
        \{col,row\}=toscreen(cos(t[#])*rlab,sin(t[#])*rlab);
        ctext(""+lab[#],col,row-textheight()/2);
      endif;
    end;
    barcolor(cl); barstyle(st);
    holding(h);
    window(w);
  endfunction
\end{eulerudf}
\begin{eulercomment}
There is no grid or axis ticks here. Moreover, we use the full window for the plot.

We call reset before we test this plot to restore the graphics defaults. This is not
necessary, if you are sure that your plot works.
\end{eulercomment}
\begin{eulerprompt}
>reset; starplot1(normal(1,10)+5,color=red,lab=1:10):
\end{eulerprompt}
\eulerimg{27}{images/EMT4Plot2D-Naela Rizqy Arofah-22305144042-145.png}
\begin{eulercomment}
Sometimes, you may want to plot something that plot2d cannot do, but almost.

In the following function, we do a logarithmic impulse plot. plot2d can do logarithmic
plots, but not for impulse bars.
\end{eulercomment}
\begin{eulerprompt}
>function logimpulseplot1 (x,y) ...
\end{eulerprompt}
\begin{eulerudf}
    \{x0,y0\}=makeimpulse(x,log(y)/log(10));
    plot2d(x0,y0,>bar,grid=0);
    h=holding(1);
    frame();
    xgrid(ticks(x));
    p=plot();
    for i=-10 to 10;
      if i<=p[4] and i>=p[3] then
         ygrid(i,yt="10^"+i);
      endif;
    end;
    holding(h);
  endfunction
\end{eulerudf}
\begin{eulercomment}
Let us test it with exponentially distributed values.
\end{eulercomment}
\begin{eulerprompt}
>aspect(1.5); x=1:10; y=-log(random(size(x)))*200; ...
>logimpulseplot1(x,y):
\end{eulerprompt}
\eulerimg{17}{images/EMT4Plot2D-Naela Rizqy Arofah-22305144042-146.png}
\begin{eulercomment}
Let us animate a 2D curve using direct plots. The plot(x,y) command
simply plots a curve into the plot window. setplot(a,b,c,d) sets this
window.

The wait(0) function forces the plot to appear on the graphics
windows. Otherwise, the redraw takes place in sparse time intervals.
\end{eulercomment}
\begin{eulerprompt}
>function animliss (n,m) ...
\end{eulerprompt}
\begin{eulerudf}
  t=linspace(0,2pi,500);
  f=0;
  c=framecolor(0);
  l=linewidth(2);
  setplot(-1,1,-1,1);
  repeat
    clg;
    plot(sin(n*t),cos(m*t+f));
    wait(0);
    if testkey() then break; endif;
    f=f+0.02;
  end;
  framecolor(c);
  linewidth(l);
  endfunction
\end{eulerudf}
\begin{eulercomment}
Press any key to stop this animation.
\end{eulercomment}
\begin{eulerprompt}
>animliss(2,3); // lihat hasilnya, jika sudah puas, tekan ENTER
\end{eulerprompt}
\eulerheading{Logarithmic Plots}
\begin{eulercomment}
EMT uses the "logplot" parameter for logarithmic scales.\\
Logarithmic plots can be plotted either using a logarithmic scale in y with logplot=1,
or using logarithmic scales in x and y with logplot=2, or in x with logplot=3.

\end{eulercomment}
\begin{eulerttcomment}
 - logplot=1: y-logarithmic
 - logplot=2: x-y-logarithmic
 - logplot=3: x-logarithmic
\end{eulerttcomment}
\begin{eulerprompt}
>plot2d("exp(x^3-x)*x^2",1,5,logplot=1):
\end{eulerprompt}
\eulerimg{17}{images/EMT4Plot2D-Naela Rizqy Arofah-22305144042-147.png}
\begin{eulerprompt}
>plot2d("exp(x+sin(x))",0,100,logplot=1):
\end{eulerprompt}
\eulerimg{17}{images/EMT4Plot2D-Naela Rizqy Arofah-22305144042-148.png}
\begin{eulerprompt}
>plot2d("exp(x+sin(x))",10,100,logplot=2):
\end{eulerprompt}
\eulerimg{17}{images/EMT4Plot2D-Naela Rizqy Arofah-22305144042-149.png}
\begin{eulerprompt}
>plot2d("gamma(x)",1,10,logplot=1):
\end{eulerprompt}
\eulerimg{17}{images/EMT4Plot2D-Naela Rizqy Arofah-22305144042-150.png}
\begin{eulerprompt}
>plot2d("log(x*(2+sin(x/100)))",10,1000,logplot=3):
\end{eulerprompt}
\eulerimg{17}{images/EMT4Plot2D-Naela Rizqy Arofah-22305144042-151.png}
\begin{eulercomment}
This does also work with data plots.
\end{eulercomment}
\begin{eulerprompt}
>x=10^(1:20); y=x^2-x;
>plot2d(x,y,logplot=2):
\end{eulerprompt}
\eulerimg{17}{images/EMT4Plot2D-Naela Rizqy Arofah-22305144042-152.png}
\eulerheading{Rujukan Lengkap Fungsi plot2d()}
\begin{eulercomment}
\end{eulercomment}
\begin{eulerttcomment}
  function plot2d (xv, yv, btest, a, b, c, d, xmin, xmax, r, n,  ..
  logplot, grid, frame, framecolor, square, color, thickness, style, ..
  auto, add, user, delta, points, addpoints, pointstyle, bar, histogram,  ..
  distribution, even, steps, own, adaptive, hue, level, contour,  ..
  nc, filled, fillcolor, outline, title, xl, yl, maps, contourcolor, ..
  contourwidth, ticks, margin, clipping, cx, cy, insimg, spectral,  ..
  cgrid, vertical, smaller, dl, niveau, levels)
\end{eulerttcomment}
\begin{eulercomment}
Multipurpose plot function for plots in the plane (2D plots). This function can do
plots of functions of one variables, data plots, curves in the plane, bar plots, grids
of complex numbers, and implicit plots of functions of two variables.

Parameters
\\
x,y       : equations, functions or data vectors\\
a,b,c,d   : Plot area (default a=-2,b=2)\\
r         : if r is set, then a=cx-r, b=cx+r, c=cy-r, d=cy+r\\
\end{eulercomment}
\begin{eulerttcomment}
            r can be a vector [rx,ry] or a vector [rx1,rx2,ry1,ry2].
\end{eulerttcomment}
\begin{eulercomment}
xmin,xmax : range of the parameter for curves\\
auto      : Determine y-range automatically (default)\\
square    : if true, try to keep square x-y-ranges\\
n         : number of intervals (default is adaptive)\\
grid      : 0 = no grid and labels,\\
\end{eulercomment}
\begin{eulerttcomment}
            1 = axis only,
            2 = normal grid (see below for the number of grid lines)
            3 = inside axis
            4 = no grid
            5 = full grid including margin
            6 = ticks at the frame
            7 = axis only
            8 = axis only, sub-ticks
\end{eulerttcomment}
\begin{eulercomment}
frame     : 0 = no frame\\
framecolor: color of the frame and the grid\\
margin    : number between 0 and 0.4 for the margin around the plot\\
color     : Color of curves. If this is a vector of colors,\\
\end{eulercomment}
\begin{eulerttcomment}
            it will be used for each row of a matrix of plots. In the case of
            point plots, it should be a column vector. If a row vector or a
            full matrix of colors is used for point plots, it will be used for
            each data point.
\end{eulerttcomment}
\begin{eulercomment}
thickness : line thickness for curves\\
\end{eulercomment}
\begin{eulerttcomment}
            This value can be smaller than 1 for very thin lines.
\end{eulerttcomment}
\begin{eulercomment}
style     : Plot style for lines, markers, and fills.\\
\end{eulercomment}
\begin{eulerttcomment}
            For points use
            "[]", "<>", ".", "..", "...",
            "*", "+", "|", "-", "o"
            "[]#", "<>#", "o#" (filled shapes)
            "[]w", "<>w", "ow" (non-transparent)
            For lines use
            "-", "--", "-.", ".", ".-.", "-.-", "->"
            For filled polygons or bar plots use
            "#", "#O", "O", "/", "\(\backslash\)", "\(\backslash\)/",
            "+", "|", "-", "t"
\end{eulerttcomment}
\begin{eulercomment}
points    : plot single points instead of line segments\\
addpoints : if true, plots line segments and points\\
add       : add the plot to the existing plot\\
user      : enable user interaction for functions\\
delta     : step size for user interaction\\
bar       : bar plot (x are the interval bounds, y the interval values)\\
histogram : plots the frequencies of x in n subintervals\\
distribution=n : plots the distribution of x with n subintervals\\
even      : use inter values for automatic histograms.\\
steps     : plots the function as a step function (steps=1,2)\\
adaptive  : use adaptive plots (n is the minimal number of steps)\\
level     : plot level lines of an implicit function of two variables\\
outline   : draws boundary of level ranges.
\\
If the level value is a 2xn matrix, ranges of levels will be drawn\\
in the color using the given fill style. If outline is true, it\\
will be drawn in the contour color. Using this feature, regions of\\
f(x,y) between limits can be marked.
\\
hue       : add hue color to the level plot to indicate the function\\
\end{eulercomment}
\begin{eulerttcomment}
            value
\end{eulerttcomment}
\begin{eulercomment}
contour   : Use level plot with automatic levels\\
nc        : number of automatic level lines\\
title     : plot title (default "")\\
xl, yl    : labels for the x- and y-axis\\
smaller   : if \textgreater{}0, there will be more space to the left for labels.\\
vertical  :\\
\end{eulercomment}
\begin{eulerttcomment}
  Turns vertical labels on or off. This changes the global variable
  verticallabels locally for one plot. The value 1 sets only vertical
  text, the value 2 uses vertical numerical labels on the y axis.
\end{eulerttcomment}
\begin{eulercomment}
filled    : fill the plot of a curve\\
fillcolor : fill color for bar and filled curves\\
outline   : boundary for filled polygons\\
logplot   : set logarithmic plots\\
\end{eulercomment}
\begin{eulerttcomment}
            1 = logplot in y,
            2 = logplot in xy,
            3 = logplot in x
\end{eulerttcomment}
\begin{eulercomment}
own       :\\
\end{eulercomment}
\begin{eulerttcomment}
  A string, which points to an own plot routine. With >user, you get
  the same user interaction as in plot2d. The range will be set
  before each call to your function.
\end{eulerttcomment}
\begin{eulercomment}
maps      : map expressions (0 is faster), functions are always mapped.\\
contourcolor : color of contour lines\\
contourwidth : width of contour lines\\
clipping  : toggles the clipping (default is true)\\
title     :\\
\end{eulercomment}
\begin{eulerttcomment}
  This can be used to describe the plot. The title will appear above
  the plot. Moreover, a label for the x and y axis can be added with
  xl="string" or yl="string". Other labels can be added with the
  functions label() or labelbox(). The title can be a unicode
  string or an image of a Latex formula.
\end{eulerttcomment}
\begin{eulercomment}
cgrid     :\\
\end{eulercomment}
\begin{eulerttcomment}
  Determines the number of grid lines for plots of complex grids.
  Should be a divisor of the the matrix size minus 1 (number of
  subintervals). cgrid can be a vector [cx,cy].
\end{eulerttcomment}
\begin{eulercomment}

Overview

The function can plot

- expressions, call collections or functions of one variable,\\
- parametric curves,\\
- x data against y data,\\
- implicit functions,\\
- bar plots,\\
- complex grids,\\
- polygons.

If a function or expression for xv is given, plot2d() will compute\\
values in the given range using the function or expression. The\\
expression must be an expression in the variable x. The range must\\
be defined in the parameters a and b unless the default range\\
[-2,2] should be used. The y-range will be computed automatically,\\
unless c and d are specified, or a radius r, which yields the range\\
[-r,r] for x and y. For plots of functions, plot2d will use an\\
adaptive evaluation of the function by default. To speed up the\\
plot for complicated functions, switch this off with \textless{}adaptive, and\\
optionally decrease the number of intervals n. Moreover, plot2d()\\
will by default use mapping. I.e., it will compute the plot element\\
for element. If your expression or your functions can handle a\\
vector x, you can switch that off with \textless{}maps for faster evaluation.

Note that adaptive plots are always computed element for element. \\
If functions or expressions for both xv and for yv are specified,\\
plot2d() will compute a curve with the xv values as x-coordinates\\
and the yv values as y-coordinates. In this case, a range should be\\
defined for the parameter using xmin, xmax. Expressions contained\\
in strings must always be expressions in the parameter variable x.
\end{eulercomment}

\chapter{EMT untuk plot 3D}

\eulersubheading{}
\begin{eulercomment}
Ini adalah pengenalan plot 3D di Euler. Kita memerlukan plot 3D untuk
memvisualisasikan fungsi dari dua variabel.

Euler menggambar fungsi-fungsi tersebut dengan menggunakan algoritme
pengurutan untuk\\
menyembunyikan bagian-bagian di latar belakang. Secara umum, Euler
menggunakan proyeksi pusat. Standarnya adalah dari kuadran x-y positif
ke arah asal x=y=z=0, tetapi sudut=0° terlihat dari arah sumbu-y.
Sudut pandang dan ketinggian dapat diubah.

Euler can plot :

- memplot - permukaan dengan garis bayangan dan garis datar,\\
- awan titik,\\
- kurva parametrik,\\
- permukaan implisit.


Plot 3D suatu fungsi menggunakan plot3d. Cara termudah adalah dengan
memplot ekspresi dalam x dan y. Parameter r mengatur rentang plot
sekitar (0,0).
\end{eulercomment}
\begin{eulerprompt}
>aspect(1.5); plot3d("x^2+sin(y)",-5,5,0,6*pi):
\end{eulerprompt}
\eulerimg{17}{images/EMT4Plot3D-Naela Rizqy Arofah-22305144042-001.png}
\begin{eulerprompt}
>plot3d("x^2+x*sin(y)",-5,5,0,6*pi):
\end{eulerprompt}
\eulerimg{17}{images/EMT4Plot3D-Naela Rizqy Arofah-22305144042-002.png}
\begin{eulercomment}
Silakan lakukan modifikasi agar gambar "talang bergelombang" tersebut tidak lurus melainkan melengkung/melingkar, baik
melingkar secara mendatar maupun melingkar turun/naik (seperti papan peluncur pada kolam renang. Temukan rumusnya.
\end{eulercomment}
\eulerheading{Fungsi dua Variabel}
\begin{eulercomment}
ntuk grafik suatu fungsi, gunakan - 

- ekspresi sederhana dalam x dan y,\\
- nama fungsi dari dua variabel,\\
- atau matriks data.

Standarnya adalah kisi-kisi kawat berisi dengan warna berbeda di kedua
sisi. Perhatikan bahwa jumlah interval kisi default adalah 10, tetapi
plot menggunakan jumlah default persegi panjang 40x40 untuk membuat
permukaannya. Ini bisa diubah.

- n=40, n=[40,40]: jumlah garis grid di setiap arah.\\
- grid=10, grid=[10,10]: : jumlah garis grid di setiap arah.

Kami menggunakan default n=40 dan grid=10.
\end{eulercomment}
\begin{eulerprompt}
>plot3d("x^2+y^2"):
\end{eulerprompt}
\eulerimg{17}{images/EMT4Plot3D-Naela Rizqy Arofah-22305144042-003.png}
\begin{eulercomment}
Interaksi pengguna dimungkinkan dengan parameter \textgreater{}pengguna. Pengguna
dapat menekan tombol berikut.

- left,right,up,down: : putar sudut pandang,\\
- +,-: zoom in or out\\
- a: menghasilkan anaglyph (lihat di bawah)\\
- l:  beralih memutar sumber cahaya(lihat di bawah)\\
- space: reset to default\\
- return: end interaction
\end{eulercomment}
\begin{eulerprompt}
>plot3d("exp(-x^2+y^2)",>user, ...
>  title="Turn with the vector keys (press return to finish)"):
\end{eulerprompt}
\eulerimg{17}{images/EMT4Plot3D-Naela Rizqy Arofah-22305144042-004.png}
\begin{eulercomment}
Rentang plot untuk fungsi dapat ditentukan dengan :

- a,b: the x-range\\
- c,d: the y-range\\
- r: a symmetric square around (0,0).\\
- n: number of subintervals for the plot.

Ada beberapa parameter untuk menskalakan fungsi atau mengubah tampilan
grafik.

fscale: scales to function values (default is \textless{}fscale).\\
scale: angka atau vektor 1x2 untuk menskalakan ke arah x dan y\\
frame:  jenis bingkai (default 1).
\end{eulercomment}
\begin{eulerprompt}
>plot3d("exp(-(x^2+y^2)/5)",r=10,n=80,fscale=4,scale=1.2,frame=3,>user):
\end{eulerprompt}
\eulerimg{17}{images/EMT4Plot3D-Naela Rizqy Arofah-22305144042-005.png}
\begin{eulercomment}
Tampilan dapat diubah dengan berbagai cara.

- distance: jarak pandang ke plot.\\
- zoom: the zoom value.\\
- sudut: the angle to the negative y-axis in radians.\\
- height: the height of the view in radians.

Nilai default dapat diperiksa atau diubah dengan fungsi view(). Ini
mengembalikan parameter dalam urutan di atas.
\end{eulercomment}
\begin{eulerprompt}
>view
\end{eulerprompt}
\begin{euleroutput}
  [5,  2.6,  2,  0.4]
\end{euleroutput}
\begin{eulercomment}
Jarak yang lebih dekat membutuhkan lebih sedikit zoom. Efeknya lebih
seperti lensa sudut lebar.

ada contoh berikut, sudut=0 dan tinggi=0 dilihat dari sumbu y negatif.
Label sumbu untuk y disembunyikan dalam kasus ini.
\end{eulercomment}
\begin{eulerprompt}
>plot3d("x^2+y",distance=3,zoom=1,angle=pi/2,height=0):
\end{eulerprompt}
\eulerimg{17}{images/EMT4Plot3D-Naela Rizqy Arofah-22305144042-006.png}
\begin{eulercomment}
Plot selalu terlihat berada di tengah kubus plot. Anda dapat
memindahkan bagian tengah dengan parameter tengah.
\end{eulercomment}
\begin{eulerprompt}
>plot3d("x^4+y^2",a=0,b=1,c=-1,d=1,angle=-20°,height=20°, ...
>  center=[0.4,0,0],zoom=5):
\end{eulerprompt}
\eulerimg{17}{images/EMT4Plot3D-Naela Rizqy Arofah-22305144042-007.png}
\begin{eulercomment}
Plotnya diskalakan agar sesuai dengan unit kubus untuk dilihat. Jadi
tidak perlu mengubah jarak atau zoom tergantung ukuran plot. Namun
labelnya mengacu pada ukuran sebenarnya.

Jika Anda mematikannya dengan scale=false, Anda harus berhati-hati
agar plot tetap masuk ke dalam jendela plotting, dengan mengubah jarak
pandang atau zoom, dan memindahkan bagian tengah.
\end{eulercomment}
\begin{eulerprompt}
>plot3d("5*exp(-x^2-y^2)",r=2,<fscale,<scale,distance=13,height=50°, ...
>  center=[0,0,-2],frame=3):
\end{eulerprompt}
\eulerimg{17}{images/EMT4Plot3D-Naela Rizqy Arofah-22305144042-008.png}
\begin{eulercomment}
Plot kutub juga tersedia. Parameter polar=true menggambar plot kutub.
Fungsi tersebut harus tetap merupakan fungsi dari x dan y. Parameter
"fscale" menskalakan fungsi dengan skalanya sendiri. Kalau tidak,
fungsinya akan diskalakan agar sesuai dengan kubus.
\end{eulercomment}
\begin{eulerprompt}
>plot3d("1/(x^2+y^2+1)",r=5,>polar, ...
>fscale=2,>hue,n=100,zoom=4,>contour,color=blue):
\end{eulerprompt}
\eulerimg{17}{images/EMT4Plot3D-Naela Rizqy Arofah-22305144042-009.png}
\begin{eulerprompt}
>function f(r) := exp(-r/2)*cos(r); ...
>plot3d("f(x^2+y^2)",>polar,scale=[1,1,0.4],r=pi,frame=3,zoom=4):
\end{eulerprompt}
\eulerimg{17}{images/EMT4Plot3D-Naela Rizqy Arofah-22305144042-010.png}
\begin{eulercomment}
Parameter memutar memutar fungsi di x di sekitar sumbu x.

- rotate=1: Uses the x-axis\\
- rotate=2: Uses the z-axis
\end{eulercomment}
\begin{eulerprompt}
>plot3d("x^2+1",a=-1,b=1,rotate=true,grid=5):
\end{eulerprompt}
\eulerimg{17}{images/EMT4Plot3D-Naela Rizqy Arofah-22305144042-011.png}
\begin{eulerprompt}
>plot3d("x^2+1",a=-1,b=1,rotate=2,grid=5):
\end{eulerprompt}
\eulerimg{17}{images/EMT4Plot3D-Naela Rizqy Arofah-22305144042-012.png}
\begin{eulerprompt}
>plot3d("sqrt(25-x^2)",a=0,b=5,rotate=1):
\end{eulerprompt}
\eulerimg{17}{images/EMT4Plot3D-Naela Rizqy Arofah-22305144042-013.png}
\begin{eulerprompt}
>plot3d("x*sin(x)",a=0,b=6pi,rotate=2):
\end{eulerprompt}
\eulerimg{17}{images/EMT4Plot3D-Naela Rizqy Arofah-22305144042-014.png}
\begin{eulercomment}
Berikut adalah plot dengan tiga fungsi.
\end{eulercomment}
\begin{eulerprompt}
>plot3d("x","x^2+y^2","y",r=2,zoom=3.5,frame=3):
\end{eulerprompt}
\eulerimg{17}{images/EMT4Plot3D-Naela Rizqy Arofah-22305144042-015.png}
\eulerheading{Plot Kontur}
\begin{eulercomment}
ntuk plotnya, Euler menambahkan garis grid. Sebaliknya dimungkinkan
untuk menggunakan garis datar dan rona satu warna atau rona warna
spektral. Euler dapat menggambar ketinggian fungsi pada plot dengan
arsiran. Di semua plot 3D, Euler dapat menghasilkan anaglyph.

- \textgreater{}hue:  Mengaktifkan bayangan cahaya, bukan kabel.

\end{eulercomment}
\begin{eulerttcomment}
 >contour: : Membuat plot garis kontur otomatis pada plot.
\end{eulerttcomment}
\begin{eulercomment}
- level=... (or levels): A Vektor nilai garis kontur.

Standarnya adalah level="auto", yang menghitung beberapa garis level
secara otomatis. Seperti yang Anda lihat di plot, level sebenarnya
adalah rentang level.

Gaya default dapat diubah. Untuk plot kontur berikut, kami menggunakan
grid yang lebih halus berukuran 100x100 poin, menskalakan fungsi dan
plot, dan menggunakan sudut pandang yang berbeda.
\end{eulercomment}
\begin{eulerprompt}
>plot3d("exp(-x^2-y^2)",r=2,n=100,level="thin", ...
> >contour,>spectral,fscale=1,scale=1.1,angle=45°,height=20°):
\end{eulerprompt}
\eulerimg{17}{images/EMT4Plot3D-Naela Rizqy Arofah-22305144042-016.png}
\begin{eulerprompt}
>plot3d("exp(x*y)",angle=100°,>contour,color=green):
\end{eulerprompt}
\eulerimg{17}{images/EMT4Plot3D-Naela Rizqy Arofah-22305144042-017.png}
\begin{eulercomment}
Bayangan defaultnya menggunakan warna abu-abu. Namun rentang warna
spektral juga tersedia. \\
- \textgreater{}spectral: Menggunakan skema spektral default\\
- color=...: Menggunakan warna khusus atau skema

spektral Untuk plot berikut, kami menggunakan skema spektral default
dan menambah jumlah titik untuk mendapatkan tampilan yang sangat
halus.
\end{eulercomment}
\begin{eulerprompt}
>plot3d("x^2+y^2",>spectral,>contour,n=100):
\end{eulerprompt}
\eulerimg{17}{images/EMT4Plot3D-Naela Rizqy Arofah-22305144042-018.png}
\begin{eulercomment}
Selain garis level otomatis, kita juga dapat menetapkan nilai garis
level. Ini akan menghasilkan garis level yang tipis, bukan rentang
level.
\end{eulercomment}
\begin{eulerprompt}
>plot3d("x^2-y^2",0,5,0,5,level=-1:0.1:1,color=redgreen):
\end{eulerprompt}
\eulerimg{17}{images/EMT4Plot3D-Naela Rizqy Arofah-22305144042-019.png}
\begin{eulercomment}
Dalam plot berikut, kita menggunakan dua pita tingkat yang sangat luas
dari -0,1 hingga 1, dan dari 0,9 hingga 1. Ini dimasukkan sebagai
matriks dengan batas tingkat sebagai kolom.

Selain itu, kami melapisi grid dengan 10 interval di setiap arah.
\end{eulercomment}
\begin{eulerprompt}
>plot3d("x^2+y^3",level=[-0.1,0.9;0,1], ...
>  >spectral,angle=30°,grid=10,contourcolor=gray):
\end{eulerprompt}
\eulerimg{17}{images/EMT4Plot3D-Naela Rizqy Arofah-22305144042-020.png}
\begin{eulercomment}
Pada contoh berikut, kita memplot himpunan, di mana :

\end{eulercomment}
\begin{eulerformula}
\[
f(x,y) = x^y-y^x = 0
\]
\end{eulerformula}
\begin{eulercomment}
Kami menggunakan satu garis tipis untuk garis level.
\end{eulercomment}
\begin{eulerprompt}
>plot3d("x^y-y^x",level=0,a=0,b=6,c=0,d=6,contourcolor=red,n=100):
\end{eulerprompt}
\eulerimg{17}{images/EMT4Plot3D-Naela Rizqy Arofah-22305144042-021.png}
\begin{eulercomment}
Dimungkinkan untuk menampilkan bidang kontur di bawah plot. Warna dan
jarak ke plot dapat ditentukan.
\end{eulercomment}
\begin{eulerprompt}
>plot3d("x^2+y^4",>cp,cpcolor=green,cpdelta=0.2):
\end{eulerprompt}
\eulerimg{17}{images/EMT4Plot3D-Naela Rizqy Arofah-22305144042-022.png}
\begin{eulercomment}
Berikut beberapa gaya lainnya. Kami selalu mematikan bingkai, dan
menggunakan berbagai skema warna untuk plot dan kisi.
\end{eulercomment}
\begin{eulerprompt}
>figure(2,2); ...
>expr="y^3-x^2"; ...
>figure(1);  ...
>  plot3d(expr,<frame,>cp,cpcolor=spectral); ...
>figure(2);  ...
>  plot3d(expr,<frame,>spectral,grid=10,cp=2); ...
>figure(3);  ...
>  plot3d(expr,<frame,>contour,color=gray,nc=5,cp=3,cpcolor=greenred); ...
>figure(4);  ...
>  plot3d(expr,<frame,>hue,grid=10,>transparent,>cp,cpcolor=gray); ...
>figure(0):
\end{eulerprompt}
\eulerimg{17}{images/EMT4Plot3D-Naela Rizqy Arofah-22305144042-023.png}
\begin{eulercomment}
Ada beberapa skema spektral lainnya, yang diberi nomor dari 1 hingga
9. Namun Anda juga dapat menggunakan warna=nilai, di mana nilai :

- spectral: untuk rentang dari biru ke merah\\
- white: untuk rentang yang lebih redup \\
- yellowblue,purplegreen,blueyellow,greenred\\
- blueyellow, greenpurple,yellowblue,redgreen
\end{eulercomment}
\begin{eulerprompt}
>figure(3,3); ...
>for i=1:9;  ...
>  figure(i); plot3d("x^2+y^2",spectral=i,>contour,>cp,<frame,zoom=4);  ...
>end; ...
>figure(0):
\end{eulerprompt}
\eulerimg{17}{images/EMT4Plot3D-Naela Rizqy Arofah-22305144042-024.png}
\begin{eulercomment}
Sumber cahaya dapat diubah dengan l dan tombol kursor selama interaksi
pengguna. Itu juga dapat diatur dengan parameter.

- light: arah\\
- amb: ambient light between 0 and 1

Catatan : program tidak membuat perbedaan antara sisi plot. Tidak ada
bayangan. Untuk ini, Anda memerlukan Povray.
\end{eulercomment}
\begin{eulerprompt}
>plot3d("-x^2-y^2", ...
>  hue=true,light=[0,1,1],amb=0,user=true, ...
>  title="Press l and cursor keys (return to exit)"):
\end{eulerprompt}
\eulerimg{17}{images/EMT4Plot3D-Naela Rizqy Arofah-22305144042-025.png}
\begin{eulercomment}
Parameter warna mengubah warna permukaan. Warna garis level juga bisa
diubah.
\end{eulercomment}
\begin{eulerprompt}
>plot3d("-x^2-y^2",color=rgb(0.2,0.2,0),hue=true,frame=false, ...
>  zoom=3,contourcolor=red,level=-2:0.1:1,dl=0.01):
\end{eulerprompt}
\eulerimg{17}{images/EMT4Plot3D-Naela Rizqy Arofah-22305144042-026.png}
\begin{eulercomment}
The color 0 gives a special rainbow effect.
\end{eulercomment}
\begin{eulerprompt}
>plot3d("x^2/(x^2+y^2+1)",color=0,hue=true,grid=10):
\end{eulerprompt}
\eulerimg{17}{images/EMT4Plot3D-Naela Rizqy Arofah-22305144042-027.png}
\begin{eulercomment}
The surface can also be transparent.
\end{eulercomment}
\begin{eulerprompt}
>plot3d("x^2+y^2",>transparent,grid=10,wirecolor=red):
\end{eulerprompt}
\eulerimg{17}{images/EMT4Plot3D-Naela Rizqy Arofah-22305144042-028.png}
\eulerheading{Plot Implisit}
\begin{eulercomment}
Ada juga plot implisit dalam tiga dimensi. Euler menghasilkan
pemotongan melalui objek. Fitur plot3d mencakup plot implisit. Plot
ini menunjukkan himpunan nol suatu fungsi dalam tiga variabel.
Permukaannya juga bisa transparan.\\
Solusi dari


atex: f(x,y,z) = 0

dapat divisualisasikan dalam potongan yang sejajar dengan bidang xy-,
xz- dan yz.

- implicit=1: cut parallel to the y-z-plane\\
- implicit=2: cut parallel to the x-z-plane\\
- implicit=4: cut parallel to the x-y-plane

Tambahkan nilai berikut, jika Anda mau. Dalam contoh kita memplot :

\end{eulercomment}
\begin{eulerformula}
\[
M = \{ (x,y,z) : x^2+y^3+zy=1 \}
\]
\end{eulerformula}
\begin{eulerprompt}
>plot3d("x^2+y^3+z*y-1",r=5,implicit=3):
\end{eulerprompt}
\eulerimg{17}{images/EMT4Plot3D-Naela Rizqy Arofah-22305144042-029.png}
\begin{eulerprompt}
>c=1; d=1;
>plot3d("((x^2+y^2-c^2)^2+(z^2-1)^2)*((y^2+z^2-c^2)^2+(x^2-1)^2)*((z^2+x^2-c^2)^2+(y^2-1)^2)-d",r=2,<frame,>implicit,>user): 
\end{eulerprompt}
\eulerimg{17}{images/EMT4Plot3D-Naela Rizqy Arofah-22305144042-030.png}
\begin{eulerprompt}
>plot3d("x^2+y^2+4*x*z+z^3",>implicit,r=2,zoom=2.5):
\end{eulerprompt}
\eulerimg{17}{images/EMT4Plot3D-Naela Rizqy Arofah-22305144042-031.png}
\eulerheading{Merencanakan Data 3D}
\begin{eulercomment}
Sama seperti plot2d, plot3d menerima data. Untuk objek 3D, Anda perlu
menyediakan matriks nilai x-, y- dan z, atau tiga fungsi atau ekspresi
fx(x,y), fy(x,y), fz(x,y).

\end{eulercomment}
\begin{eulerformula}
\[
\gamma(t,s) = (x(t,s),y(t,s),z(t,s))
\]
\end{eulerformula}
\begin{eulercomment}
Karena x,y,z adalah matriks, kita asumsikan bahwa (t,s) melewati grid
persegi. Hasilnya, Anda dapat memplot gambar persegi panjang di ruang
angkasa.\\
Anda dapat menggunakan bahasa matriks Euler untuk menghasilkan
koordinat secara efektif.

alam contoh berikut, kita menggunakan vektor nilai t dan vektor kolom
nilai s untuk membuat parameter permukaan bola. Dalam gambar kita
dapat menandai wilayah, dalam kasus kita wilayah kutub.
\end{eulercomment}
\begin{eulerprompt}
>t=linspace(0,2pi,180); s=linspace(-pi/2,pi/2,90)'; ...
>x=cos(s)*cos(t); y=cos(s)*sin(t); z=sin(s); ...
>plot3d(x,y,z,>hue, ...
>color=blue,<frame,grid=[10,20], ...
>values=s,contourcolor=red,level=[90°-24°;90°-22°], ...
>scale=1.4,height=50°):
\end{eulerprompt}
\eulerimg{17}{images/EMT4Plot3D-Naela Rizqy Arofah-22305144042-032.png}
\begin{eulercomment}
Here is an example, which is the graph of a function.
\end{eulercomment}
\begin{eulerprompt}
>t=-1:0.1:1; s=(-1:0.1:1)'; plot3d(t,s,t*s,grid=10):
\end{eulerprompt}
\eulerimg{17}{images/EMT4Plot3D-Naela Rizqy Arofah-22305144042-033.png}
\begin{eulercomment}
Namun, kita bisa membuat berbagai macam permukaan. Berikut adalah
permukaan yang sama sebagai suatu fungsi :

\end{eulercomment}
\begin{eulerformula}
\[
x = y \, z
\]
\end{eulerformula}
\begin{eulerprompt}
>plot3d(t*s,t,s,angle=180°,grid=10):
\end{eulerprompt}
\eulerimg{17}{images/EMT4Plot3D-Naela Rizqy Arofah-22305144042-034.png}
\begin{eulercomment}
Dengan lebih banyak usaha, kita dapat menghasilkan banyak permukaan.


alam contoh berikut kita membuat tampilan bayangan dari bola yang
terdistorsi. Koordinat bola yang biasa adalah

\end{eulercomment}
\begin{eulerformula}
\[
\gamma(t,s) = (\cos(t)\cos(s),\sin(t)\sin(s),\cos(s))
\]
\end{eulerformula}
\begin{eulercomment}
with

\end{eulercomment}
\begin{eulerformula}
\[
0 \le t \le 2\pi, \quad \frac{-\pi}{2} \le s \le \frac{\pi}{2}.
\]
\end{eulerformula}
\begin{eulercomment}
Kami mendistorsi ini dengan sebuah faktor

\end{eulercomment}
\begin{eulerformula}
\[
d(t,s) = \frac{\cos(4t)+\cos(8s)}{4}.
\]
\end{eulerformula}
\begin{eulerprompt}
>t=linspace(0,2pi,320); s=linspace(-pi/2,pi/2,160)'; ...
>d=1+0.2*(cos(4*t)+cos(8*s)); ...
>plot3d(cos(t)*cos(s)*d,sin(t)*cos(s)*d,sin(s)*d,hue=1, ...
>  light=[1,0,1],frame=0,zoom=5):
\end{eulerprompt}
\eulerimg{17}{images/EMT4Plot3D-Naela Rizqy Arofah-22305144042-035.png}
\begin{eulercomment}
Tentu saja, point cloud juga dimungkinkan. Untuk memplot data titik
dalam ruang, kita memerlukan tiga vektor untuk koordinat titik-titik
tersebut.

Gayanya sama seperti di plot2d dengan points=true;
\end{eulercomment}
\begin{eulerprompt}
>n=500;  ...
>  plot3d(normal(1,n),normal(1,n),normal(1,n),points=true,style="."):
\end{eulerprompt}
\eulerimg{17}{images/EMT4Plot3D-Naela Rizqy Arofah-22305144042-036.png}
\begin{eulercomment}
Dimungkinkan juga untuk memplot kurva dalam 3D. Dalam hal ini, lebih
mudah untuk menghitung terlebih dahulu titik-titik kurva. Untuk kurva
pada bidang kita menggunakan barisan koordinat dan parameter
wire=true.
\end{eulercomment}
\begin{eulerprompt}
>t=linspace(0,8pi,500); ...
>plot3d(sin(t),cos(t),t/10,>wire,zoom=3):
\end{eulerprompt}
\eulerimg{17}{images/EMT4Plot3D-Naela Rizqy Arofah-22305144042-037.png}
\begin{eulerprompt}
>t=linspace(0,4pi,1000); plot3d(cos(t),sin(t),t/2pi,>wire, ...
>linewidth=3,wirecolor=blue):
\end{eulerprompt}
\eulerimg{17}{images/EMT4Plot3D-Naela Rizqy Arofah-22305144042-038.png}
\begin{eulerprompt}
>X=cumsum(normal(3,100)); ...
> plot3d(X[1],X[2],X[3],>anaglyph,>wire):
\end{eulerprompt}
\eulerimg{17}{images/EMT4Plot3D-Naela Rizqy Arofah-22305144042-039.png}
\begin{eulercomment}
EMT juga dapat membuat plot dalam mode anaglyph. Untuk melihat plot
seperti itu, Anda memerlukan kacamata berwarna merah/cyan.
\end{eulercomment}
\begin{eulerprompt}
> plot3d("x^2+y^3",>anaglyph,>contour,angle=30°):
\end{eulerprompt}
\eulerimg{17}{images/EMT4Plot3D-Naela Rizqy Arofah-22305144042-040.png}
\begin{eulercomment}
Seringkali skema warna spektral digunakan untuk plot. Ini menekankan
ketinggian fungsinya.
\end{eulercomment}
\begin{eulerprompt}
>plot3d("x^2*y^3-y",>spectral,>contour,zoom=3.2):
\end{eulerprompt}
\eulerimg{17}{images/EMT4Plot3D-Naela Rizqy Arofah-22305144042-041.png}
\begin{eulercomment}
Euler juga dapat memplot permukaan yang diparameterisasi, jika
parameternya adalah nilai x, y, dan z dari gambar kotak persegi
panjang di ruang tersebut.

Untuk demo berikut, kami menyiapkan parameter u- dan v-, dan
menghasilkan koordinat ruang dari parameter tersebut.

\end{eulercomment}
\begin{eulerttcomment}
 t \(\backslash\)le 2\(\backslash\)pi, \(\backslash\)quad \(\backslash\)frac\{-\(\backslash\)pi\}\{2\} \(\backslash\)le s \(\backslash\)le \(\backslash\)frac\{\(\backslash\)pi\}\{2\}.
\end{eulerttcomment}
\begin{eulerprompt}
>u=linspace(-1,1,10); v=linspace(0,2*pi,50)'; ...
>X=(3+u*cos(v/2))*cos(v); Y=(3+u*cos(v/2))*sin(v); Z=u*sin(v/2); ...
>plot3d(X,Y,Z,>anaglyph,<frame,>wire,scale=2.3):
\end{eulerprompt}
\eulerimg{17}{images/EMT4Plot3D-Naela Rizqy Arofah-22305144042-042.png}
\begin{eulercomment}
Here is a more complicated example, which is majestic with red/cyan glasses.
\end{eulercomment}
\begin{eulerprompt}
>u:=linspace(-pi,pi,160); v:=linspace(-pi,pi,400)';  ...
>x:=(4*(1+.25*sin(3*v))+cos(u))*cos(2*v); ...
>y:=(4*(1+.25*sin(3*v))+cos(u))*sin(2*v); ...
> z=sin(u)+2*cos(3*v); ...
>plot3d(x,y,z,frame=0,scale=1.5,hue=1,light=[1,0,-1],zoom=2.8,>anaglyph):
\end{eulerprompt}
\eulerimg{17}{images/EMT4Plot3D-Naela Rizqy Arofah-22305144042-043.png}
\eulerheading{Plot Statistik}
\begin{eulercomment}
Plot batang juga dimungkinkan. Untuk ini, kita harus menyediakan

- x: row vector with n+1 elements\\
- y: column vector with n+1 elements\\
- z: nxn matrix of values.

z can be larger, but only nxn values will be used.

Dalam contoh ini, pertama-tama kita menghitung nilainya. Kemudian kita
sesuaikan x dan y, sehingga vektor-vektornya berpusat pada nilai yang
digunakan.
\end{eulercomment}
\begin{eulerprompt}
>x=-1:0.1:1; y=x'; z=x^2+y^2; ...
>xa=(x|1.1)-0.05; ya=(y_1.1)-0.05; ...
>plot3d(xa,ya,z,bar=true):
\end{eulerprompt}
\eulerimg{17}{images/EMT4Plot3D-Naela Rizqy Arofah-22305144042-044.png}
\begin{eulercomment}
Dimungkinkan untuk membagi plot suatu permukaan menjadi dua bagian
atau lebih.
\end{eulercomment}
\begin{eulerprompt}
>x=-1:0.1:1; y=x'; z=x+y; d=zeros(size(x)); ...
>plot3d(x,y,z,disconnect=2:2:20):
\end{eulerprompt}
\eulerimg{17}{images/EMT4Plot3D-Naela Rizqy Arofah-22305144042-045.png}
\begin{eulercomment}
Jika memuat atau menghasilkan matriks data M dari file dan perlu
memplotnya dalam 3D, Anda dapat menskalakan matriks ke [-1,1] dengan
skala(M), atau menskalakan matriks dengan \textgreater{}zscale. Hal ini dapat
dikombinasikan dengan faktor penskalaan individual yang diterapkan
sebagai tambahan.
\end{eulercomment}
\begin{eulerprompt}
>i=1:20; j=i'; ...
>plot3d(i*j^2+100*normal(20,20),>zscale,scale=[1,1,1.5],angle=-40°,zoom=1.8):
\end{eulerprompt}
\eulerimg{17}{images/EMT4Plot3D-Naela Rizqy Arofah-22305144042-046.png}
\begin{eulerprompt}
>Z=intrandom(5,100,6); v=zeros(5,6); ...
>loop 1 to 5; v[#]=getmultiplicities(1:6,Z[#]); end; ...
>columnsplot3d(v',scols=1:5,ccols=[1:5]):
\end{eulerprompt}
\eulerimg{17}{images/EMT4Plot3D-Naela Rizqy Arofah-22305144042-047.png}
\eulerheading{Permukaan Benda Putar}
\begin{eulerprompt}
>plot2d("(x^2+y^2-1)^3-x^2*y^3",r=1.3, ...
>style="#",color=red,<outline, ...
>level=[-2;0],n=100):
\end{eulerprompt}
\eulerimg{17}{images/EMT4Plot3D-Naela Rizqy Arofah-22305144042-048.png}
\begin{eulerprompt}
>ekspresi &= (x^2+y^2-1)^3-x^2*y^3; $ekspresi
\end{eulerprompt}
\begin{eulerformula}
\[
\left(y^2+x^2-1\right)^3-x^2\,y^3
\]
\end{eulerformula}
\begin{eulercomment}
Kami ingin memutar kurva hati di sekitar sumbu y. Inilah ungkapan yang
mendefinisikan hati:

\end{eulercomment}
\begin{eulerformula}
\[
f(x,y)=(x^2+y^2-1)^3-x^2.y^3.
\]
\end{eulerformula}
\begin{eulercomment}
Next we set

\end{eulercomment}
\begin{eulerformula}
\[
x=r.cos(a),\quad y=r.sin(a).
\]
\end{eulerformula}
\begin{eulerprompt}
>function fr(r,a) &= ekspresi with [x=r*cos(a),y=r*sin(a)] | trigreduce; $fr(r,a)
\end{eulerprompt}
\begin{eulerformula}
\[
\left(r^2-1\right)^3+\frac{\left(\sin \left(5\,a\right)-\sin \left(
 3\,a\right)-2\,\sin a\right)\,r^5}{16}
\]
\end{eulerformula}
\begin{eulercomment}
Hal ini memungkinkan untuk mendefinisikan fungsi numerik, yang
menyelesaikan r, jika a diberikan. Dengan fungsi tersebut kita dapat
memplot jantung yang diputar sebagai permukaan parametrik.
\end{eulercomment}
\begin{eulerprompt}
>function map f(a) := bisect("fr",0,2;a); ...
>t=linspace(-pi/2,pi/2,100); r=f(t);  ...
>s=linspace(pi,2pi,100)'; ...
>plot3d(r*cos(t)*sin(s),r*cos(t)*cos(s),r*sin(t), ...
>>hue,<frame,color=red,zoom=4,amb=0,max=0.7,grid=12,height=50°):
\end{eulerprompt}
\eulerimg{17}{images/EMT4Plot3D-Naela Rizqy Arofah-22305144042-051.png}
\begin{eulercomment}
Berikut ini adalah plot 3D dari gambar di atas yang diputar
mengelilingi sumbu z. Kami mendefinisikan fungsi yang mendeskripsikan
objek.
\end{eulercomment}
\begin{eulerprompt}
>function f(x,y,z) ...
\end{eulerprompt}
\begin{eulerudf}
  r=x^2+y^2;
  return (r+z^2-1)^3-r*z^3;
   endfunction
\end{eulerudf}
\begin{eulerprompt}
>plot3d("f(x,y,z)", ...
>xmin=0,xmax=1.2,ymin=-1.2,ymax=1.2,zmin=-1.2,zmax=1.4, ...
>implicit=1,angle=-30°,zoom=2.5,n=[10,100,60],>anaglyph):
\end{eulerprompt}
\eulerimg{17}{images/EMT4Plot3D-Naela Rizqy Arofah-22305144042-052.png}
\eulerheading{Plot 3D Khusus}
\begin{eulercomment}
Fungsi plot3d bagus untuk dimiliki, tetapi tidak memenuhi semua
kebutuhan. Selain rutinitas yang lebih mendasar, dimungkinkan untuk
mendapatkan plot berbingkai dari objek apa pun yang Anda suka.

Meskipun Euler bukan program 3D, ia dapat menggabungkan beberapa objek
dasar. Kami mencoba memvisualisasikan paraboloid dan garis
singgungnya.
\end{eulercomment}
\begin{eulerprompt}
>function myplot ...
\end{eulerprompt}
\begin{eulerudf}
    y=-1:0.01:1; x=(-1:0.01:1)';
    plot3d(x,y,0.2*(x-0.1)/2,<scale,<frame,>hue, ..
      hues=0.5,>contour,color=orange);
    h=holding(1);
    plot3d(x,y,(x^2+y^2)/2,<scale,<frame,>contour,>hue);
    holding(h);
  endfunction
\end{eulerudf}
\begin{eulercomment}
Now framedplot() provides the frames, and sets the views.
\end{eulercomment}
\begin{eulerprompt}
>framedplot("myplot",[-1,1,-1,1,0,1],height=0,angle=-30°, ...
>  center=[0,0,-0.7],zoom=3):
\end{eulerprompt}
\eulerimg{17}{images/EMT4Plot3D-Naela Rizqy Arofah-22305144042-053.png}
\begin{eulercomment}
Dengan cara yang sama, Anda dapat memplot bidang kontur secara manual.
Perhatikan bahwa plot3d() menyetel jendela ke fullwindow(), secara
default, tetapi plotcontourplane() berasumsi demikian.
\end{eulercomment}
\begin{eulerprompt}
>x=-1:0.02:1.1; y=x'; z=x^2-y^4;
>function myplot (x,y,z) ...
\end{eulerprompt}
\begin{eulerudf}
    zoom(2);
    wi=fullwindow();
    plotcontourplane(x,y,z,level="auto",<scale);
    plot3d(x,y,z,>hue,<scale,>add,color=white,level="thin");
    window(wi);
    reset();
  endfunction
\end{eulerudf}
\begin{eulerprompt}
>myplot(x,y,z):
\end{eulerprompt}
\eulerimg{27}{images/EMT4Plot3D-Naela Rizqy Arofah-22305144042-054.png}
\eulerheading{Animasi}
\begin{eulercomment}
Euler dapat menggunakan frame untuk melakukan pra-komputasi animasi.


alah satu fungsi yang memanfaatkan teknik ini adalah memutar. Itu
dapat mengubah sudut pandang dan menggambar ulang plot 3D. Fungsi ini
memanggil addpage() untuk setiap plot baru. Akhirnya ia menganimasikan
plotnya.


ilakan pelajari sumber rotasi untuk melihat lebih detail.
\end{eulercomment}
\begin{eulerprompt}
>function testplot () := plot3d("x^2+y^3"); ...
>rotate("testplot"); testplot():
\end{eulerprompt}
\eulerimg{27}{images/EMT4Plot3D-Naela Rizqy Arofah-22305144042-055.png}
\eulerheading{Menggambar Povray}
\begin{eulercomment}
Dengan bantuan file Euler povray.e, Euler dapat menghasilkan file
Povray. Hasilnya sangat bagus untuk dilihat.

Anda perlu menginstal Povray (32bit atau 64bit) dari
http://www.povray.org/, dan meletakkan sub-direktori "bin" Povray ke jalur lingkungan, atau mengatur variabel "defaultpovray" dengan jalur lengkap yang mengarah ke "pvengine.exe".


ntarmuka Povray Euler menghasilkan file Povray di direktori home
pengguna, dan memanggil Povray untuk menguraikan file-file ini. Nama
file default adalah current.pov, dan direktori default adalah
eulerhome(), biasanya c:\textbackslash{}Users\textbackslash{}Username\textbackslash{}Euler. Povray menghasilkan
file PNG, yang dapat dimuat oleh Euler ke dalam notebook. Untuk
membersihkan file-file ini, gunakan povclear().


ungsi pov3d memiliki semangat yang sama dengan plot3d. Ini dapat
menghasilkan grafik fungsi f(x,y), atau permukaan dengan koordinat
X,Y,Z dalam matriks, termasuk garis level opsional. Fungsi ini memulai
raytracer secara otomatis, dan memuat adegan ke dalam notebook Euler.


elain pov3d(), ada banyak fungsi yang menghasilkan objek Povray.
Fungsi-fungsi ini mengembalikan string, yang berisi kode Povray untuk
objek. Untuk menggunakan fungsi ini, mulai file Povray dengan
povstart(). Kemudian gunakan writeln(...) untuk menulis objek ke file
adegan. Terakhir, akhiri file dengan povend(). Secara default,
raytracer akan dimulai, dan PNG akan dimasukkan ke dalam notebook
Euler.

Fungsi objek memiliki parameter yang disebut "tampilan", yang
memerlukan string dengan kode Povray untuk tekstur dan penyelesaian
objek. Fungsi povlook() dapat digunakan untuk menghasilkan string ini.
Ini memiliki parameter untuk warna, transparansi, Phong Shading dll.


Perhatikan bahwa alam semesta Povray memiliki sistem koordinat lain.
Antarmuka ini menerjemahkan semua koordinat ke sistem Povray. Jadi
Anda dapat terus berpikir dalam sistem koordinat Euler dengan z
menunjuk vertikal ke atas, dan sumbu x,y,z di tangan kanan. Fungsi
pov3d memiliki semangat yang sama dengan plot3d. Ini dapat
menghasilkan grafik fungsi f(x,y), atau permukaan dengan koordinat
X,Y,Z dalam matriks, termasuk garis level opsional. Fungsi ini memulai
raytracer secara otomatis, dan memuat adegan ke dalam notebook Euler.\\
Anda perlu memuat file povray
\end{eulercomment}
\begin{eulerprompt}
>load povray;
\end{eulerprompt}
\begin{eulercomment}
Pastikan, direktori Povray bin ada di jalurnya. Jika tidak, edit
variabel berikut sehingga berisi jalur ke povray yang dapat
dieksekusi.
\end{eulercomment}
\begin{eulerprompt}
>defaultpovray="C:\(\backslash\)Program Files\(\backslash\)POV-Ray\(\backslash\)v3.7\(\backslash\)bin\(\backslash\)pvengine.exe"
\end{eulerprompt}
\begin{euleroutput}
  C:\(\backslash\)Program Files\(\backslash\)POV-Ray\(\backslash\)v3.7\(\backslash\)bin\(\backslash\)pvengine.exe
\end{euleroutput}
\begin{eulercomment}
Untuk kesan pertama, kami memplot fungsi sederhana. Perintah berikut
menghasilkan file povray di direktori pengguna Anda, dan menjalankan
Povray untuk penelusuran sinar file ini.


ika Anda memulai perintah berikut, GUI Povray akan terbuka,
menjalankan file, dan menutup secara otomatis. Karena alasan keamanan,
Anda akan ditanya apakah Anda ingin mengizinkan file exe dijalankan.
Anda dapat menekan batal untuk menghentikan pertanyaan lebih lanjut.
Anda mungkin harus menekan OK di jendela Povray untuk mengonfirmasi
dialog pengaktifan Povray.
\end{eulercomment}
\begin{eulerprompt}
>plot3d("x^2+y^2",zoom=2):
>pov3d("x^2+y^2",zoom=3);
\end{eulerprompt}
\begin{euleroutput}
  Function pov3d not found.
  Try list ... to find functions!
  Error in:
  pov3d("x^2+y^2",zoom=3); ...
                         ^
\end{euleroutput}
\begin{eulercomment}
Kita dapat membuat fungsinya transparan dan menambahkan penyelesaian
lainnya. Kita juga dapat menambahkan garis level ke plot fungsi.
\end{eulercomment}
\begin{eulerprompt}
>pov3d("x^2+y^3",axiscolor=red,angle=-45°,>anaglyph, ...
>  look=povlook(cyan,0.2),level=-1:0.5:1,zoom=3.8);
\end{eulerprompt}
\begin{euleroutput}
  Function povlook not found.
  Try list ... to find functions!
  Error in:
  ... ed,angle=-45°,>anaglyph,   look=povlook(cyan,0.2),level=-1:0.5 ...
                                                       ^
\end{euleroutput}
\begin{eulercomment}
Terkadang perlu untuk mencegah penskalaan fungsi, dan menskalakan
fungsi secara manual.

Kita memplot himpunan titik pada bidang kompleks, dimana hasil kali
jarak ke 1 dan -1 sama dengan 1.
\end{eulercomment}
\begin{eulerprompt}
>pov3d("((x-1)^2+y^2)*((x+1)^2+y^2)/40",r=2, ...
>  angle=-120°,level=1/40,dlevel=0.005,light=[-1,1,1],height=10°,n=50, ...
>  <fscale,zoom=3.8);
\end{eulerprompt}
\eulerheading{Merencanakan dengan Koordinat}
\begin{eulercomment}
Daripada menggunakan fungsi, kita bisa memplotnya dengan koordinat.
Seperti di plot3d, kita memerlukan tiga matriks untuk mendefinisikan
objek.

Dalam contoh ini kita memutar suatu fungsi di sekitar sumbu z.
\end{eulercomment}
\begin{eulerprompt}
>function f(x) := x^3-x+1; ...
>x=-1:0.01:1; t=linspace(0,2pi,50)'; ...
>Z=x; X=cos(t)*f(x); Y=sin(t)*f(x); ...
>pov3d(X,Y,Z,angle=40°,look=povlook(red,0.1),height=50°,axis=0,zoom=4,light=[10,5,15]);
\end{eulerprompt}
\begin{euleroutput}
  Function povlook not found.
  Try list ... to find functions!
  Error in:
  ... f(x); pov3d(X,Y,Z,angle=40°,look=povlook(red,0.1),height=50°,a ...
                                                       ^
\end{euleroutput}
\begin{eulercomment}
Pada contoh berikut, kita memplot gelombang teredam. Kami menghasilkan
gelombang dengan bahasa matriks Euler.

Kami juga menunjukkan, bagaimana objek tambahan dapat ditambahkan ke
adegan pov3d. Untuk pembuatan objek, lihat contoh berikut. Perhatikan
bahwa plot3d menskalakan plot, sehingga cocok dengan kubus satuan.
\end{eulercomment}
\begin{eulerprompt}
>r=linspace(0,1,80); phi=linspace(0,2pi,80)'; ...
>x=r*cos(phi); y=r*sin(phi); z=exp(-5*r)*cos(8*pi*r)/3;  ...
>pov3d(x,y,z,zoom=6,axis=0,height=30°,add=povsphere([0.5,0,0.25],0.15,povlook(red)), ...
>  w=500,h=300);
\end{eulerprompt}
\begin{euleroutput}
  Function povlook not found.
  Try list ... to find functions!
  Error in:
  ... =30°,add=povsphere([0.5,0,0.25],0.15,povlook(red)),   w=500,h= ...
                                                       ^
\end{euleroutput}
\begin{eulercomment}
Dengan metode peneduh canggih Povray, sangat sedikit titik yang dapat
menghasilkan permukaan yang sangat halus. Hanya pada batas-batas dan
dalam bayangan, triknya mungkin terlihat jelas.

Untuk ini, kita perlu menjumlahkan vektor normal di setiap titik
matriks.
\end{eulercomment}
\begin{eulerprompt}
>Z &= x^2*y^3
\end{eulerprompt}
\begin{euleroutput}
  
                                   2  3
                                  x  y
  
\end{euleroutput}
\begin{eulercomment}
Persamaan permukaannya adalah [x,y,Z]. Kami menghitung dua turunan
dari x dan y dan mengambil perkalian silangnya sebagai normal.
\end{eulercomment}
\begin{eulerprompt}
>dx &= diff([x,y,Z],x); dy &= diff([x,y,Z],y);
\end{eulerprompt}
\begin{eulercomment}
Kami mendefinisikan normal sebagai produk silang dari turunan ini, dan
mendefinisikan fungsi koordinat
\end{eulercomment}
\begin{eulerprompt}
>N &= crossproduct(dx,dy); NX &= N[1]; NY &= N[2]; NZ &= N[3]; N,
\end{eulerprompt}
\begin{euleroutput}
  
                                 3       2  2
                         [- 2 x y , - 3 x  y , 1]
  
\end{euleroutput}
\begin{eulercomment}
We use only 25 points.
\end{eulercomment}
\begin{eulerprompt}
>x=-1:0.5:1; y=x';
>pov3d(x,y,Z(x,y),angle=10°, ...
>  xv=NX(x,y),yv=NY(x,y),zv=NZ(x,y),<shadow);
\end{eulerprompt}
\begin{euleroutput}
  Function pov3d not found.
  Try list ... to find functions!
  Error in:
  ... =10°,   xv=NX(x,y),yv=NY(x,y),zv=NZ(x,y),<shadow); ...
                                                       ^
\end{euleroutput}
\begin{eulercomment}
The following is the Trefoil knot done by A. Busser in Povray. There
is an improved version of this in the examples.

See: Examples\textbackslash{}Trefoil Knot \textbar{} Trefoil Knot

For a good look with not too many points, we add normal vectors here.
We use Maxima to compute the normals for us. First, the three
functions for the coordinates as symbolic expressions.
\end{eulercomment}
\begin{eulerprompt}
>X &= ((4+sin(3*y))+cos(x))*cos(2*y); ...
>Y &= ((4+sin(3*y))+cos(x))*sin(2*y); ...
>Z &= sin(x)+2*cos(3*y);
\end{eulerprompt}
\begin{eulercomment}
Then the two derivative vectors to x and y.
\end{eulercomment}
\begin{eulerprompt}
>dx &= diff([X,Y,Z],x); dy &= diff([X,Y,Z],y);
\end{eulerprompt}
\begin{eulercomment}
Now the normal, which is the cross product of the two derivatives.
\end{eulercomment}
\begin{eulerprompt}
>dn &= crossproduct(dx,dy);
\end{eulerprompt}
\begin{eulercomment}
We now evaluate all this numerically.
\end{eulercomment}
\begin{eulerprompt}
>x:=linspace(-%pi,%pi,40); y:=linspace(-%pi,%pi,100)';
\end{eulerprompt}
\begin{eulercomment}
The normal vectors are evaluations of the symbolic expressions dn[i]
for i=1,2,3. The syntax for this is \&"expression"(parameters). This is
an alternative to the method in the previous example, where we defined
symbolic expressions NX, NY, NZ first.
\end{eulercomment}
\begin{eulerprompt}
>pov3d(X(x,y),Y(x,y),Z(x,y),>anaglyph,axis=0,zoom=5,w=450,h=350, ...
>  <shadow,look=povlook(blue), ...
>  xv=&"dn[1]"(x,y), yv=&"dn[2]"(x,y), zv=&"dn[3]"(x,y));
\end{eulerprompt}
\begin{eulercomment}
We can also generate a grid in 3D.
\end{eulercomment}
\begin{eulerprompt}
>povstart(zoom=4); ...
>x=-1:0.5:1; r=1-(x+1)^2/6; ...
>t=(0°:30°:360°)'; y=r*cos(t); z=r*sin(t); ...
>writeln(povgrid(x,y,z,d=0.02,dballs=0.05)); ...
>povend();
\end{eulerprompt}
\begin{euleroutput}
  exec:
      return _exec(program,param,dir,print,hidden,wait);
  povray:
      exec(program,params,defaulthome);
  Try "trace errors" to inspect local variables after errors.
  povend:
      povray(file,w,h,aspect,exit); 
\end{euleroutput}
\begin{eulercomment}
With povgrid(), curves are possible.
\end{eulercomment}
\begin{eulerprompt}
>povstart(center=[0,0,1],zoom=3.6); ...
>t=linspace(0,2,1000); r=exp(-t); ...
>x=cos(2*pi*10*t)*r; y=sin(2*pi*10*t)*r; z=t; ...
>writeln(povgrid(x,y,z,povlook(red))); ...
>writeAxis(0,2,axis=3); ...
>povend();
\end{eulerprompt}
\eulerheading{Objek Povray}
\begin{eulercomment}
Di atas, kami menggunakan pov3d untuk memplot permukaan. Antarmuka
povray di Euler juga dapat menghasilkan objek Povray. Objek ini
disimpan sebagai string di Euler, dan perlu ditulis ke file Povray.

Kami memulai output dengan povstart().
\end{eulercomment}
\begin{eulerprompt}
>povstart(zoom=4);
\end{eulerprompt}
\begin{eulercomment}
Pertama kita mendefinisikan tiga silinder, dan menyimpannya dalam
string di Euler.

Fungsi povx() dll. hanya mengembalikan vektor [1,0,0], yang dapat
digunakan sebagai gantinya.
\end{eulercomment}
\begin{eulerprompt}
>c1=povcylinder(-povx,povx,1,povlook(red)); ...
>c2=povcylinder(-povy,povy,1,povlook(yellow)); ...
>c3=povcylinder(-povz,povz,1,povlook(blue)); ...
\end{eulerprompt}
\begin{eulercomment}
String tersebut berisi kode Povray, yang tidak perlu kita pahami pada
saat itu.

Fungsi povx() dll. hanya mengembalikan vektor [1,0,0], yang dapat
digunakan sebagai gantinya.
\end{eulercomment}
\begin{eulerprompt}
>c2
\end{eulerprompt}
\begin{euleroutput}
  cylinder \{ <0,0,-1>, <0,0,1>, 1
   texture \{ pigment \{ color rgb <0.941176,0.941176,0.392157> \}  \} 
   finish \{ ambient 0.2 \} 
   \}
\end{euleroutput}
\begin{eulercomment}
As you see, we added texture to the objects in three different colors.

Hal ini dilakukan oleh povlook(), yang mengembalikan string dengan
kode Povray yang relevan. Kita dapat menggunakan warna default Euler,
atau menentukan warna kita sendiri. Kita juga dapat menambahkan
transparansi, atau mengubah cahaya sekitar.
\end{eulercomment}
\begin{eulerprompt}
>povlook(rgb(0.1,0.2,0.3),0.1,0.5)
\end{eulerprompt}
\begin{euleroutput}
   texture \{ pigment \{ color rgbf <0.101961,0.2,0.301961,0.1> \}  \} 
   finish \{ ambient 0.5 \} 
  
\end{euleroutput}
\begin{eulercomment}
Sekarang kita mendefinisikan objek persimpangan, dan menulis hasilnya
ke file.\\
i dilakukan oleh povlook(), yang mengembalikan string dengan kode
Povray yang relevan. Kita dapat menggunakan warna default Euler, atau
menentukan warna kita sendiri. Kita juga dapat menambahkan
transparansi, atau mengubah cahaya sekitar.
\end{eulercomment}
\begin{eulerprompt}
>writeln(povintersection([c1,c2,c3]));
\end{eulerprompt}
\begin{eulercomment}
Persimpangan tiga silinder sulit untuk divisualisasikan jika Anda
belum pernah melihatnya sebelumnya.
\end{eulercomment}
\begin{eulerprompt}
>povend;
\end{eulerprompt}
\begin{eulercomment}
Fungsi berikut menghasilkan fraktal secara rekursif.

Fungsi pertama menunjukkan bagaimana Euler menangani objek Povray
sederhana. Fungsi povbox() mengembalikan string, yang berisi koordinat
kotak, tekstur, dan hasil akhir.
\end{eulercomment}
\begin{eulerprompt}
>function onebox(x,y,z,d) := povbox([x,y,z],[x+d,y+d,z+d],povlook());
>function fractal (x,y,z,h,n) ...
\end{eulerprompt}
\begin{eulerudf}
   if n==1 then writeln(onebox(x,y,z,h));
   else
     h=h/3;
     fractal(x,y,z,h,n-1);
     fractal(x+2*h,y,z,h,n-1);
     fractal(x,y+2*h,z,h,n-1);
     fractal(x,y,z+2*h,h,n-1);
     fractal(x+2*h,y+2*h,z,h,n-1);
     fractal(x+2*h,y,z+2*h,h,n-1);
     fractal(x,y+2*h,z+2*h,h,n-1);
     fractal(x+2*h,y+2*h,z+2*h,h,n-1);
     fractal(x+h,y+h,z+h,h,n-1);
   endif;
  endfunction
\end{eulerudf}
\begin{eulerprompt}
>povstart(fade=10,<shadow);
>fractal(-1,-1,-1,2,4);
>povend();
\end{eulerprompt}
\begin{eulercomment}
Perbedaan memungkinkan pemisahan satu objek dari objek lainnya.
Seperti persimpangan, ada bagian dari objek CSG di Povray.
\end{eulercomment}
\begin{eulerprompt}
>povstart(light=[5,-5,5],fade=10);
\end{eulerprompt}
\begin{eulercomment}
For this demonstration, we define an object in Povray, instead of
using a string in Euler. Definitions are written to the file
immediately.

A box coordinate of -1 just means [-1,-1,-1].
\end{eulercomment}
\begin{eulerprompt}
>povdefine("mycube",povbox(-1,1));
\end{eulerprompt}
\begin{eulercomment}
We can use this object in povobject(), which returns a string as
usual.
\end{eulercomment}
\begin{eulerprompt}
>c1=povobject("mycube",povlook(red));
\end{eulerprompt}
\begin{eulercomment}
We generate a second cube, and rotate and scale it a bit.
\end{eulercomment}
\begin{eulerprompt}
>c2=povobject("mycube",povlook(yellow),translate=[1,1,1], ...
>  rotate=xrotate(10°)+yrotate(10°), scale=1.2);
\end{eulerprompt}
\begin{eulercomment}
Then we take the difference of the two objects.
\end{eulercomment}
\begin{eulerprompt}
>writeln(povdifference(c1,c2));
\end{eulerprompt}
\begin{eulercomment}
Now add three axes.
\end{eulercomment}
\begin{eulerprompt}
>writeAxis(-1.2,1.2,axis=1); ...
>writeAxis(-1.2,1.2,axis=2); ...
>writeAxis(-1.2,1.2,axis=4); ...
>povend();
\end{eulerprompt}
\eulerheading{Fungsi Implisit}
\begin{eulercomment}
Povray dapat memplot himpunan di mana f(x,y,z)=0, seperti parameter
implisit di plot3d. Namun hasilnya terlihat jauh lebih baik.


intaks untuk fungsinya sedikit berbeda. Anda tidak dapat menggunakan
keluaran ekspresi Maxima atau Euler.


Latex:((x\textasciicircum{}2+y\textasciicircum{}2-c\textasciicircum{}2)\textasciicircum{}2+(z\textasciicircum{}2-1)\textasciicircum{}2)*((y\textasciicircum{}2+z\textasciicircum{}2-c\textasciicircum{}2)\textasciicircum{}2+(x\textasciicircum{}2-1)\textasciicircum{}2)*((z\textasciicircum{}2+x\textasciicircum{}2-c\textasciicircum{}2)\textasciicircum{}2+(y\textasciicircum{}2-1)\textasciicircum{}2)=d
\end{eulercomment}
\begin{eulerprompt}
>povstart(angle=70°,height=50°,zoom=4);
>c=0.1; d=0.1; ...
>writeln(povsurface("(pow(pow(x,2)+pow(y,2)-pow(c,2),2)+pow(pow(z,2)-1,2))*(pow(pow(y,2)+pow(z,2)-pow(c,2),2)+pow(pow(x,2)-1,2))*(pow(pow(z,2)+pow(x,2)-pow(c,2),2)+pow(pow(y,2)-1,2))-d",povlook(red))); ...
>povend();
\end{eulerprompt}
\begin{euleroutput}
  Error : Povray error!
  
  Error generated by error() command
  
  povray:
      error("Povray error!");
  Try "trace errors" to inspect local variables after errors.
  povend:
      povray(file,w,h,aspect,exit); 
\end{euleroutput}
\begin{eulerprompt}
>povstart(angle=25°,height=10°); 
>writeln(povsurface("pow(x,2)+pow(y,2)*pow(z,2)-1",povlook(blue),povbox(-2,2,"")));
>povend();
>povstart(angle=70°,height=50°,zoom=4);
\end{eulerprompt}
\begin{eulercomment}
Create the implicit surface. Note the different syntax in the
expression.
\end{eulercomment}
\begin{eulerprompt}
>writeln(povsurface("pow(x,2)*y-pow(y,3)-pow(z,2)",povlook(green))); ...
>writeAxes(); ...
>povend();
\end{eulerprompt}
\eulerheading{Objek Jaring}
\begin{eulercomment}
Dalam contoh ini, kami menunjukkan cara membuat objek mesh, dan
menggambarnya dengan informasi tambahan.

Kita ingin memaksimalkan xy pada kondisi x+y=1 dan mendemonstrasikan
sentuhan tangensial garis datar.
\end{eulercomment}
\begin{eulerprompt}
>povstart(angle=-10°,center=[0.5,0.5,0.5],zoom=7);
\end{eulerprompt}
\begin{eulercomment}
We cannot store the object in a string as before, since is too large. So we define the object in a Povray file using
#declare. The function povtriangle() does this automatically. It can accept normal vectors just like pov3d().

The following defines the mesh object, and writes it immediately into the file.
\end{eulercomment}
\begin{eulerprompt}
>x=0:0.02:1; y=x'; z=x*y; vx=-y; vy=-x; vz=1;
>mesh=povtriangles(x,y,z,"",vx,vy,vz);
\end{eulerprompt}
\begin{eulercomment}
Now we define two discs, which will be intersected with the surface.
\end{eulercomment}
\begin{eulerprompt}
>cl=povdisc([0.5,0.5,0],[1,1,0],2); ...
>ll=povdisc([0,0,1/4],[0,0,1],2);
\end{eulerprompt}
\begin{eulercomment}
Write the surface minus the two discs.
\end{eulercomment}
\begin{eulerprompt}
>writeln(povdifference(mesh,povunion([cl,ll]),povlook(green)));
\end{eulerprompt}
\begin{eulercomment}
Write the two intersections.
\end{eulercomment}
\begin{eulerprompt}
>writeln(povintersection([mesh,cl],povlook(red))); ...
>writeln(povintersection([mesh,ll],povlook(gray)));
\end{eulerprompt}
\begin{eulercomment}
Write a point at the maximum.
\end{eulercomment}
\begin{eulerprompt}
>writeln(povpoint([1/2,1/2,1/4],povlook(gray),size=2*defaultpointsize));
\end{eulerprompt}
\begin{eulercomment}
Add axes and finish.
\end{eulercomment}
\begin{eulerprompt}
>writeAxes(0,1,0,1,0,1,d=0.015); ...
>povend();
\end{eulerprompt}
\eulerheading{Anaglyphs di Povray}
\begin{eulercomment}
Untuk menghasilkan anaglyph untuk kacamata merah/cyan, Povray harus
dijalankan dua kali dari posisi kamera berbeda. Ini menghasilkan dua
file Povray dan dua file PNG, yang dimuat dengan fungsi
loadanaglyph().

entu saja, Anda memerlukan kacamata berwarna merah/cyan untuk melihat
contoh berikut dengan benar.

ungsi pov3d() memiliki saklar sederhana untuk menghasilkan anaglyph.
\end{eulercomment}
\begin{eulerprompt}
>pov3d("-exp(-x^2-y^2)/2",r=2,height=45°,>anaglyph, ...
>  center=[0,0,0.5],zoom=3.5);
\end{eulerprompt}
\begin{eulercomment}
If you create a scene with objects, you need to put the generation of
the scene into a function, and run it twice with different values for
the anaglyph parameter.
\end{eulercomment}
\begin{eulerprompt}
>function myscene ...
\end{eulerprompt}
\begin{eulerudf}
    s=povsphere(povc,1);
    cl=povcylinder(-povz,povz,0.5);
    clx=povobject(cl,rotate=xrotate(90°));
    cly=povobject(cl,rotate=yrotate(90°));
    c=povbox([-1,-1,0],1);
    un=povunion([cl,clx,cly,c]);
    obj=povdifference(s,un,povlook(red));
    writeln(obj);
    writeAxes();
  endfunction
\end{eulerudf}
\begin{eulercomment}
The function povanaglyph() does all this. The parameters are like in
povstart() and povend() combined.
\end{eulercomment}
\begin{eulerprompt}
>povanaglyph("myscene",zoom=4.5);
\end{eulerprompt}
\eulerheading{Mendefinisikan Objek sendiri}
\begin{eulercomment}
ntarmuka povray Euler berisi banyak objek. Namun Anda tidak dibatasi
pada hal ini. Anda dapat membuat objek sendiri, yang menggabungkan
objek lain, atau merupakan objek yang benar-benar baru.

Kami mendemonstrasikan torus. Perintah Povray untuk ini adalah
"torus". Jadi kami mengembalikan string dengan perintah ini dan
parameternya. Perhatikan bahwa torus selalu berpusat pada titik asal.
\end{eulercomment}
\begin{eulerprompt}
>function povdonat (r1,r2,look="") ...
\end{eulerprompt}
\begin{eulerudf}
    return "torus \{"+r1+","+r2+look+"\}";
  endfunction
\end{eulerudf}
\begin{eulercomment}
Here is our first torus.
\end{eulercomment}
\begin{eulerprompt}
>t1=povdonat(0.8,0.2)
\end{eulerprompt}
\begin{euleroutput}
  torus \{0.8,0.2\}
\end{euleroutput}
\begin{eulercomment}
Let us use this object to create a second torus, translated and
rotated.
\end{eulercomment}
\begin{eulerprompt}
>t2=povobject(t1,rotate=xrotate(90°),translate=[0.8,0,0])
\end{eulerprompt}
\begin{euleroutput}
  object \{ torus \{0.8,0.2\}
   rotate 90 *x 
   translate <0.8,0,0>
   \}
\end{euleroutput}
\begin{eulercomment}
Now we place these objects into a scene. For the look, we use Phong
Shading.
\end{eulercomment}
\begin{eulerprompt}
>povstart(center=[0.4,0,0],angle=0°,zoom=3.8,aspect=1.5); ...
>writeln(povobject(t1,povlook(green,phong=1))); ...
>writeln(povobject(t2,povlook(green,phong=1))); ...
\end{eulerprompt}
\begin{eulerttcomment}
 >povend();
\end{eulerttcomment}
\begin{eulercomment}
calls the Povray program. However, in case of errors, it does not
display the error. You should therefore use

\end{eulercomment}
\begin{eulerttcomment}
 >povend(<exit);
\end{eulerttcomment}
\begin{eulercomment}

if anything did not work. This will leave the Povray window open.
\end{eulercomment}
\begin{eulerprompt}
>povend(h=320,w=480);
\end{eulerprompt}
\begin{euleroutput}
  Function povstart not found.
  Try list ... to find functions!
  Error in:
  povstart(center=[0.4,0,0],angle=0°,zoom=3.8,aspect=1.5); writeln(povobject(t1,povlook(green,phong=1))); writeln(povobj ...
                                                         ^
\end{euleroutput}
\begin{eulercomment}
Here is a more elaborate example. We solve

\end{eulercomment}
\begin{eulerformula}
\[
Ax \le b, \quad x \ge 0, \quad c.x \to \text{Max.}
\]
\end{eulerformula}
\begin{eulercomment}
and show the feasible points and the optimum in a 3D plot.
\end{eulercomment}
\begin{eulerprompt}
>A=[10,8,4;5,6,8;6,3,2;9,5,6];
>b=[10,10,10,10]';
>c=[1,1,1];
\end{eulerprompt}
\begin{eulercomment}
First, let us check, if this example has a solution at all.
\end{eulercomment}
\begin{eulerprompt}
>x=simplex(A,b,c,>max,>check)'
\end{eulerprompt}
\begin{euleroutput}
  [0,  1,  0.5]
\end{euleroutput}
\begin{eulercomment}
Yes, it has.

Next we define two objects. The first is the plane

\end{eulercomment}
\begin{eulerformula}
\[
a \cdot x \le b
\]
\end{eulerformula}
\begin{eulerprompt}
>function oneplane (a,b,look="") ...
\end{eulerprompt}
\begin{eulerudf}
    return povplane(a,b,look)
  endfunction
\end{eulerudf}
\begin{eulercomment}
Then we define the intersection of all half spaces and a cube.
\end{eulercomment}
\begin{eulerprompt}
>function adm (A, b, r, look="") ...
\end{eulerprompt}
\begin{eulerudf}
    ol=[];
    loop 1 to rows(A); ol=ol|oneplane(A[#],b[#]); end;
    ol=ol|povbox([0,0,0],[r,r,r]);
    return povintersection(ol,look);
  endfunction
\end{eulerudf}
\begin{eulercomment}
We can now plot the scene.
\end{eulercomment}
\begin{eulerprompt}
>povstart(angle=120°,center=[0.5,0.5,0.5],zoom=3.5); ...
>writeln(adm(A,b,2,povlook(green,0.4))); ...
>writeAxes(0,1.3,0,1.6,0,1.5); ...
\end{eulerprompt}
\begin{eulercomment}
The following is a circle around the optimum.
\end{eulercomment}
\begin{eulerprompt}
>writeln(povintersection([povsphere(x,0.5),povplane(c,c.x')], ...
>  povlook(red,0.9)));
\end{eulerprompt}
\begin{eulercomment}
And an error in the direction of the optimum.
\end{eulercomment}
\begin{eulerprompt}
>writeln(povarrow(x,c*0.5,povlook(red)));
\end{eulerprompt}
\begin{eulercomment}
We add text to the screen. Text is just a 3D object. We need to place
and turn it according to our view.
\end{eulercomment}
\begin{eulerprompt}
>writeln(povtext("Linear Problem",[0,0.2,1.3],size=0.05,rotate=5°)); ...
>povend();
\end{eulerprompt}
\eulerheading{Contoh Lainnya}
\begin{eulercomment}
Anda dapat menemukan beberapa contoh Povray di Euler di file berikut.


ee: Examples/Dandelin Spheres\\
See: Examples/Donat Math\\
See: Examples/Trefoil Knot\\
See: Examples/Optimization by Affine Scaling

\begin{eulercomment}
\eulerheading{Contoh Soal}
\begin{eulercomment}
1. \\
\end{eulercomment}
\begin{eulerformula}
\[
2x^3+2y^2
\]
\end{eulerformula}
\begin{eulercomment}
Fungsi ini ingin dilihat dengan distance 7, zoom 4 kali, dengan sudut
pandang dari pernglihat sebesar pi/9 dan ketinggian 1.
\end{eulercomment}
\begin{eulerprompt}
>plot3d("2*x^3+2*y^2",distance=7,zoom=4,angle=pi/9,height=1):
\end{eulerprompt}
\eulerimg{27}{images/EMT4Plot3D-Naela Rizqy Arofah-22305144042-056.png}
\begin{eulercomment}
2. Dengan fungsi dari nomer satu tunjukkan beberapa tampilan yang
berbeda baik itu dari warna maupun sudut penglihatan
\end{eulercomment}
\begin{eulerprompt}
>plot3d("2*x^3+2*y^2",angle=pi/2,>contour,>spectral,>hue):
\end{eulerprompt}
\eulerimg{27}{images/EMT4Plot3D-Naela Rizqy Arofah-22305144042-057.png}
\begin{eulerprompt}
>plot3d("2*x^3+2*y^2",angle=pi/4,>contour,color=blue,>hue):
\end{eulerprompt}
\eulerimg{27}{images/EMT4Plot3D-Naela Rizqy Arofah-22305144042-058.png}
\begin{eulerprompt}
>plot3d("2*x^3+2*y^2",angle=pi/5,color=red,>hue):
\end{eulerprompt}
\eulerimg{27}{images/EMT4Plot3D-Naela Rizqy Arofah-22305144042-059.png}

\begin{eulercomment}
\eulersubheading{3.9 Menggambar Titik pada ruang 3D}
\begin{eulercomment}
Menggambar titik pada ruang tiga dimensi (3D) merupakan proses
visualisasi titik dalam sistem koorditat tiga dimensi (3D). Ruang tiga
dimensi (3D) memiliki tiga sumbu utama : sumbu x,y, dan z yang
bersilangan tegak lurus satu sam lain. Titik dalam ruang tiga dimensi
(3D) dapat didefinisikan dengan tiga koordinat tersebut.

Untuk menggambar/memplot data titik dalam ruang,kita membutuhkan tiga\\
vektor untuk koordinat titik-titik tersebut.

Gayanya sama seperti gaya di plot2D, yaitu dengan points=true;
\end{eulercomment}
\begin{eulerprompt}
>n=400;...
>plot3d(normal(1,n),normal(1,n),normal(1,n),points=true,style="."):
\end{eulerprompt}
\eulerimg{27}{images/Plot3D-subtopik 9,10-Naela Rizqy Arofah-001.png}
\begin{eulercomment}
Penjelasan :

1. n=...; digunakan untuk menginisialisasi variabel n dengan nilai ...
variabel ini akan digunakan sebagai penentuan jumlah titik yang akan
digunakan dalam plot 3D.

2. normal(1,n), Fungsi normal digunakan untuk menghasilkan milai-nilai
acak yang terdistribusi secara normal (gaussian) dengan rata-rata 1
dan deviasi standar 1. hal ini untuk mendapatkan koordinat x,y, dan z
untuk plot 3D.

3. plot3d(...)merupakan fungsi yang digunakan untuk plot 3D dengan
parameter-parameter, antara lain yaitu :\\
- points=true (mengatur agar titik-titik data ditampilkan dalam plot
tersebut).\\
- style (untuk mengatur gaya titik yang akan ditampilkan dalam plot).

Berikut akan ditunjukkan contoh lainnya:
\end{eulercomment}
\begin{eulerprompt}
>n=10000;...
>plot3d(normal(1,n),normal(1,n),normal(1,n),points=true,style="."):
\end{eulerprompt}
\eulerimg{27}{images/Plot3D-subtopik 9,10-Naela Rizqy Arofah-002.png}
\begin{eulerprompt}
>n=200;...
>plot3d(normal(1,n),normal(1,n),normal(1,n),points=true,style=","):
\end{eulerprompt}
\eulerimg{27}{images/Plot3D-subtopik 9,10-Naela Rizqy Arofah-003.png}
\begin{eulercomment}
Pada contok berikut akan ditunjukkan titik yang ditentukan dengan
vektor baris x,y,z
\end{eulercomment}
\begin{eulerprompt}
>x=[100,200,400]; y=[150,300,450]; z=[300,500,700]; plot3d(x,y,z,points=true,style=","):
\end{eulerprompt}
\eulerimg{27}{images/Plot3D-subtopik 9,10-Naela Rizqy Arofah-004.png}
\begin{eulercomment}
Dimungkinkan juga untuk memplot kurva dalam 3D. Dalam hal ini, lebih
mudah untuk menghitung terlebih dahulu titik-titik kurva. Untuk kurva
pada bidang kita menggunakan barisan koordinat dan parameter
wire=true.
\end{eulercomment}
\begin{eulerprompt}
>t=linspace(0,8pi,500)
\end{eulerprompt}
\begin{euleroutput}
  [0,  0.0502655,  0.100531,  0.150796,  0.201062,  0.251327,  0.301593,
  0.351858,  0.402124,  0.452389,  0.502655,  0.55292,  0.603186,
  0.653451,  0.703717,  0.753982,  0.804248,  0.854513,  0.904779,
  0.955044,  1.00531,  1.05558,  1.10584,  1.15611,  1.20637,  1.25664,
  1.3069,  1.35717,  1.40743,  1.4577,  1.50796,  1.55823,  1.6085,
  1.65876,  1.70903,  1.75929,  1.80956,  1.85982,  1.91009,  1.96035,
  2.01062,  2.06088,  2.11115,  2.16142,  2.21168,  2.26195,  2.31221,
  2.36248,  2.41274,  2.46301,  2.51327,  2.56354,  2.61381,  2.66407,
  2.71434,  2.7646,  2.81487,  2.86513,  2.9154,  2.96566,  3.01593,
  3.06619,  3.11646,  3.16673,  3.21699,  3.26726,  3.31752,  3.36779,
  3.41805,  3.46832,  3.51858,  3.56885,  3.61911,  3.66938,  3.71965,
  3.76991,  3.82018,  3.87044,  3.92071,  3.97097,  4.02124,  4.0715,
  4.12177,  4.17204,  4.2223,  4.27257,  4.32283,  4.3731,  4.42336,
  4.47363,  4.52389,  4.57416,  4.62442,  4.67469,  4.72496,  4.77522,
  4.82549,  4.87575,  4.92602,  4.97628,  5.02655,  5.07681,  5.12708,
  5.17734,  5.22761,  5.27788,  5.32814,  5.37841,  5.42867,  5.47894,
  5.5292,  5.57947,  5.62973,  5.68,  5.73027,  5.78053,  5.8308,
  5.88106,  5.93133,  5.98159,  6.03186,  6.08212,  6.13239,  6.18265,
  6.23292,  6.28319,  6.33345,  6.38372,  6.43398,  6.48425,  6.53451,
  6.58478,  6.63504,  6.68531,  6.73557,  6.78584,  6.83611,  6.88637,
   ... ]
\end{euleroutput}
\begin{eulerprompt}
>t=linspace(0,8pi,500);...
>plot3d(sin(t),cos(t),t/10,>wire,zoom=2):
\end{eulerprompt}
\eulerimg{27}{images/Plot3D-subtopik 9,10-Naela Rizqy Arofah-005.png}
\begin{eulercomment}
Contoh lainnya :\\
pertama kita tentukan terlebih dahulu titik-titik kurvanya
\end{eulercomment}
\begin{eulerprompt}
>t=linspace(0,6pi,700)
\end{eulerprompt}
\begin{euleroutput}
  [0,  0.0269279,  0.0538559,  0.0807838,  0.107712,  0.13464,  0.161568,
  0.188496,  0.215423,  0.242351,  0.269279,  0.296207,  0.323135,
  0.350063,  0.376991,  0.403919,  0.430847,  0.457775,  0.484703,
  0.511631,  0.538559,  0.565487,  0.592415,  0.619343,  0.64627,
  0.673198,  0.700126,  0.727054,  0.753982,  0.78091,  0.807838,
  0.834766,  0.861694,  0.888622,  0.91555,  0.942478,  0.969406,
  0.996334,  1.02326,  1.05019,  1.07712,  1.10405,  1.13097,  1.1579,
  1.18483,  1.21176,  1.23869,  1.26561,  1.29254,  1.31947,  1.3464,
  1.37332,  1.40025,  1.42718,  1.45411,  1.48104,  1.50796,  1.53489,
  1.56182,  1.58875,  1.61568,  1.6426,  1.66953,  1.69646,  1.72339,
  1.75032,  1.77724,  1.80417,  1.8311,  1.85803,  1.88496,  1.91188,
  1.93881,  1.96574,  1.99267,  2.0196,  2.04652,  2.07345,  2.10038,
  2.12731,  2.15423,  2.18116,  2.20809,  2.23502,  2.26195,  2.28887,
  2.3158,  2.34273,  2.36966,  2.39659,  2.42351,  2.45044,  2.47737,
  2.5043,  2.53123,  2.55815,  2.58508,  2.61201,  2.63894,  2.66587,
  2.69279,  2.71972,  2.74665,  2.77358,  2.80051,  2.82743,  2.85436,
  2.88129,  2.90822,  2.93515,  2.96207,  2.989,  3.01593,  3.04286,
  3.06978,  3.09671,  3.12364,  3.15057,  3.1775,  3.20442,  3.23135,
  3.25828,  3.28521,  3.31214,  3.33906,  3.36599,  3.39292,  3.41985,
  3.44678,  3.4737,  3.50063,  3.52756,  3.55449,  3.58142,  3.60834,
   ... ]
\end{euleroutput}
\begin{eulerprompt}
>t=linspace(0,6pi,700);...
>plot3d(sin(t),cos(t),t/3pi,>wire,zoom=2):
\end{eulerprompt}
\eulerimg{27}{images/Plot3D-subtopik 9,10-Naela Rizqy Arofah-006.png}
\begin{eulerprompt}
>X=cumsum(normal(7,70)); plot3d(X[1],X[2],X[3],>anaglyph,>wire,zoom=2):
\end{eulerprompt}
\eulerimg{27}{images/Plot3D-subtopik 9,10-Naela Rizqy Arofah-007.png}
\eulersubheading{ 3.10 Mengatur tampilan, warna, dan sudut pandang gambar}
\begin{eulercomment}
Ada beberapa parameter untuk menskalakan fungsi atau mengubah tampilan
grafik.

- fscale : menskalakan ke nilai fungsi (default is \textless{}fscale)\\
- scale : angka atau vektor 1x2 untuk menskalakan ke arah x dan y\\
- frame : jenis bingkai (default 1)
\end{eulercomment}
\begin{eulerprompt}
>plot3d("exp(-(x^2+y^2)/5)",r=10,n=80,fscale=4,scale=1.2,frame=3,>user):
\end{eulerprompt}
\eulerimg{27}{images/Plot3D-subtopik 9,10-Naela Rizqy Arofah-008.png}
\begin{eulercomment}
contoh lainnya :
\end{eulercomment}
\begin{eulerprompt}
>plot3d("x^2+y^2",r=2,scale=[1.5,1,0.5]):
\end{eulerprompt}
\eulerimg{27}{images/Plot3D-subtopik 9,10-Naela Rizqy Arofah-009.png}
\begin{eulercomment}
Untuk mengatur tampilan pada bangun ruang 3D di euler yaitu dengan
berbagai cara, anta lain yaitu :

- distance : jarak pandang ke plot\\
- zoom : zoom hasilnya\\
- angle : sudut terhadap sumbu y\\
- height : ketingiian pandangan dalam radian

Nilai default dapat diperiksa atau diubah dengan fungsi view(). Ini
mengembalikan parameter dalam urutan di atas.
\end{eulercomment}
\begin{eulerprompt}
>view
\end{eulerprompt}
\begin{euleroutput}
  [5,  2.6,  2,  0.4]
\end{euleroutput}
\begin{eulercomment}
Misalkan kita ambil contoh

\end{eulercomment}
\begin{eulerformula}
\[
x^3+y^2
\]
\end{eulerformula}
\begin{eulercomment}
Lalu kita akan membuat distance 4, zoom 2, dengan sudut pi/3, dan
height 0, maka dapat kita tuliskan seperti :
\end{eulercomment}
\begin{eulerprompt}
>plot3d("x^3+y^2",distance=4,zoom=2,angle=pi/3,height=0):
\end{eulerprompt}
\eulerimg{27}{images/Plot3D-subtopik 9,10-Naela Rizqy Arofah-010.png}
\begin{eulercomment}
Contoh kedua :

\end{eulercomment}
\begin{eulerformula}
\[
2x^3+2y^2
\]
\end{eulerformula}
\begin{eulercomment}
dengan distance 7,zoom=4, dengan sudut pandang pi/9,height 1\\
maka akan terlihat seperti :
\end{eulercomment}
\begin{eulerprompt}
>plot3d("2*x^3+2*y^2",distance=7,zoom=4,angle=pi/9,height=1):
\end{eulerprompt}
\eulerimg{27}{images/Plot3D-subtopik 9,10-Naela Rizqy Arofah-011.png}
\begin{eulercomment}
Plot akan selalu terlihat berda di tengah-tengah kubus plot. Agar
dapat terlihat berbeda kita dapat memindahkan bagian tengahnya dengan
parameter tengah "center"
\end{eulercomment}
\begin{eulerprompt}
>plot3d("2*x^3+2*y^2",distance=7,zoom=4,angle=pi/9,height=1,...
> center=[0.4,1,0]):
\end{eulerprompt}
\eulerimg{27}{images/Plot3D-subtopik 9,10-Naela Rizqy Arofah-012.png}
\begin{eulercomment}
Plotnya diskalakan agar sesuai dengan unit kubus unuk dilihat. Jadi
tidak perlu mengubah jarak atau zoom tergantung ukuran plot.Namun
labelnya mengacu pada ukuran sebenarnya.

Jika ingin mematikannya dapat menggunakam scale=false, kita harus
berhati-hati agar plot tetap masuk ke dalam jendela plotting, dengan
mengubah jarak pandang atau zoom, dan memindahkan bagian tengah.
\end{eulercomment}
\begin{eulerprompt}
>plot3d("5*exp(-x^4-y)",r=2,<fscale,<scale;...
>distance=20,height=50°,center=[0,0,-2],frame=3,zoom=2):
\end{eulerprompt}
\eulerimg{27}{images/Plot3D-subtopik 9,10-Naela Rizqy Arofah-013.png}
\begin{eulercomment}
Selain itu cara untuk mengatur tampilan agar terlihat dari sisi lain\\
kita dapat menggunakan parameter memutar fungsi atau rotate.

- rotate=1 : untuk memutar pada sumbu x\\
- rotate=2 : untuk memutar pada sumbu z
\end{eulercomment}
\begin{eulerprompt}
>plot3d("2*x",a=1,b=3,rotate=2,grid=5):
\end{eulerprompt}
\eulerimg{27}{images/Plot3D-subtopik 9,10-Naela Rizqy Arofah-014.png}
\begin{eulercomment}
Selanjutnya untuk mengubah rona suatu warna atau rona warna spektral.
Euler dapat menggambar ketinggian fungsi pada plot dengan arsiran. Di
semua plot 3D,Euler dapat menghasilkan anaglyph.

- \textgreater{}hue :Mengaktifkan bayangan cahaya.\\
- \textgreater{}contour : Membuat plot garis kontur otomatis pada plot.\\
- \textgreater{}spectral : Membuat warna spektral pada plot\\
- level=...(atau levels): A Vektor nilai garis kontur.\\
- color : mengubah warna selain warna spektral

Sebagai contohnya :

Kita akan menggambarkan\\
\end{eulercomment}
\begin{eulerformula}
\[
sin(x)^2
\]
\end{eulerformula}
\begin{eulercomment}
Dengan menggunakan \textgreater{}hue,\textgreater{}contour,\textgreater{}spectral dan color=... untuk
mengubah warna serta memberi bayangan pada gambar tersebut.
\end{eulercomment}
\begin{eulerprompt}
>plot3d("sin(x)^2",angle=pi/6,>contour,color=green,>hue):
\end{eulerprompt}
\eulerimg{27}{images/Plot3D-subtopik 9,10-Naela Rizqy Arofah-015.png}
\begin{eulercomment}
Spektral untuk plot berikut, kita akan menggunakan skema sprektral
default (\textgreater{}spectral) dan menambah jumlah titik untuk mendapatkan
tampilan yang lebih halus.
\end{eulercomment}
\begin{eulerprompt}
>plot3d("x^3+2*y^2",>spectral,>contour,n=100):
\end{eulerprompt}
\eulerimg{27}{images/Plot3D-subtopik 9,10-Naela Rizqy Arofah-016.png}
\begin{eulercomment}
Selain garis level otomatis, kita juga dapat menetapkan nilai garis
level. Ini akan menghasilkan garis level yang tipis (bukan rentang
level).
\end{eulercomment}
\begin{eulerprompt}
>plot3d("2*x^3+y^2",0,5,0,5,level=-6:0.3:1,color=redgreen):
\end{eulerprompt}
\eulerimg{27}{images/Plot3D-subtopik 9,10-Naela Rizqy Arofah-017.png}
\begin{eulercomment}
Dalam plot berikut, kita menggunakan pita tingkat yang sangat luas
dari -0,1 hingga 1 dan dari 0,9 hingga 1. Ini dimasukkan sebagai
matrks dengan batas tingkat sebagai kolom.

Selain itu, kami melapisi grid dengan 10 interval di setiap arah.\\
dengan menggunakan fungsi :

\end{eulercomment}
\begin{eulerformula}
\[
x+y^2
\]
\end{eulerformula}
\begin{eulerprompt}
>plot3d("x+y^2",level=[-0.1,0.9;0,1],...
>>spectral,angle=60°,grid=10,contourcolor=gray):
\end{eulerprompt}
\eulerimg{27}{images/Plot3D-subtopik 9,10-Naela Rizqy Arofah-018.png}
\begin{eulercomment}
Pada plot berikut, kita akan memplot suatu himpunan, dimana :

\end{eulercomment}
\begin{eulerformula}
\[
x^y-y^x=0
\]
\end{eulerformula}
\begin{eulercomment}
kita akan menggunakan satu garis tipis untuk garis level.
\end{eulercomment}
\begin{eulerprompt}
>plot3d("x^y-y^x",level=0,a=0,b=6,c=0,d=6,contourcolor=green,n=50):
\end{eulerprompt}
\eulerimg{27}{images/Plot3D-subtopik 9,10-Naela Rizqy Arofah-019.png}
\begin{eulercomment}
Dimungkinkan juga untuk menampilkan bidang kontur di bawah plot. Warna
dan jarak ke plot dapat ditentukan. (\textgreater{}cp)
\end{eulercomment}
\begin{eulerprompt}
>plot3d("x^2+y^3",>cp,cpcolor=blue,cpdelta=0.2):
\end{eulerprompt}
\eulerimg{27}{images/Plot3D-subtopik 9,10-Naela Rizqy Arofah-020.png}
\begin{eulercomment}
Sumber cahaya dapat diubah dengan l dan tombol kursor selama interaksi
pengguna. Itu juga dapat diatur dengan parameter.

- light: arah\\
- amb: cahaya sekitar antara 0 and 1

Catatan : program tidak membuat perbedaan antara sisi plot. Tidak ada
bayangan. Untuk ini, Anda memerlukan Povray.
\end{eulercomment}
\begin{eulerprompt}
>plot3d("-x^2-y^2", ...
>hue=true,light=[0,1,1],amb=0,user=true, ...
> title="Press l and cursor keys (return to exit)"):
\end{eulerprompt}
\eulerimg{27}{images/Plot3D-subtopik 9,10-Naela Rizqy Arofah-021.png}
\begin{eulercomment}
Parameter warna mengubah warna permukaan. Warna garis level juga bisa
diubah.
\end{eulercomment}
\begin{eulerprompt}
>plot3d("x^4+y^2",color=rgb(0.2,0.2,0),hue=true,frame=false, ...
>zoom=3,contourcolor=blue,level=-2:0.1:1,dl=0.01):
\end{eulerprompt}
\eulerimg{27}{images/Plot3D-subtopik 9,10-Naela Rizqy Arofah-022.png}
\begin{eulercomment}
Kemudian jika color=0 maka akan memberikan efek warna pelangi. seperti
berikut :
\end{eulercomment}
\begin{eulerprompt}
>plot3d("x^2/(x^4+y^3+5)",color=0,hue=true,grid=10):
\end{eulerprompt}
\eulerimg{27}{images/Plot3D-subtopik 9,10-Naela Rizqy Arofah-023.png}
\begin{eulercomment}
Kita juga dapat menampilkan dalam bentuk transparan, seperti berikut :
\end{eulercomment}
\begin{eulerprompt}
>plot3d("x^2/(x^4+y^3+5)",>transparent,grid=10,wirecolor=blue):
\end{eulerprompt}
\eulerimg{27}{images/Plot3D-subtopik 9,10-Naela Rizqy Arofah-024.png}
\documentclass{article}

\usepackage{eumat}

\chapter{EMT untuk Kalkulus}
\eulersubheading{}
\begin{eulercomment}
Materi Kalkulus mencakup di antaranya:

- Fungsi (fungsi aljabar, trigonometri, eksponensial, logaritma,
komposisi fungsi)\\
- Limit Fungsi,\\
- Turunan Fungsi,\\
- Integral Tak Tentu,\\
- Integral Tentu dan Aplikasinya,\\
- Barisan dan Deret (kekonvergenan barisan dan deret).

EMT (bersama Maxima) dapat digunakan untuk melakukan semua perhitungan
di dalam kalkulus, baik secara numerik maupun analitik (eksak).

\end{eulercomment}
\eulersubheading{Mendefinisikan Fungsi}
\begin{eulercomment}
Terdapat beberapa cara mendefinisikan fungsi pada EMT, yakni:

- Menggunakan format nama\_fungsi := rumus fungsi (untuk fungsi
numerik),\\
- Menggunakan format nama\_fungsi \&= rumus fungsi (untuk fungsi
simbolik, namun dapat dihitung secara numerik),\\
- Menggunakan format nama\_fungsi \&\&= rumus fungsi (untuk fungsi
simbolik murni, tidak dapat dihitung langsung),\\
- Fungsi sebagai program EMT.

Setiap format harus diawali dengan perintah function (bukan sebagai
ekspresi).

Berikut adalah adalah beberapa contoh cara mendefinisikan fungsi:

\end{eulercomment}
\begin{eulerformula}
\[
f(x)=2x^2+e^{\sin(x)}.
\]
\end{eulerformula}
\begin{eulerprompt}
>function f(x) := 2*x^2+exp(sin(x)) // fungsi numerik
>f(0), f(1), f(pi)
\end{eulerprompt}
\begin{euleroutput}
  1
  4.31977682472
  20.7392088022
\end{euleroutput}
\begin{eulerprompt}
>f(a) // tidak dapat dihitung nilainya
\end{eulerprompt}
\begin{euleroutput}
  Variable or function a not found.
  Error in:
  f(a) // tidak dapat dihitung nilainya ...
     ^
\end{euleroutput}
\begin{eulercomment}
Silakan Anda plot kurva fungsi di atas!

Berikutnya kita definisikan fungsi:

\end{eulercomment}
\begin{eulerformula}
\[
g(x)=\frac{\sqrt{x^2-3x}}{x+1}.
\]
\end{eulerformula}
\begin{eulerprompt}
>function g(x) := sqrt(x^2-3*x)/(x+1)
>g(3)
\end{eulerprompt}
\begin{euleroutput}
  0
\end{euleroutput}
\begin{eulerprompt}
>g(0)
\end{eulerprompt}
\begin{euleroutput}
  0
\end{euleroutput}
\begin{eulerprompt}
>g(1) // kompleks, tidak dapat dihitung oleh fungsi numerik
\end{eulerprompt}
\begin{euleroutput}
  Floating point error!
  Error in sqrt
  Try "trace errors" to inspect local variables after errors.
  g:
      useglobal; return sqrt(x^2-3*x)/(x+1) 
  Error in:
  g(1) // kompleks, tidak dapat dihitung oleh fungsi numerik ...
      ^
\end{euleroutput}
\begin{eulercomment}
Silakan Anda plot kurva fungsi di atas!
\end{eulercomment}
\begin{eulerprompt}
>f(g(5)) // komposisi fungsi
\end{eulerprompt}
\begin{euleroutput}
  2.20920171961
\end{euleroutput}
\begin{eulerprompt}
>g(f(5))
\end{eulerprompt}
\begin{euleroutput}
  0.950898070639
\end{euleroutput}
\begin{eulerprompt}
>function h(x) := f(g(x)) // definisi komposisi fungsi 
>h(5) // sama dengan f(g(5))
\end{eulerprompt}
\begin{euleroutput}
  2.20920171961
\end{euleroutput}
\begin{eulercomment}
Silakan Anda plot kurva fungsi komposisi fungsi f dan g:

\end{eulercomment}
\begin{eulerformula}
\[
h(x)=f(g(x))
\]
\end{eulerformula}
\begin{eulercomment}
dan

\end{eulercomment}
\begin{eulerformula}
\[
u(x)=g(f(x))
\]
\end{eulerformula}
\begin{eulercomment}
bersama-sama kurva fungsi f dan g dalam satu bidang koordinat.
\end{eulercomment}
\begin{eulerprompt}
>f(0:10) // nilai-nilai f(0), f(1), f(2), ..., f(10)
\end{eulerprompt}
\begin{euleroutput}
  [1,  4.31978,  10.4826,  19.1516,  32.4692,  50.3833,  72.7562,
  99.929,  130.69,  163.51,  200.58]
\end{euleroutput}
\begin{eulerprompt}
>fmap(0:10) // sama dengan f(0:10), berlaku untuk semua fungsi
\end{eulerprompt}
\begin{euleroutput}
  [1,  4.31978,  10.4826,  19.1516,  32.4692,  50.3833,  72.7562,
  99.929,  130.69,  163.51,  200.58]
\end{euleroutput}
\begin{eulerprompt}
>gmap(200:210)
\end{eulerprompt}
\begin{euleroutput}
  [0.987534,  0.987596,  0.987657,  0.987718,  0.987778,  0.987837,
  0.987896,  0.987954,  0.988012,  0.988069,  0.988126]
\end{euleroutput}
\begin{eulercomment}
Misalkan kita akan mendefinisikan fungsi

\end{eulercomment}
\begin{eulerformula}
\[
f(x) = \begin{cases} x^3 & x>0 \\ x^2 & x\le 0. \end{cases}
\]
\end{eulerformula}
\begin{eulercomment}
Fungsi tersebut tidak dapat didefinisikan sebagai fungsi numerik secara "inline" menggunakan
format :=, melainkan didefinisikan sebagai program. Perhatikan, kata "map" digunakan agar fungsi
dapat menerima vektor sebagai input, dan hasilnya berupa vektor. Jika tanpa kata "map" fungsinya
hanya dapat menerima input satu nilai.
\end{eulercomment}
\begin{eulerprompt}
>function map f(x) ...
\end{eulerprompt}
\begin{eulerudf}
    if x>0 then return x^3
    else return x^2
    endif;
  endfunction
\end{eulerudf}
\begin{eulerprompt}
>f(1)
\end{eulerprompt}
\begin{euleroutput}
  1
\end{euleroutput}
\begin{eulerprompt}
>f(-2)
\end{eulerprompt}
\begin{euleroutput}
  4
\end{euleroutput}
\begin{eulerprompt}
>f(-5:5)
\end{eulerprompt}
\begin{euleroutput}
  [25,  16,  9,  4,  1,  0,  1,  8,  27,  64,  125]
\end{euleroutput}
\begin{eulerprompt}
>aspect(1.5); plot2d("f(x)",-5,5):
\end{eulerprompt}
\eulerimg{17}{images/EMT4Kalkulus-Naela Rizqy Arofah-22305144042-001.png}
\begin{eulerprompt}
>function f(x) &= 2*E^x // fungsi simbolik
\end{eulerprompt}
\begin{euleroutput}
  
                                      x
                                   2 E
  
\end{euleroutput}
\begin{eulerprompt}
>$f(a) // nilai fungsi secara simbolik
\end{eulerprompt}
\begin{eulerformula}
\[
2\,e^{a}
\]
\end{eulerformula}
\begin{eulerprompt}
>f(E) // nilai fungsi berupa bilangan desimal
\end{eulerprompt}
\begin{euleroutput}
  30.308524483
\end{euleroutput}
\begin{eulerprompt}
>$f(E), $float(%)
\end{eulerprompt}
\begin{eulerformula}
\[
2\,e^{e}
\]
\end{eulerformula}
\begin{eulerformula}
\[
30.30852448295852
\]
\end{eulerformula}
\begin{eulerprompt}
>function g(x) &= 3*x+1
\end{eulerprompt}
\begin{euleroutput}
  
                                 3 x + 1
  
\end{euleroutput}
\begin{eulerprompt}
>function h(x) &= f(g(x)) // komposisi fungsi
\end{eulerprompt}
\begin{euleroutput}
  
                                   3 x + 1
                                2 E
  
\end{euleroutput}
\begin{eulerprompt}
>plot2d("h(x)",-1,1):
\end{eulerprompt}
\eulerimg{17}{images/EMT4Kalkulus-Naela Rizqy Arofah-22305144042-005.png}
\eulerheading{Latihan}
\begin{eulercomment}
Bukalah buku Kalkulus. Cari dan pilih beberapa (paling sedikit 5
fungsi berbeda tipe/bentuk/jenis) fungsi dari buku tersebut, kemudian
definisikan fungsi-fungsi tersebut dan komposisinya di EMT pada
baris-baris perintah berikut (jika perlu tambahkan lagi). Untuk setiap
fungsi, hitung beberapa nilainya, baik untuk satu nilai maupun vektor.
Gambar grafik fungsi-fungsi tersebut dan komposisi-komposisi 2 fungsi.

Juga, carilah fungsi beberapa (dua) variabel. Lakukan hal sama seperti
di atas.

\end{eulercomment}
\eulersubheading{}
\begin{eulercomment}
\begin{eulercomment}
\eulerheading{1}
\begin{eulerprompt}
>function f(x) := -x^2-2*x+2
>function g(x) := 4^x+2
>f(1), f(-2), f(4), g(2), g(4), g(-1)
\end{eulerprompt}
\begin{euleroutput}
  -1
  2
  -22
  18
  258
  2.25
\end{euleroutput}
\eulerheading{2}
\begin{eulerprompt}
>f(3), f(-1), f(2), g(3), g(7), g(-1)
\end{eulerprompt}
\begin{euleroutput}
  -13
  3
  -6
  66
  16386
  2.25
\end{euleroutput}
\eulerheading{3}
\begin{eulerprompt}
>plot2d("f(x)", color=green), plot2d("g(x)", color=red,>add):
\end{eulerprompt}
\eulerimg{17}{images/EMT4Kalkulus-Naela Rizqy Arofah-22305144042-006.png}
\eulerheading{4}
\begin{eulerprompt}
>function h(x) := f(g(x))
>function j(x) := g(f(x))
>h(1), h(-2), h(4), j(2), j(4), j(-1)
\end{eulerprompt}
\begin{euleroutput}
  -46
  -6.37890625
  -67078
  2.00024414063
  2
  66
\end{euleroutput}
\eulerheading{5}
\begin{eulerprompt}
>plot2d("h(x)", color=blue), plot2d("j(x)",>spectral,>add):
\end{eulerprompt}
\eulerimg{17}{images/EMT4Kalkulus-Naela Rizqy Arofah-22305144042-007.png}
\begin{eulercomment}
\begin{eulercomment}
\eulerheading{Menghitung Limit}
\begin{eulercomment}
Perhitungan limit pada EMT dapat dilakukan dengan menggunakan fungsi Maxima, yakni "limit".
Fungsi "limit" dapat digunakan untuk menghitung limit fungsi dalam bentuk ekspresi maupun fungsi
yang sudah didefinisikan sebelumnya. Nilai limit dapat dihitung pada sebarang nilai atau pada tak
hingga (-inf, minf, dan inf). Limit kiri dan limit kanan juga dapat dihitung, dengan cara memberi
opsi "plus" atau "minus". Hasil limit dapat berupa nilai, "und" (tak definisi), "ind" (tak tentu
namun terbatas), "infinity" (kompleks tak hingga).

Perhatikan beberapa contoh berikut. Perhatikan cara menampilkan perhitungan secara lengkap, tidak
hanya menampilkan hasilnya saja.
\end{eulercomment}
\begin{eulerprompt}
>$showev('limit(sqrt(x^2-3*x)/(x+1),x,inf))
\end{eulerprompt}
\begin{eulerformula}
\[
\lim_{x\rightarrow \infty }{\frac{\sqrt{x^2-3\,x}}{x+1}}=1
\]
\end{eulerformula}
\begin{eulerprompt}
>$limit((x^3-13*x^2+51*x-63)/(x^3-4*x^2-3*x+18),x,3)
\end{eulerprompt}
\begin{eulerformula}
\[
-\frac{4}{5}
\]
\end{eulerformula}
\begin{eulercomment}
maxima: 'limit((x\textasciicircum{}3-13*x\textasciicircum{}2+51*x-63)/(x\textasciicircum{}3-4*x\textasciicircum{}2-3*x+18),x,3)=limit((x\textasciicircum{}3-13*x\textasciicircum{}2+51*x-63)/(x\textasciicircum{}3-4*x\textasciicircum{}2-3*x+18),x,3)

Fungsi tersebut diskontinu di titik x=3. Berikut adalah grafik fungsinya.
\end{eulercomment}
\begin{eulerprompt}
>aspect(1.5); plot2d("(x^3-13*x^2+51*x-63)/(x^3-4*x^2-3*x+18)",0,4); plot2d(3,-4/5,>points,style="ow",>add):
\end{eulerprompt}
\eulerimg{17}{images/EMT4Kalkulus-Naela Rizqy Arofah-22305144042-010.png}
\begin{eulerprompt}
>$limit(2*x*sin(x)/(1-cos(x)),x,0)
\end{eulerprompt}
\begin{eulerformula}
\[
4
\]
\end{eulerformula}
\begin{eulercomment}
maxima: 'limit(2*x*sin(x)/(1-cos(x)),x,0)=limit(2*x*sin(x)/(1-cos(x)),x,0)

Fungsi tersebut diskontinu di titik x=0. Berikut adalah grafik fungsinya.
\end{eulercomment}
\begin{eulerprompt}
>plot2d("2*x*sin(x)/(1-cos(x))",-pi,pi); plot2d(0,4,>points,style="ow",>add):
\end{eulerprompt}
\eulerimg{17}{images/EMT4Kalkulus-Naela Rizqy Arofah-22305144042-012.png}
\begin{eulerprompt}
>$limit(cot(7*h)/cot(5*h),h,0)
\end{eulerprompt}
\begin{eulerformula}
\[
\frac{5}{7}
\]
\end{eulerformula}
\begin{eulercomment}
maxima: showev('limit(cot(7*h)/cot(5*h),h,0))

Fungsi tersebut juga diskontinu (karena tidak terdefinisi) di x=0. Berikut adalah grafiknya.
\end{eulercomment}
\begin{eulerprompt}
>plot2d("cot(7*x)/cot(5*x)",-0.001,0.001); plot2d(0,5/7,>points,style="ow",>add):
\end{eulerprompt}
\eulerimg{17}{images/EMT4Kalkulus-Naela Rizqy Arofah-22305144042-014.png}
\begin{eulerprompt}
>$showev('limit(((x/8)^(1/3)-1)/(x-8),x,8))
\end{eulerprompt}
\begin{eulerformula}
\[
\lim_{x\rightarrow 8}{\frac{\frac{x^{\frac{1}{3}}}{2}-1}{x-8}}=
 \frac{1}{24}
\]
\end{eulerformula}
\begin{eulercomment}
Tunjukkan limit tersebut dengan grafik, seperti contoh-contoh sebelumnya.
\end{eulercomment}
\begin{eulerprompt}
>$showev('limit(1/(2*x-1),x,0))
\end{eulerprompt}
\begin{eulerformula}
\[
\lim_{x\rightarrow 0}{\frac{1}{2\,x-1}}=-1
\]
\end{eulerformula}
\begin{eulercomment}
Tunjukkan limit tersebut dengan grafik, seperti contoh-contoh sebelumnya.
\end{eulercomment}
\begin{eulerprompt}
>$showev('limit((x^2-3*x-10)/(x-5),x,5))
\end{eulerprompt}
\begin{eulerformula}
\[
\lim_{x\rightarrow 5}{\frac{x^2-3\,x-10}{x-5}}=7
\]
\end{eulerformula}
\begin{eulercomment}
Tunjukkan limit tersebut dengan grafik, seperti contoh-contoh sebelumnya.
\end{eulercomment}
\begin{eulerprompt}
>$showev('limit(sqrt(x^2+x)-x,x,inf))
\end{eulerprompt}
\begin{eulerformula}
\[
\lim_{x\rightarrow \infty }{\sqrt{x^2+x}-x}=\frac{1}{2}
\]
\end{eulerformula}
\begin{eulercomment}
Tunjukkan limit tersebut dengan grafik, seperti contoh-contoh sebelumnya.
\end{eulercomment}
\begin{eulerprompt}
>$showev('limit(abs(x-1)/(x-1),x,1,minus))
\end{eulerprompt}
\begin{eulerformula}
\[
\lim_{x\uparrow 1}{\frac{\left| x-1\right| }{x-1}}=-1
\]
\end{eulerformula}
\begin{eulercomment}
Hitung limit di atas untuk x menuju 1 dari kanan.\\
Tunjukkan limit tersebut dengan grafik, seperti contoh-contoh sebelumnya.
\end{eulercomment}
\begin{eulerprompt}
>$showev('limit(sin(x)/x,x,0))
\end{eulerprompt}
\begin{eulerformula}
\[
\lim_{x\rightarrow 0}{\frac{\sin x}{x}}=1
\]
\end{eulerformula}
\begin{eulerprompt}
>plot2d("sin(x)/x",-pi,pi); plot2d(0,1,>points,style="ow",>add):
\end{eulerprompt}
\eulerimg{17}{images/EMT4Kalkulus-Naela Rizqy Arofah-22305144042-021.png}
\begin{eulerprompt}
>$showev('limit(sin(x^3)/x,x,0))
\end{eulerprompt}
\begin{eulerformula}
\[
\lim_{x\rightarrow 0}{\frac{\sin x^3}{x}}=0
\]
\end{eulerformula}
\begin{eulercomment}
Tunjukkan limit tersebut dengan grafik, seperti contoh-contoh sebelumnya.
\end{eulercomment}
\begin{eulerprompt}
>$showev('limit(log(x), x, minf))
\end{eulerprompt}
\begin{eulerformula}
\[
\lim_{x\rightarrow  -\infty }{\log x}={\it infinity}
\]
\end{eulerformula}
\begin{eulerprompt}
>$showev('limit((-2)^x,x, inf))
\end{eulerprompt}
\begin{eulerformula}
\[
\lim_{x\rightarrow \infty }{\left(-2\right)^{x}}={\it infinity}
\]
\end{eulerformula}
\begin{eulerprompt}
>$showev('limit(t-sqrt(2-t),t,2,minus))
\end{eulerprompt}
\begin{eulerformula}
\[
\lim_{t\uparrow 2}{t-\sqrt{2-t}}=2
\]
\end{eulerformula}
\begin{eulerprompt}
>$showev('limit(t-sqrt(2-t),t,2,plus))
\end{eulerprompt}
\begin{eulerformula}
\[
\lim_{t\downarrow 2}{t-\sqrt{2-t}}=2
\]
\end{eulerformula}
\begin{eulerprompt}
>$showev('limit(t-sqrt(2-t),t,5,plus)) // Perhatikan hasilnya
\end{eulerprompt}
\begin{eulerformula}
\[
\lim_{t\downarrow 5}{t-\sqrt{2-t}}=5-\sqrt{3}\,i
\]
\end{eulerformula}
\begin{eulerprompt}
>plot2d("x-sqrt(2-x)",0,2):
\end{eulerprompt}
\eulerimg{17}{images/EMT4Kalkulus-Naela Rizqy Arofah-22305144042-028.png}
\begin{eulerprompt}
>$showev('limit((x^2-9)/(2*x^2-5*x-3),x,3))
\end{eulerprompt}
\begin{eulerformula}
\[
\lim_{x\rightarrow 3}{\frac{x^2-9}{2\,x^2-5\,x-3}}=\frac{6}{7}
\]
\end{eulerformula}
\begin{eulercomment}
Tunjukkan limit tersebut dengan grafik, seperti contoh-contoh sebelumnya.
\end{eulercomment}
\begin{eulerprompt}
>$showev('limit((1-cos(x))/x,x,0))
\end{eulerprompt}
\begin{eulerformula}
\[
\lim_{x\rightarrow 0}{\frac{1-\cos x}{x}}=0
\]
\end{eulerformula}
\begin{eulercomment}
Tunjukkan limit tersebut dengan grafik, seperti contoh-contoh sebelumnya.
\end{eulercomment}
\begin{eulerprompt}
>$showev('limit((x^2+abs(x))/(x^2-abs(x)),x,0))
\end{eulerprompt}
\begin{eulerformula}
\[
\lim_{x\rightarrow 0}{\frac{\left| x\right| +x^2}{x^2-\left| x
 \right| }}=-1
\]
\end{eulerformula}
\begin{eulercomment}
Tunjukkan limit tersebut dengan grafik, seperti contoh-contoh sebelumnya.
\end{eulercomment}
\begin{eulerprompt}
>$showev('limit((1+1/x)^x,x,inf))
\end{eulerprompt}
\begin{eulerformula}
\[
\lim_{x\rightarrow \infty }{\left(\frac{1}{x}+1\right)^{x}}=e
\]
\end{eulerformula}
\begin{eulerprompt}
>plot2d("(1+1/x)^x",0,1000):
\end{eulerprompt}
\eulerimg{17}{images/EMT4Kalkulus-Naela Rizqy Arofah-22305144042-033.png}
\begin{eulerprompt}
>$showev('limit((1+k/x)^x,x,inf))
\end{eulerprompt}
\begin{eulerformula}
\[
\lim_{x\rightarrow \infty }{\left(\frac{k}{x}+1\right)^{x}}=e^{k}
\]
\end{eulerformula}
\begin{eulerprompt}
>$showev('limit((1+x)^(1/x),x,0))
\end{eulerprompt}
\begin{eulerformula}
\[
\lim_{x\rightarrow 0}{\left(x+1\right)^{\frac{1}{x}}}=e
\]
\end{eulerformula}
\begin{eulercomment}
Tunjukkan limit tersebut dengan grafik, seperti contoh-contoh sebelumnya.
\end{eulercomment}
\begin{eulerprompt}
>$showev('limit((x/(x+k))^x,x,inf))
\end{eulerprompt}
\begin{eulerformula}
\[
\lim_{x\rightarrow \infty }{\left(\frac{x}{x+k}\right)^{x}}=e^ {- k
  }
\]
\end{eulerformula}
\begin{eulerprompt}
>$showev('limit((E^x-E^2)/(x-2),x,2))
\end{eulerprompt}
\begin{eulerformula}
\[
\lim_{x\rightarrow 2}{\frac{e^{x}-e^2}{x-2}}=e^2
\]
\end{eulerformula}
\begin{eulercomment}
Tunjukkan limit tersebut dengan grafik, seperti contoh-contoh sebelumnya.
\end{eulercomment}
\begin{eulerprompt}
>$showev('limit(sin(1/x),x,0))
\end{eulerprompt}
\begin{eulerformula}
\[
\lim_{x\rightarrow 0}{\sin \left(\frac{1}{x}\right)}={\it ind}
\]
\end{eulerformula}
\begin{eulerprompt}
>$showev('limit(sin(1/x),x,inf))
\end{eulerprompt}
\begin{eulerformula}
\[
\lim_{x\rightarrow \infty }{\sin \left(\frac{1}{x}\right)}=0
\]
\end{eulerformula}
\begin{eulerprompt}
>plot2d("sin(1/x)",-0.001,0.001):
\end{eulerprompt}
\eulerimg{17}{images/EMT4Kalkulus-Naela Rizqy Arofah-22305144042-040.png}
\eulerheading{Latihan}
\begin{eulercomment}
Bukalah buku Kalkulus. Cari dan pilih beberapa (paling sedikit 5
fungsi berbeda tipe/bentuk/jenis) fungsi dari buku tersebut, kemudian
definisikan di EMT pada baris-baris perintah berikut (jika perlu
tambahkan lagi). Untuk setiap fungsi, hitung nilai limit fungsi
tersebut di beberapa nilai dan di tak hingga. Gambar grafik fungsi
tersebut untuk mengkonfirmasi nilai-nilai limit tersebut.

\end{eulercomment}
\eulersubheading{}
\begin{eulercomment}
\begin{eulercomment}
\eulerheading{1}
\begin{eulerprompt}
>$showev('limit(sin(2*x),x,0))
\end{eulerprompt}
\begin{eulerformula}
\[
\lim_{x\rightarrow 0}{\sin \left(2\,x\right)}=0
\]
\end{eulerformula}
\begin{eulerprompt}
>$showev('limit(sin(2*x),x,pi/3))
\end{eulerprompt}
\begin{eulerformula}
\[
\lim_{x\rightarrow \frac{\pi}{3}}{\sin \left(2\,x\right)}=\frac{
 \sqrt{3}}{2}
\]
\end{eulerformula}
\begin{eulerprompt}
>plot2d("sin(2*x)",-pi,pi):
\end{eulerprompt}
\eulerimg{17}{images/EMT4Kalkulus-Naela Rizqy Arofah-22305144042-043.png}
\eulerheading{2}
\begin{eulerprompt}
>$showev('limit(sin(x)^2,x,0))
\end{eulerprompt}
\begin{eulerformula}
\[
\lim_{x\rightarrow 0}{\sin ^2x}=0
\]
\end{eulerformula}
\begin{eulerprompt}
>$showev('limit(sin(x)^2,x,-2))
\end{eulerprompt}
\begin{eulerformula}
\[
\lim_{x\rightarrow -2}{\sin ^2x}=\sin ^22
\]
\end{eulerformula}
\begin{eulerprompt}
>$showev('limit(sin(x)^2,x,-5))
\end{eulerprompt}
\begin{eulerformula}
\[
\lim_{x\rightarrow -5}{\sin ^2x}=\sin ^25
\]
\end{eulerformula}
\begin{eulerprompt}
>plot2d("sin(x^2)",-50,50):
\end{eulerprompt}
\eulerimg{17}{images/EMT4Kalkulus-Naela Rizqy Arofah-22305144042-047.png}
\eulerheading{3}
\begin{eulerprompt}
>$showev('limit(3*x^2+2*x+1,x,0))
\end{eulerprompt}
\begin{eulerformula}
\[
\lim_{x\rightarrow 0}{3\,x^2+2\,x+1}=1
\]
\end{eulerformula}
\begin{eulerprompt}
>$showev('limit(3*x^2+2*x+1,x,-5))
\end{eulerprompt}
\begin{eulerformula}
\[
\lim_{x\rightarrow -5}{3\,x^2+2\,x+1}=66
\]
\end{eulerformula}
\begin{eulerprompt}
>$showev('limit(3*x^2+2*x+1,x,-3))
\end{eulerprompt}
\begin{eulerformula}
\[
\lim_{x\rightarrow -3}{3\,x^2+2\,x+1}=22
\]
\end{eulerformula}
\begin{eulerprompt}
>plot2d("3*x^2+2*x+1",-30,30,color=blue):
\end{eulerprompt}
\eulerimg{17}{images/EMT4Kalkulus-Naela Rizqy Arofah-22305144042-051.png}
\eulerheading{4}
\begin{eulerprompt}
>$showev('limit(cos(x),x,0))
\end{eulerprompt}
\begin{eulerformula}
\[
\lim_{x\rightarrow 0}{\cos x}=1
\]
\end{eulerformula}
\begin{eulerprompt}
>$showev('limit(cos(x),x,5))
\end{eulerprompt}
\begin{eulerformula}
\[
\lim_{x\rightarrow 5}{\cos x}=\cos 5
\]
\end{eulerformula}
\begin{eulerprompt}
>plot2d("cos(x)",-30,30,color=blue):
\end{eulerprompt}
\eulerimg{17}{images/EMT4Kalkulus-Naela Rizqy Arofah-22305144042-054.png}
\eulerheading{5}
\begin{eulerprompt}
>plot2d("3*x^2+2*x+1",-5,5,color=red):
\end{eulerprompt}
\eulerimg{17}{images/EMT4Kalkulus-Naela Rizqy Arofah-22305144042-055.png}
\begin{eulercomment}
\begin{eulercomment}
\eulerheading{Turunan Fungsi}
\begin{eulercomment}
Definisi turunan:

\end{eulercomment}
\begin{eulerformula}
\[
f'(x) = \lim_{h\to 0} \frac{f(x+h)-f(x)}{h}
\]
\end{eulerformula}
\begin{eulercomment}
Berikut adalah contoh-contoh menentukan turunan fungsi dengan menggunakan definisi turunan
(limit).
\end{eulercomment}
\begin{eulerprompt}
>$showev('limit(((x+h)^2-x^2)/h,h,0)) // turunan x^2
\end{eulerprompt}
\begin{eulerformula}
\[
\lim_{h\rightarrow 0}{\frac{\left(x+h\right)^2-x^2}{h}}=2\,x
\]
\end{eulerformula}
\begin{eulerprompt}
>p &= expand((x+h)^2-x^2)|simplify; $p //pembilang dijabarkan dan disederhanakan
\end{eulerprompt}
\begin{eulerformula}
\[
2\,h\,x+h^2
\]
\end{eulerformula}
\begin{eulerprompt}
>q &=ratsimp(p/h); $q // ekspresi yang akan dihitung limitnya disederhanakan
\end{eulerprompt}
\begin{eulerformula}
\[
2\,x+h
\]
\end{eulerformula}
\begin{eulerprompt}
>$limit(q,h,0) // nilai limit sebagai turunan
\end{eulerprompt}
\begin{eulerformula}
\[
2\,x
\]
\end{eulerformula}
\begin{eulerprompt}
>$showev('limit(((x+h)^n-x^n)/h,h,0)) // turunan x^n
\end{eulerprompt}
\begin{eulerformula}
\[
\lim_{h\rightarrow 0}{\frac{\left(x+h\right)^{n}-x^{n}}{h}}=n\,x^{n
 -1}
\]
\end{eulerformula}
\begin{eulercomment}
Mengapa hasilnya seperti itu? Tuliskan atau tunjukkan bahwa hasil limit tersebut
benar, sehingga benar turunan fungsinya benar.  Tulis penjelasan Anda di komentar
ini.

Sebagai petunjuk, ekspansikan (x+h)\textasciicircum{}n dengan menggunakan teorema binomial.
\end{eulercomment}
\begin{eulerprompt}
>$showev('limit((sin(x+h)-sin(x))/h,h,0)) // turunan sin(x)
\end{eulerprompt}
\begin{eulerformula}
\[
\lim_{h\rightarrow 0}{\frac{\sin \left(x+h\right)-\sin x}{h}}=\cos 
 x
\]
\end{eulerformula}
\begin{eulercomment}
Mengapa hasilnya seperti itu? Tuliskan atau tunjukkan bahwa hasil limit tersebut\\
benar, sehingga benar turunan fungsinya benar.  Tulis penjelasan Anda di komentar
ini.

Sebagai petunjuk, ekspansikan sin(x+h) dengan menggunakan rumus jumlah dua sudut.
\end{eulercomment}
\begin{eulerprompt}
>$showev('limit((log(x+h)-log(x))/h,h,0)) // turunan log(x)
\end{eulerprompt}
\begin{eulerformula}
\[
\lim_{h\rightarrow 0}{\frac{\log \left(x+h\right)-\log x}{h}}=
 \frac{1}{x}
\]
\end{eulerformula}
\begin{eulercomment}
Mengapa hasilnya seperti itu? Tuliskan atau tunjukkan bahwa hasil limit tersebut\\
benar, sehingga benar turunan fungsinya benar.  Tulis penjelasan Anda di komentar
ini.

Sebagai petunjuk, gunakan sifat-sifat logaritma dan hasil limit pada bagian
sebelumnya di atas.
\end{eulercomment}
\begin{eulerprompt}
>$showev('limit((1/(x+h)-1/x)/h,h,0)) // turunan 1/x
\end{eulerprompt}
\begin{eulerformula}
\[
\lim_{h\rightarrow 0}{\frac{\frac{1}{x+h}-\frac{1}{x}}{h}}=-\frac{1
 }{x^2}
\]
\end{eulerformula}
\begin{eulerprompt}
>$showev('limit((E^(x+h)-E^x)/h,h,0)) // turunan f(x)=e^x
\end{eulerprompt}
\begin{euleroutput}
  Answering "Is x an integer?" with "integer"
  Answering "Is x an integer?" with "integer"
  Answering "Is x an integer?" with "integer"
  Answering "Is x an integer?" with "integer"
  Answering "Is x an integer?" with "integer"
  Maxima is asking
  Acceptable answers are: yes, y, Y, no, n, N, unknown, uk
  Is x an integer?
  
  Use assume!
  Error in:
  $showev('limit((E^(x+h)-E^x)/h,h,0)) // turunan f(x)=e^x ...
                                       ^
\end{euleroutput}
\begin{eulercomment}
Maxima bermasalah dengan limit:

\end{eulercomment}
\begin{eulerformula}
\[
\lim_{h\to 0}\frac{e^{x+h}-e^x}{h}.
\]
\end{eulerformula}
\begin{eulercomment}
Oleh karena itu diperlukan trik khusus agar hasilnya benar.
\end{eulercomment}
\begin{eulerprompt}
>$showev('limit((E^h-1)/h,h,0))
\end{eulerprompt}
\begin{eulerformula}
\[
\lim_{h\rightarrow 0}{\frac{e^{h}-1}{h}}=1
\]
\end{eulerformula}
\begin{eulerprompt}
>$showev('factor(E^(x+h)-E^x))
\end{eulerprompt}
\begin{eulerformula}
\[
{\it factor}\left(e^{x+h}-e^{x}\right)=\left(e^{h}-1\right)\,e^{x}
\]
\end{eulerformula}
\begin{eulerprompt}
>$showev('limit(factor((E^(x+h)-E^x)/h),h,0)) // turunan f(x)=e^x
\end{eulerprompt}
\begin{eulerformula}
\[
\left(\lim_{h\rightarrow 0}{\frac{e^{h}-1}{h}}\right)\,e^{x}=e^{x}
\]
\end{eulerformula}
\begin{eulerprompt}
>function f(x) &= x^x
\end{eulerprompt}
\begin{euleroutput}
  
                                     x
                                    x
  
\end{euleroutput}
\begin{eulerprompt}
>$showev('limit(f(x),x,0))
\end{eulerprompt}
\begin{eulerformula}
\[
\lim_{x\rightarrow 0}{x^{x}}=1
\]
\end{eulerformula}
\begin{eulercomment}
Silakan Anda gambar kurva

\end{eulercomment}
\begin{eulerformula}
\[
y=x^x.
\]
\end{eulerformula}
\begin{eulerprompt}
>$showef('limit((f(x+h)-f(x))/h,h,0))// merupakan turunan f(x)=x^x
\end{eulerprompt}
\begin{eulerformula}
\[
{\it showef}\left(\lim_{h\rightarrow 0}{\frac{\left(x+h\right)^{x+h
 }-x^{x}}{h}}\right)
\]
\end{eulerformula}
\begin{eulerprompt}
>$showev('limit((f(x+h)-f(x))/h,h,0)) // turunan f(x)=x^x
\end{eulerprompt}
\begin{eulerformula}
\[
\lim_{h\rightarrow 0}{\frac{\left(x+h\right)^{x+h}-x^{x}}{h}}=
 {\it infinity}
\]
\end{eulerformula}
\begin{eulercomment}
Di sini Maxima juga bermasalah terkait limit:

\end{eulercomment}
\begin{eulerformula}
\[
\lim_{h\to 0} \frac{(x+h)^{x+h}-x^x}{h}.
\]
\end{eulerformula}
\begin{eulercomment}
Dalam hal ini diperlukan asumsi nilai x.
\end{eulercomment}
\begin{eulerprompt}
>&assume(x>0); $showev('limit((f(x+h)-f(x))/h,h,0)) // turunan f(x)=x^x
\end{eulerprompt}
\begin{eulerformula}
\[
\lim_{h\rightarrow 0}{\frac{\left(x+h\right)^{x+h}-x^{x}}{h}}=x^{x}
 \,\left(\log x+1\right)
\]
\end{eulerformula}
\begin{eulercomment}
Mengapa hasilnya seperti itu? Tuliskan atau tunjukkan bahwa hasil limit tersebut benar, sehingga benar turunan fungsinya benar.
Tulis penjelasan Anda di komentar ini.
\end{eulercomment}
\begin{eulerprompt}
>&forget(x>0) // jangan lupa, lupakan asumsi untuk kembali ke semula
\end{eulerprompt}
\begin{euleroutput}
  
                                 [x > 0]
  
\end{euleroutput}
\begin{eulerprompt}
>&forget(x<0)
\end{eulerprompt}
\begin{euleroutput}
  
                                 [x < 0]
  
\end{euleroutput}
\begin{eulerprompt}
>&facts()
\end{eulerprompt}
\begin{euleroutput}
  
                                    []
  
\end{euleroutput}
\begin{eulerprompt}
>$showev('limit((asin(x+h)-asin(x))/h,h,0)) // turunan arcsin(x)
\end{eulerprompt}
\begin{eulerformula}
\[
\lim_{h\rightarrow 0}{\frac{\arcsin \left(x+h\right)-\arcsin x}{h}}=
 \frac{1}{\sqrt{1-x^2}}
\]
\end{eulerformula}
\begin{eulercomment}
Mengapa hasilnya seperti itu? Tuliskan atau tunjukkan bahwa hasil limit tersebut benar, sehingga
benar turunan fungsinya benar. Tulis penjelasan Anda di komentar ini.
\end{eulercomment}
\begin{eulerprompt}
>$showev('limit((tan(x+h)-tan(x))/h,h,0)) // turunan tan(x)
\end{eulerprompt}
\begin{eulerformula}
\[
\lim_{h\rightarrow 0}{\frac{\tan \left(x+h\right)-\tan x}{h}}=
 \frac{1}{\cos ^2x}
\]
\end{eulerformula}
\begin{eulercomment}
Mengapa hasilnya seperti itu? Tuliskan atau tunjukkan bahwa hasil limit tersebut benar, sehingga
benar turunan fungsinya benar. Tulis penjelasan Anda di komentar ini.
\end{eulercomment}
\begin{eulerprompt}
>function f(x) &= sinh(x) // definisikan f(x)=sinh(x)
\end{eulerprompt}
\begin{euleroutput}
  
                                 sinh(x)
  
\end{euleroutput}
\begin{eulerprompt}
>function df(x) &= limit((f(x+h)-f(x))/h,h,0); $df(x) // df(x) = f'(x)
\end{eulerprompt}
\begin{eulerformula}
\[
\frac{e^ {- x }\,\left(e^{2\,x}+1\right)}{2}
\]
\end{eulerformula}
\begin{eulercomment}
Hasilnya adalah cosh(x), karena

\end{eulercomment}
\begin{eulerformula}
\[
\frac{e^x+e^{-x}}{2}=\cosh(x).
\]
\end{eulerformula}
\begin{eulerprompt}
>plot2d(["f(x)","df(x)"],-pi,pi,color=[blue,red]):
\end{eulerprompt}
\eulerimg{17}{images/EMT4Kalkulus-Naela Rizqy Arofah-22305144042-074.png}
\begin{eulerprompt}
>function f(x) &= sin(3*x^5+7)^2
\end{eulerprompt}
\begin{euleroutput}
  
                                 2    5
                              sin (3 x  + 7)
  
\end{euleroutput}
\begin{eulerprompt}
>diff(f,3), diffc(f,3)
\end{eulerprompt}
\begin{euleroutput}
  1198.32948904
  1198.72863721
\end{euleroutput}
\begin{eulercomment}
Apakah perbedaan diff dan diffc?
\end{eulercomment}
\begin{eulerprompt}
>$showev('diff(f(x),x))
\end{eulerprompt}
\begin{eulerformula}
\[
\frac{d}{d\,x}\,\sin ^2\left(3\,x^5+7\right)=30\,x^4\,\cos \left(3
 \,x^5+7\right)\,\sin \left(3\,x^5+7\right)
\]
\end{eulerformula}
\begin{eulerprompt}
>$% with x=3
\end{eulerprompt}
\begin{eulerformula}
\[
{\it \%at}\left(\frac{d}{d\,x}\,\sin ^2\left(3\,x^5+7\right) , x=3
 \right)=2430\,\cos 736\,\sin 736
\]
\end{eulerformula}
\begin{eulerprompt}
>$float(%)
\end{eulerprompt}
\begin{eulerformula}
\[
{\it \%at}\left(\frac{d^{1.0}}{d\,x^{1.0}}\,\sin ^2\left(3.0\,x^5+
 7.0\right) , x=3.0\right)=1198.728637211748
\]
\end{eulerformula}
\begin{eulerprompt}
>plot2d(f,0,3.1):
\end{eulerprompt}
\eulerimg{17}{images/EMT4Kalkulus-Naela Rizqy Arofah-22305144042-078.png}
\begin{eulerprompt}
>function f(x) &=5*cos(2*x)-2*x*sin(2*x) // mendifinisikan fungsi f
\end{eulerprompt}
\begin{euleroutput}
  
                        5 cos(2 x) - 2 x sin(2 x)
  
\end{euleroutput}
\begin{eulerprompt}
>function df(x) &=diff(f(x),x) // fd(x) = f'(x)
\end{eulerprompt}
\begin{euleroutput}
  
                       - 12 sin(2 x) - 4 x cos(2 x)
  
\end{euleroutput}
\begin{eulerprompt}
>$'f(1)=f(1), $float(f(1)), $'f(2)=f(2), $float(f(2)) // nilai f(1) dan f(2)
\end{eulerprompt}
\begin{eulerformula}
\[
f\left(1\right)=5\,\cos 2-2\,\sin 2
\]
\end{eulerformula}
\begin{eulerformula}
\[
-3.899329036387075
\]
\end{eulerformula}
\begin{eulerformula}
\[
f\left(2\right)=5\,\cos 4-4\,\sin 4
\]
\end{eulerformula}
\begin{eulerformula}
\[
-0.2410081230863468
\]
\end{eulerformula}
\begin{eulerprompt}
>xp=solve("df(x)",1,2,0) // solusi f'(x)=0 pada interval [1, 2]
\end{eulerprompt}
\begin{euleroutput}
  1.35822987384
\end{euleroutput}
\begin{eulerprompt}
>df(xp), f(xp) // cek bahwa f'(xp)=0 dan nilai ekstrim di titik tersebut
\end{eulerprompt}
\begin{euleroutput}
  0
  -5.67530133759
\end{euleroutput}
\begin{eulerprompt}
>plot2d(["f(x)","df(x)"],0,2*pi,color=[blue,red]): //grafik fungsi dan turunannya
\end{eulerprompt}
\eulerimg{17}{images/EMT4Kalkulus-Naela Rizqy Arofah-22305144042-083.png}
\begin{eulercomment}
Perhatikan titik-titik "puncak" grafik y=f(x) dan nilai turunan pada saat grafik fungsinya mencapai titik "puncak" tersebut.
\end{eulercomment}
\eulerheading{Latihan}
\begin{eulercomment}
Bukalah buku Kalkulus. Cari dan pilih beberapa (paling sedikit 5
fungsi berbeda tipe/bentuk/jenis) fungsi dari buku tersebut, kemudian
definisikan di EMT pada baris-baris perintah berikut (jika perlu
tambahkan lagi). Untuk setiap fungsi, tentukan turunannya dengan
menggunakan definisi turunan (limit), menggunakan perintah diff, dan
secara manual (langkah demi langkah yang dihitung dengan Maxima)
seperti contoh-contoh di atas. Gambar grafik fungsi asli dan fungsi
turunannya pada sumbu koordinat yang sama.\\
\end{eulercomment}
\eulersubheading{}
\begin{eulercomment}
\begin{eulercomment}
\eulerheading{1}
\begin{eulerprompt}
> function f(x)&= sin(x^2+6*x)^2
\end{eulerprompt}
\begin{euleroutput}
  
                                 2  2
                              sin (x  + 6 x)
  
\end{euleroutput}
\begin{eulerprompt}
>$showev('limit(f(x),x,0))
\end{eulerprompt}
\begin{eulerformula}
\[
\lim_{x\rightarrow 0}{\sin ^2\left(x^2+6\,x\right)}=0
\]
\end{eulerformula}
\begin{eulerprompt}
>$showev('limit(((x+h)^n-x^n)/h,h,0))
\end{eulerprompt}
\begin{eulerformula}
\[
\lim_{h\rightarrow 0}{\frac{\left(x+h\right)^{n}-x^{n}}{h}}=n\,x^{n
 -1}
\]
\end{eulerformula}
\begin{eulerprompt}
>$df(x)
\end{eulerprompt}
\begin{eulerformula}
\[
-12\,\sin \left(2\,x\right)-4\,x\,\cos \left(2\,x\right)
\]
\end{eulerformula}
\begin{eulerprompt}
>plot2d(["f(x)","df(x)"],0,2*pi):
\end{eulerprompt}
\eulerimg{17}{images/EMT4Kalkulus-Naela Rizqy Arofah-22305144042-087.png}
\eulerheading{2}
\begin{eulerprompt}
>plot2d(["f(x)","df(x)"],0,2*pi,color=[blue,red]):
\end{eulerprompt}
\eulerimg{17}{images/EMT4Kalkulus-Naela Rizqy Arofah-22305144042-088.png}
\eulerheading{3}
\begin{eulerprompt}
>function f(x)&= sin(x)^2
\end{eulerprompt}
\begin{euleroutput}
  
                                    2
                                 sin (x)
  
\end{euleroutput}
\begin{eulerprompt}
>$showev('limit(f(x),x,0))
\end{eulerprompt}
\begin{eulerformula}
\[
\lim_{x\rightarrow 0}{\sin ^2x}=0
\]
\end{eulerformula}
\begin{eulerprompt}
>$showev('limit(f(x),x,-3))
\end{eulerprompt}
\begin{eulerformula}
\[
\lim_{x\rightarrow -3}{\sin ^2x}=\sin ^23
\]
\end{eulerformula}
\begin{eulerprompt}
>$df(x)
\end{eulerprompt}
\begin{eulerformula}
\[
-12\,\sin \left(2\,x\right)-4\,x\,\cos \left(2\,x\right)
\]
\end{eulerformula}
\begin{eulerprompt}
>plot2d(["f(x)","df(x)"],0,8*pi):
\end{eulerprompt}
\eulerimg{17}{images/EMT4Kalkulus-Naela Rizqy Arofah-22305144042-092.png}
\eulerheading{4}
\begin{eulerprompt}
>plot2d(["f(x)","df(x)"],0,2*pi,color=[blue,red]):
\end{eulerprompt}
\eulerimg{17}{images/EMT4Kalkulus-Naela Rizqy Arofah-22305144042-093.png}
\eulerheading{5}
\begin{eulerprompt}
>plot2d(["f(x)","df(x)"],0,2*pi,color=[green,red]):
\end{eulerprompt}
\eulerimg{17}{images/EMT4Kalkulus-Naela Rizqy Arofah-22305144042-094.png}
\eulerheading{6}
\begin{eulerprompt}
>plot2d(["f(x)","df(x)"],0,2*pi,color=[blue,green]):
\end{eulerprompt}
\eulerimg{17}{images/EMT4Kalkulus-Naela Rizqy Arofah-22305144042-095.png}
\eulerheading{Integral}
\begin{eulercomment}
EMT dapat digunakan untuk menghitung integral, baik integral tak tentu maupun
integral tentu. Untuk integral tak tentu (simbolik) sudah tentu EMT menggunakan
Maxima, sedangkan untuk perhitungan integral tentu EMT sudah menyediakan beberapa
fungsi yang mengimplementasikan algoritma kuadratur (perhitungan integral tentu
menggunakan metode numerik).

Pada notebook ini akan ditunjukkan perhitungan integral tentu dengan menggunakan
Teorema Dasar Kalkulus:

\end{eulercomment}
\begin{eulerformula}
\[
\int_a^b f(x)\ dx = F(b)-F(a), \quad \text{ dengan  } F'(x) = f(x).
\]
\end{eulerformula}
\begin{eulercomment}
Fungsi untuk menentukan integral adalah integrate. Fungsi ini dapat digunakan untuk
menentukan, baik integral tentu maupun tak tentu (jika fungsinya memiliki
antiderivatif). Untuk perhitungan integral tentu fungsi integrate menggunakan metode
numerik (kecuali fungsinya tidak integrabel, kita tidak akan menggunakan metode ini).
\end{eulercomment}
\begin{eulerprompt}
>$showev('integrate(x^n,x))
\end{eulerprompt}
\begin{euleroutput}
  Answering "Is n equal to -1?" with "no"
\end{euleroutput}
\begin{eulerformula}
\[
\int {x^{n}}{\;dx}=\frac{x^{n+1}}{n+1}
\]
\end{eulerformula}
\begin{eulerprompt}
>$showev('integrate(1/(1+x),x))
\end{eulerprompt}
\begin{eulerformula}
\[
\int {\frac{1}{x+1}}{\;dx}=\log \left(x+1\right)
\]
\end{eulerformula}
\begin{eulerprompt}
>$showev('integrate(1/(1+x^2),x))
\end{eulerprompt}
\begin{eulerformula}
\[
\int {\frac{1}{x^2+1}}{\;dx}=\arctan x
\]
\end{eulerformula}
\begin{eulerprompt}
>$showev('integrate(1/sqrt(1-x^2),x))
\end{eulerprompt}
\begin{eulerformula}
\[
\int {\frac{1}{\sqrt{1-x^2}}}{\;dx}=\arcsin x
\]
\end{eulerformula}
\begin{eulerprompt}
>$showev('integrate(sin(x),x,0,pi))
\end{eulerprompt}
\begin{eulerformula}
\[
\int_{0}^{\pi}{\sin x\;dx}=2
\]
\end{eulerformula}
\begin{eulerprompt}
>plot2d("sin(x)",0,2*pi):
\end{eulerprompt}
\eulerimg{17}{images/EMT4Kalkulus-Naela Rizqy Arofah-22305144042-101.png}
\begin{eulerprompt}
>$showev('integrate(sin(x),x,a,b))
\end{eulerprompt}
\begin{eulerformula}
\[
\int_{a}^{b}{\sin x\;dx}=\cos a-\cos b
\]
\end{eulerformula}
\begin{eulerprompt}
>$showev('integrate(x^n,x,a,b))
\end{eulerprompt}
\begin{euleroutput}
  Answering "Is n positive, negative or zero?" with "positive"
\end{euleroutput}
\begin{eulerformula}
\[
\int_{a}^{b}{x^{n}\;dx}=\frac{b^{n+1}}{n+1}-\frac{a^{n+1}}{n+1}
\]
\end{eulerformula}
\begin{eulerprompt}
>$showev('integrate(x^2*sqrt(2*x+1),x))
\end{eulerprompt}
\begin{eulerformula}
\[
\int {x^2\,\sqrt{2\,x+1}}{\;dx}=\frac{\left(2\,x+1\right)^{\frac{7
 }{2}}}{28}-\frac{\left(2\,x+1\right)^{\frac{5}{2}}}{10}+\frac{\left(
 2\,x+1\right)^{\frac{3}{2}}}{12}
\]
\end{eulerformula}
\begin{eulerprompt}
>$showev('integrate(x^2*sqrt(2*x+1),x,0,2))
\end{eulerprompt}
\begin{eulerformula}
\[
\int_{0}^{2}{x^2\,\sqrt{2\,x+1}\;dx}=\frac{2\,5^{\frac{5}{2}}}{21}-
 \frac{2}{105}
\]
\end{eulerformula}
\begin{eulerprompt}
>$ratsimp(%)
\end{eulerprompt}
\begin{eulerformula}
\[
\int_{0}^{2}{x^2\,\sqrt{2\,x+1}\;dx}=\frac{2\,5^{\frac{7}{2}}-2}{
 105}
\]
\end{eulerformula}
\begin{eulerprompt}
>$showev('integrate((sin(sqrt(x)+a)*E^sqrt(x))/sqrt(x),x,0,pi^2))
\end{eulerprompt}
\begin{eulerformula}
\[
\int_{0}^{\pi^2}{\frac{\sin \left(\sqrt{x}+a\right)\,e^{\sqrt{x}}}{
 \sqrt{x}}\;dx}=\left(-e^{\pi}-1\right)\,\sin a+\left(e^{\pi}+1
 \right)\,\cos a
\]
\end{eulerformula}
\begin{eulerprompt}
>$factor(%)
\end{eulerprompt}
\begin{eulerformula}
\[
\int_{0}^{\pi^2}{\frac{\sin \left(\sqrt{x}+a\right)\,e^{\sqrt{x}}}{
 \sqrt{x}}\;dx}=\left(-e^{\pi}-1\right)\,\left(\sin a-\cos a\right)
\]
\end{eulerformula}
\begin{eulerprompt}
>function map f(x) &= E^(-x^2)
\end{eulerprompt}
\begin{euleroutput}
  
                                      2
                                   - x
                                  E
  
\end{euleroutput}
\begin{eulerprompt}
>$showev('integrate(f(x),x))
\end{eulerprompt}
\begin{eulerformula}
\[
\int {e^ {- x^2 }}{\;dx}=\frac{\sqrt{\pi}\,\mathrm{erf}\left(x
 \right)}{2}
\]
\end{eulerformula}
\begin{eulercomment}
Fungsi f tidak memiliki antiturunan, integralnya masih memuat integral lain.

\end{eulercomment}
\begin{eulerformula}
\[
erf(x) = \int \frac{e^{-x^2}}{\sqrt{\pi}} \ dx.
\]
\end{eulerformula}
\begin{eulercomment}
Kita tidak dapat menggunakan teorema Dasar kalkulus untuk menghitung integral tentu fungsi tersebut jika semua batasnya berhingga.
Dalam hal ini dapat digunakan metode numerik (rumus kuadratur).

Misalkan kita akan menghitung:

maxima: 'integrate(f(x),x,0,pi)
\end{eulercomment}
\begin{eulerprompt}
>x=0:0.1:pi-0.1; plot2d(x,f(x+0.1),>bar); plot2d("f(x)",0,pi,>add):
\end{eulerprompt}
\eulerimg{17}{images/EMT4Kalkulus-Naela Rizqy Arofah-22305144042-110.png}
\begin{eulercomment}
Integral tentu

maxima: 'integrate(f(x),x,0,pi)

dapat dihampiri dengan jumlah luas persegi-persegi panjang di bawah kurva y=f(x)
tersebut. Langkah-langkahnya adalah sebagai berikut.
\end{eulercomment}
\begin{eulerprompt}
>t &= makelist(a,a,0,pi-0.1,0.1); // t sebagai list untuk menyimpan nilai-nilai x
>fx &= makelist(f(t[i]+0.1),i,1,length(t)); // simpan nilai-nilai f(x)
>// jangan menggunakan x sebagai list, kecuali Anda pakar Maxima!
\end{eulerprompt}
\begin{eulercomment}
Hasilnya adalah:

maxima: 'integrate(f(x),x,0,pi) = 0.1*sum(fx[i],i,1,length(fx))

Jumlah tersebut diperoleh dari hasil kali lebar sub-subinterval (=0.1) dan jumlah nilai-nilai f(x) untuk
x = 0.1, 0.2, 0.3, ..., 3.2.
\end{eulercomment}
\begin{eulerprompt}
>0.1*sum(f(x+0.1)) // cek langsung dengan perhitungan numerik EMT
\end{eulerprompt}
\begin{euleroutput}
  0.836219610253
\end{euleroutput}
\begin{eulercomment}
Untuk mendapatkan nilai integral tentu yang mendekati nilai sebenarnya, lebar
sub-intervalnya dapat diperkecil lagi, sehingga daerah di bawah kurva tertutup
semuanya, misalnya dapat digunakan lebar subinterval 0.001. (Silakan dicoba!)

Meskipun Maxima tidak dapat menghitung integral tentu fungsi tersebut untuk
batas-batas yang berhingga, namun integral tersebut dapat dihitung secara eksak jika
batas-batasnya tak hingga. Ini adalah salah satu keajaiban di dalam matematika, yang
terbatas tidak dapat dihitung secara eksak, namun yang tak hingga malah dapat
dihitung secara eksak.
\end{eulercomment}
\begin{eulerprompt}
>$showev('integrate(f(x),x,0,inf))
\end{eulerprompt}
\begin{eulerformula}
\[
\int_{0}^{\infty }{e^ {- x^2 }\;dx}=\frac{\sqrt{\pi}}{2}
\]
\end{eulerformula}
\begin{eulercomment}
Tunjukkan kebenaran hasil di atas!

Berikut adalah contoh lain fungsi yang tidak memiliki antiderivatif, sehingga integral tentunya hanya
dapat dihitung dengan metode numerik.
\end{eulercomment}
\begin{eulerprompt}
>function f(x) &= x^x
\end{eulerprompt}
\begin{euleroutput}
  
                                     x
                                    x
  
\end{euleroutput}
\begin{eulerprompt}
>$showev('integrate(f(x),x,0,1))
\end{eulerprompt}
\begin{eulerformula}
\[
\int_{0}^{1}{x^{x}\;dx}=\int_{0}^{1}{x^{x}\;dx}
\]
\end{eulerformula}
\begin{eulerprompt}
>x=0:0.1:1-0.01; plot2d(x,f(x+0.01),>bar); plot2d("f(x)",0,1,>add):
\end{eulerprompt}
\eulerimg{17}{images/EMT4Kalkulus-Naela Rizqy Arofah-22305144042-113.png}
\begin{eulercomment}
Maxima gagal menghitung integral tentu tersebut secara langsung menggunakan perintah
integrate. Berikut kita lakukan seperti contoh sebelumnya untuk mendapat hasil atau
pendekatan nilai integral tentu tersebut.
\end{eulercomment}
\begin{eulerprompt}
>t &= makelist(a,a,0,1-0.01,0.01);
>fx &= makelist(f(t[i]+0.01),i,1,length(t));
\end{eulerprompt}
\begin{eulercomment}
maxima: 'integrate(f(x),x,0,1) = 0.01*sum(fx[i],i,1,length(fx))

Apakah hasil tersebut cukup baik? perhatikan gambarnya.
\end{eulercomment}
\begin{eulerprompt}
>function f(x) &= sin(3*x^5+7)^2
\end{eulerprompt}
\begin{euleroutput}
  
                                 2    5
                              sin (3 x  + 7)
  
\end{euleroutput}
\begin{eulerprompt}
>integrate(f,0,1)
\end{eulerprompt}
\begin{euleroutput}
  0.542581176074
\end{euleroutput}
\begin{eulerprompt}
>&showev('integrate(f(x),x,0,1))
\end{eulerprompt}
\begin{euleroutput}
  
           1                           1              pi
          /                      gamma(-) sin(14) sin(--)
          [     2    5                 5              10
          I  sin (3 x  + 7) dx = ------------------------
          ]                                  1/5
          /                              10 6
           0
         4/5                  1          4/5                  1
   - (((6    gamma_incomplete(-, 6 I) + 6    gamma_incomplete(-, - 6 I))
                              5                               5
               4/5                    1
   sin(14) + (6    I gamma_incomplete(-, 6 I)
                                      5
      4/5                    1                       pi
   - 6    I gamma_incomplete(-, - 6 I)) cos(14)) sin(--) - 60)/120
                             5                       10
  
\end{euleroutput}
\begin{eulerprompt}
>&float(%)
\end{eulerprompt}
\begin{euleroutput}
  
           1.0
          /
          [       2      5
          I    sin (3.0 x  + 7.0) dx = 
          ]
          /
           0.0
  0.09820784258795788 - 0.008333333333333333
   (0.3090169943749474 (0.1367372182078336
   (4.192962712629476 I gamma__incomplete(0.2, 6.0 I)
   - 4.192962712629476 I gamma__incomplete(0.2, - 6.0 I))
   + 0.9906073556948704 (4.192962712629476 gamma__incomplete(0.2, 6.0 I)
   + 4.192962712629476 gamma__incomplete(0.2, - 6.0 I))) - 60.0)
  
\end{euleroutput}
\begin{eulerprompt}
>$showev('integrate(x*exp(-x),x,0,1)) // Integral tentu (eksak)
\end{eulerprompt}
\begin{eulerformula}
\[
\int_{0}^{1}{x\,e^ {- x }\;dx}=1-2\,e^ {- 1 }
\]
\end{eulerformula}
\eulerheading{Aplikasi Integral Tentu}
\begin{eulerprompt}
>plot2d("x^3-x",-0.1,1.1); plot2d("-x^2",>add);  ...
>b=solve("x^3-x+x^2",0.5); x=linspace(0,b,200); xi=flipx(x); ...
>plot2d(x|xi,x^3-x|-xi^2,>filled,style="|",fillcolor=1,>add): // Plot daerah antara 2 kurva
\end{eulerprompt}
\eulerimg{17}{images/EMT4Kalkulus-Naela Rizqy Arofah-22305144042-115.png}
\begin{eulerprompt}
>a=solve("x^3-x+x^2",0), b=solve("x^3-x+x^2",1) // absis titik-titik potong kedua kurva
\end{eulerprompt}
\begin{euleroutput}
  0
  0.61803398875
\end{euleroutput}
\begin{eulerprompt}
>integrate("(-x^2)-(x^3-x)",a,b) // luas daerah yang diarsir
\end{eulerprompt}
\begin{euleroutput}
  0.0758191713542
\end{euleroutput}
\begin{eulercomment}
Hasil tersebut akan kita bandingkan dengan perhitungan secara analitik.
\end{eulercomment}
\begin{eulerprompt}
>a &= solve((-x^2)-(x^3-x),x); $a // menentukan absis titik potong kedua kurva secara eksak
\end{eulerprompt}
\begin{eulerformula}
\[
\left[ x=\frac{-\sqrt{5}-1}{2} , x=\frac{\sqrt{5}-1}{2} , x=0
  \right] 
\]
\end{eulerformula}
\begin{eulerprompt}
>$showev('integrate(-x^2-x^3+x,x,0,(sqrt(5)-1)/2)) // Nilai integral secara eksak
\end{eulerprompt}
\begin{eulerformula}
\[
\int_{0}^{\frac{\sqrt{5}-1}{2}}{-x^3-x^2+x\;dx}=\frac{13-5^{\frac{3
 }{2}}}{24}
\]
\end{eulerformula}
\begin{eulerprompt}
>$float(%)
\end{eulerprompt}
\begin{eulerformula}
\[
\int_{0.0}^{0.6180339887498949}{-1.0\,x^3-1.0\,x^2+x\;dx}=
 0.07581917135421037
\]
\end{eulerformula}
\eulersubheading{Panjang Kurva}
\begin{eulercomment}
Hitunglah panjang kurva berikut ini dan luas daerah di dalam kurva tersebut.

\end{eulercomment}
\begin{eulerformula}
\[
\gamma(t) = (r(t) \cos(t), r(t) \sin(t))
\]
\end{eulerformula}
\begin{eulercomment}
dengan

\end{eulercomment}
\begin{eulerformula}
\[
r(t) = 1 + \dfrac{\sin(3t)}{2},\quad 0\le t\le 2\pi.
\]
\end{eulerformula}
\begin{eulerprompt}
>t=linspace(0,2pi,1000); r=1+sin(3*t)/2; x=r*cos(t); y=r*sin(t); ...
>plot2d(x,y,>filled,fillcolor=red,style="/",r=1.5): // Kita gambar kurvanya terlebih dahulu
\end{eulerprompt}
\eulerimg{17}{images/EMT4Kalkulus-Naela Rizqy Arofah-22305144042-119.png}
\begin{eulerprompt}
>function r(t) &= 1+sin(3*t)/2; $'r(t)=r(t)
\end{eulerprompt}
\begin{eulerformula}
\[
r\left(\left[ 0 , 0.01 , 0.02 , 0.03 , 0.04 , 0.05 , 0.06 , 0.07 , 
 0.08 , 0.09 , 0.1 , 0.11 , 0.12 , 0.13 , 0.14 , 0.15 , 0.16 , 0.17
  , 0.18 , 0.19 , 0.2 , 0.21 , 0.2200000000000001 , 
 0.2300000000000001 , 0.2400000000000001 , 0.2500000000000001 , 
 0.2600000000000001 , 0.2700000000000001 , 0.2800000000000001 , 
 0.2900000000000001 , 0.3000000000000001 , 0.3100000000000001 , 
 0.3200000000000001 , 0.3300000000000001 , 0.3400000000000001 , 
 0.3500000000000001 , 0.3600000000000002 , 0.3700000000000002 , 
 0.3800000000000002 , 0.3900000000000002 , 0.4000000000000002 , 
 0.4100000000000002 , 0.4200000000000002 , 0.4300000000000002 , 
 0.4400000000000002 , 0.4500000000000002 , 0.4600000000000002 , 
 0.4700000000000003 , 0.4800000000000003 , 0.4900000000000003 , 
 0.5000000000000002 , 0.5100000000000002 , 0.5200000000000002 , 
 0.5300000000000002 , 0.5400000000000003 , 0.5500000000000003 , 
 0.5600000000000003 , 0.5700000000000003 , 0.5800000000000003 , 
 0.5900000000000003 , 0.6000000000000003 , 0.6100000000000003 , 
 0.6200000000000003 , 0.6300000000000003 , 0.6400000000000003 , 
 0.6500000000000004 , 0.6600000000000004 , 0.6700000000000004 , 
 0.6800000000000004 , 0.6900000000000004 , 0.7000000000000004 , 
 0.7100000000000004 , 0.7200000000000004 , 0.7300000000000004 , 
 0.7400000000000004 , 0.7500000000000004 , 0.7600000000000005 , 
 0.7700000000000005 , 0.7800000000000005 , 0.7900000000000005 , 
 0.8000000000000005 , 0.8100000000000005 , 0.8200000000000005 , 
 0.8300000000000005 , 0.8400000000000005 , 0.8500000000000005 , 
 0.8600000000000005 , 0.8700000000000006 , 0.8800000000000006 , 
 0.8900000000000006 , 0.9000000000000006 , 0.9100000000000006 , 
 0.9200000000000006 , 0.9300000000000006 , 0.9400000000000006 , 
 0.9500000000000006 , 0.9600000000000006 , 0.9700000000000006 , 
 0.9800000000000006 , 0.9900000000000007 \right] \right)=\left[ 1 , 
 1.014997750101248 , 1.029982003239722 , 1.044939274599006 , 
 1.05985610364446 , 1.0747190662368 , 1.089514786712912 , 
 1.10422994992305 , 1.118851313213567 , 1.133365718344415 , 
 1.14776010333067 , 1.162021514197434 , 1.176137116637545 , 
 1.190094207561581 , 1.203880226529785 , 1.217482767055615 , 
 1.230889587770742 , 1.244088623441454 , 1.257067995826556 , 
 1.269816024366985 , 1.282321236697518 , 1.294572378971135 , 
 1.306558425986717 , 1.318268591110984 , 1.329692335985737 , 
 1.340819380011667 , 1.351639709600205 , 1.362143587185071 , 
 1.37232155998543 , 1.382164468512753 , 1.391663454813742 , 
 1.400809970441889 , 1.409595784150499 , 1.41801298930026 , 
 1.426054010974682 , 1.433711612797009 , 1.440978903442474 , 
 1.447849342840024 , 1.454316748057942 , 1.460375298868068 , 
 1.466019542983613 , 1.471244400965849 , 1.476045170795258 , 
 1.480417532103036 , 1.484357550059133 , 1.48786167891333 , 
 1.49092676518618 , 1.493550050506925 , 1.495729174095843 , 
 1.49746217488879 , 1.498747493302027 , 1.499583972635738 , 
 1.499970860114983 , 1.499907807567145 , 1.499394871735262 , 
 1.498432514226959 , 1.497021601099038 , 1.495163402078079 , 
 1.492859589417777 , 1.490112236394023 , 1.486923815439098 , 
 1.483297195916649 , 1.479235641539457 , 1.474742807432315 , 
 1.469822736842662 , 1.464479857501934 , 1.458718977640905 , 
 1.4525452816626 , 1.44596432547669 , 1.438982031499539 , 
 1.431604683324436 , 1.423838920066784 , 1.415691730389341 , 
 1.407170446212898 , 1.398282736118043 , 1.38903659844396 , 
 1.379440354090461 , 1.369502639029735 , 1.359232396534563 , 
 1.348638869129968 , 1.337731590275575 , 1.326520375786132 , 
 1.315015314997945 , 1.303226761689157 , 1.29116532476204 , 
 1.278841858695708 , 1.26626745377781 , 1.253453426124026 , 
 1.240411307494323 , 1.227152834915152 , 1.213689940116914 , 
 1.200034738796209 , 1.186199519712527 , 1.172196733629194 , 
 1.158038982108526 , 1.143739006171271 , 1.129309674830555 , 
 1.114763973510631 , 1.100114992360884 , 1.085375914475572 \right] 
\]
\end{eulerformula}
\begin{eulerprompt}
>function fx(t) &= r(t)*cos(t); $'fx(t)=fx(t)
\end{eulerprompt}
\begin{eulerformula}
\[
{\it fx}\left(\left[ 0 , 0.01 , 0.02 , 0.03 , 0.04 , 0.05 , 0.06 , 
 0.07 , 0.08 , 0.09 , 0.1 , 0.11 , 0.12 , 0.13 , 0.14 , 0.15 , 0.16
  , 0.17 , 0.18 , 0.19 , 0.2 , 0.21 , 0.2200000000000001 , 
 0.2300000000000001 , 0.2400000000000001 , 0.2500000000000001 , 
 0.2600000000000001 , 0.2700000000000001 , 0.2800000000000001 , 
 0.2900000000000001 , 0.3000000000000001 , 0.3100000000000001 , 
 0.3200000000000001 , 0.3300000000000001 , 0.3400000000000001 , 
 0.3500000000000001 , 0.3600000000000002 , 0.3700000000000002 , 
 0.3800000000000002 , 0.3900000000000002 , 0.4000000000000002 , 
 0.4100000000000002 , 0.4200000000000002 , 0.4300000000000002 , 
 0.4400000000000002 , 0.4500000000000002 , 0.4600000000000002 , 
 0.4700000000000003 , 0.4800000000000003 , 0.4900000000000003 , 
 0.5000000000000002 , 0.5100000000000002 , 0.5200000000000002 , 
 0.5300000000000002 , 0.5400000000000003 , 0.5500000000000003 , 
 0.5600000000000003 , 0.5700000000000003 , 0.5800000000000003 , 
 0.5900000000000003 , 0.6000000000000003 , 0.6100000000000003 , 
 0.6200000000000003 , 0.6300000000000003 , 0.6400000000000003 , 
 0.6500000000000004 , 0.6600000000000004 , 0.6700000000000004 , 
 0.6800000000000004 , 0.6900000000000004 , 0.7000000000000004 , 
 0.7100000000000004 , 0.7200000000000004 , 0.7300000000000004 , 
 0.7400000000000004 , 0.7500000000000004 , 0.7600000000000005 , 
 0.7700000000000005 , 0.7800000000000005 , 0.7900000000000005 , 
 0.8000000000000005 , 0.8100000000000005 , 0.8200000000000005 , 
 0.8300000000000005 , 0.8400000000000005 , 0.8500000000000005 , 
 0.8600000000000005 , 0.8700000000000006 , 0.8800000000000006 , 
 0.8900000000000006 , 0.9000000000000006 , 0.9100000000000006 , 
 0.9200000000000006 , 0.9300000000000006 , 0.9400000000000006 , 
 0.9500000000000006 , 0.9600000000000006 , 0.9700000000000006 , 
 0.9800000000000006 , 0.9900000000000007 \right] \right)=\left[ 1 , 
 1.014947000636657 , 1.029776013705529 , 1.044469087191079 , 
 1.059008331806833 , 1.073375947255439 , 1.087554248364218 , 
 1.101525691055367 , 1.11527289811021 , 1.128778684687222 , 
 1.142026083553954 , 1.154998369993414 , 1.16767908634602 , 
 1.180052066148761 , 1.192101457833886 , 1.203811747950136 , 
 1.215167783870255 , 1.226154795949382 , 1.236758419099762 , 
 1.246964713748154 , 1.256760186143285 , 1.266131807981756 , 
 1.275067035321848 , 1.283553826755846 , 1.29158066081265 , 
 1.29913655256367 , 1.306211069406282 , 1.312794346000405 , 
 1.318877098335118 , 1.324450636903608 , 1.329506878966172 , 
 1.334038359882425 , 1.338038243495345 , 1.341500331551311 , 
 1.344419072141793 , 1.346789567153917 , 1.348607578718725 , 
 1.349869534647481 , 1.350572532848044 , 1.350714344714907 , 
 1.350293417488142 , 1.349308875578123 , 1.347760520854542 , 
 1.345648831899879 , 1.342974962229111 , 1.339740737479097 , 
 1.335948651572729 , 1.331601861864506 , 1.326704183275865 , 
 1.321260081430156 , 1.315274664798767 , 1.308753675871437 , 
 1.301703481365363 , 1.294131061489226 , 1.286043998279732 , 
 1.277450463029762 , 1.268359202828647 , 1.25877952623647 , 
 1.248721288115691 , 1.238194873644713 , 1.227211181539273 , 
 1.215781606508839 , 1.203918020976346 , 1.191632756090801 , 
 1.17893858206338 , 1.165848687858719 , 1.152376660274093 , 
 1.138536462440146 , 1.124342411777761 , 1.10980915744646 , 
 1.094951657320579 , 1.079785154530145 , 1.064325153604093 , 
 1.04858739625406 , 1.032587836837555 , 1.0163426175398 , 
 0.999868043313951 , 0.9831805566197906 , 0.9662967120012925 , 
 0.9492331505436565 , 0.932006574250646 , 0.9146337203831 , 
 0.897131335799599 , 0.8795161513401855 , 0.8618048562939812 , 
 0.8440140729913906 , 0.8261603315613344 , 0.8082600448937051 , 
 0.7903294838468643 , 0.7723847527396025 , 0.754441765166499 , 
 0.7365162201750889 , 0.7186235788426429 , 0.7007790412897039 , 
 0.6829975241668103 , 0.6652936386500562 , 0.6476816689803099 , 
 0.6301755515800127 , 0.6127888547805567 , 0.595534759192214 \right] 
\]
\end{eulerformula}
\begin{eulerprompt}
>function fy(t) &= r(t)*sin(t); $'fy(t)=fy(t)
\end{eulerprompt}
\begin{eulerformula}
\[
{\it fy}\left(\left[ 0 , 0.01 , 0.02 , 0.03 , 0.04 , 0.05 , 0.06 , 
 0.07 , 0.08 , 0.09 , 0.1 , 0.11 , 0.12 , 0.13 , 0.14 , 0.15 , 0.16
  , 0.17 , 0.18 , 0.19 , 0.2 , 0.21 , 0.2200000000000001 , 
 0.2300000000000001 , 0.2400000000000001 , 0.2500000000000001 , 
 0.2600000000000001 , 0.2700000000000001 , 0.2800000000000001 , 
 0.2900000000000001 , 0.3000000000000001 , 0.3100000000000001 , 
 0.3200000000000001 , 0.3300000000000001 , 0.3400000000000001 , 
 0.3500000000000001 , 0.3600000000000002 , 0.3700000000000002 , 
 0.3800000000000002 , 0.3900000000000002 , 0.4000000000000002 , 
 0.4100000000000002 , 0.4200000000000002 , 0.4300000000000002 , 
 0.4400000000000002 , 0.4500000000000002 , 0.4600000000000002 , 
 0.4700000000000003 , 0.4800000000000003 , 0.4900000000000003 , 
 0.5000000000000002 , 0.5100000000000002 , 0.5200000000000002 , 
 0.5300000000000002 , 0.5400000000000003 , 0.5500000000000003 , 
 0.5600000000000003 , 0.5700000000000003 , 0.5800000000000003 , 
 0.5900000000000003 , 0.6000000000000003 , 0.6100000000000003 , 
 0.6200000000000003 , 0.6300000000000003 , 0.6400000000000003 , 
 0.6500000000000004 , 0.6600000000000004 , 0.6700000000000004 , 
 0.6800000000000004 , 0.6900000000000004 , 0.7000000000000004 , 
 0.7100000000000004 , 0.7200000000000004 , 0.7300000000000004 , 
 0.7400000000000004 , 0.7500000000000004 , 0.7600000000000005 , 
 0.7700000000000005 , 0.7800000000000005 , 0.7900000000000005 , 
 0.8000000000000005 , 0.8100000000000005 , 0.8200000000000005 , 
 0.8300000000000005 , 0.8400000000000005 , 0.8500000000000005 , 
 0.8600000000000005 , 0.8700000000000006 , 0.8800000000000006 , 
 0.8900000000000006 , 0.9000000000000006 , 0.9100000000000006 , 
 0.9200000000000006 , 0.9300000000000006 , 0.9400000000000006 , 
 0.9500000000000006 , 0.9600000000000006 , 0.9700000000000006 , 
 0.9800000000000006 , 0.9900000000000007 \right] \right)=\left[ 0 , 
 0.01014980833556662 , 0.02059826678292271 , 0.03134347622283015 , 
 0.04238293991838228 , 0.05371356612987439 , 0.06533167172990376 , 
 0.07723298681299934 , 0.08941266029246918 , 0.1018652664755576 , 
 0.1145848126064173 , 0.1275647473648353 , 0.1407979703071057 , 
 0.1542768422339107 , 0.1679931964685752 , 0.1819383510275811 , 
 0.1961031216637831 , 0.2104778357613507 , 0.2250523470600841 , 
 0.2398160511854019 , 0.2547579019589912 , 0.2698664284638497 , 
 0.2851297528362152 , 0.3005356087557041 , 0.3160713606038417 , 
 0.3317240232600813 , 0.3474802825033731 , 0.3633265159863522 , 
 0.3792488147482899 , 0.3952330052320643 , 0.411264671769591 , 
 0.4273291794993832 , 0.4434116976792021 , 0.4594972233561165 , 
 0.4755706053556919 , 0.4916165685515136 , 0.5076197383757777 , 
 0.5235646655312819 , 0.5394358508648145 , 0.5552177703616642 , 
 0.5708949002207642 , 0.5864517419698421 , 0.6018728475798654 , 
 0.6171428445380648 , 0.6322464608388652 , 0.6471685498521687 , 
 0.6618941150286309 , 0.6764083344018014 , 0.6906965848473219 , 
 0.704744466059751 , 0.7185378242080237 , 0.7320627752310482 , 
 0.7453057277355214 , 0.7582534054586558 , 0.7708928692592016 , 
 0.7832115386008901 , 0.7951972124932317 , 0.8068380898554457 , 
 0.8181227892702304 , 0.8290403680950348 , 0.8395803408995157 , 
 0.8497326971989371 , 0.8594879184543822 , 0.8688369943118147 , 
 0.877771438053233 , 0.8862833012344233 , 0.894365187485098 , 
 0.9020102654485477 , 0.9092122808393135 , 0.91596556759876 , 
 0.9222650581299157 , 0.9281062925943645 , 0.9334854272555032 , 
 0.9383992418539865 , 0.9428451460027243 , 0.9468211845903713 , 
 0.9503260421838114 , 0.9533590464217597 , 0.9559201703932094 , 
 0.9580100339960551 , 0.9596299042728891 , 0.9607816947225576 , 
 0.9614679635877484 , 0.9616919111204768 , 0.9614573758289937 , 
 0.9607688297112769 , 0.9596313724818526 , 0.9580507248003547 , 
 0.9560332205117796 , 0.9535857979100135 , 0.950715990037748 , 
 0.9474319140374602 , 0.9437422595696462 , 0.9396562763159917 , 
 0.9351837605866338 , 0.9303350410521015 , 0.9251209636219332 , 
 0.9195528754933222 , 0.9136426083945087 , 0.9074024610488752
  \right] 
\]
\end{eulerformula}
\begin{eulerprompt}
>function ds(t) &= trigreduce(radcan(sqrt(diff(fx(t),t)^2+diff(fy(t),t)^2))); $'ds(t)=ds(t)
\end{eulerprompt}
\begin{euleroutput}
  Maxima said:
  diff: second argument must be a variable; found errexp1
   -- an error. To debug this try: debugmode(true);
  
  Error in:
  ... e(radcan(sqrt(diff(fx(t),t)^2+diff(fy(t),t)^2))); $'ds(t)=ds(t ...
                                                       ^
\end{euleroutput}
\begin{eulerprompt}
>$integrate(ds(x),x,0,2*pi) //panjang (keliling) kurva
\end{eulerprompt}
\begin{eulerformula}
\[
\int_{0}^{2\,\pi}{{\it ds}\left(x\right)\;dx}
\]
\end{eulerformula}
\begin{eulercomment}
Maxima gagal melakukan perhitungan eksak integral tersebut.

Berikut kita hitung integralnya secara umerik dengan perintah EMT.
\end{eulercomment}
\begin{eulerprompt}
>integrate("ds(x)",0,2*pi)
\end{eulerprompt}
\begin{euleroutput}
  Function ds not found.
  Try list ... to find functions!
  Error in expression: ds(x)
  %mapexpression1:
      return expr(x,args());
  Error in map.
  %evalexpression:
      if maps then return %mapexpression1(x,f$;args());
  gauss:
      if maps then y=%evalexpression(f$,a+h-(h*xn)',maps;args());
  adaptivegauss:
      t1=gauss(f$,c,c+h;args(),=maps);
  Try "trace errors" to inspect local variables after errors.
  integrate:
      return adaptivegauss(f$,a,b,eps*1000;args(),=maps);
\end{euleroutput}
\begin{eulercomment}
Spiral Logaritmik

\end{eulercomment}
\begin{eulerformula}
\[
x=e^{ax}\cos x,\ y=e^{ax}\sin x.
\]
\end{eulerformula}
\begin{eulerprompt}
>a=0.1; plot2d("exp(a*x)*cos(x)","exp(a*x)*sin(x)",r=2,xmin=0,xmax=2*pi):
\end{eulerprompt}
\eulerimg{17}{images/EMT4Kalkulus-Naela Rizqy Arofah-22305144042-124.png}
\begin{eulerprompt}
>&kill(a) // hapus expresi a
\end{eulerprompt}
\begin{euleroutput}
  
                                   done
  
\end{euleroutput}
\begin{eulerprompt}
>function fx(t) &= exp(a*t)*cos(t); $'fx(t)=fx(t)
\end{eulerprompt}
\begin{eulerformula}
\[
{\it fx}\left(\left[ 0 , 0.01 , 0.02 , 0.03 , 0.04 , 0.05 , 0.06 , 
 0.07 , 0.08 , 0.09 , 0.1 , 0.11 , 0.12 , 0.13 , 0.14 , 0.15 , 0.16
  , 0.17 , 0.18 , 0.19 , 0.2 , 0.21 , 0.2200000000000001 , 
 0.2300000000000001 , 0.2400000000000001 , 0.2500000000000001 , 
 0.2600000000000001 , 0.2700000000000001 , 0.2800000000000001 , 
 0.2900000000000001 , 0.3000000000000001 , 0.3100000000000001 , 
 0.3200000000000001 , 0.3300000000000001 , 0.3400000000000001 , 
 0.3500000000000001 , 0.3600000000000002 , 0.3700000000000002 , 
 0.3800000000000002 , 0.3900000000000002 , 0.4000000000000002 , 
 0.4100000000000002 , 0.4200000000000002 , 0.4300000000000002 , 
 0.4400000000000002 , 0.4500000000000002 , 0.4600000000000002 , 
 0.4700000000000003 , 0.4800000000000003 , 0.4900000000000003 , 
 0.5000000000000002 , 0.5100000000000002 , 0.5200000000000002 , 
 0.5300000000000002 , 0.5400000000000003 , 0.5500000000000003 , 
 0.5600000000000003 , 0.5700000000000003 , 0.5800000000000003 , 
 0.5900000000000003 , 0.6000000000000003 , 0.6100000000000003 , 
 0.6200000000000003 , 0.6300000000000003 , 0.6400000000000003 , 
 0.6500000000000004 , 0.6600000000000004 , 0.6700000000000004 , 
 0.6800000000000004 , 0.6900000000000004 , 0.7000000000000004 , 
 0.7100000000000004 , 0.7200000000000004 , 0.7300000000000004 , 
 0.7400000000000004 , 0.7500000000000004 , 0.7600000000000005 , 
 0.7700000000000005 , 0.7800000000000005 , 0.7900000000000005 , 
 0.8000000000000005 , 0.8100000000000005 , 0.8200000000000005 , 
 0.8300000000000005 , 0.8400000000000005 , 0.8500000000000005 , 
 0.8600000000000005 , 0.8700000000000006 , 0.8800000000000006 , 
 0.8900000000000006 , 0.9000000000000006 , 0.9100000000000006 , 
 0.9200000000000006 , 0.9300000000000006 , 0.9400000000000006 , 
 0.9500000000000006 , 0.9600000000000006 , 0.9700000000000006 , 
 0.9800000000000006 , 0.9900000000000007 \right] \right)=\left[ 1 , 
 0.9999500004166653\,e^{0.01\,a} , 0.9998000066665778\,e^{0.02\,a} , 
 0.9995500337489875\,e^{0.03\,a} , 0.9992001066609779\,e^{0.04\,a} , 
 0.9987502603949663\,e^{0.05\,a} , 0.9982005399352042\,e^{0.06\,a} , 
 0.9975510002532796\,e^{0.07\,a} , 0.9968017063026194\,e^{0.08\,a} , 
 0.9959527330119943\,e^{0.09\,a} , 0.9950041652780258\,e^{0.1\,a} , 
 0.9939560979566968\,e^{0.11\,a} , 0.9928086358538663\,e^{0.12\,a} , 
 0.9915618937147881\,e^{0.13\,a} , 0.9902159962126372\,e^{0.14\,a} , 
 0.9887710779360422\,e^{0.15\,a} , 0.9872272833756269\,e^{0.16\,a} , 
 0.9855847669095608\,e^{0.17\,a} , 0.9838436927881214\,e^{0.18\,a} , 
 0.9820042351172703\,e^{0.19\,a} , 0.9800665778412416\,e^{0.2\,a} , 
 0.9780309147241483\,e^{0.21\,a} , 0.9758974493306055\,e^{
 0.2200000000000001\,a} , 0.9736663950053748\,e^{0.2300000000000001\,
 a} , 0.9713379748520296\,e^{0.2400000000000001\,a} , 
 0.9689124217106447\,e^{0.2500000000000001\,a} , 0.9663899781345132\,
 e^{0.2600000000000001\,a} , 0.9637708963658905\,e^{
 0.2700000000000001\,a} , 0.9610554383107709\,e^{0.2800000000000001\,
 a} , 0.9582438755126972\,e^{0.2900000000000001\,a} , 
 0.955336489125606\,e^{0.3000000000000001\,a} , 0.9523335698857134\,e
 ^{0.3100000000000001\,a} , 0.9492354180824408\,e^{0.3200000000000001
 \,a} , 0.9460423435283869\,e^{0.3300000000000001\,a} , 
 0.9427546655283462\,e^{0.3400000000000001\,a} , 0.9393727128473789\,
 e^{0.3500000000000001\,a} , 0.9358968236779348\,e^{
 0.3600000000000002\,a} , 0.9323273456060344\,e^{0.3700000000000002\,
 a} , 0.9286646355765101\,e^{0.3800000000000002\,a} , 
 0.924909059857313\,e^{0.3900000000000002\,a} , 0.921060994002885\,e
 ^{0.4000000000000002\,a} , 0.917120822816605\,e^{0.4100000000000002
 \,a} , 0.9130889403123081\,e^{0.4200000000000002\,a} , 
 0.9089657496748851\,e^{0.4300000000000002\,a} , 0.9047516632199634\,
 e^{0.4400000000000002\,a} , 0.9004471023526768\,e^{
 0.4500000000000002\,a} , 0.8960524975255252\,e^{0.4600000000000002\,
 a} , 0.8915682881953289\,e^{0.4700000000000003\,a} , 
 0.886994922779284\,e^{0.4800000000000003\,a} , 0.8823328586101213\,e
 ^{0.4900000000000003\,a} , 0.8775825618903726\,e^{0.5000000000000002
 \,a} , 0.8727445076457512\,e^{0.5100000000000002\,a} , 
 0.8678191796776498\,e^{0.5200000000000002\,a} , 0.8628070705147609\,
 e^{0.5300000000000002\,a} , 0.857708681363824\,e^{0.5400000000000003
 \,a} , 0.8525245220595056\,e^{0.5500000000000003\,a} , 
 0.847255111013416\,e^{0.5600000000000003\,a} , 0.8419009751622686\,e
 ^{0.5700000000000003\,a} , 0.8364626499151868\,e^{0.5800000000000003
 \,a} , 0.8309406791001633\,e^{0.5900000000000003\,a} , 
 0.8253356149096781\,e^{0.6000000000000003\,a} , 0.8196480178454794\,
 e^{0.6100000000000003\,a} , 0.8138784566625338\,e^{
 0.6200000000000003\,a} , 0.8080275083121516\,e^{0.6300000000000003\,
 a} , 0.8020957578842924\,e^{0.6400000000000003\,a} , 
 0.7960837985490556\,e^{0.6500000000000004\,a} , 0.7899922314973649\,
 e^{0.6600000000000004\,a} , 0.783821665880849\,e^{0.6700000000000004
 \,a} , 0.7775727187509277\,e^{0.6800000000000004\,a} , 
 0.7712460149971063\,e^{0.6900000000000004\,a} , 0.7648421872844882\,
 e^{0.7000000000000004\,a} , 0.7583618759905079\,e^{
 0.7100000000000004\,a} , 0.7518057291408947\,e^{0.7200000000000004\,
 a} , 0.7451744023448701\,e^{0.7300000000000004\,a} , 
 0.7384685587295876\,e^{0.7400000000000004\,a} , 0.7316888688738206\,
 e^{0.7500000000000004\,a} , 0.7248360107409049\,e^{
 0.7600000000000005\,a} , 0.7179106696109431\,e^{0.7700000000000005\,
 a} , 0.7109135380122771\,e^{0.7800000000000005\,a} , 
 0.7038453156522357\,e^{0.7900000000000005\,a} , 0.696706709347165\,e
 ^{0.8000000000000005\,a} , 0.6894984329517466\,e^{0.8100000000000005
 \,a} , 0.6822212072876132\,e^{0.8200000000000005\,a} , 
 0.6748757600712667\,e^{0.8300000000000005\,a} , 0.6674628258413078\,
 e^{0.8400000000000005\,a} , 0.6599831458849817\,e^{
 0.8500000000000005\,a} , 0.6524374681640515\,e^{0.8600000000000005\,
 a} , 0.6448265472400008\,e^{0.8700000000000006\,a} , 
 0.6371511441985798\,e^{0.8800000000000006\,a} , 0.6294120265736964\,
 e^{0.8900000000000006\,a} , 0.6216099682706641\,e^{
 0.9000000000000006\,a} , 0.6137457494888111\,e^{0.9100000000000006\,
 a} , 0.6058201566434623\,e^{0.9200000000000006\,a} , 
 0.5978339822872978\,e^{0.9300000000000006\,a} , 0.5897880250310977\,
 e^{0.9400000000000006\,a} , 0.581683089463883\,e^{0.9500000000000006
 \,a} , 0.5735199860724561\,e^{0.9600000000000006\,a} , 
 0.5652995311603538\,e^{0.9700000000000006\,a} , 0.5570225467662168\,
 e^{0.9800000000000006\,a} , 0.548689860581587\,e^{0.9900000000000007
 \,a} \right] 
\]
\end{eulerformula}
\begin{eulerprompt}
>function fy(t) &= exp(a*t)*sin(t); $'fy(t)=fy(t)
\end{eulerprompt}
\begin{eulerformula}
\[
{\it fy}\left(\left[ 0 , 0.01 , 0.02 , 0.03 , 0.04 , 0.05 , 0.06 , 
 0.07 , 0.08 , 0.09 , 0.1 , 0.11 , 0.12 , 0.13 , 0.14 , 0.15 , 0.16
  , 0.17 , 0.18 , 0.19 , 0.2 , 0.21 , 0.2200000000000001 , 
 0.2300000000000001 , 0.2400000000000001 , 0.2500000000000001 , 
 0.2600000000000001 , 0.2700000000000001 , 0.2800000000000001 , 
 0.2900000000000001 , 0.3000000000000001 , 0.3100000000000001 , 
 0.3200000000000001 , 0.3300000000000001 , 0.3400000000000001 , 
 0.3500000000000001 , 0.3600000000000002 , 0.3700000000000002 , 
 0.3800000000000002 , 0.3900000000000002 , 0.4000000000000002 , 
 0.4100000000000002 , 0.4200000000000002 , 0.4300000000000002 , 
 0.4400000000000002 , 0.4500000000000002 , 0.4600000000000002 , 
 0.4700000000000003 , 0.4800000000000003 , 0.4900000000000003 , 
 0.5000000000000002 , 0.5100000000000002 , 0.5200000000000002 , 
 0.5300000000000002 , 0.5400000000000003 , 0.5500000000000003 , 
 0.5600000000000003 , 0.5700000000000003 , 0.5800000000000003 , 
 0.5900000000000003 , 0.6000000000000003 , 0.6100000000000003 , 
 0.6200000000000003 , 0.6300000000000003 , 0.6400000000000003 , 
 0.6500000000000004 , 0.6600000000000004 , 0.6700000000000004 , 
 0.6800000000000004 , 0.6900000000000004 , 0.7000000000000004 , 
 0.7100000000000004 , 0.7200000000000004 , 0.7300000000000004 , 
 0.7400000000000004 , 0.7500000000000004 , 0.7600000000000005 , 
 0.7700000000000005 , 0.7800000000000005 , 0.7900000000000005 , 
 0.8000000000000005 , 0.8100000000000005 , 0.8200000000000005 , 
 0.8300000000000005 , 0.8400000000000005 , 0.8500000000000005 , 
 0.8600000000000005 , 0.8700000000000006 , 0.8800000000000006 , 
 0.8900000000000006 , 0.9000000000000006 , 0.9100000000000006 , 
 0.9200000000000006 , 0.9300000000000006 , 0.9400000000000006 , 
 0.9500000000000006 , 0.9600000000000006 , 0.9700000000000006 , 
 0.9800000000000006 , 0.9900000000000007 \right] \right)=\left[ 0 , 
 0.009999833334166664\,e^{0.01\,a} , 0.01999866669333308\,e^{0.02\,a}
  , 0.02999550020249566\,e^{0.03\,a} , 0.03998933418663416\,e^{0.04\,
 a} , 0.04997916927067833\,e^{0.05\,a} , 0.0599640064794446\,e^{0.06
 \,a} , 0.06994284733753277\,e^{0.07\,a} , 0.0799146939691727\,e^{
 0.08\,a} , 0.08987854919801104\,e^{0.09\,a} , 0.09983341664682814\,e
 ^{0.1\,a} , 0.1097783008371748\,e^{0.11\,a} , 0.1197122072889193\,e
 ^{0.12\,a} , 0.1296341426196948\,e^{0.13\,a} , 0.1395431146442365\,e
 ^{0.14\,a} , 0.1494381324735992\,e^{0.15\,a} , 0.159318206614246\,e
 ^{0.16\,a} , 0.169182349066996\,e^{0.17\,a} , 0.1790295734258242\,e
 ^{0.18\,a} , 0.1888588949765006\,e^{0.19\,a} , 0.1986693307950612\,e
 ^{0.2\,a} , 0.2084598998460996\,e^{0.21\,a} , 0.2182296230808694\,e
 ^{0.2200000000000001\,a} , 0.2279775235351885\,e^{0.2300000000000001
 \,a} , 0.2377026264271347\,e^{0.2400000000000001\,a} , 
 0.247403959254523\,e^{0.2500000000000001\,a} , 0.2570805518921552\,e
 ^{0.2600000000000001\,a} , 0.2667314366888312\,e^{0.2700000000000001
 \,a} , 0.2763556485641138\,e^{0.2800000000000001\,a} , 
 0.2859522251048356\,e^{0.2900000000000001\,a} , 0.2955202066613397\,
 e^{0.3000000000000001\,a} , 0.3050586364434436\,e^{
 0.3100000000000001\,a} , 0.3145665606161179\,e^{0.3200000000000001\,
 a} , 0.3240430283948685\,e^{0.3300000000000001\,a} , 
 0.3334870921408145\,e^{0.3400000000000001\,a} , 0.3428978074554515\,
 e^{0.3500000000000001\,a} , 0.3522742332750901\,e^{
 0.3600000000000002\,a} , 0.3616154319649622\,e^{0.3700000000000002\,
 a} , 0.3709204694129828\,e^{0.3800000000000002\,a} , 
 0.3801884151231616\,e^{0.3900000000000002\,a} , 0.3894183423086507\,
 e^{0.4000000000000002\,a} , 0.3986093279844231\,e^{
 0.4100000000000002\,a} , 0.4077604530595704\,e^{0.4200000000000002\,
 a} , 0.416870802429211\,e^{0.4300000000000002\,a} , 
 0.4259394650659998\,e^{0.4400000000000002\,a} , 0.4349655341112304\,
 e^{0.4500000000000002\,a} , 0.44394810696552\,e^{0.4600000000000002
 \,a} , 0.4528862853790685\,e^{0.4700000000000003\,a} , 
 0.4617791755414831\,e^{0.4800000000000003\,a} , 0.4706258881711582\,
 e^{0.4900000000000003\,a} , 0.4794255386042032\,e^{
 0.5000000000000002\,a} , 0.4881772468829077\,e^{0.5100000000000002\,
 a} , 0.4968801378437369\,e^{0.5200000000000002\,a} , 
 0.5055333412048472\,e^{0.5300000000000002\,a} , 0.5141359916531133\,
 e^{0.5400000000000003\,a} , 0.5226872289306594\,e^{
 0.5500000000000003\,a} , 0.5311861979208836\,e^{0.5600000000000003\,
 a} , 0.5396320487339695\,e^{0.5700000000000003\,a} , 
 0.5480239367918738\,e^{0.5800000000000003\,a} , 0.556361022912784\,e
 ^{0.5900000000000003\,a} , 0.5646424733950356\,e^{0.6000000000000003
 \,a} , 0.5728674601004815\,e^{0.6100000000000003\,a} , 
 0.5810351605373053\,e^{0.6200000000000003\,a} , 0.5891447579422698\,
 e^{0.6300000000000003\,a} , 0.5971954413623923\,e^{
 0.6400000000000003\,a} , 0.6051864057360399\,e^{0.6500000000000004\,
 a} , 0.6131168519734341\,e^{0.6600000000000004\,a} , 
 0.6209859870365599\,e^{0.6700000000000004\,a} , 0.6287930240184688\,
 e^{0.6800000000000004\,a} , 0.6365371822219682\,e^{
 0.6900000000000004\,a} , 0.6442176872376913\,e^{0.7000000000000004\,
 a} , 0.651833771021537\,e^{0.7100000000000004\,a} , 
 0.6593846719714734\,e^{0.7200000000000004\,a} , 0.6668696350036982\,
 e^{0.7300000000000004\,a} , 0.6742879116281454\,e^{
 0.7400000000000004\,a} , 0.6816387600233345\,e^{0.7500000000000004\,
 a} , 0.6889214451105516\,e^{0.7600000000000005\,a} , 
 0.696135238627357\,e^{0.7700000000000005\,a} , 0.7032794192004105\,e
 ^{0.7800000000000005\,a} , 0.7103532724176082\,e^{0.7900000000000005
 \,a} , 0.7173560908995231\,e^{0.8000000000000005\,a} , 
 0.7242871743701429\,e^{0.8100000000000005\,a} , 0.7311458297268962\,
 e^{0.8200000000000005\,a} , 0.7379313711099631\,e^{
 0.8300000000000005\,a} , 0.7446431199708596\,e^{0.8400000000000005\,
 a} , 0.751280405140293\,e^{0.8500000000000005\,a} , 
 0.7578425628952773\,e^{0.8600000000000005\,a} , 0.7643289370255054\,
 e^{0.8700000000000006\,a} , 0.7707388788989696\,e^{
 0.8800000000000006\,a} , 0.7770717475268242\,e^{0.8900000000000006\,
 a} , 0.7833269096274837\,e^{0.9000000000000006\,a} , 
 0.7895037396899508\,e^{0.9100000000000006\,a} , 0.7956016200363664\,
 e^{0.9200000000000006\,a} , 0.8016199408837775\,e^{
 0.9300000000000006\,a} , 0.8075581004051147\,e^{0.9400000000000006\,
 a} , 0.8134155047893741\,e^{0.9500000000000006\,a} , 
 0.8191915683009986\,e^{0.9600000000000006\,a} , 0.8248857133384504\,
 e^{0.9700000000000006\,a} , 0.8304973704919708\,e^{
 0.9800000000000006\,a} , 0.8360259786005209\,e^{0.9900000000000007\,
 a} \right] 
\]
\end{eulerformula}
\begin{eulerprompt}
>function df(t) &= trigreduce(radcan(sqrt(diff(fx(t),t)^2+diff(fy(t),t)^2))); $'df(t)=df(t)
\end{eulerprompt}
\begin{euleroutput}
  Maxima said:
  diff: second argument must be a variable; found errexp1
   -- an error. To debug this try: debugmode(true);
  
  Error in:
  ... e(radcan(sqrt(diff(fx(t),t)^2+diff(fy(t),t)^2))); $'df(t)=df(t ...
                                                       ^
\end{euleroutput}
\begin{eulerprompt}
>S &=integrate(df(t),t,0,2*%pi); $S // panjang kurva (spiral)
\end{eulerprompt}
\begin{euleroutput}
  Maxima said:
  defint: variable of integration cannot be a constant; found errexp1
   -- an error. To debug this try: debugmode(true);
  
  Error in:
  S &=integrate(df(t),t,0,2*%pi); $S // panjang kurva (spiral) ...
                                ^
\end{euleroutput}
\begin{eulerprompt}
>S(a=0.1) // Panjang kurva untuk a=0.1
\end{eulerprompt}
\begin{euleroutput}
  Function S not found.
  Try list ... to find functions!
  Error in:
  S(a=0.1) // Panjang kurva untuk a=0.1 ...
          ^
\end{euleroutput}
\begin{eulercomment}
Soal:

Tunjukkan bahwa keliling lingkaran dengan jari-jari r adalah K=2.pi.r.

Berikut adalah contoh menghitung panjang parabola.
\end{eulercomment}
\begin{eulerprompt}
>plot2d("x^2",xmin=-1,xmax=1):
\end{eulerprompt}
\eulerimg{17}{images/EMT4Kalkulus-Naela Rizqy Arofah-22305144042-127.png}
\begin{eulerprompt}
>$showev('integrate(sqrt(1+diff(x^2,x)^2),x,-1,1))
\end{eulerprompt}
\begin{eulerformula}
\[
\int_{-1}^{1}{\sqrt{4\,x^2+1}\;dx}=\frac{{\rm asinh}\; 2+2\,\sqrt{5
 }}{2}
\]
\end{eulerformula}
\begin{eulerprompt}
>$float(%)
\end{eulerprompt}
\begin{eulerformula}
\[
\int_{-1.0}^{1.0}{\sqrt{4.0\,x^2+1.0}\;dx}=2.957885715089195
\]
\end{eulerformula}
\begin{eulerprompt}
>x=-1:0.2:1; y=x^2; plot2d(x,y);  ...
>  plot2d(x,y,points=1,style="o#",add=1):
\end{eulerprompt}
\eulerimg{17}{images/EMT4Kalkulus-Naela Rizqy Arofah-22305144042-130.png}
\begin{eulercomment}
Panjang tersebut dapat dihampiri dengan menggunakan jumlah panjang ruas-ruas garis yang menghubungkan titik-titik pada parabola
tersebut.
\end{eulercomment}
\begin{eulerprompt}
>i=1:cols(x)-1; sum(sqrt((x[i+1]-x[i])^2+(y[i+1]-y[i])^2))
\end{eulerprompt}
\begin{euleroutput}
  2.95191957027
\end{euleroutput}
\begin{eulercomment}
Hasilnya mendekati panjang yang dihitung secara eksak. Untuk mendapatkan hampiran yang cukup akurat, jarak antar titik dapat
diperkecil, misalnya 0.1, 0.05, 0.01, dan seterusnya. Cobalah Anda ulangi perhitungannya dengan nilai-nilai tersebut.

\end{eulercomment}
\eulersubheading{Koordinat Kartesius}
\begin{eulercomment}
Berikut diberikan contoh perhitungan panjang kurva menggunakan koordinat Kartesius. Kita akan hitung panjang kurva dengan
persamaan implisit:

\end{eulercomment}
\begin{eulerformula}
\[
x^3+y^3-3xy=0.
\]
\end{eulerformula}
\begin{eulerprompt}
>z &= x^3+y^3-3*x*y; $z
\end{eulerprompt}
\begin{eulerformula}
\[
y^3-3\,x\,y+x^3
\]
\end{eulerformula}
\begin{eulerprompt}
>plot2d(z,r=2,level=0,n=100):
\end{eulerprompt}
\eulerimg{17}{images/EMT4Kalkulus-Naela Rizqy Arofah-22305144042-132.png}
\begin{eulercomment}
Kita tertarik pada kurva di kuadran pertama.
\end{eulercomment}
\begin{eulerprompt}
>plot2d(z,a=0,b=2,c=0,d=2,level=[-10;0],n=100,contourwidth=3,style="/"):
\end{eulerprompt}
\eulerimg{17}{images/EMT4Kalkulus-Naela Rizqy Arofah-22305144042-133.png}
\begin{eulercomment}
Kita selesaikan persamaannya untuk x.
\end{eulercomment}
\begin{eulerprompt}
>$z with y=l*x, sol &= solve(%,x); $sol
\end{eulerprompt}
\begin{eulerformula}
\[
l^3\,x^3+x^3-3\,l\,x^2
\]
\end{eulerformula}
\begin{eulerformula}
\[
\left[ x=\frac{3\,l}{l^3+1} , x=0 \right] 
\]
\end{eulerformula}
\begin{eulercomment}
Kita gunakan solusi tersebut untuk mendefinisikan fungsi dengan Maxima.
\end{eulercomment}
\begin{eulerprompt}
>function f(l) &= rhs(sol[1]); $'f(l)=f(l)
\end{eulerprompt}
\begin{eulerformula}
\[
f\left(l\right)=\frac{3\,l}{l^3+1}
\]
\end{eulerformula}
\begin{eulercomment}
Fungsi tersebut juga dapat digunaka untuk menggambar kurvanya. Ingat, bahwa fungsi tersebut adalah nilai x dan nilai y=l*x, yakni
x=f(l) dan y=l*f(l).
\end{eulercomment}
\begin{eulerprompt}
>plot2d(&f(x),&x*f(x),xmin=-0.5,xmax=2,a=0,b=2,c=0,d=2,r=1.5):
\end{eulerprompt}
\begin{eulercomment}
Elemen panjang kurva adalah:

\end{eulercomment}
\begin{eulerformula}
\[
ds=\sqrt{f'(l)^2+(lf'(l)+f(l))^2}.
\]
\end{eulerformula}
\begin{eulerprompt}
>function ds(l) &= ratsimp(sqrt(diff(f(l),l)^2+diff(l*f(l),l)^2)); $'ds(l)=ds(l)
\end{eulerprompt}
\begin{eulerformula}
\[
{\it ds}\left(l\right)=\frac{\sqrt{9\,l^8+36\,l^6-36\,l^5-36\,l^3+
 36\,l^2+9}}{\sqrt{l^{12}+4\,l^9+6\,l^6+4\,l^3+1}}
\]
\end{eulerformula}
\begin{eulerprompt}
>$integrate(ds(l),l,0,1)
\end{eulerprompt}
\begin{eulerformula}
\[
\int_{0}^{1}{\frac{\sqrt{9\,l^8+36\,l^6-36\,l^5-36\,l^3+36\,l^2+9}
 }{\sqrt{l^{12}+4\,l^9+6\,l^6+4\,l^3+1}}\;dl}
\]
\end{eulerformula}
\begin{eulercomment}
Integral tersebut tidak dapat dihitung secara eksak menggunakan Maxima. Kita hitung integral etrsebut secara numerik dengan Euler.
Karena kurva simetris, kita hitung untuk nilai variabel integrasi dari 0 sampai 1, kemudian hasilnya dikalikan 2. 
\end{eulercomment}
\begin{eulerprompt}
>2*integrate("ds(x)",0,1)
\end{eulerprompt}
\begin{euleroutput}
  4.91748872168
\end{euleroutput}
\begin{eulerprompt}
>2*romberg(&ds(x),0,1)// perintah Euler lain untuk menghitung nilai hampiran integral yang sama
\end{eulerprompt}
\begin{euleroutput}
  4.91748872168
\end{euleroutput}
\begin{eulercomment}
Perhitungan di datas dapat dilakukan untuk sebarang fungsi x dan y dengan mendefinisikan fungsi EMT, misalnya kita beri nama
panjangkurva. Fungsi ini selalu memanggil Maxima untuk menurunkan fungsi yang diberikan.
\end{eulercomment}
\begin{eulerprompt}
>function panjangkurva(fx,fy,a,b) ...
\end{eulerprompt}
\begin{eulerudf}
  ds=mxm("sqrt(diff(@fx,x)^2+diff(@fy,x)^2)");
  return romberg(ds,a,b);
  endfunction
\end{eulerudf}
\begin{eulerprompt}
>panjangkurva("x","x^2",-1,1) // cek untuk menghitung panjang kurva parabola sebelumnya
\end{eulerprompt}
\begin{euleroutput}
  2.95788571509
\end{euleroutput}
\begin{eulercomment}
Bandingkan dengan nilai eksak di atas.
\end{eulercomment}
\begin{eulerprompt}
>2*panjangkurva(mxm("f(x)"),mxm("x*f(x)"),0,1) // cek contoh terakhir, bandingkan hasilnya!
\end{eulerprompt}
\begin{euleroutput}
  4.91748872168
\end{euleroutput}
\begin{eulercomment}
Kita hitung panjang spiral Archimides berikut ini dengan fungsi tersebut.
\end{eulercomment}
\begin{eulerprompt}
>plot2d("x*cos(x)","x*sin(x)",xmin=0,xmax=2*pi,square=1):
\end{eulerprompt}
\eulerimg{17}{images/EMT4Kalkulus-Naela Rizqy Arofah-22305144042-139.png}
\begin{eulerprompt}
>panjangkurva("x*cos(x)","x*sin(x)",0,2*pi)
\end{eulerprompt}
\begin{euleroutput}
  21.2562941482
\end{euleroutput}
\begin{eulercomment}
Berikut kita definisikan fungsi yang sama namun dengan Maxima, untuk perhitungan eksak.
\end{eulercomment}
\begin{eulerprompt}
>&kill(ds,x,fx,fy)
\end{eulerprompt}
\begin{euleroutput}
  
                                   done
  
\end{euleroutput}
\begin{eulerprompt}
>function ds(fx,fy) &&= sqrt(diff(fx,x)^2+diff(fy,x)^2)
\end{eulerprompt}
\begin{euleroutput}
  
                             2              2
                    sqrt(diff (fy, x) + diff (fx, x))
  
\end{euleroutput}
\begin{eulerprompt}
>sol &= ds(x*cos(x),x*sin(x)); $sol // Kita gunakan untuk menghitung panjang kurva terakhir di atas
\end{eulerprompt}
\begin{eulerformula}
\[
\sqrt{\left(\cos x-x\,\sin x\right)^2+\left(\sin x+x\,\cos x\right)
 ^2}
\]
\end{eulerformula}
\begin{eulerprompt}
>$sol | trigreduce | expand, $integrate(%,x,0,2*pi), %()
\end{eulerprompt}
\begin{eulerformula}
\[
\sqrt{x^2+1}
\]
\end{eulerformula}
\begin{eulerformula}
\[
\frac{{\rm asinh}\; \left(2\,\pi\right)+2\,\pi\,\sqrt{4\,\pi^2+1}}{
 2}
\]
\end{eulerformula}
\begin{euleroutput}
  21.2562941482
\end{euleroutput}
\begin{eulercomment}
Hasilnya sama dengan perhitungan menggunakan fungsi EMT.

Berikut adalah contoh lain penggunaan fungsi Maxima tersebut.
\end{eulercomment}
\begin{eulerprompt}
>plot2d("3*x^2-1","3*x^3-1",xmin=-1/sqrt(3),xmax=1/sqrt(3),square=1):
\end{eulerprompt}
\eulerimg{17}{images/EMT4Kalkulus-Naela Rizqy Arofah-22305144042-143.png}
\begin{eulerprompt}
>sol &= radcan(ds(3*x^2-1,3*x^3-1)); $sol
\end{eulerprompt}
\begin{eulerformula}
\[
3\,x\,\sqrt{9\,x^2+4}
\]
\end{eulerformula}
\begin{eulerprompt}
>$showev('integrate(sol,x,0,1/sqrt(3))), $2*float(%) // panjang kurva di atas
\end{eulerprompt}
\begin{eulerformula}
\[
3\,\int_{0}^{\frac{1}{\sqrt{3}}}{x\,\sqrt{9\,x^2+4}\;dx}=3\,\left(
 \frac{7^{\frac{3}{2}}}{27}-\frac{8}{27}\right)
\]
\end{eulerformula}
\begin{eulerformula}
\[
6.0\,\int_{0.0}^{0.5773502691896258}{x\,\sqrt{9.0\,x^2+4.0}\;dx}=
 2.337835372767141
\]
\end{eulerformula}
\eulersubheading{Sikloid}
\begin{eulercomment}
Berikut kita akan menghitung panjang kurva lintasan (sikloid) suatu titik pada lingkaran yang berputar ke kanan pada permukaan
datar. Misalkan jari-jari lingkaran tersebut adalah r. Posisi titik pusat lingkaran pada saat t adalah:

\end{eulercomment}
\begin{eulerformula}
\[
(rt,r).
\]
\end{eulerformula}
\begin{eulercomment}
Misalkan posisi titik pada lingkaran tersebut mula-mula (0,0) dan posisinya pada saat t adalah:

\end{eulercomment}
\begin{eulerformula}
\[
(r(t-\sin(t)),r(1-\cos(t))).
\]
\end{eulerformula}
\begin{eulercomment}
Berikut kita plot lintasan tersebut dan beberapa posisi lingkaran ketika t=0, t=pi/2, t=r*pi.
\end{eulercomment}
\begin{eulerprompt}
>x &= r*(t-sin(t))
\end{eulerprompt}
\begin{euleroutput}
  
          [0, 1.66665833335744e-7 r, 1.33330666692022e-6 r, 
  4.499797504338432e-6 r, 1.066581336583994e-5 r, 
  2.083072932167196e-5 r, 3.599352055540239e-5 r, 
  5.71526624672386e-5 r, 8.530603082730626e-5 r, 
  1.214508019889565e-4 r, 1.665833531718508e-4 r, 
  2.216991628251896e-4 r, 2.877927110806339e-4 r, 
  3.658573803051457e-4 r, 4.568853557635201e-4 r, 
  5.618675264007778e-4 r, 6.817933857540259e-4 r, 
  8.176509330039827e-4 r, 9.704265741758145e-4 r, 
  0.001141105023499428 r, 0.001330669204938795 r, 
  0.001540100153900437 r, 0.001770376919130678 r, 
  0.002022476464811601 r, 0.002297373572865413 r, 
  0.002596040745477063 r, 0.002919448107844891 r, 
  0.003268563311168871 r, 0.003644351435886262 r, 
  0.004047774895164447 r, 0.004479793338660443 r, 0.0049413635565565 r, 
  0.005433439383882244 r, 0.005956971605131645 r, 
  0.006512907859185624 r, 0.007102192544548636 r, 
  0.007725766724910044 r, 0.00838456803503801 r, 
  0.009079530587017326 r, 0.009811584876838586 r, 0.0105816576913495 r, 
  0.01139067201557714 r, 0.01223954694042984 r, 0.01312919757078923 r, 
  0.01406053493400045 r, 0.01503446588876983 r, 0.01605189303448024 r, 
  0.01711371462093175 r, 0.01822082445851714 r, 0.01937411182884202 r, 
  0.02057446139579705 r, 0.02182275311709253 r, 0.02311986215626333 r, 
  0.02446665879515308 r, 0.02586400834688696 r, 0.02731277106934082 r, 
  0.02881380207911666 r, 0.03036795126603076 r, 0.03197606320812652 r, 
  0.0336389770872163 r, 0.03535752660496472 r, 0.03713253989951881 r, 
  0.03896483946269502 r, 0.0408552420577305 r, 0.04280455863760801 r, 
  0.04481359426396048 r, 0.04688314802656623 r, 0.04901401296344043 r, 
  0.05120697598153157 r, 0.05346281777803219 r, 0.05578231276230905 r, 
  0.05816622897846346 r, 0.06061532802852698 r, 0.0631303649963022 r, 
  0.06571208837185505 r, 0.06836123997666599 r, 0.07107855488944881 r, 
  0.07386476137264342 r, 0.07672058079958999 r, 0.07964672758239233 r, 
  0.08264390910047736 r, 0.0857128256298576 r, 0.08885417027310427 r, 
  0.09206862889003742 r, 0.09535688002914089 r, 0.0987195948597075 r, 
  0.1021574371047232 r, 0.1056710629744951 r, 0.1092611211010309 r, 
  0.1129282524731764 r, 0.1166730903725168 r, 0.1204962603100498 r, 
  0.1243983799636342 r, 0.1283800591162231 r, 0.1324418995948859 r, 
  0.1365844952106265 r, 0.140808431699002 r, 0.1451142866615502 r, 
  0.1495026295080298 r, 0.1539740213994798 r]
  
\end{euleroutput}
\begin{eulerprompt}
>y &= r*(1-cos(t))
\end{eulerprompt}
\begin{euleroutput}
  
          [0, 4.999958333473664e-5 r, 1.999933334222437e-4 r, 
  4.499662510124569e-4 r, 7.998933390220841e-4 r, 
  0.001249739605033717 r, 0.00179946006479581 r, 
  0.002448999746720415 r, 0.003198293697380561 r, 
  0.004047266988005727 r, 0.004995834721974179 r, 
  0.006043902043303184 r, 0.00719136414613375 r, 0.00843810628521191 r, 
  0.009784003787362772 r, 0.01122892206395776 r, 0.01277271662437307 r, 
  0.01441523309043924 r, 0.01615630721187855 r, 0.01799576488272969 r, 
  0.01993342215875837 r, 0.02196908527585173 r, 0.02410255066939448 r, 
  0.02633360499462523 r, 0.02866202514797045 r, 0.03108757828935527 r, 
  0.03361002186548678 r, 0.03622910363410947 r, 0.03894456168922911 r, 
  0.04175612448730281 r, 0.04466351087439402 r, 0.04766643011428662 r, 
  0.05076458191755917 r, 0.0539576564716131 r, 0.05724533447165381 r, 
  0.06062728715262111 r, 0.06410317632206519 r, 0.06767265439396564 r, 
  0.07133536442348987 r, 0.07509094014268702 r, 0.07893900599711501 r, 
  0.08287917718339499 r, 0.08691105968769186 r, 0.09103425032511492 r, 
  0.09524833678003664 r, 0.09955289764732322 r, 0.1039475024744748 r, 
  0.1084317118046711 r, 0.113005077220716 r, 0.1176671413898787 r, 
  0.1224174381096274 r, 0.1272554923542488 r, 0.1321808203223502 r, 
  0.1371929294852391 r, 0.1422913186361759 r, 0.1474754779404944 r, 
  0.152744888986584 r, 0.1580990248377314 r, 0.1635373500848132 r, 
  0.1690593208998367 r, 0.1746643850903219 r, 0.1803519821545206 r, 
  0.1861215433374662 r, 0.1919724916878484 r, 0.1979042421157076 r, 
  0.2039162014509444 r, 0.2100077685026351 r, 0.216178334119151 r, 
  0.2224272812490723 r, 0.2287539850028937 r, 0.2351578127155118 r, 
  0.2416381240094921 r, 0.2481942708591053 r, 0.2548255976551299 r, 
  0.2615314412704124 r, 0.2683111311261794 r, 0.2751639892590951 r, 
  0.2820893303890569 r, 0.2890864619877229 r, 0.2961546843477643 r, 
  0.3032932906528349 r, 0.3105015670482534 r, 0.3177787927123868 r, 
  0.3251242399287333 r, 0.3325371741586922 r, 0.3400168541150183 r, 
  0.3475625318359485 r, 0.3551734527599992 r, 0.3628488558014202 r, 
  0.3705879734263036 r, 0.3783900317293359 r, 0.3862542505111889 r, 
  0.3941798433565377 r, 0.4021660177127022 r, 0.4102119749689023 r, 
  0.418316910536117 r, 0.4264800139275439 r, 0.4347004688396462 r, 
  0.4429774532337832 r, 0.451310139418413 r]
  
\end{euleroutput}
\begin{eulercomment}
Berikut kita gambar sikloid untuk r=1.
\end{eulercomment}
\begin{eulerprompt}
>ex &= x-sin(x); ey &= 1-cos(x); aspect(1);
>plot2d(ex,ey,xmin=0,xmax=4pi,square=1); ...
>  plot2d("2+cos(x)","1+sin(x)",xmin=0,xmax=2pi,>add,color=blue); ...
>  plot2d([2,ex(2)],[1,ey(2)],color=red,>add); ...
>  plot2d(ex(2),ey(2),>points,>add,color=red); ...
>  plot2d("2pi+cos(x)","1+sin(x)",xmin=0,xmax=2pi,>add,color=blue); ...
>  plot2d([2pi,ex(2pi)],[1,ey(2pi)],color=red,>add);  ...
>  plot2d(ex(2pi),ey(2pi),>points,>add,color=red):
\end{eulerprompt}
\begin{euleroutput}
  Error : [0,1.66665833335744e-7*r-sin(1.66665833335744e-7*r),1.33330666692022e-6*r-sin(1.33330666692022e-6*r),4.499797504338432e-6*r-sin(4.499797504338432e-6*r),1.066581336583994e-5*r-sin(1.066581336583994e-5*r),2.083072932167196e-5*r-sin(2.083072932167196e-5*r),3.599352055540239e-5*r-sin(3.599352055540239e-5*r),5.71526624672386e-5*r-sin(5.71526624672386e-5*r),8.530603082730626e-5*r-sin(8.530603082730626e-5*r),1.214508019889565e-4*r-sin(1.214508019889565e-4*r),1.665833531718508e-4*r-sin(1.665833531718508e-4*r),2.216991628251896e-4*r-sin(2.216991628251896e-4*r),2.877927110806339e-4*r-sin(2.877927110806339e-4*r),3.658573803051457e-4*r-sin(3.658573803051457e-4*r),4.5688535576352e-4*r-sin(4.5688535576352e-4*r),5.618675264007778e-4*r-sin(5.618675264007778e-4*r),6.817933857540259e-4*r-sin(6.817933857540259e-4*r),8.176509330039827e-4*r-sin(8.176509330039827e-4*r),9.704265741758145e-4*r-sin(9.704265741758145e-4*r),0.001141105023499428*r-sin(0.001141105023499428*r),0.001330669204938795*r-sin(0.001330669204938795*r),0.001540100153900437*r-sin(0.001540100153900437*r),0.001770376919130678*r-sin(0.001770376919130678*r),0.002022476464811601*r-sin(0.002022476464811601*r),0.002297373572865413*r-sin(0.002297373572865413*r),0.002596040745477063*r-sin(0.002596040745477063*r),0.002919448107844891*r-sin(0.002919448107844891*r),0.003268563311168871*r-sin(0.003268563311168871*r),0.003644351435886262*r-sin(0.003644351435886262*r),0.004047774895164447*r-sin(0.004047774895164447*r),0.004479793338660443*r-sin(0.004479793338660443*r),0.0049413635565565*r-sin(0.0049413635565565*r),0.005433439383882244*r-sin(0.005433439383882244*r),0.005956971605131645*r-sin(0.005956971605131645*r),0.006512907859185624*r-sin(0.006512907859185624*r),0.007102192544548636*r-sin(0.007102192544548636*r),0.007725766724910044*r-sin(0.007725766724910044*r),0.00838456803503801*r-sin(0.00838456803503801*r),0.009079530587017326*r-sin(0.009079530587017326*r),0.009811584876838586*r-sin(0.009811584876838586*r),0.0105816576913495*r-sin(0.0105816576913495*r),0.01139067201557714*r-sin(0.01139067201557714*r),0.01223954694042984*r-sin(0.01223954694042984*r),0.01312919757078923*r-sin(0.01312919757078923*r),0.01406053493400045*r-sin(0.01406053493400045*r),0.01503446588876983*r-sin(0.01503446588876983*r),0.01605189303448024*r-sin(0.01605189303448024*r),0.01711371462093175*r-sin(0.01711371462093175*r),0.01822082445851714*r-sin(0.01822082445851714*r),0.01937411182884202*r-sin(0.01937411182884202*r),0.02057446139579705*r-sin(0.02057446139579705*r),0.02182275311709253*r-sin(0.02182275311709253*r),0.02311986215626333*r-sin(0.02311986215626333*r),0.02446665879515308*r-sin(0.02446665879515308*r),0.02586400834688696*r-sin(0.02586400834688696*r),0.02731277106934082*r-sin(0.02731277106934082*r),0.02881380207911666*r-sin(0.02881380207911666*r),0.03036795126603076*r-sin(0.03036795126603076*r),0.03197606320812652*r-sin(0.03197606320812652*r),0.0336389770872163*r-sin(0.0336389770872163*r),0.03535752660496472*r-sin(0.03535752660496472*r),0.03713253989951881*r-sin(0.03713253989951881*r),0.03896483946269502*r-sin(0.03896483946269502*r),0.0408552420577305*r-sin(0.0408552420577305*r),0.04280455863760801*r-sin(0.04280455863760801*r),0.04481359426396048*r-sin(0.04481359426396048*r),0.04688314802656623*r-sin(0.04688314802656623*r),0.04901401296344043*r-sin(0.04901401296344043*r),0.05120697598153157*r-sin(0.05120697598153157*r),0.05346281777803219*r-sin(0.05346281777803219*r),0.05578231276230905*r-sin(0.05578231276230905*r),0.05816622897846346*r-sin(0.05816622897846346*r),0.06061532802852698*r-sin(0.06061532802852698*r),0.0631303649963022*r-sin(0.0631303649963022*r),0.06571208837185505*r-sin(0.06571208837185505*r),0.06836123997666599*r-sin(0.06836123997666599*r),0.07107855488944881*r-sin(0.07107855488944881*r),0.07386476137264342*r-sin(0.07386476137264342*r),0.07672058079958999*r-sin(0.07672058079958999*r),0.07964672758239233*r-sin(0.07964672758239233*r),0.08264390910047736*r-sin(0.08264390910047736*r),0.0857128256298576*r-sin(0.0857128256298576*r),0.08885417027310427*r-sin(0.08885417027310427*r),0.09206862889003742*r-sin(0.09206862889003742*r),0.09535688002914089*r-sin(0.09535688002914089*r),0.0987195948597075*r-sin(0.0987195948597075*r),0.1021574371047232*r-sin(0.1021574371047232*r),0.1056710629744951*r-sin(0.1056710629744951*r),0.1092611211010309*r-sin(0.1092611211010309*r),0.1129282524731764*r-sin(0.1129282524731764*r),0.1166730903725168*r-sin(0.1166730903725168*r),0.1204962603100498*r-sin(0.1204962603100498*r),0.1243983799636342*r-sin(0.1243983799636342*r),0.1283800591162231*r-sin(0.1283800591162231*r),0.1324418995948859*r-sin(0.1324418995948859*r),0.1365844952106265*r-sin(0.1365844952106265*r),0.140808431699002*r-sin(0.140808431699002*r),0.1451142866615502*r-sin(0.1451142866615502*r),0.1495026295080298*r-sin(0.1495026295080298*r),0.1539740213994798*r-sin(0.1539740213994798*r)] does not produce a real or column vector
  
  Error generated by error() command
  
  adaptiveeval:
      error(f$|" does not produce a real or column vector"); 
  Try "trace errors" to inspect local variables after errors.
  plot2d:
      dw/n,dw/n^2,dw/n;args());
\end{euleroutput}
\begin{eulercomment}
Berikut dihitung panjang lintasan untuk 1 putaran penuh. (Jangan salah menduga bahwa panjang lintasan 1 putaran penuh sama dengan
keliling lingkaran!)
\end{eulercomment}
\begin{eulerprompt}
>ds &= radcan(sqrt(diff(ex,x)^2+diff(ey,x)^2)); $ds=trigsimp(ds) // elemen panjang kurva sikloid
\end{eulerprompt}
\begin{euleroutput}
  Maxima said:
  diff: second argument must be a variable; found errexp1
   -- an error. To debug this try: debugmode(true);
  
  Error in:
  ds &= radcan(sqrt(diff(ex,x)^2+diff(ey,x)^2)); $ds=trigsimp(ds ...
                                               ^
\end{euleroutput}
\begin{eulerprompt}
>ds &= trigsimp(ds); $ds
>$showev('integrate(ds,x,0,2*pi)) // hitung panjang sikloid satu putaran penuh
\end{eulerprompt}
\begin{euleroutput}
  Maxima said:
  defint: variable of integration must be a simple or subscripted variable.
  defint: found errexp1
  #0: showev(f='integrate(ds,[0,1.66665833335744e-7*r,1.33330666692022e-6*r,4.499797504338432e-6*r,1.06658133658399...)
   -- an error. To debug this try: debugmode(true);
  
  Error in:
  $showev('integrate(ds,x,0,2*pi)) // hitung panjang sikloid sat ...
                                   ^
\end{euleroutput}
\begin{eulerprompt}
>integrate(mxm("ds"),0,2*pi) // hitung secara numerik
\end{eulerprompt}
\begin{euleroutput}
  Illegal function result in map.
  %evalexpression:
      if maps then return %mapexpression1(x,f$;args());
  gauss:
      if maps then y=%evalexpression(f$,a+h-(h*xn)',maps;args());
  adaptivegauss:
      t1=gauss(f$,c,c+h;args(),=maps);
  Try "trace errors" to inspect local variables after errors.
  integrate:
      return adaptivegauss(f$,a,b,eps*1000;args(),=maps);
\end{euleroutput}
\begin{eulerprompt}
>romberg(mxm("ds"),0,2*pi) // cara lain hitung secara numerik
\end{eulerprompt}
\begin{euleroutput}
  Wrong argument!
  
  Cannot combine a symbolic expression here.
  Did you want to create a symbolic expression?
  Then start with &.
  
  Try "trace errors" to inspect local variables after errors.
  romberg:
      if cols(y)==1 then return y*(b-a); endif;
  Error in:
  romberg(mxm("ds"),0,2*pi) // cara lain hitung secara numerik ...
                           ^
\end{euleroutput}
\begin{eulercomment}
Perhatikan, seperti terlihat pada gambar, panjang sikloid lebih besar daripada keliling lingkarannya, yakni:

\end{eulercomment}
\begin{eulerformula}
\[
2\pi.
\]
\end{eulerformula}
\eulersubheading{Kurvatur (Kelengkungan) Kurva}
\begin{eulercomment}
image: Osculating.png

Aslinya, kelengkungan kurva diferensiabel (yakni, kurva mulus yang tidak lancip) di titik P didefinisikan melalui lingkaran
oskulasi (yaitu, lingkaran yang melalui titik P dan terbaik memperkirakan, paling banyak menyinggung kurva di sekitar P). Pusat
dan radius kelengkungan kurva di P adalah pusat dan radius lingkaran oskulasi. Kelengkungan adalah kebalikan dari radius
kelengkungan:

\end{eulercomment}
\begin{eulerformula}
\[
\kappa =\frac {1}{R}
\]
\end{eulerformula}
\begin{eulercomment}
dengan R adalah radius kelengkungan. (Setiap lingkaran memiliki kelengkungan ini pada setiap titiknya, dapat diartikan, setiap
lingkaran berputar 2pi sejauh 2piR.)\\
Definisi ini sulit dimanipulasi dan dinyatakan ke dalam rumus untuk kurva umum. Oleh karena itu digunakan definisi lain yang
ekivalen.

\end{eulercomment}
\eulersubheading{Definisi Kurvatur dengan Fungsi Parametrik Panjang Kurva}
\begin{eulercomment}
Setiap kurva diferensiabel dapat dinyatakan dengan persamaan parametrik terhadap panjang kurva s:

\end{eulercomment}
\begin{eulerformula}
\[
\gamma(s) = (x(s),\ y(s)),
\]
\end{eulerformula}
\begin{eulercomment}
dengan x dan y adalah fungsi riil yang diferensiabel, yang memenuhi:

\end{eulercomment}
\begin{eulerformula}
\[
\|\gamma'(s)\|=\sqrt{x'(s)^2+y'(s)^2}=1.
\]
\end{eulerformula}
\begin{eulercomment}
Ini berarti bahwa vektor singgung


\end{eulercomment}
\begin{eulerformula}
\[
\mathbf{T}(s)=(x'(s),\ y'(s))
\]
\end{eulerformula}
\begin{eulercomment}
memiliki norm 1 dan merupakan vektor singgung satuan.

Apabila kurvanya memiliki turunan kedua, artinya turunan kedua x dan y ada, maka T'(s) ada. Vektor ini merupakan normal kurva yang
arahnya menuju pusat kurvatur, norm-nya merupakan nilai kurvatur (kelengkungan):

\end{eulercomment}
\begin{eulerformula}
\[
 \begin{aligned}\mathbf{T}(s) &= \mathbf{\gamma}'(s),\\ \mathbf{T}^{2}(s) &=1\ \text{(konstanta)}\Rightarrow \mathbf{T}'(s)\cdot \mathbf{T}(s)=0\\ \kappa(s) &=\|\mathbf {T}'(s)\|= \|\mathbf{\gamma}''(s)\|=\sqrt{x''(s)^{2}+y''(s)^{2}}.\end{aligned}
\]
\end{eulerformula}
\begin{eulercomment}
Nilai

\end{eulercomment}
\begin{eulerformula}
\[
R(s)=\frac{1}{\kappa(s)}
\]
\end{eulerformula}
\begin{eulercomment}
disebut jari-jari (radius) kelengkungan kurva.

Bilangan riil

\end{eulercomment}
\begin{eulerformula}
\[
 k(s) = \pm\kappa(s)
\]
\end{eulerformula}
\begin{eulercomment}
disebut nilai kelengkungan bertanda.

Contoh:\\
Akan ditentukan kurvatur lingkaran

\end{eulercomment}
\begin{eulerformula}
\[
x=r\cos t,\ y= r\sin t.
\]
\end{eulerformula}
\begin{eulerprompt}
>fx &= r*cos(t); fy &=r*sin(t);
>&assume(t>0,r>0); s &=integrate(sqrt(diff(fx,t)^2+diff(fy,t)^2),t,0,t); s // elemen panjang kurva, panjang busur lingkaran (s)
\end{eulerprompt}
\begin{euleroutput}
  Maxima said:
  diff: second argument must be a variable; found errexp1
   -- an error. To debug this try: debugmode(true);
  
  Error in:
  ... =integrate(sqrt(diff(fx,t)^2+diff(fy,t)^2),t,0,t); s // elemen ...
                                                       ^
\end{euleroutput}
\begin{eulerprompt}
>&kill(s); fx &= r*cos(s/r); fy &=r*sin(s/r); // definisi ulang persamaan parametrik terhadap s dengan substitusi t=s/r
>k &= trigsimp(sqrt(diff(fx,s,2)^2+diff(fy,s,2)^2)); $k // nilai kurvatur lingkaran dengan menggunakan definisi di atas
\end{eulerprompt}
\begin{eulerformula}
\[
\frac{1}{r}
\]
\end{eulerformula}
\begin{eulercomment}
Untuk representasi parametrik umum, misalkan

\end{eulercomment}
\begin{eulerformula}
\[
x = x(t),\ y= y(t)
\]
\end{eulerformula}
\begin{eulercomment}
merupakan persamaan parametrik untuk kurva bidang yang terdiferensialkan dua kali. Kurvatur untuk kurva tersebut didefinisikan
sebagai

\end{eulercomment}
\begin{eulerformula}
\[
\begin{aligned}\kappa &= \frac{d\phi}{ds}=\frac{\frac{d\phi}{dt}}{\frac{ds}{dt}}\quad (\phi \text{ adalah sudut kemiringan garis singgung dan }s \text{ adalah panjang kurva})\\ &=\frac{\frac{d\phi}{dt}}{\sqrt{(\frac{dx}{dt})^2+(\frac{dy}{dt})^2}}= \frac{\frac{d\phi}{dt}}{\sqrt{x'(t)^2+y'(t)^2}}.\end{aligned}.
\]
\end{eulerformula}
\begin{eulercomment}
Selanjutnya, pembilang pada persamaan di atas dapat dicari sebagai berikut.

\end{eulercomment}
\begin{eulerformula}
\[
\begin{aligned}\sec^2\phi\frac{d\phi}{dt} &= \frac{d}{dt}\left(\tan\phi\right)= \frac{d}{dt}\left(\frac{dy}{dx}\right)= \frac{d}{dt}\left(\frac{dy/dt}{dx/dt}\right)= \frac{d}{dt}\left(\frac{y'(t)}{x'(t)}\right)=\frac{x'(t)y''(t)-x''(t)y'(t)}{x'(t)^2}.\\ & \\ \frac{d\phi}{dt} &= \frac{1}{\sec^2\phi}\frac{x'(t)y''(t)-x''(t)y'(t)}{x'(t)^2}\\ &= \frac{1}{1+\tan^2\phi}\frac{x'(t)y''(t)-x''(t)y'(t)}{x'(t)^2}\\ &= \frac{1}{1+\left(\frac{y'(t)}{x'(t)}\right)^2}\frac{x'(t)y''(t)-x''(t)y'(t)}{x'(t)^2}\\ &= \frac{x'(t)y''(t)-x''(t)y'(t)}{x'(t)^2+y'(t)^2}.\end{aligned}
\]
\end{eulerformula}
\begin{eulercomment}
Jadi, rumus kurvatur untuk kurva parametrik

\end{eulercomment}
\begin{eulerformula}
\[
x=x(t),\ y=y(t)
\]
\end{eulerformula}
\begin{eulercomment}
adalah

\end{eulercomment}
\begin{eulerformula}
\[
\kappa(t) = \frac{x'(t)y''(t)-x''(t)y'(t)}{\left(x'(t)^2+y'(t)^2\right)^{3/2}}.
\]
\end{eulerformula}
\begin{eulercomment}
Jika kurvanya dinyatakan dengan persamaan parametrik pada koordinat kutub

\end{eulercomment}
\begin{eulerformula}
\[
x=r(\theta)\cos\theta,\ y=r(\theta)\sin\theta,
\]
\end{eulerformula}
\begin{eulercomment}
maka rumus kurvaturnya adalah

\end{eulercomment}
\begin{eulerformula}
\[
\kappa(\theta) = \frac{r(\theta)^2+2r'(\theta)^2-r(\theta)r''(\theta)}{\left(r'(\theta)^2+r'(\theta)^2\right)^{3/2}}.
\]
\end{eulerformula}
\begin{eulercomment}
(Silakan Anda turunkan rumus tersebut!)

Contoh:\\
Lingkaran dengan pusat (0,0) dan jari-jari r dapat dinyatakan dengan persamaan parametrik

\end{eulercomment}
\begin{eulerformula}
\[
x=r\cos t,\ y=r\sin t.
\]
\end{eulerformula}
\begin{eulercomment}
Nilai kelengkungan lingkaran tersebut adalah

\end{eulercomment}
\begin{eulerformula}
\[
\kappa(t)=\frac{x'(t)y''(t)-x''(t)y'(t)}{\left(x'(t)^2+y'(t)^2\right)^{3/2}}=\frac{r^2}{r^3}=\frac 1 r.
\]
\end{eulerformula}
\begin{eulercomment}
Hasil cocok dengan definisi kurvatur suatu kelengkungan.
\end{eulercomment}
\begin{eulercomment}
Kurva

\end{eulercomment}
\begin{eulerformula}
\[
y=f(x)
\]
\end{eulerformula}
\begin{eulercomment}
dapat dinyatakan ke dalam persamaan parametrik

\end{eulercomment}
\begin{eulerformula}
\[
x=t,\ y=f(t),\ \text{ dengan } x'(t)=1,\ x''(t)=0,
\]
\end{eulerformula}
\begin{eulercomment}
sehingga kurvaturnya adalah

\end{eulercomment}
\begin{eulerformula}
\[
\kappa(t) = \frac{y''(t)}{\left(1+y'(t)^2\right)^{3/2}}.
\]
\end{eulerformula}
\begin{eulercomment}
Contoh:\\
Akan ditentukan kurvatur parabola

\end{eulercomment}
\begin{eulerformula}
\[
y=ax^2+bx+c.
\]
\end{eulerformula}
\begin{eulerprompt}
>function f(x) &= a*x^2+b*x+c; $y=f(x)
\end{eulerprompt}
\begin{eulerformula}
\[
\left[ 0 , 4.999958333473664 \times 10^{-5}\,r , 
 1.999933334222437 \times 10^{-4}\,r , 
 4.499662510124569 \times 10^{-4}\,r , 
 7.998933390220841 \times 10^{-4}\,r , 0.001249739605033717\,r , 
 0.00179946006479581\,r , 0.002448999746720415\,r , 
 0.003198293697380561\,r , 0.004047266988005727\,r , 
 0.004995834721974179\,r , 0.006043902043303184\,r , 
 0.00719136414613375\,r , 0.00843810628521191\,r , 
 0.009784003787362772\,r , 0.01122892206395776\,r , 
 0.01277271662437307\,r , 0.01441523309043924\,r , 
 0.01615630721187855\,r , 0.01799576488272969\,r , 
 0.01993342215875837\,r , 0.02196908527585173\,r , 
 0.02410255066939448\,r , 0.02633360499462523\,r , 
 0.02866202514797045\,r , 0.03108757828935527\,r , 
 0.03361002186548678\,r , 0.03622910363410947\,r , 
 0.03894456168922911\,r , 0.04175612448730281\,r , 
 0.04466351087439402\,r , 0.04766643011428662\,r , 
 0.05076458191755917\,r , 0.0539576564716131\,r , 0.05724533447165381
 \,r , 0.06062728715262111\,r , 0.06410317632206519\,r , 
 0.06767265439396564\,r , 0.07133536442348987\,r , 
 0.07509094014268702\,r , 0.07893900599711501\,r , 
 0.08287917718339499\,r , 0.08691105968769186\,r , 
 0.09103425032511492\,r , 0.09524833678003664\,r , 
 0.09955289764732322\,r , 0.1039475024744748\,r , 0.1084317118046711
 \,r , 0.113005077220716\,r , 0.1176671413898787\,r , 
 0.1224174381096274\,r , 0.1272554923542488\,r , 0.1321808203223502\,
 r , 0.1371929294852391\,r , 0.1422913186361759\,r , 
 0.1474754779404944\,r , 0.152744888986584\,r , 0.1580990248377314\,r
  , 0.1635373500848132\,r , 0.1690593208998367\,r , 
 0.1746643850903219\,r , 0.1803519821545206\,r , 0.1861215433374662\,
 r , 0.1919724916878484\,r , 0.1979042421157076\,r , 
 0.2039162014509444\,r , 0.2100077685026351\,r , 0.216178334119151\,r
  , 0.2224272812490723\,r , 0.2287539850028937\,r , 
 0.2351578127155118\,r , 0.2416381240094921\,r , 0.2481942708591053\,
 r , 0.2548255976551299\,r , 0.2615314412704124\,r , 
 0.2683111311261794\,r , 0.2751639892590951\,r , 0.2820893303890569\,
 r , 0.2890864619877229\,r , 0.2961546843477643\,r , 
 0.3032932906528349\,r , 0.3105015670482534\,r , 0.3177787927123868\,
 r , 0.3251242399287333\,r , 0.3325371741586922\,r , 
 0.3400168541150183\,r , 0.3475625318359485\,r , 0.3551734527599992\,
 r , 0.3628488558014202\,r , 0.3705879734263036\,r , 
 0.3783900317293359\,r , 0.3862542505111889\,r , 0.3941798433565377\,
 r , 0.4021660177127022\,r , 0.4102119749689023\,r , 
 0.418316910536117\,r , 0.4264800139275439\,r , 0.4347004688396462\,r
  , 0.4429774532337832\,r , 0.451310139418413\,r \right] =\left[ c , 
 2.7777500001498 \times 10^{-14}\,a\,r^2+
 1.66665833335744 \times 10^{-7}\,b\,r+c , 
 1.777706668053906 \times 10^{-12}\,a\,r^2+
 1.33330666692022 \times 10^{-6}\,b\,r+c , 
 2.024817758005038 \times 10^{-11}\,a\,r^2+
 4.499797504338432 \times 10^{-6}\,b\,r+c , 
 1.137595747549299 \times 10^{-10}\,a\,r^2+
 1.066581336583994 \times 10^{-5}\,b\,r+c , 
 4.339192840727639 \times 10^{-10}\,a\,r^2+
 2.083072932167196 \times 10^{-5}\,b\,r+c , 
 1.295533521972174 \times 10^{-9}\,a\,r^2+
 3.599352055540239 \times 10^{-5}\,b\,r+c , 
 3.266426827094104 \times 10^{-9}\,a\,r^2+
 5.71526624672386 \times 10^{-5}\,b\,r+c , 
 7.277118895509326 \times 10^{-9}\,a\,r^2+
 8.530603082730626 \times 10^{-5}\,b\,r+c , 
 1.475029730376073 \times 10^{-8}\,a\,r^2+
 1.214508019889565 \times 10^{-4}\,b\,r+c , 
 2.775001355397757 \times 10^{-8}\,a\,r^2+
 1.665833531718508 \times 10^{-4}\,b\,r+c , 
 4.915051879738995 \times 10^{-8}\,a\,r^2+
 2.216991628251896 \times 10^{-4}\,b\,r+c , 
 8.28246445511412 \times 10^{-8}\,a\,r^2+
 2.877927110806339 \times 10^{-4}\,b\,r+c , 
 1.33851622723744 \times 10^{-7}\,a\,r^2+
 3.658573803051457 \times 10^{-4}\,b\,r+c , 
 2.087442283111582 \times 10^{-7}\,a\,r^2+
 4.568853557635201 \times 10^{-4}\,b\,r+c , 
 3.156951172237287 \times 10^{-7}\,a\,r^2+
 5.618675264007778 \times 10^{-4}\,b\,r+c , 
 4.64842220857938 \times 10^{-7}\,a\,r^2+
 6.817933857540259 \times 10^{-4}\,b\,r+c , 
 6.685530482422835 \times 10^{-7}\,a\,r^2+
 8.176509330039827 \times 10^{-4}\,b\,r+c , 
 9.417277358666075 \times 10^{-7}\,a\,r^2+
 9.704265741758145 \times 10^{-4}\,b\,r+c , 
 1.30212067465563 \times 10^{-6}\,a\,r^2+0.001141105023499428\,b\,r+c
  , 1.770680532972444 \times 10^{-6}\,a\,r^2+0.001330669204938795\,b
 \,r+c , 2.371908484044149 \times 10^{-6}\,a\,r^2+
 0.001540100153900437\,b\,r+c , 3.134234435790633 \times 10^{-6}\,a\,
 r^2+0.001770376919130678\,b\,r+c , 4.090411050716832 \times 10^{-6}
 \,a\,r^2+0.002022476464811601\,b\,r+c , 
 5.277925333300395 \times 10^{-6}\,a\,r^2+0.002297373572865413\,b\,r+
 c , 6.739427552177103 \times 10^{-6}\,a\,r^2+0.002596040745477063\,b
 \,r+c , 8.523177254399114 \times 10^{-6}\,a\,r^2+
 0.002919448107844891\,b\,r+c , 1.068350611911921 \times 10^{-5}\,a\,
 r^2+0.003268563311168871\,b\,r+c , 1.328129738824626 \times 10^{-5}
 \,a\,r^2+0.003644351435886262\,b\,r+c , 
 1.638448160192355 \times 10^{-5}\,a\,r^2+0.004047774895164447\,b\,r+
 c , 2.006854835710647 \times 10^{-5}\,a\,r^2+0.004479793338660443\,b
 \,r+c , 2.44170737980647 \times 10^{-5}\,a\,r^2+0.0049413635565565\,
 b\,r+c , 2.952226353832265 \times 10^{-5}\,a\,r^2+
 0.005433439383882244\,b\,r+c , 3.548551070434468 \times 10^{-5}\,a\,
 r^2+0.005956971605131645\,b\,r+c , 4.241796878224187 \times 10^{-5}
 \,a\,r^2+0.006512907859185624\,b\,r+c , 
 5.044113893984222 \times 10^{-5}\,a\,r^2+0.007102192544548636\,b\,r+
 c , 5.968747148772726 \times 10^{-5}\,a\,r^2+0.007725766724910044\,b
 \,r+c , 7.030098113418114 \times 10^{-5}\,a\,r^2+0.00838456803503801
 \,b\,r+c , 8.243787568058321 \times 10^{-5}\,a\,r^2+
 0.009079530587017326\,b\,r+c , 9.626719779540763 \times 10^{-5}\,a\,
 r^2+0.009811584876838586\,b\,r+c , 1.11971479496896 \times 10^{-4}\,
 a\,r^2+0.0105816576913495\,b\,r+c , 1.297474089664522 \times 10^{-4}
 \,a\,r^2+0.01139067201557714\,b\,r+c , 
 1.498065093069853 \times 10^{-4}\,a\,r^2+0.01223954694042984\,b\,r+c
  , 1.723758288528179 \times 10^{-4}\,a\,r^2+0.01312919757078923\,b\,
 r+c , 1.976986426302469 \times 10^{-4}\,a\,r^2+0.01406053493400045\,
 b\,r+c , 2.260351645605837 \times 10^{-4}\,a\,r^2+
 0.01503446588876983\,b\,r+c , 2.576632699903951 \times 10^{-4}\,a\,r
 ^2+0.01605189303448024\,b\,r+c , 2.928792281266932 \times 10^{-4}\,a
 \,r^2+0.01711371462093175\,b\,r+c , 3.319984439480964 \times 10^{-4}
 \,a\,r^2+0.01822082445851714\,b\,r+c , 
 3.753562091564763 \times 10^{-4}\,a\,r^2+0.01937411182884202\,b\,r+c
  , 4.233084617271431 \times 10^{-4}\,a\,r^2+0.02057446139579705\,b\,
 r+c , 4.762325536095718 \times 10^{-4}\,a\,r^2+0.02182275311709253\,
 b\,r+c , 5.34528026124617 \times 10^{-4}\,a\,r^2+0.02311986215626333
 \,b\,r+c , 5.986173925984417 \times 10^{-4}\,a\,r^2+
 0.02446665879515308\,b\,r+c , 6.689469277678383 \times 10^{-4}\,a\,r
 ^2+0.02586400834688696\,b\,r+c , 7.459874634862211 \times 10^{-4}\,a
 \,r^2+0.02731277106934082\,b\,r+c , 8.302351902545073 \times 10^{-4}
 \,a\,r^2+0.02881380207911666\,b\,r+c , 
 9.222124640960191 \times 10^{-4}\,a\,r^2+0.03036795126603076\,b\,r+c
  , 0.001022468618290102\,a\,r^2+0.03197606320812652\,b\,r+c , 
 0.001131580779474263\,a\,r^2+0.0336389770872163\,b\,r+c , 
 0.001250154687620788\,a\,r^2+0.03535752660496472\,b\,r+c , 
 0.001378825519389357\,a\,r^2+0.03713253989951881\,b\,r+c , 
 0.001518258714353595\,a\,r^2+0.03896483946269502\,b\,r+c , 
 0.001669150803595751\,a\,r^2+0.0408552420577305\,b\,r+c , 
 0.001832230240160423\,a\,r^2+0.04280455863760801\,b\,r+c , 
 0.002008258230854871\,a\,r^2+0.04481359426396048\,b\,r+c , 
 0.002198029568880921\,a\,r^2+0.04688314802656623\,b\,r+c , 
 0.002402373466780307\,a\,r^2+0.04901401296344043\,b\,r+c , 
 0.002622154389173151\,a\,r^2+0.05120697598153157\,b\,r+c , 
 0.002858272884767075\,a\,r^2+0.05346281777803219\,b\,r+c , 
 0.003111666417112067\,a\,r^2+0.05578231276230905\,b\,r+c , 
 0.003383310193575043\,a\,r^2+0.05816622897846346\,b\,r+c , 
 0.003674217992005929\,a\,r^2+0.06061532802852698\,b\,r+c , 
 0.003985442984566339\,a\,r^2+0.0631303649963022\,b\,r+c , 
 0.004318078558190487\,a\,r^2+0.06571208837185505\,b\,r+c , 
 0.004673259131147316\,a\,r^2+0.06836123997666599\,b\,r+c , 
 0.005052160965172387\,a\,r^2+0.07107855488944881\,b\,r+c , 
 0.005456002972637555\,a\,r^2+0.07386476137264342\,b\,r+c , 
 0.005886047518226416\,a\,r^2+0.07672058079958999\,b\,r+c , 
 0.006343601214583815\,a\,r^2+0.07964672758239233\,b\,r+c , 
 0.006830015711407966\,a\,r^2+0.08264390910047736\,b\,r+c , 
 0.007346688477454374\,a\,r^2+0.0857128256298576\,b\,r+c , 
 0.007895063574921807\,a\,r^2+0.08885417027310427\,b\,r+c , 
 0.008476632425691433\,a\,r^2+0.09206862889003742\,b\,r+c , 
 0.009092934568891969\,a\,r^2+0.09535688002914089\,b\,r+c , 
 0.009745558409264787\,a\,r^2+0.0987195948597075\,b\,r+c , 
 0.01043614195580549\,a\,r^2+0.1021574371047232\,b\,r+c , 
 0.01116637355015972\,a\,r^2+0.1056710629744951\,b\,r+c , 
 0.01193799258425414\,a\,r^2+0.1092611211010309\,b\,r+c , 
 0.01275279020664547\,a\,r^2+0.1129282524731764\,b\,r+c , 
 0.01361261001707348\,a\,r^2+0.1166730903725168\,b\,r+c , 
 0.01451934874870728\,a\,r^2+0.1204962603100498\,b\,r+c , 
 0.01547495693757671\,a\,r^2+0.1243983799636342\,b\,r+c , 
 0.01648143957868493\,a\,r^2+0.1283800591162231\,b\,r+c , 
 0.01754085676830185\,a\,r^2+0.1324418995948859\,b\,r+c , 
 0.01865532433194167\,a\,r^2+0.1365844952106265\,b\,r+c , 
 0.01982701443753252\,a\,r^2+0.140808431699002\,b\,r+c , 
 0.02105815619329058\,a\,r^2+0.1451142866615502\,b\,r+c , 
 0.02235103622981523\,a\,r^2+0.1495026295080298\,b\,r+c , 
 0.02370799926592746\,a\,r^2+0.1539740213994798\,b\,r+c \right] 
\]
\end{eulerformula}
\begin{eulerprompt}
>function k(x) &= (diff(f(x),x,2))/(1+diff(f(x),x)^2)^(3/2); $'k(x)=k(x) // kelengkungan parabola 
\end{eulerprompt}
\begin{euleroutput}
  Maxima said:
  diff: second argument must be a variable; found errexp1
   -- an error. To debug this try: debugmode(true);
  
  Error in:
  ... (x) &= (diff(f(x),x,2))/(1+diff(f(x),x)^2)^(3/2); $'k(x)=k(x)  ...
                                                       ^
\end{euleroutput}
\begin{eulerprompt}
>function f(x) &= x^2+x+1; $y=f(x) // akan kita plot kelengkungan parabola untuk a=b=c=1
\end{eulerprompt}
\begin{eulerformula}
\[
\left[ 0 , 4.999958333473664 \times 10^{-5}\,r , 
 1.999933334222437 \times 10^{-4}\,r , 
 4.499662510124569 \times 10^{-4}\,r , 
 7.998933390220841 \times 10^{-4}\,r , 0.001249739605033717\,r , 
 0.00179946006479581\,r , 0.002448999746720415\,r , 
 0.003198293697380561\,r , 0.004047266988005727\,r , 
 0.004995834721974179\,r , 0.006043902043303184\,r , 
 0.00719136414613375\,r , 0.00843810628521191\,r , 
 0.009784003787362772\,r , 0.01122892206395776\,r , 
 0.01277271662437307\,r , 0.01441523309043924\,r , 
 0.01615630721187855\,r , 0.01799576488272969\,r , 
 0.01993342215875837\,r , 0.02196908527585173\,r , 
 0.02410255066939448\,r , 0.02633360499462523\,r , 
 0.02866202514797045\,r , 0.03108757828935527\,r , 
 0.03361002186548678\,r , 0.03622910363410947\,r , 
 0.03894456168922911\,r , 0.04175612448730281\,r , 
 0.04466351087439402\,r , 0.04766643011428662\,r , 
 0.05076458191755917\,r , 0.0539576564716131\,r , 0.05724533447165381
 \,r , 0.06062728715262111\,r , 0.06410317632206519\,r , 
 0.06767265439396564\,r , 0.07133536442348987\,r , 
 0.07509094014268702\,r , 0.07893900599711501\,r , 
 0.08287917718339499\,r , 0.08691105968769186\,r , 
 0.09103425032511492\,r , 0.09524833678003664\,r , 
 0.09955289764732322\,r , 0.1039475024744748\,r , 0.1084317118046711
 \,r , 0.113005077220716\,r , 0.1176671413898787\,r , 
 0.1224174381096274\,r , 0.1272554923542488\,r , 0.1321808203223502\,
 r , 0.1371929294852391\,r , 0.1422913186361759\,r , 
 0.1474754779404944\,r , 0.152744888986584\,r , 0.1580990248377314\,r
  , 0.1635373500848132\,r , 0.1690593208998367\,r , 
 0.1746643850903219\,r , 0.1803519821545206\,r , 0.1861215433374662\,
 r , 0.1919724916878484\,r , 0.1979042421157076\,r , 
 0.2039162014509444\,r , 0.2100077685026351\,r , 0.216178334119151\,r
  , 0.2224272812490723\,r , 0.2287539850028937\,r , 
 0.2351578127155118\,r , 0.2416381240094921\,r , 0.2481942708591053\,
 r , 0.2548255976551299\,r , 0.2615314412704124\,r , 
 0.2683111311261794\,r , 0.2751639892590951\,r , 0.2820893303890569\,
 r , 0.2890864619877229\,r , 0.2961546843477643\,r , 
 0.3032932906528349\,r , 0.3105015670482534\,r , 0.3177787927123868\,
 r , 0.3251242399287333\,r , 0.3325371741586922\,r , 
 0.3400168541150183\,r , 0.3475625318359485\,r , 0.3551734527599992\,
 r , 0.3628488558014202\,r , 0.3705879734263036\,r , 
 0.3783900317293359\,r , 0.3862542505111889\,r , 0.3941798433565377\,
 r , 0.4021660177127022\,r , 0.4102119749689023\,r , 
 0.418316910536117\,r , 0.4264800139275439\,r , 0.4347004688396462\,r
  , 0.4429774532337832\,r , 0.451310139418413\,r \right] =\left[ 1 , 
 2.7777500001498 \times 10^{-14}\,r^2+1.66665833335744 \times 10^{-7}
 \,r+1 , 1.777706668053906 \times 10^{-12}\,r^2+
 1.33330666692022 \times 10^{-6}\,r+1 , 
 2.024817758005038 \times 10^{-11}\,r^2+
 4.499797504338432 \times 10^{-6}\,r+1 , 
 1.137595747549299 \times 10^{-10}\,r^2+
 1.066581336583994 \times 10^{-5}\,r+1 , 
 4.339192840727639 \times 10^{-10}\,r^2+
 2.083072932167196 \times 10^{-5}\,r+1 , 
 1.295533521972174 \times 10^{-9}\,r^2+
 3.599352055540239 \times 10^{-5}\,r+1 , 
 3.266426827094104 \times 10^{-9}\,r^2+
 5.71526624672386 \times 10^{-5}\,r+1 , 
 7.277118895509326 \times 10^{-9}\,r^2+
 8.530603082730626 \times 10^{-5}\,r+1 , 
 1.475029730376073 \times 10^{-8}\,r^2+
 1.214508019889565 \times 10^{-4}\,r+1 , 
 2.775001355397757 \times 10^{-8}\,r^2+
 1.665833531718508 \times 10^{-4}\,r+1 , 
 4.915051879738995 \times 10^{-8}\,r^2+
 2.216991628251896 \times 10^{-4}\,r+1 , 
 8.28246445511412 \times 10^{-8}\,r^2+
 2.877927110806339 \times 10^{-4}\,r+1 , 
 1.33851622723744 \times 10^{-7}\,r^2+
 3.658573803051457 \times 10^{-4}\,r+1 , 
 2.087442283111582 \times 10^{-7}\,r^2+
 4.568853557635201 \times 10^{-4}\,r+1 , 
 3.156951172237287 \times 10^{-7}\,r^2+
 5.618675264007778 \times 10^{-4}\,r+1 , 
 4.64842220857938 \times 10^{-7}\,r^2+
 6.817933857540259 \times 10^{-4}\,r+1 , 
 6.685530482422835 \times 10^{-7}\,r^2+
 8.176509330039827 \times 10^{-4}\,r+1 , 
 9.417277358666075 \times 10^{-7}\,r^2+
 9.704265741758145 \times 10^{-4}\,r+1 , 
 1.30212067465563 \times 10^{-6}\,r^2+0.001141105023499428\,r+1 , 
 1.770680532972444 \times 10^{-6}\,r^2+0.001330669204938795\,r+1 , 
 2.371908484044149 \times 10^{-6}\,r^2+0.001540100153900437\,r+1 , 
 3.134234435790633 \times 10^{-6}\,r^2+0.001770376919130678\,r+1 , 
 4.090411050716832 \times 10^{-6}\,r^2+0.002022476464811601\,r+1 , 
 5.277925333300395 \times 10^{-6}\,r^2+0.002297373572865413\,r+1 , 
 6.739427552177103 \times 10^{-6}\,r^2+0.002596040745477063\,r+1 , 
 8.523177254399114 \times 10^{-6}\,r^2+0.002919448107844891\,r+1 , 
 1.068350611911921 \times 10^{-5}\,r^2+0.003268563311168871\,r+1 , 
 1.328129738824626 \times 10^{-5}\,r^2+0.003644351435886262\,r+1 , 
 1.638448160192355 \times 10^{-5}\,r^2+0.004047774895164447\,r+1 , 
 2.006854835710647 \times 10^{-5}\,r^2+0.004479793338660443\,r+1 , 
 2.44170737980647 \times 10^{-5}\,r^2+0.0049413635565565\,r+1 , 
 2.952226353832265 \times 10^{-5}\,r^2+0.005433439383882244\,r+1 , 
 3.548551070434468 \times 10^{-5}\,r^2+0.005956971605131645\,r+1 , 
 4.241796878224187 \times 10^{-5}\,r^2+0.006512907859185624\,r+1 , 
 5.044113893984222 \times 10^{-5}\,r^2+0.007102192544548636\,r+1 , 
 5.968747148772726 \times 10^{-5}\,r^2+0.007725766724910044\,r+1 , 
 7.030098113418114 \times 10^{-5}\,r^2+0.00838456803503801\,r+1 , 
 8.243787568058321 \times 10^{-5}\,r^2+0.009079530587017326\,r+1 , 
 9.626719779540763 \times 10^{-5}\,r^2+0.009811584876838586\,r+1 , 
 1.11971479496896 \times 10^{-4}\,r^2+0.0105816576913495\,r+1 , 
 1.297474089664522 \times 10^{-4}\,r^2+0.01139067201557714\,r+1 , 
 1.498065093069853 \times 10^{-4}\,r^2+0.01223954694042984\,r+1 , 
 1.723758288528179 \times 10^{-4}\,r^2+0.01312919757078923\,r+1 , 
 1.976986426302469 \times 10^{-4}\,r^2+0.01406053493400045\,r+1 , 
 2.260351645605837 \times 10^{-4}\,r^2+0.01503446588876983\,r+1 , 
 2.576632699903951 \times 10^{-4}\,r^2+0.01605189303448024\,r+1 , 
 2.928792281266932 \times 10^{-4}\,r^2+0.01711371462093175\,r+1 , 
 3.319984439480964 \times 10^{-4}\,r^2+0.01822082445851714\,r+1 , 
 3.753562091564763 \times 10^{-4}\,r^2+0.01937411182884202\,r+1 , 
 4.233084617271431 \times 10^{-4}\,r^2+0.02057446139579705\,r+1 , 
 4.762325536095718 \times 10^{-4}\,r^2+0.02182275311709253\,r+1 , 
 5.34528026124617 \times 10^{-4}\,r^2+0.02311986215626333\,r+1 , 
 5.986173925984417 \times 10^{-4}\,r^2+0.02446665879515308\,r+1 , 
 6.689469277678383 \times 10^{-4}\,r^2+0.02586400834688696\,r+1 , 
 7.459874634862211 \times 10^{-4}\,r^2+0.02731277106934082\,r+1 , 
 8.302351902545073 \times 10^{-4}\,r^2+0.02881380207911666\,r+1 , 
 9.222124640960191 \times 10^{-4}\,r^2+0.03036795126603076\,r+1 , 
 0.001022468618290102\,r^2+0.03197606320812652\,r+1 , 
 0.001131580779474263\,r^2+0.0336389770872163\,r+1 , 
 0.001250154687620788\,r^2+0.03535752660496472\,r+1 , 
 0.001378825519389357\,r^2+0.03713253989951881\,r+1 , 
 0.001518258714353595\,r^2+0.03896483946269502\,r+1 , 
 0.001669150803595751\,r^2+0.0408552420577305\,r+1 , 
 0.001832230240160423\,r^2+0.04280455863760801\,r+1 , 
 0.002008258230854871\,r^2+0.04481359426396048\,r+1 , 
 0.002198029568880921\,r^2+0.04688314802656623\,r+1 , 
 0.002402373466780307\,r^2+0.04901401296344043\,r+1 , 
 0.002622154389173151\,r^2+0.05120697598153157\,r+1 , 
 0.002858272884767075\,r^2+0.05346281777803219\,r+1 , 
 0.003111666417112067\,r^2+0.05578231276230905\,r+1 , 
 0.003383310193575043\,r^2+0.05816622897846346\,r+1 , 
 0.003674217992005929\,r^2+0.06061532802852698\,r+1 , 
 0.003985442984566339\,r^2+0.0631303649963022\,r+1 , 
 0.004318078558190487\,r^2+0.06571208837185505\,r+1 , 
 0.004673259131147316\,r^2+0.06836123997666599\,r+1 , 
 0.005052160965172387\,r^2+0.07107855488944881\,r+1 , 
 0.005456002972637555\,r^2+0.07386476137264342\,r+1 , 
 0.005886047518226416\,r^2+0.07672058079958999\,r+1 , 
 0.006343601214583815\,r^2+0.07964672758239233\,r+1 , 
 0.006830015711407966\,r^2+0.08264390910047736\,r+1 , 
 0.007346688477454374\,r^2+0.0857128256298576\,r+1 , 
 0.007895063574921807\,r^2+0.08885417027310427\,r+1 , 
 0.008476632425691433\,r^2+0.09206862889003742\,r+1 , 
 0.009092934568891969\,r^2+0.09535688002914089\,r+1 , 
 0.009745558409264787\,r^2+0.0987195948597075\,r+1 , 
 0.01043614195580549\,r^2+0.1021574371047232\,r+1 , 
 0.01116637355015972\,r^2+0.1056710629744951\,r+1 , 
 0.01193799258425414\,r^2+0.1092611211010309\,r+1 , 
 0.01275279020664547\,r^2+0.1129282524731764\,r+1 , 
 0.01361261001707348\,r^2+0.1166730903725168\,r+1 , 
 0.01451934874870728\,r^2+0.1204962603100498\,r+1 , 
 0.01547495693757671\,r^2+0.1243983799636342\,r+1 , 
 0.01648143957868493\,r^2+0.1283800591162231\,r+1 , 
 0.01754085676830185\,r^2+0.1324418995948859\,r+1 , 
 0.01865532433194167\,r^2+0.1365844952106265\,r+1 , 
 0.01982701443753252\,r^2+0.140808431699002\,r+1 , 
 0.02105815619329058\,r^2+0.1451142866615502\,r+1 , 
 0.02235103622981523\,r^2+0.1495026295080298\,r+1 , 
 0.02370799926592746\,r^2+0.1539740213994798\,r+1 \right] 
\]
\end{eulerformula}
\begin{eulerprompt}
>function k(x) &= (diff(f(x),x,2))/(1+diff(f(x),x)^2)^(3/2); $'k(x)=k(x) // kelengkungan parabola 
\end{eulerprompt}
\begin{euleroutput}
  Maxima said:
  diff: second argument must be a variable; found errexp1
   -- an error. To debug this try: debugmode(true);
  
  Error in:
  ... (x) &= (diff(f(x),x,2))/(1+diff(f(x),x)^2)^(3/2); $'k(x)=k(x)  ...
                                                       ^
\end{euleroutput}
\begin{eulercomment}
Berikut kita gambar parabola tersebut beserta kurva kelengkungan, kurva jari-jari kelengkungan dan salah satu lingkaran oskulasi
di titik puncak parabola. Perhatikan, puncak parabola dan jari-jari lingkaran oskulasi di puncak parabola adalah

\end{eulercomment}
\begin{eulerformula}
\[
(-1/2,3/4),\ 1/k(2)=1/2,
\]
\end{eulerformula}
\begin{eulercomment}
sehingga pusat lingkaran oskulasi adalah (-1/2, 5/4).
\end{eulercomment}
\begin{eulerprompt}
>plot2d(["f(x)", "k(x)"],-2,1, color=[blue,red]); plot2d("1/k(x)",-1.5,1,color=green,>add); ...
>plot2d("-1/2+1/k(-1/2)*cos(x)","5/4+1/k(-1/2)*sin(x)",xmin=0,xmax=2pi,>add,color=blue):
\end{eulerprompt}
\begin{euleroutput}
  Error : f(x) does not produce a real or column vector
  
  Error generated by error() command
  
  %ploteval:
      error(f$|" does not produce a real or column vector"); 
  adaptiveevalone:
      s=%ploteval(g$,t;args());
  Try "trace errors" to inspect local variables after errors.
  plot2d:
      dw/n,dw/n^2,dw/n,auto;args());
\end{euleroutput}
\begin{eulercomment}
Untuk kurva yang dinyatakan dengan fungsi implisit

\end{eulercomment}
\begin{eulerformula}
\[
F(x,y)=0
\]
\end{eulerformula}
\begin{eulercomment}
dengan turunan-turunan parsial

\end{eulercomment}
\begin{eulerformula}
\[
F_x=\frac{\partial F}{\partial x},\ F_y=\frac{\partial F}{\partial y},\ F_{xy}=\frac{\partial}{\partial y}\left(\frac{\partial F}{\partial x}\right),\ F_{xx}=\frac{\partial}{\partial x}\left(\frac{\partial F}{\partial x}\right),\ F_{yy}=\frac{\partial}{\partial y}\left(\frac{\partial F}{\partial y}\right),
\]
\end{eulerformula}
\begin{eulercomment}
berlaku

\end{eulercomment}
\begin{eulerformula}
\[
F_x dx+ F_y dy = 0\text{ atau } \frac{dy}{dx}=-\frac{F_x}{F_y},
\]
\end{eulerformula}
\begin{eulercomment}
sehingga kurvaturnya adalah

\end{eulercomment}
\begin{eulerformula}
\[
\kappa =\frac {F_y^2F_{xx}-2F_xF_yF_{xy}+F_x^2F_{yy}}{\left(F_x^2+F_y^2\right)^{3/2}}.
\]
\end{eulerformula}
\begin{eulercomment}
(Silakan Anda turunkan sendiri!)

Contoh 1:\\
Parabola 

\end{eulercomment}
\begin{eulerformula}
\[
y=ax^2+bx+c
\]
\end{eulerformula}
\begin{eulercomment}
dapat dinyatakan ke dalam persamaan implisit

\end{eulercomment}
\begin{eulerformula}
\[
ax^2+bx+c-y=0.
\]
\end{eulerformula}
\begin{eulerprompt}
>function F(x,y) &=a*x^2+b*x+c-y; $F(x,y)
\end{eulerprompt}
\begin{eulerformula}
\[
\left[ c , 2.7777500001498 \times 10^{-14}\,a\,r^2+
 1.66665833335744 \times 10^{-7}\,b\,r-
 4.999958333473664 \times 10^{-5}\,r+c , 
 1.777706668053906 \times 10^{-12}\,a\,r^2+
 1.33330666692022 \times 10^{-6}\,b\,r-
 1.999933334222437 \times 10^{-4}\,r+c , 
 2.024817758005038 \times 10^{-11}\,a\,r^2+
 4.499797504338432 \times 10^{-6}\,b\,r-
 4.499662510124569 \times 10^{-4}\,r+c , 
 1.137595747549299 \times 10^{-10}\,a\,r^2+
 1.066581336583994 \times 10^{-5}\,b\,r-
 7.998933390220841 \times 10^{-4}\,r+c , 
 4.339192840727639 \times 10^{-10}\,a\,r^2+
 2.083072932167196 \times 10^{-5}\,b\,r-0.001249739605033717\,r+c , 
 1.295533521972174 \times 10^{-9}\,a\,r^2+
 3.599352055540239 \times 10^{-5}\,b\,r-0.00179946006479581\,r+c , 
 3.266426827094104 \times 10^{-9}\,a\,r^2+
 5.71526624672386 \times 10^{-5}\,b\,r-0.002448999746720415\,r+c , 
 7.277118895509326 \times 10^{-9}\,a\,r^2+
 8.530603082730626 \times 10^{-5}\,b\,r-0.003198293697380561\,r+c , 
 1.475029730376073 \times 10^{-8}\,a\,r^2+
 1.214508019889565 \times 10^{-4}\,b\,r-0.004047266988005727\,r+c , 
 2.775001355397757 \times 10^{-8}\,a\,r^2+
 1.665833531718508 \times 10^{-4}\,b\,r-0.004995834721974179\,r+c , 
 4.915051879738995 \times 10^{-8}\,a\,r^2+
 2.216991628251896 \times 10^{-4}\,b\,r-0.006043902043303184\,r+c , 
 8.28246445511412 \times 10^{-8}\,a\,r^2+
 2.877927110806339 \times 10^{-4}\,b\,r-0.00719136414613375\,r+c , 
 1.33851622723744 \times 10^{-7}\,a\,r^2+
 3.658573803051457 \times 10^{-4}\,b\,r-0.00843810628521191\,r+c , 
 2.087442283111582 \times 10^{-7}\,a\,r^2+
 4.568853557635201 \times 10^{-4}\,b\,r-0.009784003787362772\,r+c , 
 3.156951172237287 \times 10^{-7}\,a\,r^2+
 5.618675264007778 \times 10^{-4}\,b\,r-0.01122892206395776\,r+c , 
 4.64842220857938 \times 10^{-7}\,a\,r^2+
 6.817933857540259 \times 10^{-4}\,b\,r-0.01277271662437307\,r+c , 
 6.685530482422835 \times 10^{-7}\,a\,r^2+
 8.176509330039827 \times 10^{-4}\,b\,r-0.01441523309043924\,r+c , 
 9.417277358666075 \times 10^{-7}\,a\,r^2+
 9.704265741758145 \times 10^{-4}\,b\,r-0.01615630721187855\,r+c , 
 1.30212067465563 \times 10^{-6}\,a\,r^2+0.001141105023499428\,b\,r-
 0.01799576488272969\,r+c , 1.770680532972444 \times 10^{-6}\,a\,r^2+
 0.001330669204938795\,b\,r-0.01993342215875837\,r+c , 
 2.371908484044149 \times 10^{-6}\,a\,r^2+0.001540100153900437\,b\,r-
 0.02196908527585173\,r+c , 3.134234435790633 \times 10^{-6}\,a\,r^2+
 0.001770376919130678\,b\,r-0.02410255066939448\,r+c , 
 4.090411050716832 \times 10^{-6}\,a\,r^2+0.002022476464811601\,b\,r-
 0.02633360499462523\,r+c , 5.277925333300395 \times 10^{-6}\,a\,r^2+
 0.002297373572865413\,b\,r-0.02866202514797045\,r+c , 
 6.739427552177103 \times 10^{-6}\,a\,r^2+0.002596040745477063\,b\,r-
 0.03108757828935527\,r+c , 8.523177254399114 \times 10^{-6}\,a\,r^2+
 0.002919448107844891\,b\,r-0.03361002186548678\,r+c , 
 1.068350611911921 \times 10^{-5}\,a\,r^2+0.003268563311168871\,b\,r-
 0.03622910363410947\,r+c , 1.328129738824626 \times 10^{-5}\,a\,r^2+
 0.003644351435886262\,b\,r-0.03894456168922911\,r+c , 
 1.638448160192355 \times 10^{-5}\,a\,r^2+0.004047774895164447\,b\,r-
 0.04175612448730281\,r+c , 2.006854835710647 \times 10^{-5}\,a\,r^2+
 0.004479793338660443\,b\,r-0.04466351087439402\,r+c , 
 2.44170737980647 \times 10^{-5}\,a\,r^2+0.0049413635565565\,b\,r-
 0.04766643011428662\,r+c , 2.952226353832265 \times 10^{-5}\,a\,r^2+
 0.005433439383882244\,b\,r-0.05076458191755917\,r+c , 
 3.548551070434468 \times 10^{-5}\,a\,r^2+0.005956971605131645\,b\,r-
 0.0539576564716131\,r+c , 4.241796878224187 \times 10^{-5}\,a\,r^2+
 0.006512907859185624\,b\,r-0.05724533447165381\,r+c , 
 5.044113893984222 \times 10^{-5}\,a\,r^2+0.007102192544548636\,b\,r-
 0.06062728715262111\,r+c , 5.968747148772726 \times 10^{-5}\,a\,r^2+
 0.007725766724910044\,b\,r-0.06410317632206519\,r+c , 
 7.030098113418114 \times 10^{-5}\,a\,r^2+0.00838456803503801\,b\,r-
 0.06767265439396564\,r+c , 8.243787568058321 \times 10^{-5}\,a\,r^2+
 0.009079530587017326\,b\,r-0.07133536442348987\,r+c , 
 9.626719779540763 \times 10^{-5}\,a\,r^2+0.009811584876838586\,b\,r-
 0.07509094014268702\,r+c , 1.11971479496896 \times 10^{-4}\,a\,r^2+
 0.0105816576913495\,b\,r-0.07893900599711501\,r+c , 
 1.297474089664522 \times 10^{-4}\,a\,r^2+0.01139067201557714\,b\,r-
 0.08287917718339499\,r+c , 1.498065093069853 \times 10^{-4}\,a\,r^2+
 0.01223954694042984\,b\,r-0.08691105968769186\,r+c , 
 1.723758288528179 \times 10^{-4}\,a\,r^2+0.01312919757078923\,b\,r-
 0.09103425032511492\,r+c , 1.976986426302469 \times 10^{-4}\,a\,r^2+
 0.01406053493400045\,b\,r-0.09524833678003664\,r+c , 
 2.260351645605837 \times 10^{-4}\,a\,r^2+0.01503446588876983\,b\,r-
 0.09955289764732322\,r+c , 2.576632699903951 \times 10^{-4}\,a\,r^2+
 0.01605189303448024\,b\,r-0.1039475024744748\,r+c , 
 2.928792281266932 \times 10^{-4}\,a\,r^2+0.01711371462093175\,b\,r-
 0.1084317118046711\,r+c , 3.319984439480964 \times 10^{-4}\,a\,r^2+
 0.01822082445851714\,b\,r-0.113005077220716\,r+c , 
 3.753562091564763 \times 10^{-4}\,a\,r^2+0.01937411182884202\,b\,r-
 0.1176671413898787\,r+c , 4.233084617271431 \times 10^{-4}\,a\,r^2+
 0.02057446139579705\,b\,r-0.1224174381096274\,r+c , 
 4.762325536095718 \times 10^{-4}\,a\,r^2+0.02182275311709253\,b\,r-
 0.1272554923542488\,r+c , 5.34528026124617 \times 10^{-4}\,a\,r^2+
 0.02311986215626333\,b\,r-0.1321808203223502\,r+c , 
 5.986173925984417 \times 10^{-4}\,a\,r^2+0.02446665879515308\,b\,r-
 0.1371929294852391\,r+c , 6.689469277678383 \times 10^{-4}\,a\,r^2+
 0.02586400834688696\,b\,r-0.1422913186361759\,r+c , 
 7.459874634862211 \times 10^{-4}\,a\,r^2+0.02731277106934082\,b\,r-
 0.1474754779404944\,r+c , 8.302351902545073 \times 10^{-4}\,a\,r^2+
 0.02881380207911666\,b\,r-0.152744888986584\,r+c , 
 9.222124640960191 \times 10^{-4}\,a\,r^2+0.03036795126603076\,b\,r-
 0.1580990248377314\,r+c , 0.001022468618290102\,a\,r^2+
 0.03197606320812652\,b\,r-0.1635373500848132\,r+c , 
 0.001131580779474263\,a\,r^2+0.0336389770872163\,b\,r-
 0.1690593208998367\,r+c , 0.001250154687620788\,a\,r^2+
 0.03535752660496472\,b\,r-0.1746643850903219\,r+c , 
 0.001378825519389357\,a\,r^2+0.03713253989951881\,b\,r-
 0.1803519821545206\,r+c , 0.001518258714353595\,a\,r^2+
 0.03896483946269502\,b\,r-0.1861215433374662\,r+c , 
 0.001669150803595751\,a\,r^2+0.0408552420577305\,b\,r-
 0.1919724916878484\,r+c , 0.001832230240160423\,a\,r^2+
 0.04280455863760801\,b\,r-0.1979042421157076\,r+c , 
 0.002008258230854871\,a\,r^2+0.04481359426396048\,b\,r-
 0.2039162014509444\,r+c , 0.002198029568880921\,a\,r^2+
 0.04688314802656623\,b\,r-0.2100077685026351\,r+c , 
 0.002402373466780307\,a\,r^2+0.04901401296344043\,b\,r-
 0.216178334119151\,r+c , 0.002622154389173151\,a\,r^2+
 0.05120697598153157\,b\,r-0.2224272812490723\,r+c , 
 0.002858272884767075\,a\,r^2+0.05346281777803219\,b\,r-
 0.2287539850028937\,r+c , 0.003111666417112067\,a\,r^2+
 0.05578231276230905\,b\,r-0.2351578127155118\,r+c , 
 0.003383310193575043\,a\,r^2+0.05816622897846346\,b\,r-
 0.2416381240094921\,r+c , 0.003674217992005929\,a\,r^2+
 0.06061532802852698\,b\,r-0.2481942708591053\,r+c , 
 0.003985442984566339\,a\,r^2+0.0631303649963022\,b\,r-
 0.2548255976551299\,r+c , 0.004318078558190487\,a\,r^2+
 0.06571208837185505\,b\,r-0.2615314412704124\,r+c , 
 0.004673259131147316\,a\,r^2+0.06836123997666599\,b\,r-
 0.2683111311261794\,r+c , 0.005052160965172387\,a\,r^2+
 0.07107855488944881\,b\,r-0.2751639892590951\,r+c , 
 0.005456002972637555\,a\,r^2+0.07386476137264342\,b\,r-
 0.2820893303890569\,r+c , 0.005886047518226416\,a\,r^2+
 0.07672058079958999\,b\,r-0.2890864619877229\,r+c , 
 0.006343601214583815\,a\,r^2+0.07964672758239233\,b\,r-
 0.2961546843477643\,r+c , 0.006830015711407966\,a\,r^2+
 0.08264390910047736\,b\,r-0.3032932906528349\,r+c , 
 0.007346688477454374\,a\,r^2+0.0857128256298576\,b\,r-
 0.3105015670482534\,r+c , 0.007895063574921807\,a\,r^2+
 0.08885417027310427\,b\,r-0.3177787927123868\,r+c , 
 0.008476632425691433\,a\,r^2+0.09206862889003742\,b\,r-
 0.3251242399287333\,r+c , 0.009092934568891969\,a\,r^2+
 0.09535688002914089\,b\,r-0.3325371741586922\,r+c , 
 0.009745558409264787\,a\,r^2+0.0987195948597075\,b\,r-
 0.3400168541150183\,r+c , 0.01043614195580549\,a\,r^2+
 0.1021574371047232\,b\,r-0.3475625318359485\,r+c , 
 0.01116637355015972\,a\,r^2+0.1056710629744951\,b\,r-
 0.3551734527599992\,r+c , 0.01193799258425414\,a\,r^2+
 0.1092611211010309\,b\,r-0.3628488558014202\,r+c , 
 0.01275279020664547\,a\,r^2+0.1129282524731764\,b\,r-
 0.3705879734263036\,r+c , 0.01361261001707348\,a\,r^2+
 0.1166730903725168\,b\,r-0.3783900317293359\,r+c , 
 0.01451934874870728\,a\,r^2+0.1204962603100498\,b\,r-
 0.3862542505111889\,r+c , 0.01547495693757671\,a\,r^2+
 0.1243983799636342\,b\,r-0.3941798433565377\,r+c , 
 0.01648143957868493\,a\,r^2+0.1283800591162231\,b\,r-
 0.4021660177127022\,r+c , 0.01754085676830185\,a\,r^2+
 0.1324418995948859\,b\,r-0.4102119749689023\,r+c , 
 0.01865532433194167\,a\,r^2+0.1365844952106265\,b\,r-
 0.418316910536117\,r+c , 0.01982701443753252\,a\,r^2+
 0.140808431699002\,b\,r-0.4264800139275439\,r+c , 
 0.02105815619329058\,a\,r^2+0.1451142866615502\,b\,r-
 0.4347004688396462\,r+c , 0.02235103622981523\,a\,r^2+
 0.1495026295080298\,b\,r-0.4429774532337832\,r+c , 
 0.02370799926592746\,a\,r^2+0.1539740213994798\,b\,r-
 0.451310139418413\,r+c \right] 
\]
\end{eulerformula}
\begin{eulerprompt}
>Fx &= diff(F(x,y),x), Fxx &=diff(F(x,y),x,2), Fy &=diff(F(x,y),y), Fxy &=diff(diff(F(x,y),x),y), Fyy &=diff(F(x,y),y,2)  
\end{eulerprompt}
\begin{euleroutput}
  Maxima said:
  diff: second argument must be a variable; found errexp1
   -- an error. To debug this try: debugmode(true);
  
  Error in:
  Fx &= diff(F(x,y),x), Fxx &=diff(F(x,y),x,2), Fy &=diff(F(x,y) ...
                      ^
\end{euleroutput}
\begin{eulerprompt}
>function k(x) &= (Fy^2*Fxx-2*Fx*Fy*Fxy+Fx^2*Fyy)/(Fx^2+Fy^2)^(3/2); $'k(x)=k(x) // kurvatur parabola tersebut
\end{eulerprompt}
\begin{eulerformula}
\[
k\left(\left[ 0 , 1.66665833335744 \times 10^{-7}\,r , 
 1.33330666692022 \times 10^{-6}\,r , 
 4.499797504338432 \times 10^{-6}\,r , 
 1.066581336583994 \times 10^{-5}\,r , 
 2.083072932167196 \times 10^{-5}\,r , 
 3.599352055540239 \times 10^{-5}\,r , 
 5.71526624672386 \times 10^{-5}\,r , 
 8.530603082730626 \times 10^{-5}\,r , 
 1.214508019889565 \times 10^{-4}\,r , 
 1.665833531718508 \times 10^{-4}\,r , 
 2.216991628251896 \times 10^{-4}\,r , 
 2.877927110806339 \times 10^{-4}\,r , 
 3.658573803051457 \times 10^{-4}\,r , 
 4.568853557635201 \times 10^{-4}\,r , 
 5.618675264007778 \times 10^{-4}\,r , 
 6.817933857540259 \times 10^{-4}\,r , 
 8.176509330039827 \times 10^{-4}\,r , 
 9.704265741758145 \times 10^{-4}\,r , 0.001141105023499428\,r , 
 0.001330669204938795\,r , 0.001540100153900437\,r , 
 0.001770376919130678\,r , 0.002022476464811601\,r , 
 0.002297373572865413\,r , 0.002596040745477063\,r , 
 0.002919448107844891\,r , 0.003268563311168871\,r , 
 0.003644351435886262\,r , 0.004047774895164447\,r , 
 0.004479793338660443\,r , 0.0049413635565565\,r , 
 0.005433439383882244\,r , 0.005956971605131645\,r , 
 0.006512907859185624\,r , 0.007102192544548636\,r , 
 0.007725766724910044\,r , 0.00838456803503801\,r , 
 0.009079530587017326\,r , 0.009811584876838586\,r , 
 0.0105816576913495\,r , 0.01139067201557714\,r , 0.01223954694042984
 \,r , 0.01312919757078923\,r , 0.01406053493400045\,r , 
 0.01503446588876983\,r , 0.01605189303448024\,r , 
 0.01711371462093175\,r , 0.01822082445851714\,r , 
 0.01937411182884202\,r , 0.02057446139579705\,r , 
 0.02182275311709253\,r , 0.02311986215626333\,r , 
 0.02446665879515308\,r , 0.02586400834688696\,r , 
 0.02731277106934082\,r , 0.02881380207911666\,r , 
 0.03036795126603076\,r , 0.03197606320812652\,r , 0.0336389770872163
 \,r , 0.03535752660496472\,r , 0.03713253989951881\,r , 
 0.03896483946269502\,r , 0.0408552420577305\,r , 0.04280455863760801
 \,r , 0.04481359426396048\,r , 0.04688314802656623\,r , 
 0.04901401296344043\,r , 0.05120697598153157\,r , 
 0.05346281777803219\,r , 0.05578231276230905\,r , 
 0.05816622897846346\,r , 0.06061532802852698\,r , 0.0631303649963022
 \,r , 0.06571208837185505\,r , 0.06836123997666599\,r , 
 0.07107855488944881\,r , 0.07386476137264342\,r , 
 0.07672058079958999\,r , 0.07964672758239233\,r , 
 0.08264390910047736\,r , 0.0857128256298576\,r , 0.08885417027310427
 \,r , 0.09206862889003742\,r , 0.09535688002914089\,r , 
 0.0987195948597075\,r , 0.1021574371047232\,r , 0.1056710629744951\,
 r , 0.1092611211010309\,r , 0.1129282524731764\,r , 
 0.1166730903725168\,r , 0.1204962603100498\,r , 0.1243983799636342\,
 r , 0.1283800591162231\,r , 0.1324418995948859\,r , 
 0.1365844952106265\,r , 0.140808431699002\,r , 0.1451142866615502\,r
  , 0.1495026295080298\,r , 0.1539740213994798\,r \right] \right)=
 \frac{{\it Fx}^2\,{\it Fyy}+{\it Fxx}\,{\it Fy}^2-2\,{\it Fx}\,
 {\it Fxy}\,{\it Fy}}{\left({\it Fy}^2+{\it Fx}^2\right)^{\frac{3}{2}
 }}
\]
\end{eulerformula}
\begin{eulercomment}
Hasilnya sama dengan sebelumnya yang menggunakan persamaan parabola biasa.
\end{eulercomment}
\eulerheading{Latihan}
\begin{eulercomment}
- Bukalah buku Kalkulus.\\
- Cari dan pilih beberapa (paling sedikit 5 fungsi berbeda tipe/bentuk/jenis) fungsi dari buku tersebut, kemudian definisikan di
EMT pada baris-baris perintah berikut (jika perlu tambahkan lagi).\\
- Untuk setiap fungsi, tentukan anti turunannya (jika ada), hitunglah integral tentu dengan batas-batas yang menarik (Anda
tentukan sendiri), seperti contoh-contoh tersebut.\\
- Lakukan hal yang sama untuk fungsi-fungsi yang tidak dapat diintegralkan (cari sedikitnya 3 fungsi).\\
- Gambar grafik fungsi dan daerah integrasinya pada sumbu koordinat yang sama.\\
- Gunakan integral tentu untuk mencari luas daerah yang dibatasi oleh dua kurva yang berpotongan di dua titik. (Cari dan gambar
kedua kurva dan arsir (warnai) daerah yang dibatasi oleh keduanya.)\\
- Gunakan integral tentu untuk menghitung volume benda putar kurva y= f(x) yang diputar mengelilingi sumbu x dari x=a sampai x=b,
yakni

\end{eulercomment}
\begin{eulerformula}
\[
V = \int_a^b \pi (f(x)^2\ dx.
\]
\end{eulerformula}
\begin{eulercomment}
(Pilih fungsinya dan gambar kurva dan benda putar yang dihasilkan. Anda dapat mencari contoh-contoh bagaimana cara menggambar
benda hasil perputaran suatu kurva.)\\
- Gunakan integral tentu untuk menghitung panjang kurva y=f(x) dari x=a sampai x=b dengan menggunakan rumus:

\end{eulercomment}
\begin{eulerformula}
\[
S = \int_a^b \sqrt{1+(f'(x))^2} \ dx.
\]
\end{eulerformula}
\begin{eulercomment}
(Pilih fungsi dan gambar kurvanya.)\\
- Apabila fungsi dinyatakan dalam koordinat kutub x=f(r,t), y=g(r,t), r=h(t), x=a bersesuaian dengan t=t0 dan x=b bersesuian
dengan t=t1, maka rumus di atas akan menjadi:

\end{eulercomment}
\begin{eulerformula}
\[
S=\int_{t_0}^{t_1} \sqrt{x'(t)^2+y'(t)^2}\ dt.
\]
\end{eulerformula}
\begin{eulercomment}
- Pilih beberapa kurva menarik (selain lingkaran dan parabola) dari buku  kalkulus. Nyatakan setiap kurva tersebut dalam bentuk:\\
\end{eulercomment}
\begin{eulerttcomment}
  a. koordinat Kartesius (persamaan y=f(x))
  b. koordinat kutub ( r=r(theta))
  c. persamaan parametrik x=x(t), y=y(t)
  d. persamaan implit F(x,y)=0
\end{eulerttcomment}
\begin{eulercomment}
- Tentukan kurvatur masing-masing kurva dengan menggunakan keempat representasi tersebut (hasilnya harus sama).\\
- Gambarlah kurva asli, kurva kurvatur, kurva jari-jari lingkaran oskulasi, dan salah satu lingkaran oskulasinya.
\end{eulercomment}
\eulerheading{Barisan dan Deret}
\begin{eulercomment}
(Catatan: bagian ini belum lengkap. Anda dapat membaca contoh-contoh pengguanaan EMT dan
Maxima untuk menghitung limit barisan, rumus jumlah parsial suatu deret, jumlah tak hingga
suatu deret konvergen, dan sebagainya. Anda dapat mengeksplor contoh-contoh di EMT atau
perbagai panduan penggunaan Maxima di software Maxima atau dari Internet.)

Barisan dapat didefinisikan dengan beberapa cara di dalam EMT, di antaranya:

- dengan cara yang sama seperti mendefinisikan vektor dengan elemen-elemen beraturan
(menggunakan titik dua ":");\\
- menggunakan perintah "sequence" dan rumus barisan (suku ke -n);\\
- menggunakan perintah "iterate" atau "niterate";\\
- menggunakan fungsi Maxima "create\_list" atau "makelist" untuk menghasilkan barisan
simbolik;\\
- menggunakan fungsi biasa yang inputnya vektor atau barisan;\\
- menggunakan fungsi rekursif.

EMT menyediakan beberapa perintah (fungsi) terkait barisan, yakni:

- sum: menghitung jumlah semua elemen suatu barisan\\
- cumsum: jumlah kumulatif suatu barisan\\
- differences: selisih antar elemen-elemen berturutan

EMT juga dapat digunakan untuk menghitung jumlah deret berhingga maupun deret tak hingga,
dengan menggunakan perintah (fungsi) "sum". Perhitungan dapat dilakukan secara numerik
maupun simbolik dan eksak.

Berikut adalah beberapa contoh perhitungan barisan dan deret menggunakan EMT.
\end{eulercomment}
\begin{eulerprompt}
>1:10 // barisan sederhana
\end{eulerprompt}
\begin{euleroutput}
  [1,  2,  3,  4,  5,  6,  7,  8,  9,  10]
\end{euleroutput}
\begin{eulerprompt}
>1:2:30
\end{eulerprompt}
\begin{euleroutput}
  [1,  3,  5,  7,  9,  11,  13,  15,  17,  19,  21,  23,  25,  27,  29]
\end{euleroutput}
\eulerheading{Iterasi dan Barisan}
\begin{eulercomment}
EMT menyediakan fungsi iterate("g(x)", x0, n) untuk melakukan iterasi

\end{eulercomment}
\begin{eulerformula}
\[
x_{k+1}=g(x_k), \ x_0=x_0, k= 1, 2, 3, ..., n.
\]
\end{eulerformula}
\begin{eulercomment}
Berikut ini disajikan contoh-contoh penggunaan iterasi dan rekursi dengan EMT. Contoh
pertama menunjukkan pertumbuhan dari nilai awal 1000 dengan laju pertambahan 5\%, selama 10
periode.
\end{eulercomment}
\begin{eulerprompt}
>q=1.05; iterate("x*q",1000,n=10)'
\end{eulerprompt}
\begin{euleroutput}
           1000 
           1050 
         1102.5 
        1157.63 
        1215.51 
        1276.28 
         1340.1 
         1407.1 
        1477.46 
        1551.33 
        1628.89 
\end{euleroutput}
\begin{eulercomment}
Contoh berikutnya memperlihatkan bahaya menabung di bank pada masa sekarang! Dengan bunga
tabungan sebesar 6\% per tahun atau 0.5\% per bulan dipotong pajak 20\%, dan biaya administrasi
10000 per bulan, tabungan sebesar 1 juta tanpa diambil selama sekitar 10 tahunan akan habis
diambil oleh bank!
\end{eulercomment}
\begin{eulerprompt}
>r=0.005; plot2d(iterate("(1+0.8*r)*x-10000",1000000,n=130)):
\end{eulerprompt}
\eulerimg{27}{images/EMT4Kalkulus-Naela Rizqy Arofah-22305144042-152.png}
\begin{eulercomment}
Silakan Anda coba-coba, dengan tabungan minimal berapa agar tidak akan habis diambil oleh
bank dengan ketentuan bunga dan biaya administrasi seperti di atas.

Berikut adalah perhitungan minimal tabungan agar aman di bank dengan bunga sebesar r dan
biaya administrasi a, pajak bunga 20\%.
\end{eulercomment}
\begin{eulerprompt}
>$solve(0.8*r*A-a,A), $% with [r=0.005, a=10] 
\end{eulerprompt}
\begin{eulerformula}
\[
\left[ A=\frac{5\,a}{4\,r} \right] 
\]
\end{eulerformula}
\begin{eulerformula}
\[
\left[ A=2500.0 \right] 
\]
\end{eulerformula}
\begin{eulercomment}
Berikut didefinisikan fungsi untuk menghitung saldo tabungan, kemudian dilakukan iterasi.
\end{eulercomment}
\begin{eulerprompt}
>function saldo(x,r,a) := round((1+0.8*r)*x-a,2);
>iterate(\{\{"saldo",0.005,10\}\},1000,n=6)
\end{eulerprompt}
\begin{euleroutput}
  [1000,  994,  987.98,  981.93,  975.86,  969.76,  963.64]
\end{euleroutput}
\begin{eulerprompt}
>iterate(\{\{"saldo",0.005,10\}\},2000,n=6)
\end{eulerprompt}
\begin{euleroutput}
  [2000,  1998,  1995.99,  1993.97,  1991.95,  1989.92,  1987.88]
\end{euleroutput}
\begin{eulerprompt}
>iterate(\{\{"saldo",0.005,10\}\},2500,n=6)
\end{eulerprompt}
\begin{euleroutput}
  [2500,  2500,  2500,  2500,  2500,  2500,  2500]
\end{euleroutput}
\begin{eulercomment}
Tabungan senilai 2,5 juta akan aman dan tidak akan berubah nilai (jika tidak ada penarikan),
sedangkan jika tabungan awal kurang dari 2,5 juta, lama kelamaan akan berkurang meskipun
tidak pernah dilakukan penarikan uang tabungan.
\end{eulercomment}
\begin{eulerprompt}
>iterate(\{\{"saldo",0.005,10\}\},3000,n=6)
\end{eulerprompt}
\begin{euleroutput}
  [3000,  3002,  3004.01,  3006.03,  3008.05,  3010.08,  3012.12]
\end{euleroutput}
\begin{eulercomment}
Tabungan yang lebih dari 2,5 juta baru akan bertambah jika tidak ada penarikan.

Untuk barisan yang lebih kompleks dapat digunakan fungsi "sequence()". Fungsi ini menghitung
nilai-nilai x[n] dari semua nilai sebelumnya, x[1],...,x[n-1] yang diketahui.\\
Berikut adalah contoh barisan Fibonacci.

\end{eulercomment}
\begin{eulerformula}
\[
x_n = x_{n-1}+x_{n-2}, \quad x_1=1, \quad x_2 =1
\]
\end{eulerformula}
\begin{eulerprompt}
>sequence("x[n-1]+x[n-2]",[1,1],15)
\end{eulerprompt}
\begin{euleroutput}
  [1,  1,  2,  3,  5,  8,  13,  21,  34,  55,  89,  144,  233,  377,  610]
\end{euleroutput}
\begin{eulercomment}
Barisan Fibonacci memiliki banyak sifat menarik, salah satunya adalah akar pangkat ke-n suku
ke-n akan konvergen ke pecahan emas:
\end{eulercomment}
\begin{eulerprompt}
>$'(1+sqrt(5))/2=float((1+sqrt(5))/2)
\end{eulerprompt}
\begin{eulerformula}
\[
\frac{\sqrt{5}+1}{2}=1.618033988749895
\]
\end{eulerformula}
\begin{eulerprompt}
>plot2d(sequence("x[n-1]+x[n-2]",[1,1],250)^(1/(1:250))):
\end{eulerprompt}
\eulerimg{27}{images/EMT4Kalkulus-Naela Rizqy Arofah-22305144042-156.png}
\begin{eulercomment}
Barisan yang sama juga dapat dihasilkan dengan menggunakan loop.
\end{eulercomment}
\begin{eulerprompt}
>x=ones(500); for k=3 to 500; x[k]=x[k-1]+x[k-2]; end;
\end{eulerprompt}
\begin{eulercomment}
Rekursi dapat dilakukan dengan menggunakan rumus yang tergantung pada semua elemen
sebelumnya. Pada contoh berikut, elemen ke-n merupakan jumlah (n-1) elemen sebelumnya,
dimulai dengan 1 (elemen ke-1). Jelas, nilai elemen ke-n adalah 2\textasciicircum{}(n-2), untuk n=2, 4, 5,
....
\end{eulercomment}
\begin{eulerprompt}
>sequence("sum(x)",1,10)
\end{eulerprompt}
\begin{euleroutput}
  [1,  1,  2,  4,  8,  16,  32,  64,  128,  256]
\end{euleroutput}
\begin{eulercomment}
Selain menggunakan ekspresi dalam x dan n, kita juga dapat menggunakan fungsi.

Pada contoh berikut, digunakan iterasi

\end{eulercomment}
\begin{eulerformula}
\[
x_n =A \cdot x_{n-1},
\]
\end{eulerformula}
\begin{eulercomment}
dengan A suatu matriks 2x2, dan setiap x[n] merupakan matriks/vektor 2x1.
\end{eulercomment}
\begin{eulerprompt}
>A=[1,1;1,2]; function suku(x,n) := A.x[,n-1]
>sequence("suku",[1;1],6)
\end{eulerprompt}
\begin{euleroutput}
  Real 2 x 6 matrix
  
              1             2             5            13     ...
              1             3             8            21     ...
\end{euleroutput}
\begin{eulercomment}
Hasil yang sama juga dapat diperoleh dengan menggunakan fungsi perpangkatan matriks
"matrixpower()". Cara ini lebih cepat, karena hanya menggunakan perkalian matriks sebanyak
log\_2(n).

\end{eulercomment}
\begin{eulerformula}
\[
x_n=A.x_{n-1}=A^2.x_{n-2}=A^3.x_{n-3}= ... = A^{n-1}.x_1.
\]
\end{eulerformula}
\begin{eulerprompt}
>sequence("matrixpower(A,n).[1;1]",1,6)
\end{eulerprompt}
\begin{euleroutput}
  Real 2 x 6 matrix
  
              1             5            13            34     ...
              1             8            21            55     ...
\end{euleroutput}
\eulerheading{Spiral Theodorus}
\begin{eulercomment}
image: Spiral\_of\_Theodorus.png\\
Spiral Theodorus (spiral segitiga siku-siku) dapat digambar secara rekursif. Rumus
rekursifnya adalah:

\end{eulercomment}
\begin{eulerformula}
\[
x_n = \left( 1 + \frac{i}{\sqrt{n-1}} \right) \, x_{n-1}, \quad x_1=1,
\]
\end{eulerformula}
\begin{eulercomment}
yang menghasilkan barisan bilangan kompleks.
\end{eulercomment}
\begin{eulerprompt}
>function g(n) := 1+I/sqrt(n)
\end{eulerprompt}
\begin{eulercomment}
Rekursinya dapat dijalankan sebanyak 17 untuk menghasilkan barisan 17 bilangan kompleks,
kemudian digambar bilangan-bilangan kompleksnya.
\end{eulercomment}
\begin{eulerprompt}
>x=sequence("g(n-1)*x[n-1]",1,17); plot2d(x,r=3.5); textbox(latex("Spiral\(\backslash\) Theodorus"),0.4):
\end{eulerprompt}
\eulerimg{27}{images/EMT4Kalkulus-Naela Rizqy Arofah-22305144042-157.png}
\begin{eulercomment}
Selanjutnya dihubungan titik 0 dengan titik-titik kompleks tersebut menggunakan loop.
\end{eulercomment}
\begin{eulerprompt}
>for i=1:cols(x); plot2d([0,x[i]],>add); end:
\end{eulerprompt}
\eulerimg{27}{images/EMT4Kalkulus-Naela Rizqy Arofah-22305144042-158.png}
\begin{eulerprompt}
> 
\end{eulerprompt}
\begin{eulercomment}
Spiral tersebut juga dapat didefinisikan menggunakan fungsi rekursif, yang tidak memmerlukan
indeks dan bilangan kompleks. Dalam hal ini diigunakan vektor kolom pada bidang.
\end{eulercomment}
\begin{eulerprompt}
>function gstep (v) ...
\end{eulerprompt}
\begin{eulerudf}
  w=[-v[2];v[1]];
  return v+w/norm(w);
  endfunction
\end{eulerudf}
\begin{eulercomment}
Jika dilakukan iterasi 16 kali dimulai dari [1;0] akan didapatkan matriks yang memuat
vektor-vektor dari setiap iterasi.
\end{eulercomment}
\begin{eulerprompt}
>x=iterate("gstep",[1;0],16); plot2d(x[1],x[2],r=3.5,>points):
\end{eulerprompt}
\eulerimg{27}{images/EMT4Kalkulus-Naela Rizqy Arofah-22305144042-159.png}
\begin{eulercomment}
\begin{eulercomment}
\eulerheading{Kekonvergenan}
\begin{eulercomment}
Terkadang kita ingin melakukan iterasi sampai konvergen. Apabila iterasinya tidak konvergen
setelah ditunggu lama, Anda dapat menghentikannya dengan menekan tombol [ESC].
\end{eulercomment}
\begin{eulerprompt}
>iterate("cos(x)",1) // iterasi x(n+1)=cos(x(n)), dengan x(0)=1.
\end{eulerprompt}
\begin{euleroutput}
  0.739085133216
\end{euleroutput}
\begin{eulercomment}
Iterasi tersebut konvergen ke penyelesaian persamaan

\end{eulercomment}
\begin{eulerformula}
\[
x = \cos(x).
\]
\end{eulerformula}
\begin{eulercomment}
Iterasi ini juga dapat dilakukan pada interval, hasilnya adalah barisan interval yang memuat
akar tersebut.
\end{eulercomment}
\begin{eulerprompt}
>hasil := iterate("cos(x)",~1,2~) //iterasi x(n+1)=cos(x(n)), dengan interval awal (1, 2)
\end{eulerprompt}
\begin{euleroutput}
  ~0.739085133211,0.7390851332133~
\end{euleroutput}
\begin{eulercomment}
Jika interval hasil tersebut sedikit diperlebar, akan terlihat bahwa interval tersebut
memuat akar persamaan x=cos(x).
\end{eulercomment}
\begin{eulerprompt}
>h=expand(hasil,100), cos(h) << h
\end{eulerprompt}
\begin{euleroutput}
  ~0.73908513309,0.73908513333~
  1
\end{euleroutput}
\begin{eulercomment}
Iterasi juga dapat digunakan pada fungsi yang didefinisikan.
\end{eulercomment}
\begin{eulerprompt}
>function f(x) := (x+2/x)/2
\end{eulerprompt}
\begin{eulercomment}
Iterasi x(n+1)=f(x(n)) akan konvergen ke akar kuadrat 2.
\end{eulercomment}
\begin{eulerprompt}
>iterate("f",2), sqrt(2)
\end{eulerprompt}
\begin{euleroutput}
  1.41421356237
  1.41421356237
\end{euleroutput}
\begin{eulercomment}
Jika pada perintah iterate diberikan tambahan parameter n, maka hasil iterasinya akan
ditampilkan mulai dari iterasi pertama sampai ke-n.
\end{eulercomment}
\begin{eulerprompt}
>iterate("f",2,5)
\end{eulerprompt}
\begin{euleroutput}
  [2,  1.5,  1.41667,  1.41422,  1.41421,  1.41421]
\end{euleroutput}
\begin{eulercomment}
Untuk iterasi ini tidak dapat dilakukan terhadap interval.
\end{eulercomment}
\begin{eulerprompt}
>niterate("f",~1,2~,5)
\end{eulerprompt}
\begin{euleroutput}
  [ ~1,2~,  ~1,2~,  ~1,2~,  ~1,2~,  ~1,2~,  ~1,2~ ]
\end{euleroutput}
\begin{eulercomment}
Perhatikan, hasil iterasinya sama dengan interval awal. Alasannya adalah perhitungan dengan
interval bersifat terlalu longgar. Untuk meingkatkan perhitungan pada ekspresi dapat
digunakan pembagian intervalnya, menggunakan fungsi ieval().
\end{eulercomment}
\begin{eulerprompt}
>function s(x) := ieval("(x+2/x)/2",x,10)
\end{eulerprompt}
\begin{eulercomment}
Selanjutnya dapat dilakukan iterasi hingga diperoleh hasil optimal, dan intervalnya tidak
semakin mengecil. Hasilnya berupa interval yang memuat akar persamaan:

\end{eulercomment}
\begin{eulerformula}
\[
x = \frac{1}{2} \left( x + \frac{2}{x} \right).
\]
\end{eulerformula}
\begin{eulercomment}
Satu-satunya solusi adalah\\
\end{eulercomment}
\begin{eulerformula}
\[
x = \sqrt2.
\]
\end{eulerformula}
\begin{eulerprompt}
>iterate("s",~1,2~)
\end{eulerprompt}
\begin{euleroutput}
  ~1.41421356236,1.41421356239~
\end{euleroutput}
\begin{eulercomment}
Fungsi "iterate()" juga dapat bekerja pada vektor. Berikut adalah contoh fungsi vektor, yang
menghasilkan rata-rata aritmetika dan rata-rata geometri.

\end{eulercomment}
\begin{eulerformula}
\[
(a_{n+1},b_{n+1}) = \left( \frac{a_n+b_n}{2}, \sqrt{a_nb_n} \right)
\]
\end{eulerformula}
\begin{eulercomment}
Iterasi ke-n disimpan pada vektor kolom x[n].
\end{eulercomment}
\begin{eulerprompt}
>function g(x) := [(x[1]+x[2])/2;sqrt(x[1]*x[2])]
\end{eulerprompt}
\begin{eulercomment}
Iterasi dengan menggunakan fungsi tersebut akan konvergen ke rata-rata aritmetika dan
geometri dari nilai-nilai awal. 
\end{eulercomment}
\begin{eulerprompt}
>iterate("g",[1;5])
\end{eulerprompt}
\begin{euleroutput}
        2.60401 
        2.60401 
\end{euleroutput}
\begin{eulercomment}
Hasil tersebut konvergen agak cepat, seperti kita cek sebagai berikut.
\end{eulercomment}
\begin{eulerprompt}
>iterate("g",[1;5],4)
\end{eulerprompt}
\begin{euleroutput}
              1             3       2.61803       2.60403       2.60401 
              5       2.23607       2.59002       2.60399       2.60401 
\end{euleroutput}
\begin{eulercomment}
Iterasi pada interval dapat dilakukan dan stabil, namun tidak menunjukkan bahwa limitnya
pada batas-batas yang dihitung.
\end{eulercomment}
\begin{eulerprompt}
>iterate("g",[~1~;~5~],4)
\end{eulerprompt}
\begin{euleroutput}
  Interval 2 x 5 matrix
  
  ~0.999999999999999778,1.00000000000000022~     ...
  ~4.99999999999999911,5.00000000000000089~     ...
\end{euleroutput}
\begin{eulercomment}
Iterasi berikut konvergen sangat lambat.

\end{eulercomment}
\begin{eulerformula}
\[
x_{n+1} = \sqrt{x_n}.
\]
\end{eulerformula}
\begin{eulerprompt}
>iterate("sqrt(x)",2,10)
\end{eulerprompt}
\begin{euleroutput}
  [2,  1.41421,  1.18921,  1.09051,  1.04427,  1.0219,  1.01089,
  1.00543,  1.00271,  1.00135,  1.00068]
\end{euleroutput}
\begin{eulercomment}
Kekonvergenan iterasi tersebut dapat dipercepatdengan percepatan Steffenson:
\end{eulercomment}
\begin{eulerprompt}
>steffenson("sqrt(x)",2,10)
\end{eulerprompt}
\begin{euleroutput}
  [1.04888,  1.00028,  1,  1]
\end{euleroutput}
\eulerheading{Iterasi menggunakan Loop yang ditulis Langsung}
\begin{eulercomment}
Berikut adalah beberapa contoh penggunaan loop untuk melakukan iterasi yang ditulis langsung
pada baris perintah.
\end{eulercomment}
\begin{eulerprompt}
>x=2; repeat x=(x+2/x)/2; until x^2~=2; end; x,
\end{eulerprompt}
\begin{euleroutput}
  1.41421356237
\end{euleroutput}
\begin{eulercomment}
Penggabungan matriks menggunakan tanda "\textbar{}" dapat digunakan untuk menyimpan semua hasil
iterasi.
\end{eulercomment}
\begin{eulerprompt}
>v=[1]; for i=2 to 8; v=v|(v[i-1]*i); end; v,
\end{eulerprompt}
\begin{euleroutput}
  [1,  2,  6,  24,  120,  720,  5040,  40320]
\end{euleroutput}
\begin{eulercomment}
hasil iterasi juga dapat disimpan pada vektor yang sudah ada.
\end{eulercomment}
\begin{eulerprompt}
>v=ones(1,100); for i=2 to cols(v); v[i]=v[i-1]*i; end; ...
>plot2d(v,logplot=1); textbox(latex(&log(n)),x=0.5):
\end{eulerprompt}
\eulerimg{27}{images/EMT4Kalkulus-Naela Rizqy Arofah-22305144042-160.png}
\begin{eulerprompt}
>A =[0.5,0.2;0.7,0.1]; b=[2;2]; ...
>x=[1;1]; repeat xnew=A.x-b; until all(xnew~=x); x=xnew; end; ...
>x,
\end{eulerprompt}
\begin{euleroutput}
       -7.09677 
       -7.74194 
\end{euleroutput}
\eulerheading{Iterasi di dalam Fungsi}
\begin{eulercomment}
Fungsi atau program juga dapat menggunakan iterasi dan dapat digunakan untuk melakukan iterasi. Berikut adalah beberapa contoh
iterasi di dalam fungsi.

Contoh berikut adalah suatu fungsi untuk menghitung berapa lama suatu iterasi konvergen. Nilai fungsi tersebut adalah hasil akhir
iterasi dan banyak iterasi sampai konvergen.
\end{eulercomment}
\begin{eulerprompt}
>function map hiter(f$,x0) ...
\end{eulerprompt}
\begin{eulerudf}
  x=x0;
  maxiter=0;
  repeat
    xnew=f$(x);
    maxiter=maxiter+1;
    until xnew~=x;
    x=xnew;
  end;
  return maxiter;
  endfunction
\end{eulerudf}
\begin{eulercomment}
Misalnya, berikut adalah iterasi untuk mendapatkan hampiran akar kuadrat 2, cukup cepat,
konvergen pada iterasi ke-5, jika dimulai dari hampiran awal 2.
\end{eulercomment}
\begin{eulerprompt}
>hiter("(x+2/x)/2",2)
\end{eulerprompt}
\begin{euleroutput}
  5
\end{euleroutput}
\begin{eulercomment}
Karena fungsinya didefinisikan menggunakan "map". maka nilai awalnya dapat berupa vektor.
\end{eulercomment}
\begin{eulerprompt}
>x=1.5:0.1:10; hasil=hiter("(x+2/x)/2",x); ...
>  plot2d(x,hasil):
\end{eulerprompt}
\eulerimg{27}{images/EMT4Kalkulus-Naela Rizqy Arofah-22305144042-161.png}
\begin{eulercomment}
Dari gambar di atas terlihat bahwa kekonvergenan iterasinya semakin lambat, untuk nilai awal
semakin besar, namun penambahnnya tidak kontinu. Kita dapat menemukan kapan maksimum
iterasinya bertambah.
\end{eulercomment}
\begin{eulerprompt}
>hasil[1:10]
\end{eulerprompt}
\begin{euleroutput}
  [4,  5,  5,  5,  5,  5,  6,  6,  6,  6]
\end{euleroutput}
\begin{eulerprompt}
>x[nonzeros(differences(hasil))]
\end{eulerprompt}
\begin{euleroutput}
  [1.5,  2,  3.4,  6.6]
\end{euleroutput}
\begin{eulercomment}
maksimum iterasi sampai konvergen meningkat pada saat nilai awalnya 1.5, 2, 3.4, dan 6.6.

Contoh berikutnya adalah metode Newton pada polinomial kompleks berderajat 3.
\end{eulercomment}
\begin{eulerprompt}
>p &= x^3-1; newton &= x-p/diff(p,x); $newton
\end{eulerprompt}
\begin{euleroutput}
  Maxima said:
  diff: second argument must be a variable; found errexp1
   -- an error. To debug this try: debugmode(true);
  
  Error in:
  p &= x^3-1; newton &= x-p/diff(p,x); $newton ...
                                     ^
\end{euleroutput}
\begin{eulercomment}
Selanjutnya didefinisikan fungsi untuk melakukan iterasi (aslinya 10 kali).
\end{eulercomment}
\begin{eulerprompt}
>function iterasi(f$,x,n=10) ...
\end{eulerprompt}
\begin{eulerudf}
  loop 1 to n; x=f$(x); end;
  return x;
  endfunction
\end{eulerudf}
\begin{eulercomment}
Kita mulai dengan menentukan titik-titik grid pada bidang kompleksnya.
\end{eulercomment}
\begin{eulerprompt}
>r=1.5; x=linspace(-r,r,501); Z=x+I*x'; W=iterasi(newton,Z);
\end{eulerprompt}
\begin{euleroutput}
  Function newton needs at least 3 arguments!
  Use: newton (f$: call, df$: call, x: scalar complex \{, y: number, eps: none\}) 
  Error in:
  ...  x=linspace(-r,r,501); Z=x+I*x'; W=iterasi(newton,Z); ...
                                                       ^
\end{euleroutput}
\begin{eulercomment}
Berikut adalah akar-akar polinomial di atas.
\end{eulercomment}
\begin{eulerprompt}
>z=&solve(p)()
\end{eulerprompt}
\begin{euleroutput}
  Maxima said:
  solve: more equations than unknowns.
  Unknowns given :  
  [r]
  Equations given:  
  errexp1
   -- an error. To debug this try: debugmode(true);
  
  Error in:
  z=&solve(p)() ...
             ^
\end{euleroutput}
\begin{eulercomment}
Untuk menggambar hasil iterasinya, dihitung jarak dari hasil iterasi ke-10 ke masing-masing
akar, kemudian digunakan untuk menghitung warna yang akan digambar, yang menunjukkan limit
untuk masing-masing nilai awal. 

Fungsi plotrgb() menggunakan jendela gambar terkini untuk menggambar warna RGB sebagai
matriks.
\end{eulercomment}
\begin{eulerprompt}
>C=rgb(max(abs(W-z[1]),1),max(abs(W-z[2]),1),max(abs(W-z[3]),1)); ...
>  plot2d(none,-r,r,-r,r); plotrgb(C):
\end{eulerprompt}
\begin{euleroutput}
  Variable W not found!
  Error in:
  C=rgb(max(abs(W-z[1]),1),max(abs(W-z[2]),1),max(abs(W-z[3]),1) ...
                      ^
\end{euleroutput}
\eulerheading{Iterasi Simbolik}
\begin{eulercomment}
Seperti sudah dibahas sebelumnya, untuk menghasilkan barisan ekspresi simbolik dengan Maxima
dapat digunakan fungsi makelist().
\end{eulercomment}
\begin{eulerprompt}
>&powerdisp:true // untuk menampilkan deret pangkat mulai dari suku berpangkat terkecil
\end{eulerprompt}
\begin{euleroutput}
  
                                   true
  
\end{euleroutput}
\begin{eulerprompt}
>deret &= makelist(taylor(exp(x),x,0,k),k,1,3); $deret // barisan deret Taylor untuk e^x
\end{eulerprompt}
\begin{euleroutput}
  Maxima said:
  taylor: 0.1539740213994798*r cannot be a variable.
   -- an error. To debug this try: debugmode(true);
  
  Error in:
  deret &= makelist(taylor(exp(x),x,0,k),k,1,3); $deret // baris ...
                                               ^
\end{euleroutput}
\begin{eulercomment}
Untuk mengubah barisan deret tersebut menjadi vektor string di EMT digunakan fungsi
mxm2str(). Selanjutnya, vektor string/ekspresi hasilnya dapat digambar seperti menggambar
vektor eskpresi pada EMT.
\end{eulercomment}
\begin{eulerprompt}
>plot2d("exp(x)",0,3); // plot fungsi aslinya, e^x
>plot2d(mxm2str("deret"),>add,color=4:6): // plot ketiga deret taylor hampiran fungsi tersebut
\end{eulerprompt}
\begin{euleroutput}
  Maxima said:
  length: argument cannot be a symbol; found deret
   -- an error. To debug this try: debugmode(true);
  
  mxmeval:
      return evaluate(mxm(s));
  Try "trace errors" to inspect local variables after errors.
  mxm2str:
      n=mxmeval("length(VVV)");
\end{euleroutput}
\begin{eulercomment}
Selain cara di atas dapat juga dengan cara menggunakan indeks pada vektor/list yang
dihasilkan.
\end{eulercomment}
\begin{eulerprompt}
>$deret[3]
\end{eulerprompt}
\begin{eulerformula}
\[
{\it deret}_{3}
\]
\end{eulerformula}
\begin{eulerprompt}
>plot2d(["exp(x)",&deret[1],&deret[2],&deret[3]],0,3,color=1:4):
\end{eulerprompt}
\begin{euleroutput}
  deret is not a variable!
  Error in expression: deret[1]
  %ploteval:
      y0=f$(x[1],args());
  Try "trace errors" to inspect local variables after errors.
  plot2d:
      u=u_(%ploteval(xx[#],t;args()));
\end{euleroutput}
\begin{eulerprompt}
>$sum(sin(k*x)/k,k,1,5)
\end{eulerprompt}
\begin{eulerformula}
\[
\left[ 0 , \sin \left(1.66665833335744 \times 10^{-7}\,r\right)+
 \frac{\sin \left(3.333316666714881 \times 10^{-7}\,r\right)}{2}+
 \frac{\sin \left(4.999975000072321 \times 10^{-7}\,r\right)}{3}+
 \frac{\sin \left(6.666633333429761 \times 10^{-7}\,r\right)}{4}+
 \frac{\sin \left(8.333291666787201 \times 10^{-7}\,r\right)}{5} , 
 \sin \left(1.33330666692022 \times 10^{-6}\,r\right)+\frac{\sin 
 \left(2.66661333384044 \times 10^{-6}\,r\right)}{2}+\frac{\sin 
 \left(3.999920000760659 \times 10^{-6}\,r\right)}{3}+\frac{\sin 
 \left(5.333226667680879 \times 10^{-6}\,r\right)}{4}+\frac{\sin 
 \left(6.666533334601099 \times 10^{-6}\,r\right)}{5} , \sin \left(
 4.499797504338432 \times 10^{-6}\,r\right)+\frac{\sin \left(
 8.999595008676864 \times 10^{-6}\,r\right)}{2}+\frac{\sin \left(
 1.34993925130153 \times 10^{-5}\,r\right)}{3}+\frac{\sin \left(
 1.799919001735373 \times 10^{-5}\,r\right)}{4}+\frac{\sin \left(
 2.249898752169216 \times 10^{-5}\,r\right)}{5} , \sin \left(
 1.066581336583994 \times 10^{-5}\,r\right)+\frac{\sin \left(
 2.133162673167988 \times 10^{-5}\,r\right)}{2}+\frac{\sin \left(
 3.199744009751981 \times 10^{-5}\,r\right)}{3}+\frac{\sin \left(
 4.266325346335975 \times 10^{-5}\,r\right)}{4}+\frac{\sin \left(
 5.332906682919969 \times 10^{-5}\,r\right)}{5} , \sin \left(
 2.083072932167196 \times 10^{-5}\,r\right)+\frac{\sin \left(
 4.166145864334392 \times 10^{-5}\,r\right)}{2}+\frac{\sin \left(
 6.249218796501588 \times 10^{-5}\,r\right)}{3}+\frac{\sin \left(
 8.332291728668784 \times 10^{-5}\,r\right)}{4}+\frac{\sin \left(
 1.041536466083598 \times 10^{-4}\,r\right)}{5} , \sin \left(
 3.599352055540239 \times 10^{-5}\,r\right)+\frac{\sin \left(
 7.198704111080478 \times 10^{-5}\,r\right)}{2}+\frac{\sin \left(
 1.079805616662072 \times 10^{-4}\,r\right)}{3}+\frac{\sin \left(
 1.439740822216096 \times 10^{-4}\,r\right)}{4}+\frac{\sin \left(
 1.79967602777012 \times 10^{-4}\,r\right)}{5} , \sin \left(
 5.71526624672386 \times 10^{-5}\,r\right)+\frac{\sin \left(
 1.143053249344772 \times 10^{-4}\,r\right)}{2}+\frac{\sin \left(
 1.714579874017158 \times 10^{-4}\,r\right)}{3}+\frac{\sin \left(
 2.286106498689544 \times 10^{-4}\,r\right)}{4}+\frac{\sin \left(
 2.85763312336193 \times 10^{-4}\,r\right)}{5} , \sin \left(
 8.530603082730626 \times 10^{-5}\,r\right)+\frac{\sin \left(
 1.706120616546125 \times 10^{-4}\,r\right)}{2}+\frac{\sin \left(
 2.559180924819188 \times 10^{-4}\,r\right)}{3}+\frac{\sin \left(
 3.41224123309225 \times 10^{-4}\,r\right)}{4}+\frac{\sin \left(
 4.265301541365313 \times 10^{-4}\,r\right)}{5} , \sin \left(
 1.214508019889565 \times 10^{-4}\,r\right)+\frac{\sin \left(
 2.42901603977913 \times 10^{-4}\,r\right)}{2}+\frac{\sin \left(
 3.643524059668696 \times 10^{-4}\,r\right)}{3}+\frac{\sin \left(
 4.858032079558261 \times 10^{-4}\,r\right)}{4}+\frac{\sin \left(
 6.072540099447826 \times 10^{-4}\,r\right)}{5} , \sin \left(
 1.665833531718508 \times 10^{-4}\,r\right)+\frac{\sin \left(
 3.331667063437016 \times 10^{-4}\,r\right)}{2}+\frac{\sin \left(
 4.997500595155524 \times 10^{-4}\,r\right)}{3}+\frac{\sin \left(
 6.663334126874032 \times 10^{-4}\,r\right)}{4}+\frac{\sin \left(
 8.32916765859254 \times 10^{-4}\,r\right)}{5} , \sin \left(
 2.216991628251896 \times 10^{-4}\,r\right)+\frac{\sin \left(
 4.433983256503793 \times 10^{-4}\,r\right)}{2}+\frac{\sin \left(
 6.650974884755689 \times 10^{-4}\,r\right)}{3}+\frac{\sin \left(
 8.867966513007586 \times 10^{-4}\,r\right)}{4}+\frac{\sin \left(
 0.001108495814125948\,r\right)}{5} , \sin \left(
 2.877927110806339 \times 10^{-4}\,r\right)+\frac{\sin \left(
 5.755854221612677 \times 10^{-4}\,r\right)}{2}+\frac{\sin \left(
 8.633781332419016 \times 10^{-4}\,r\right)}{3}+\frac{\sin \left(
 0.001151170844322535\,r\right)}{4}+\frac{\sin \left(
 0.001438963555403169\,r\right)}{5} , \sin \left(
 3.658573803051457 \times 10^{-4}\,r\right)+\frac{\sin \left(
 7.317147606102914 \times 10^{-4}\,r\right)}{2}+\frac{\sin \left(
 0.001097572140915437\,r\right)}{3}+\frac{\sin \left(
 0.001463429521220583\,r\right)}{4}+\frac{\sin \left(
 0.001829286901525728\,r\right)}{5} , \sin \left(
 4.568853557635201 \times 10^{-4}\,r\right)+\frac{\sin \left(
 9.137707115270399 \times 10^{-4}\,r\right)}{2}+\frac{\sin \left(
 0.00137065606729056\,r\right)}{3}+\frac{\sin \left(
 0.00182754142305408\,r\right)}{4}+\frac{\sin \left(
 0.0022844267788176\,r\right)}{5} , \sin \left(
 5.618675264007778 \times 10^{-4}\,r\right)+\frac{\sin \left(
 0.001123735052801556\,r\right)}{2}+\frac{\sin \left(
 0.001685602579202333\,r\right)}{3}+\frac{\sin \left(
 0.002247470105603111\,r\right)}{4}+\frac{\sin \left(
 0.002809337632003889\,r\right)}{5} , \sin \left(
 6.817933857540259 \times 10^{-4}\,r\right)+\frac{\sin \left(
 0.001363586771508052\,r\right)}{2}+\frac{\sin \left(
 0.002045380157262078\,r\right)}{3}+\frac{\sin \left(
 0.002727173543016104\,r\right)}{4}+\frac{\sin \left(
 0.00340896692877013\,r\right)}{5} , \sin \left(
 8.176509330039827 \times 10^{-4}\,r\right)+\frac{\sin \left(
 0.001635301866007965\,r\right)}{2}+\frac{\sin \left(
 0.002452952799011948\,r\right)}{3}+\frac{\sin \left(
 0.003270603732015931\,r\right)}{4}+\frac{\sin \left(
 0.004088254665019914\,r\right)}{5} , \sin \left(
 9.704265741758145 \times 10^{-4}\,r\right)+\frac{\sin \left(
 0.001940853148351629\,r\right)}{2}+\frac{\sin \left(
 0.002911279722527443\,r\right)}{3}+\frac{\sin \left(
 0.003881706296703258\,r\right)}{4}+\frac{\sin \left(
 0.004852132870879072\,r\right)}{5} , \sin \left(0.001141105023499428
 \,r\right)+\frac{\sin \left(0.002282210046998856\,r\right)}{2}+
 \frac{\sin \left(0.003423315070498284\,r\right)}{3}+\frac{\sin 
 \left(0.004564420093997712\,r\right)}{4}+\frac{\sin \left(
 0.00570552511749714\,r\right)}{5} , \sin \left(0.001330669204938795
 \,r\right)+\frac{\sin \left(0.002661338409877589\,r\right)}{2}+
 \frac{\sin \left(0.003992007614816384\,r\right)}{3}+\frac{\sin 
 \left(0.005322676819755179\,r\right)}{4}+\frac{\sin \left(
 0.006653346024693974\,r\right)}{5} , \sin \left(0.001540100153900437
 \,r\right)+\frac{\sin \left(0.003080200307800873\,r\right)}{2}+
 \frac{\sin \left(0.00462030046170131\,r\right)}{3}+\frac{\sin \left(
 0.006160400615601747\,r\right)}{4}+\frac{\sin \left(
 0.007700500769502183\,r\right)}{5} , \sin \left(0.001770376919130678
 \,r\right)+\frac{\sin \left(0.003540753838261357\,r\right)}{2}+
 \frac{\sin \left(0.005311130757392035\,r\right)}{3}+\frac{\sin 
 \left(0.007081507676522714\,r\right)}{4}+\frac{\sin \left(
 0.008851884595653392\,r\right)}{5} , \sin \left(0.002022476464811601
 \,r\right)+\frac{\sin \left(0.004044952929623202\,r\right)}{2}+
 \frac{\sin \left(0.006067429394434803\,r\right)}{3}+\frac{\sin 
 \left(0.008089905859246405\,r\right)}{4}+\frac{\sin \left(
 0.01011238232405801\,r\right)}{5} , \sin \left(0.002297373572865413
 \,r\right)+\frac{\sin \left(0.004594747145730826\,r\right)}{2}+
 \frac{\sin \left(0.00689212071859624\,r\right)}{3}+\frac{\sin \left(
 0.009189494291461653\,r\right)}{4}+\frac{\sin \left(
 0.01148686786432707\,r\right)}{5} , \sin \left(0.002596040745477063
 \,r\right)+\frac{\sin \left(0.005192081490954126\,r\right)}{2}+
 \frac{\sin \left(0.007788122236431189\,r\right)}{3}+\frac{\sin 
 \left(0.01038416298190825\,r\right)}{4}+\frac{\sin \left(
 0.01298020372738531\,r\right)}{5} , \sin \left(0.002919448107844891
 \,r\right)+\frac{\sin \left(0.005838896215689782\,r\right)}{2}+
 \frac{\sin \left(0.008758344323534673\,r\right)}{3}+\frac{\sin 
 \left(0.01167779243137956\,r\right)}{4}+\frac{\sin \left(
 0.01459724053922445\,r\right)}{5} , \sin \left(0.003268563311168871
 \,r\right)+\frac{\sin \left(0.006537126622337741\,r\right)}{2}+
 \frac{\sin \left(0.009805689933506612\,r\right)}{3}+\frac{\sin 
 \left(0.01307425324467548\,r\right)}{4}+\frac{\sin \left(
 0.01634281655584435\,r\right)}{5} , \sin \left(0.003644351435886262
 \,r\right)+\frac{\sin \left(0.007288702871772523\,r\right)}{2}+
 \frac{\sin \left(0.01093305430765878\,r\right)}{3}+\frac{\sin \left(
 0.01457740574354505\,r\right)}{4}+\frac{\sin \left(
 0.01822175717943131\,r\right)}{5} , \sin \left(0.004047774895164447
 \,r\right)+\frac{\sin \left(0.008095549790328893\,r\right)}{2}+
 \frac{\sin \left(0.01214332468549334\,r\right)}{3}+\frac{\sin \left(
 0.01619109958065779\,r\right)}{4}+\frac{\sin \left(
 0.02023887447582223\,r\right)}{5} , \sin \left(0.004479793338660443
 \,r\right)+\frac{\sin \left(0.008959586677320885\,r\right)}{2}+
 \frac{\sin \left(0.01343938001598133\,r\right)}{3}+\frac{\sin \left(
 0.01791917335464177\,r\right)}{4}+\frac{\sin \left(
 0.02239896669330221\,r\right)}{5} , \sin \left(0.0049413635565565\,r
 \right)+\frac{\sin \left(0.009882727113112999\,r\right)}{2}+\frac{
 \sin \left(0.0148240906696695\,r\right)}{3}+\frac{\sin \left(
 0.019765454226226\,r\right)}{4}+\frac{\sin \left(0.0247068177827825
 \,r\right)}{5} , \sin \left(0.005433439383882244\,r\right)+\frac{
 \sin \left(0.01086687876776449\,r\right)}{2}+\frac{\sin \left(
 0.01630031815164673\,r\right)}{3}+\frac{\sin \left(
 0.02173375753552897\,r\right)}{4}+\frac{\sin \left(
 0.02716719691941122\,r\right)}{5} , \sin \left(0.005956971605131645
 \,r\right)+\frac{\sin \left(0.01191394321026329\,r\right)}{2}+\frac{
 \sin \left(0.01787091481539493\,r\right)}{3}+\frac{\sin \left(
 0.02382788642052658\,r\right)}{4}+\frac{\sin \left(
 0.02978485802565822\,r\right)}{5} , \sin \left(0.006512907859185624
 \,r\right)+\frac{\sin \left(0.01302581571837125\,r\right)}{2}+\frac{
 \sin \left(0.01953872357755687\,r\right)}{3}+\frac{\sin \left(
 0.0260516314367425\,r\right)}{4}+\frac{\sin \left(
 0.03256453929592812\,r\right)}{5} , \sin \left(0.007102192544548636
 \,r\right)+\frac{\sin \left(0.01420438508909727\,r\right)}{2}+\frac{
 \sin \left(0.02130657763364591\,r\right)}{3}+\frac{\sin \left(
 0.02840877017819454\,r\right)}{4}+\frac{\sin \left(
 0.03551096272274318\,r\right)}{5} , \sin \left(0.007725766724910044
 \,r\right)+\frac{\sin \left(0.01545153344982009\,r\right)}{2}+\frac{
 \sin \left(0.02317730017473013\,r\right)}{3}+\frac{\sin \left(
 0.03090306689964017\,r\right)}{4}+\frac{\sin \left(
 0.03862883362455022\,r\right)}{5} , \sin \left(0.00838456803503801\,
 r\right)+\frac{\sin \left(0.01676913607007602\,r\right)}{2}+\frac{
 \sin \left(0.02515370410511403\,r\right)}{3}+\frac{\sin \left(
 0.03353827214015204\,r\right)}{4}+\frac{\sin \left(
 0.04192284017519005\,r\right)}{5} , \sin \left(0.009079530587017326
 \,r\right)+\frac{\sin \left(0.01815906117403465\,r\right)}{2}+\frac{
 \sin \left(0.02723859176105198\,r\right)}{3}+\frac{\sin \left(
 0.0363181223480693\,r\right)}{4}+\frac{\sin \left(
 0.04539765293508663\,r\right)}{5} , \sin \left(0.009811584876838586
 \,r\right)+\frac{\sin \left(0.01962316975367717\,r\right)}{2}+\frac{
 \sin \left(0.02943475463051576\,r\right)}{3}+\frac{\sin \left(
 0.03924633950735434\,r\right)}{4}+\frac{\sin \left(
 0.04905792438419293\,r\right)}{5} , \sin \left(0.0105816576913495\,r
 \right)+\frac{\sin \left(0.021163315382699\,r\right)}{2}+\frac{\sin 
 \left(0.0317449730740485\,r\right)}{3}+\frac{\sin \left(
 0.042326630765398\,r\right)}{4}+\frac{\sin \left(0.0529082884567475
 \,r\right)}{5} , \sin \left(0.01139067201557714\,r\right)+\frac{
 \sin \left(0.02278134403115428\,r\right)}{2}+\frac{\sin \left(
 0.03417201604673142\,r\right)}{3}+\frac{\sin \left(
 0.04556268806230857\,r\right)}{4}+\frac{\sin \left(
 0.05695336007788571\,r\right)}{5} , \sin \left(0.01223954694042984\,
 r\right)+\frac{\sin \left(0.02447909388085967\,r\right)}{2}+\frac{
 \sin \left(0.03671864082128951\,r\right)}{3}+\frac{\sin \left(
 0.04895818776171934\,r\right)}{4}+\frac{\sin \left(
 0.06119773470214918\,r\right)}{5} , \sin \left(0.01312919757078923\,
 r\right)+\frac{\sin \left(0.02625839514157846\,r\right)}{2}+\frac{
 \sin \left(0.03938759271236769\,r\right)}{3}+\frac{\sin \left(
 0.05251679028315692\,r\right)}{4}+\frac{\sin \left(
 0.06564598785394615\,r\right)}{5} , \sin \left(0.01406053493400045\,
 r\right)+\frac{\sin \left(0.02812106986800089\,r\right)}{2}+\frac{
 \sin \left(0.04218160480200134\,r\right)}{3}+\frac{\sin \left(
 0.05624213973600178\,r\right)}{4}+\frac{\sin \left(
 0.07030267467000223\,r\right)}{5} , \sin \left(0.01503446588876983\,
 r\right)+\frac{\sin \left(0.03006893177753966\,r\right)}{2}+\frac{
 \sin \left(0.0451033976663095\,r\right)}{3}+\frac{\sin \left(
 0.06013786355507933\,r\right)}{4}+\frac{\sin \left(
 0.07517232944384916\,r\right)}{5} , \sin \left(0.01605189303448024\,
 r\right)+\frac{\sin \left(0.03210378606896047\,r\right)}{2}+\frac{
 \sin \left(0.04815567910344071\,r\right)}{3}+\frac{\sin \left(
 0.06420757213792094\,r\right)}{4}+\frac{\sin \left(
 0.08025946517240118\,r\right)}{5} , \sin \left(0.01711371462093175\,
 r\right)+\frac{\sin \left(0.03422742924186351\,r\right)}{2}+\frac{
 \sin \left(0.05134114386279526\,r\right)}{3}+\frac{\sin \left(
 0.06845485848372701\,r\right)}{4}+\frac{\sin \left(
 0.08556857310465876\,r\right)}{5} , \sin \left(0.01822082445851714\,
 r\right)+\frac{\sin \left(0.03644164891703428\,r\right)}{2}+\frac{
 \sin \left(0.05466247337555141\,r\right)}{3}+\frac{\sin \left(
 0.07288329783406855\,r\right)}{4}+\frac{\sin \left(
 0.09110412229258569\,r\right)}{5} , \sin \left(0.01937411182884202\,
 r\right)+\frac{\sin \left(0.03874822365768404\,r\right)}{2}+\frac{
 \sin \left(0.05812233548652607\,r\right)}{3}+\frac{\sin \left(
 0.07749644731536809\,r\right)}{4}+\frac{\sin \left(
 0.09687055914421011\,r\right)}{5} , \sin \left(0.02057446139579705\,
 r\right)+\frac{\sin \left(0.0411489227915941\,r\right)}{2}+\frac{
 \sin \left(0.06172338418739115\,r\right)}{3}+\frac{\sin \left(
 0.0822978455831882\,r\right)}{4}+\frac{\sin \left(0.1028723069789853
 \,r\right)}{5} , \sin \left(0.02182275311709253\,r\right)+\frac{
 \sin \left(0.04364550623418506\,r\right)}{2}+\frac{\sin \left(
 0.06546825935127759\,r\right)}{3}+\frac{\sin \left(
 0.08729101246837012\,r\right)}{4}+\frac{\sin \left(
 0.1091137655854627\,r\right)}{5} , \sin \left(0.02311986215626333\,r
 \right)+\frac{\sin \left(0.04623972431252665\,r\right)}{2}+\frac{
 \sin \left(0.06935958646878998\,r\right)}{3}+\frac{\sin \left(
 0.0924794486250533\,r\right)}{4}+\frac{\sin \left(0.1155993107813166
 \,r\right)}{5} , \sin \left(0.02446665879515308\,r\right)+\frac{
 \sin \left(0.04893331759030617\,r\right)}{2}+\frac{\sin \left(
 0.07339997638545925\,r\right)}{3}+\frac{\sin \left(
 0.09786663518061234\,r\right)}{4}+\frac{\sin \left(
 0.1223332939757654\,r\right)}{5} , \sin \left(0.02586400834688696\,r
 \right)+\frac{\sin \left(0.05172801669377391\,r\right)}{2}+\frac{
 \sin \left(0.07759202504066087\,r\right)}{3}+\frac{\sin \left(
 0.1034560333875478\,r\right)}{4}+\frac{\sin \left(0.1293200417344348
 \,r\right)}{5} , \sin \left(0.02731277106934082\,r\right)+\frac{
 \sin \left(0.05462554213868165\,r\right)}{2}+\frac{\sin \left(
 0.08193831320802247\,r\right)}{3}+\frac{\sin \left(
 0.1092510842773633\,r\right)}{4}+\frac{\sin \left(0.1365638553467041
 \,r\right)}{5} , \sin \left(0.02881380207911666\,r\right)+\frac{
 \sin \left(0.05762760415823331\,r\right)}{2}+\frac{\sin \left(
 0.08644140623734997\,r\right)}{3}+\frac{\sin \left(
 0.1152552083164666\,r\right)}{4}+\frac{\sin \left(0.1440690103955833
 \,r\right)}{5} , \sin \left(0.03036795126603076\,r\right)+\frac{
 \sin \left(0.06073590253206151\,r\right)}{2}+\frac{\sin \left(
 0.09110385379809227\,r\right)}{3}+\frac{\sin \left(0.121471805064123
 \,r\right)}{4}+\frac{\sin \left(0.1518397563301538\,r\right)}{5} , 
 \sin \left(0.03197606320812652\,r\right)+\frac{\sin \left(
 0.06395212641625303\,r\right)}{2}+\frac{\sin \left(
 0.09592818962437955\,r\right)}{3}+\frac{\sin \left(
 0.1279042528325061\,r\right)}{4}+\frac{\sin \left(0.1598803160406326
 \,r\right)}{5} , \sin \left(0.0336389770872163\,r\right)+\frac{\sin 
 \left(0.06727795417443261\,r\right)}{2}+\frac{\sin \left(
 0.1009169312616489\,r\right)}{3}+\frac{\sin \left(0.1345559083488652
 \,r\right)}{4}+\frac{\sin \left(0.1681948854360815\,r\right)}{5} , 
 \sin \left(0.03535752660496472\,r\right)+\frac{\sin \left(
 0.07071505320992943\,r\right)}{2}+\frac{\sin \left(
 0.1060725798148942\,r\right)}{3}+\frac{\sin \left(0.1414301064198589
 \,r\right)}{4}+\frac{\sin \left(0.1767876330248236\,r\right)}{5} , 
 \sin \left(0.03713253989951881\,r\right)+\frac{\sin \left(
 0.07426507979903763\,r\right)}{2}+\frac{\sin \left(
 0.1113976196985564\,r\right)}{3}+\frac{\sin \left(0.1485301595980753
 \,r\right)}{4}+\frac{\sin \left(0.1856626994975941\,r\right)}{5} , 
 \sin \left(0.03896483946269502\,r\right)+\frac{\sin \left(
 0.07792967892539004\,r\right)}{2}+\frac{\sin \left(
 0.1168945183880851\,r\right)}{3}+\frac{\sin \left(0.1558593578507801
 \,r\right)}{4}+\frac{\sin \left(0.1948241973134751\,r\right)}{5} , 
 \sin \left(0.0408552420577305\,r\right)+\frac{\sin \left(
 0.081710484115461\,r\right)}{2}+\frac{\sin \left(0.1225657261731915
 \,r\right)}{3}+\frac{\sin \left(0.163420968230922\,r\right)}{4}+
 \frac{\sin \left(0.2042762102886525\,r\right)}{5} , \sin \left(
 0.04280455863760801\,r\right)+\frac{\sin \left(0.08560911727521603\,
 r\right)}{2}+\frac{\sin \left(0.128413675912824\,r\right)}{3}+\frac{
 \sin \left(0.1712182345504321\,r\right)}{4}+\frac{\sin \left(
 0.2140227931880401\,r\right)}{5} , \sin \left(0.04481359426396048\,r
 \right)+\frac{\sin \left(0.08962718852792095\,r\right)}{2}+\frac{
 \sin \left(0.1344407827918814\,r\right)}{3}+\frac{\sin \left(
 0.1792543770558419\,r\right)}{4}+\frac{\sin \left(0.2240679713198024
 \,r\right)}{5} , \sin \left(0.04688314802656623\,r\right)+\frac{
 \sin \left(0.09376629605313247\,r\right)}{2}+\frac{\sin \left(
 0.1406494440796987\,r\right)}{3}+\frac{\sin \left(0.1875325921062649
 \,r\right)}{4}+\frac{\sin \left(0.2344157401328312\,r\right)}{5} , 
 \sin \left(0.04901401296344043\,r\right)+\frac{\sin \left(
 0.09802802592688087\,r\right)}{2}+\frac{\sin \left(
 0.1470420388903213\,r\right)}{3}+\frac{\sin \left(0.1960560518537617
 \,r\right)}{4}+\frac{\sin \left(0.2450700648172022\,r\right)}{5} , 
 \sin \left(0.05120697598153157\,r\right)+\frac{\sin \left(
 0.1024139519630631\,r\right)}{2}+\frac{\sin \left(0.1536209279445947
 \,r\right)}{3}+\frac{\sin \left(0.2048279039261263\,r\right)}{4}+
 \frac{\sin \left(0.2560348799076578\,r\right)}{5} , \sin \left(
 0.05346281777803219\,r\right)+\frac{\sin \left(0.1069256355560644\,r
 \right)}{2}+\frac{\sin \left(0.1603884533340966\,r\right)}{3}+\frac{
 \sin \left(0.2138512711121288\,r\right)}{4}+\frac{\sin \left(
 0.267314088890161\,r\right)}{5} , \sin \left(0.05578231276230905\,r
 \right)+\frac{\sin \left(0.1115646255246181\,r\right)}{2}+\frac{
 \sin \left(0.1673469382869271\,r\right)}{3}+\frac{\sin \left(
 0.2231292510492362\,r\right)}{4}+\frac{\sin \left(0.2789115638115452
 \,r\right)}{5} , \sin \left(0.05816622897846346\,r\right)+\frac{
 \sin \left(0.1163324579569269\,r\right)}{2}+\frac{\sin \left(
 0.1744986869353904\,r\right)}{3}+\frac{\sin \left(0.2326649159138539
 \,r\right)}{4}+\frac{\sin \left(0.2908311448923173\,r\right)}{5} , 
 \sin \left(0.06061532802852698\,r\right)+\frac{\sin \left(
 0.121230656057054\,r\right)}{2}+\frac{\sin \left(0.1818459840855809
 \,r\right)}{3}+\frac{\sin \left(0.2424613121141079\,r\right)}{4}+
 \frac{\sin \left(0.3030766401426349\,r\right)}{5} , \sin \left(
 0.0631303649963022\,r\right)+\frac{\sin \left(0.1262607299926044\,r
 \right)}{2}+\frac{\sin \left(0.1893910949889066\,r\right)}{3}+\frac{
 \sin \left(0.2525214599852088\,r\right)}{4}+\frac{\sin \left(
 0.315651824981511\,r\right)}{5} , \sin \left(0.06571208837185505\,r
 \right)+\frac{\sin \left(0.1314241767437101\,r\right)}{2}+\frac{
 \sin \left(0.1971362651155651\,r\right)}{3}+\frac{\sin \left(
 0.2628483534874202\,r\right)}{4}+\frac{\sin \left(0.3285604418592752
 \,r\right)}{5} , \sin \left(0.06836123997666599\,r\right)+\frac{
 \sin \left(0.136722479953332\,r\right)}{2}+\frac{\sin \left(
 0.205083719929998\,r\right)}{3}+\frac{\sin \left(0.273444959906664\,
 r\right)}{4}+\frac{\sin \left(0.3418061998833299\,r\right)}{5} , 
 \sin \left(0.07107855488944881\,r\right)+\frac{\sin \left(
 0.1421571097788976\,r\right)}{2}+\frac{\sin \left(0.2132356646683464
 \,r\right)}{3}+\frac{\sin \left(0.2843142195577952\,r\right)}{4}+
 \frac{\sin \left(0.355392774447244\,r\right)}{5} , \sin \left(
 0.07386476137264342\,r\right)+\frac{\sin \left(0.1477295227452868\,r
 \right)}{2}+\frac{\sin \left(0.2215942841179303\,r\right)}{3}+\frac{
 \sin \left(0.2954590454905737\,r\right)}{4}+\frac{\sin \left(
 0.3693238068632171\,r\right)}{5} , \sin \left(0.07672058079958999\,r
 \right)+\frac{\sin \left(0.15344116159918\,r\right)}{2}+\frac{\sin 
 \left(0.23016174239877\,r\right)}{3}+\frac{\sin \left(
 0.30688232319836\,r\right)}{4}+\frac{\sin \left(0.3836029039979499\,
 r\right)}{5} , \sin \left(0.07964672758239233\,r\right)+\frac{\sin 
 \left(0.1592934551647847\,r\right)}{2}+\frac{\sin \left(
 0.238940182747177\,r\right)}{3}+\frac{\sin \left(0.3185869103295693
 \,r\right)}{4}+\frac{\sin \left(0.3982336379119616\,r\right)}{5} , 
 \sin \left(0.08264390910047736\,r\right)+\frac{\sin \left(
 0.1652878182009547\,r\right)}{2}+\frac{\sin \left(0.2479317273014321
 \,r\right)}{3}+\frac{\sin \left(0.3305756364019095\,r\right)}{4}+
 \frac{\sin \left(0.4132195455023868\,r\right)}{5} , \sin \left(
 0.0857128256298576\,r\right)+\frac{\sin \left(0.1714256512597152\,r
 \right)}{2}+\frac{\sin \left(0.2571384768895728\,r\right)}{3}+\frac{
 \sin \left(0.3428513025194304\,r\right)}{4}+\frac{\sin \left(
 0.428564128149288\,r\right)}{5} , \sin \left(0.08885417027310427\,r
 \right)+\frac{\sin \left(0.1777083405462085\,r\right)}{2}+\frac{
 \sin \left(0.2665625108193128\,r\right)}{3}+\frac{\sin \left(
 0.3554166810924171\,r\right)}{4}+\frac{\sin \left(0.4442708513655214
 \,r\right)}{5} , \sin \left(0.09206862889003742\,r\right)+\frac{
 \sin \left(0.1841372577800748\,r\right)}{2}+\frac{\sin \left(
 0.2762058866701123\,r\right)}{3}+\frac{\sin \left(0.3682745155601497
 \,r\right)}{4}+\frac{\sin \left(0.4603431444501871\,r\right)}{5} , 
 \sin \left(0.09535688002914089\,r\right)+\frac{\sin \left(
 0.1907137600582818\,r\right)}{2}+\frac{\sin \left(0.2860706400874227
 \,r\right)}{3}+\frac{\sin \left(0.3814275201165636\,r\right)}{4}+
 \frac{\sin \left(0.4767844001457044\,r\right)}{5} , \sin \left(
 0.0987195948597075\,r\right)+\frac{\sin \left(0.197439189719415\,r
 \right)}{2}+\frac{\sin \left(0.2961587845791225\,r\right)}{3}+\frac{
 \sin \left(0.39487837943883\,r\right)}{4}+\frac{\sin \left(
 0.4935979742985375\,r\right)}{5} , \sin \left(0.1021574371047232\,r
 \right)+\frac{\sin \left(0.2043148742094465\,r\right)}{2}+\frac{
 \sin \left(0.3064723113141697\,r\right)}{3}+\frac{\sin \left(
 0.408629748418893\,r\right)}{4}+\frac{\sin \left(0.5107871855236162
 \,r\right)}{5} , \sin \left(0.1056710629744951\,r\right)+\frac{\sin 
 \left(0.2113421259489903\,r\right)}{2}+\frac{\sin \left(
 0.3170131889234854\,r\right)}{3}+\frac{\sin \left(0.4226842518979805
 \,r\right)}{4}+\frac{\sin \left(0.5283553148724757\,r\right)}{5} , 
 \sin \left(0.1092611211010309\,r\right)+\frac{\sin \left(
 0.2185222422020618\,r\right)}{2}+\frac{\sin \left(0.3277833633030928
 \,r\right)}{3}+\frac{\sin \left(0.4370444844041237\,r\right)}{4}+
 \frac{\sin \left(0.5463056055051546\,r\right)}{5} , \sin \left(
 0.1129282524731764\,r\right)+\frac{\sin \left(0.2258565049463528\,r
 \right)}{2}+\frac{\sin \left(0.3387847574195292\,r\right)}{3}+\frac{
 \sin \left(0.4517130098927056\,r\right)}{4}+\frac{\sin \left(
 0.564641262365882\,r\right)}{5} , \sin \left(0.1166730903725168\,r
 \right)+\frac{\sin \left(0.2333461807450337\,r\right)}{2}+\frac{
 \sin \left(0.3500192711175505\,r\right)}{3}+\frac{\sin \left(
 0.4666923614900673\,r\right)}{4}+\frac{\sin \left(0.5833654518625842
 \,r\right)}{5} , \sin \left(0.1204962603100498\,r\right)+\frac{\sin 
 \left(0.2409925206200996\,r\right)}{2}+\frac{\sin \left(
 0.3614887809301494\,r\right)}{3}+\frac{\sin \left(0.4819850412401991
 \,r\right)}{4}+\frac{\sin \left(0.6024813015502489\,r\right)}{5} , 
 \sin \left(0.1243983799636342\,r\right)+\frac{\sin \left(
 0.2487967599272685\,r\right)}{2}+\frac{\sin \left(0.3731951398909027
 \,r\right)}{3}+\frac{\sin \left(0.4975935198545369\,r\right)}{4}+
 \frac{\sin \left(0.6219918998181712\,r\right)}{5} , \sin \left(
 0.1283800591162231\,r\right)+\frac{\sin \left(0.2567601182324462\,r
 \right)}{2}+\frac{\sin \left(0.3851401773486692\,r\right)}{3}+\frac{
 \sin \left(0.5135202364648923\,r\right)}{4}+\frac{\sin \left(
 0.6419002955811154\,r\right)}{5} , \sin \left(0.1324418995948859\,r
 \right)+\frac{\sin \left(0.2648837991897719\,r\right)}{2}+\frac{
 \sin \left(0.3973256987846578\,r\right)}{3}+\frac{\sin \left(
 0.5297675983795438\,r\right)}{4}+\frac{\sin \left(0.6622094979744297
 \,r\right)}{5} , \sin \left(0.1365844952106265\,r\right)+\frac{\sin 
 \left(0.2731689904212531\,r\right)}{2}+\frac{\sin \left(
 0.4097534856318796\,r\right)}{3}+\frac{\sin \left(0.5463379808425062
 \,r\right)}{4}+\frac{\sin \left(0.6829224760531327\,r\right)}{5} , 
 \sin \left(0.140808431699002\,r\right)+\frac{\sin \left(
 0.2816168633980041\,r\right)}{2}+\frac{\sin \left(0.4224252950970061
 \,r\right)}{3}+\frac{\sin \left(0.5632337267960081\,r\right)}{4}+
 \frac{\sin \left(0.7040421584950102\,r\right)}{5} , \sin \left(
 0.1451142866615502\,r\right)+\frac{\sin \left(0.2902285733231005\,r
 \right)}{2}+\frac{\sin \left(0.4353428599846507\,r\right)}{3}+\frac{
 \sin \left(0.580457146646201\,r\right)}{4}+\frac{\sin \left(
 0.7255714333077512\,r\right)}{5} , \sin \left(0.1495026295080298\,r
 \right)+\frac{\sin \left(0.2990052590160597\,r\right)}{2}+\frac{
 \sin \left(0.4485078885240895\,r\right)}{3}+\frac{\sin \left(
 0.5980105180321194\,r\right)}{4}+\frac{\sin \left(0.7475131475401492
 \,r\right)}{5} , \sin \left(0.1539740213994798\,r\right)+\frac{\sin 
 \left(0.3079480427989596\,r\right)}{2}+\frac{\sin \left(
 0.4619220641984394\,r\right)}{3}+\frac{\sin \left(0.6158960855979192
 \,r\right)}{4}+\frac{\sin \left(0.769870106997399\,r\right)}{5}
  \right] 
\]
\end{eulerformula}
\begin{eulercomment}
Berikut adalah cara menggambar kurva

\end{eulercomment}
\begin{eulerformula}
\[
y=\sin(x) + \dfrac{\sin 3x}{3} + \dfrac{\sin 5x}{5} + \ldots.
\]
\end{eulerformula}
\begin{eulerprompt}
>plot2d(&sum(sin((2*k+1)*x)/(2*k+1),k,0,20),0,2pi):
\end{eulerprompt}
\begin{euleroutput}
  
  Maxima output too long!
  Error in:
  plot2d(&sum(sin((2*k+1)*x)/(2*k+1),k,0,20),0,2pi): ...
                                            ^
\end{euleroutput}
\begin{eulercomment}
Hal serupa juga dapat dilakukan dengan menggunakan matriks, misalkan kita akan menggambar
kurva

\end{eulercomment}
\begin{eulerformula}
\[
y = \sum_{k=1}^{100} \dfrac{\sin(kx)}{k},\quad 0\le x\le 2\pi.
\]
\end{eulerformula}
\begin{eulercomment}
\end{eulercomment}
\begin{eulerprompt}
>x=linspace(0,2pi,1000); k=1:100; y=sum(sin(k*x')/k)'; plot2d(x,y):
\end{eulerprompt}
\eulerimg{27}{images/EMT4Kalkulus-Naela Rizqy Arofah-22305144042-164.png}
\eulerheading{Tabel Fungsi}
\begin{eulercomment}
Terdapat cara menarik untuk menghasilkan barisan dengan ekspresi Maxima. Perintah
mxmtable() berguna untuk menampilkan dan menggambar barisan dan menghasilkan barisan sebagai
vektor kolom. 

Sebagai contoh berikut adalah barisan turunan ke-n x\textasciicircum{}x di x=1.
\end{eulercomment}
\begin{eulerprompt}
>mxmtable("diffat(x^x,x=1,n)","n",1,8,frac=1);
\end{eulerprompt}
\begin{euleroutput}
  Maxima said:
  diff: second argument must be a variable; found errexp1
  #0: diffat(expr=[0,1.66665833335744e-7*r,1.33330666692022e-6*r,4.499797504338432e-6*r,1.066581336583994e-5*r,2.08307...,x=[[0,1.66665833335744e-7*r,1.33330666692022e-6*r,4.499797504338432e-6*r,1.066581336583994e-5*r,2.0830...)
   -- an error. To debug this try: debugmode(true);
  
  %mxmevtable:
      return mxm("@expr,@var=@value")();
  Try "trace errors" to inspect local variables after errors.
  mxmtable:
      y[#,1]=%mxmevtable(expr,var,x[#]);
\end{euleroutput}
\begin{eulerprompt}
>$'sum(k, k, 1, n) = factor(ev(sum(k, k, 1, n),simpsum=true)) // simpsum:menghitung deret secara simbolik
\end{eulerprompt}
\begin{eulerformula}
\[
\sum_{k=1}^{n}{k}=\frac{n\,\left(1+n\right)}{2}
\]
\end{eulerformula}
\begin{eulerprompt}
>$'sum(1/(3^k+k), k, 0, inf) = factor(ev(sum(1/(3^k+k), k, 0, inf),simpsum=true))
\end{eulerprompt}
\begin{eulerformula}
\[
\sum_{k=0}^{\infty }{\frac{1}{k+3^{k}}}=\sum_{k=0}^{\infty }{\frac{
 1}{k+3^{k}}}
\]
\end{eulerformula}
\begin{eulercomment}
Di sini masih gagal, hasilnya tidak dihitung.
\end{eulercomment}
\begin{eulerprompt}
>$'sum(1/x^2, x, 1, inf)= ev(sum(1/x^2, x, 1, inf),simpsum=true) // ev: menghitung nilai ekspresi
\end{eulerprompt}
\begin{eulerformula}
\[
\sum_{x=1}^{\infty }{\frac{1}{x^2}}=\frac{\pi^2}{6}
\]
\end{eulerformula}
\begin{eulerprompt}
>$'sum((-1)^(k-1)/k, k, 1, inf) = factor(ev(sum((-1)^(x-1)/x, x, 1, inf),simpsum=true))
\end{eulerprompt}
\begin{eulerformula}
\[
\sum_{k=1}^{\infty }{\frac{\left(-1\right)^{-1+k}}{k}}=-\sum_{x=1
 }^{\infty }{\frac{\left(-1\right)^{x}}{x}}
\]
\end{eulerformula}
\begin{eulercomment}
Di sini masih gagal, hasilnya tidak dihitung.
\end{eulercomment}
\begin{eulerprompt}
>$'sum((-1)^k/(2*k-1), k, 1, inf) = factor(ev(sum((-1)^k/(2*k-1), k, 1, inf),simpsum=true))
\end{eulerprompt}
\begin{eulerformula}
\[
\sum_{k=1}^{\infty }{\frac{\left(-1\right)^{k}}{-1+2\,k}}=\sum_{k=1
 }^{\infty }{\frac{\left(-1\right)^{k}}{-1+2\,k}}
\]
\end{eulerformula}
\begin{eulerprompt}
>$ev(sum(1/n!, n, 0, inf),simpsum=true)
\end{eulerprompt}
\begin{eulerformula}
\[
\sum_{n=0}^{\infty }{\frac{1}{n!}}
\]
\end{eulerformula}
\begin{eulercomment}
Di sini masih gagal, hasilnya tidak dihitung, harusnya hasilnya e.
\end{eulercomment}
\begin{eulerprompt}
>&assume(abs(x)<1); $'sum(a*x^k, k, 0, inf)=ev(sum(a*x^k, k, 0, inf),simpsum=true), &forget(abs(x)<1);
\end{eulerprompt}
\begin{euleroutput}
  Answering "Is -94914474571+15819*r positive, negative or zero?" with "positive"
  Maxima said:
  sum: sum is divergent.
   -- an error. To debug this try: debugmode(true);
  
  Error in:
  ... k, 0, inf)=ev(sum(a*x^k, k, 0, inf),simpsum=true), &forget(abs ...
                                                       ^
\end{euleroutput}
\begin{eulercomment}
Deret geometri tak hingga, dengan asumsi rasional antara -1 dan 1.
\end{eulercomment}
\begin{eulerprompt}
>$'sum(x^k/k!,k,0,inf)=ev(sum(x^k/k!,k,0,inf),simpsum=true)
\end{eulerprompt}
\begin{eulerformula}
\[
\left[ 0 , \sum_{k=0}^{\infty }{\frac{\left(
 1.66665833335744 \times 10^{-7}\right)^{k}\,r^{k}}{k!}} , \sum_{k=0
 }^{\infty }{\frac{\left(1.33330666692022 \times 10^{-6}\right)^{k}\,
 r^{k}}{k!}} , \sum_{k=0}^{\infty }{\frac{\left(
 4.499797504338432 \times 10^{-6}\right)^{k}\,r^{k}}{k!}} , \sum_{k=0
 }^{\infty }{\frac{\left(1.066581336583994 \times 10^{-5}\right)^{k}
 \,r^{k}}{k!}} , \sum_{k=0}^{\infty }{\frac{\left(
 2.083072932167196 \times 10^{-5}\right)^{k}\,r^{k}}{k!}} , \sum_{k=0
 }^{\infty }{\frac{\left(3.599352055540239 \times 10^{-5}\right)^{k}
 \,r^{k}}{k!}} , \sum_{k=0}^{\infty }{\frac{\left(
 5.71526624672386 \times 10^{-5}\right)^{k}\,r^{k}}{k!}} , \sum_{k=0
 }^{\infty }{\frac{\left(8.530603082730626 \times 10^{-5}\right)^{k}
 \,r^{k}}{k!}} , \sum_{k=0}^{\infty }{\frac{\left(
 1.214508019889565 \times 10^{-4}\right)^{k}\,r^{k}}{k!}} , \sum_{k=0
 }^{\infty }{\frac{\left(1.665833531718508 \times 10^{-4}\right)^{k}
 \,r^{k}}{k!}} , \sum_{k=0}^{\infty }{\frac{\left(
 2.216991628251896 \times 10^{-4}\right)^{k}\,r^{k}}{k!}} , \sum_{k=0
 }^{\infty }{\frac{\left(2.877927110806339 \times 10^{-4}\right)^{k}
 \,r^{k}}{k!}} , \sum_{k=0}^{\infty }{\frac{\left(
 3.658573803051457 \times 10^{-4}\right)^{k}\,r^{k}}{k!}} , \sum_{k=0
 }^{\infty }{\frac{\left(4.568853557635201 \times 10^{-4}\right)^{k}
 \,r^{k}}{k!}} , \sum_{k=0}^{\infty }{\frac{\left(
 5.618675264007778 \times 10^{-4}\right)^{k}\,r^{k}}{k!}} , \sum_{k=0
 }^{\infty }{\frac{\left(6.817933857540259 \times 10^{-4}\right)^{k}
 \,r^{k}}{k!}} , \sum_{k=0}^{\infty }{\frac{\left(
 8.176509330039827 \times 10^{-4}\right)^{k}\,r^{k}}{k!}} , \sum_{k=0
 }^{\infty }{\frac{\left(9.704265741758145 \times 10^{-4}\right)^{k}
 \,r^{k}}{k!}} , \sum_{k=0}^{\infty }{\frac{0.001141105023499428^{k}
 \,r^{k}}{k!}} , \sum_{k=0}^{\infty }{\frac{0.001330669204938795^{k}
 \,r^{k}}{k!}} , \sum_{k=0}^{\infty }{\frac{0.001540100153900437^{k}
 \,r^{k}}{k!}} , \sum_{k=0}^{\infty }{\frac{0.001770376919130678^{k}
 \,r^{k}}{k!}} , \sum_{k=0}^{\infty }{\frac{0.002022476464811601^{k}
 \,r^{k}}{k!}} , \sum_{k=0}^{\infty }{\frac{0.002297373572865413^{k}
 \,r^{k}}{k!}} , \sum_{k=0}^{\infty }{\frac{0.002596040745477063^{k}
 \,r^{k}}{k!}} , \sum_{k=0}^{\infty }{\frac{0.002919448107844891^{k}
 \,r^{k}}{k!}} , \sum_{k=0}^{\infty }{\frac{0.003268563311168871^{k}
 \,r^{k}}{k!}} , \sum_{k=0}^{\infty }{\frac{0.003644351435886262^{k}
 \,r^{k}}{k!}} , \sum_{k=0}^{\infty }{\frac{0.004047774895164447^{k}
 \,r^{k}}{k!}} , \sum_{k=0}^{\infty }{\frac{0.004479793338660443^{k}
 \,r^{k}}{k!}} , \sum_{k=0}^{\infty }{\frac{0.0049413635565565^{k}\,r
 ^{k}}{k!}} , \sum_{k=0}^{\infty }{\frac{0.005433439383882244^{k}\,r
 ^{k}}{k!}} , \sum_{k=0}^{\infty }{\frac{0.005956971605131645^{k}\,r
 ^{k}}{k!}} , \sum_{k=0}^{\infty }{\frac{0.006512907859185624^{k}\,r
 ^{k}}{k!}} , \sum_{k=0}^{\infty }{\frac{0.007102192544548636^{k}\,r
 ^{k}}{k!}} , \sum_{k=0}^{\infty }{\frac{0.007725766724910044^{k}\,r
 ^{k}}{k!}} , \sum_{k=0}^{\infty }{\frac{0.00838456803503801^{k}\,r^{
 k}}{k!}} , \sum_{k=0}^{\infty }{\frac{0.009079530587017326^{k}\,r^{k
 }}{k!}} , \sum_{k=0}^{\infty }{\frac{0.009811584876838586^{k}\,r^{k}
 }{k!}} , \sum_{k=0}^{\infty }{\frac{0.0105816576913495^{k}\,r^{k}}{k
 !}} , \sum_{k=0}^{\infty }{\frac{0.01139067201557714^{k}\,r^{k}}{k!}
 } , \sum_{k=0}^{\infty }{\frac{0.01223954694042984^{k}\,r^{k}}{k!}}
  , \sum_{k=0}^{\infty }{\frac{0.01312919757078923^{k}\,r^{k}}{k!}}
  , \sum_{k=0}^{\infty }{\frac{0.01406053493400045^{k}\,r^{k}}{k!}}
  , \sum_{k=0}^{\infty }{\frac{0.01503446588876983^{k}\,r^{k}}{k!}}
  , \sum_{k=0}^{\infty }{\frac{0.01605189303448024^{k}\,r^{k}}{k!}}
  , \sum_{k=0}^{\infty }{\frac{0.01711371462093175^{k}\,r^{k}}{k!}}
  , \sum_{k=0}^{\infty }{\frac{0.01822082445851714^{k}\,r^{k}}{k!}}
  , \sum_{k=0}^{\infty }{\frac{0.01937411182884202^{k}\,r^{k}}{k!}}
  , \sum_{k=0}^{\infty }{\frac{0.02057446139579705^{k}\,r^{k}}{k!}}
  , \sum_{k=0}^{\infty }{\frac{0.02182275311709253^{k}\,r^{k}}{k!}}
  , \sum_{k=0}^{\infty }{\frac{0.02311986215626333^{k}\,r^{k}}{k!}}
  , \sum_{k=0}^{\infty }{\frac{0.02446665879515308^{k}\,r^{k}}{k!}}
  , \sum_{k=0}^{\infty }{\frac{0.02586400834688696^{k}\,r^{k}}{k!}}
  , \sum_{k=0}^{\infty }{\frac{0.02731277106934082^{k}\,r^{k}}{k!}}
  , \sum_{k=0}^{\infty }{\frac{0.02881380207911666^{k}\,r^{k}}{k!}}
  , \sum_{k=0}^{\infty }{\frac{0.03036795126603076^{k}\,r^{k}}{k!}}
  , \sum_{k=0}^{\infty }{\frac{0.03197606320812652^{k}\,r^{k}}{k!}}
  , \sum_{k=0}^{\infty }{\frac{0.0336389770872163^{k}\,r^{k}}{k!}} , 
 \sum_{k=0}^{\infty }{\frac{0.03535752660496472^{k}\,r^{k}}{k!}} , 
 \sum_{k=0}^{\infty }{\frac{0.03713253989951881^{k}\,r^{k}}{k!}} , 
 \sum_{k=0}^{\infty }{\frac{0.03896483946269502^{k}\,r^{k}}{k!}} , 
 \sum_{k=0}^{\infty }{\frac{0.0408552420577305^{k}\,r^{k}}{k!}} , 
 \sum_{k=0}^{\infty }{\frac{0.04280455863760801^{k}\,r^{k}}{k!}} , 
 \sum_{k=0}^{\infty }{\frac{0.04481359426396048^{k}\,r^{k}}{k!}} , 
 \sum_{k=0}^{\infty }{\frac{0.04688314802656623^{k}\,r^{k}}{k!}} , 
 \sum_{k=0}^{\infty }{\frac{0.04901401296344043^{k}\,r^{k}}{k!}} , 
 \sum_{k=0}^{\infty }{\frac{0.05120697598153157^{k}\,r^{k}}{k!}} , 
 \sum_{k=0}^{\infty }{\frac{0.05346281777803219^{k}\,r^{k}}{k!}} , 
 \sum_{k=0}^{\infty }{\frac{0.05578231276230905^{k}\,r^{k}}{k!}} , 
 \sum_{k=0}^{\infty }{\frac{0.05816622897846346^{k}\,r^{k}}{k!}} , 
 \sum_{k=0}^{\infty }{\frac{0.06061532802852698^{k}\,r^{k}}{k!}} , 
 \sum_{k=0}^{\infty }{\frac{0.0631303649963022^{k}\,r^{k}}{k!}} , 
 \sum_{k=0}^{\infty }{\frac{0.06571208837185505^{k}\,r^{k}}{k!}} , 
 \sum_{k=0}^{\infty }{\frac{0.06836123997666599^{k}\,r^{k}}{k!}} , 
 \sum_{k=0}^{\infty }{\frac{0.07107855488944881^{k}\,r^{k}}{k!}} , 
 \sum_{k=0}^{\infty }{\frac{0.07386476137264342^{k}\,r^{k}}{k!}} , 
 \sum_{k=0}^{\infty }{\frac{0.07672058079958999^{k}\,r^{k}}{k!}} , 
 \sum_{k=0}^{\infty }{\frac{0.07964672758239233^{k}\,r^{k}}{k!}} , 
 \sum_{k=0}^{\infty }{\frac{0.08264390910047736^{k}\,r^{k}}{k!}} , 
 \sum_{k=0}^{\infty }{\frac{0.0857128256298576^{k}\,r^{k}}{k!}} , 
 \sum_{k=0}^{\infty }{\frac{0.08885417027310427^{k}\,r^{k}}{k!}} , 
 \sum_{k=0}^{\infty }{\frac{0.09206862889003742^{k}\,r^{k}}{k!}} , 
 \sum_{k=0}^{\infty }{\frac{0.09535688002914089^{k}\,r^{k}}{k!}} , 
 \sum_{k=0}^{\infty }{\frac{0.0987195948597075^{k}\,r^{k}}{k!}} , 
 \sum_{k=0}^{\infty }{\frac{0.1021574371047232^{k}\,r^{k}}{k!}} , 
 \sum_{k=0}^{\infty }{\frac{0.1056710629744951^{k}\,r^{k}}{k!}} , 
 \sum_{k=0}^{\infty }{\frac{0.1092611211010309^{k}\,r^{k}}{k!}} , 
 \sum_{k=0}^{\infty }{\frac{0.1129282524731764^{k}\,r^{k}}{k!}} , 
 \sum_{k=0}^{\infty }{\frac{0.1166730903725168^{k}\,r^{k}}{k!}} , 
 \sum_{k=0}^{\infty }{\frac{0.1204962603100498^{k}\,r^{k}}{k!}} , 
 \sum_{k=0}^{\infty }{\frac{0.1243983799636342^{k}\,r^{k}}{k!}} , 
 \sum_{k=0}^{\infty }{\frac{0.1283800591162231^{k}\,r^{k}}{k!}} , 
 \sum_{k=0}^{\infty }{\frac{0.1324418995948859^{k}\,r^{k}}{k!}} , 
 \sum_{k=0}^{\infty }{\frac{0.1365844952106265^{k}\,r^{k}}{k!}} , 
 \sum_{k=0}^{\infty }{\frac{0.140808431699002^{k}\,r^{k}}{k!}} , 
 \sum_{k=0}^{\infty }{\frac{0.1451142866615502^{k}\,r^{k}}{k!}} , 
 \sum_{k=0}^{\infty }{\frac{0.1495026295080298^{k}\,r^{k}}{k!}} , 
 \sum_{k=0}^{\infty }{\frac{0.1539740213994798^{k}\,r^{k}}{k!}}
  \right] =\left[ 0 , \sum_{k=0}^{\infty }{\frac{\left(
 1.66665833335744 \times 10^{-7}\right)^{k}\,r^{k}}{k!}} , \sum_{k=0
 }^{\infty }{\frac{\left(1.33330666692022 \times 10^{-6}\right)^{k}\,
 r^{k}}{k!}} , \sum_{k=0}^{\infty }{\frac{\left(
 4.499797504338432 \times 10^{-6}\right)^{k}\,r^{k}}{k!}} , \sum_{k=0
 }^{\infty }{\frac{\left(1.066581336583994 \times 10^{-5}\right)^{k}
 \,r^{k}}{k!}} , \sum_{k=0}^{\infty }{\frac{\left(
 2.083072932167196 \times 10^{-5}\right)^{k}\,r^{k}}{k!}} , \sum_{k=0
 }^{\infty }{\frac{\left(3.599352055540239 \times 10^{-5}\right)^{k}
 \,r^{k}}{k!}} , \sum_{k=0}^{\infty }{\frac{\left(
 5.71526624672386 \times 10^{-5}\right)^{k}\,r^{k}}{k!}} , \sum_{k=0
 }^{\infty }{\frac{\left(8.530603082730626 \times 10^{-5}\right)^{k}
 \,r^{k}}{k!}} , \sum_{k=0}^{\infty }{\frac{\left(
 1.214508019889565 \times 10^{-4}\right)^{k}\,r^{k}}{k!}} , \sum_{k=0
 }^{\infty }{\frac{\left(1.665833531718508 \times 10^{-4}\right)^{k}
 \,r^{k}}{k!}} , \sum_{k=0}^{\infty }{\frac{\left(
 2.216991628251896 \times 10^{-4}\right)^{k}\,r^{k}}{k!}} , \sum_{k=0
 }^{\infty }{\frac{\left(2.877927110806339 \times 10^{-4}\right)^{k}
 \,r^{k}}{k!}} , \sum_{k=0}^{\infty }{\frac{\left(
 3.658573803051457 \times 10^{-4}\right)^{k}\,r^{k}}{k!}} , \sum_{k=0
 }^{\infty }{\frac{\left(4.568853557635201 \times 10^{-4}\right)^{k}
 \,r^{k}}{k!}} , \sum_{k=0}^{\infty }{\frac{\left(
 5.618675264007778 \times 10^{-4}\right)^{k}\,r^{k}}{k!}} , \sum_{k=0
 }^{\infty }{\frac{\left(6.817933857540259 \times 10^{-4}\right)^{k}
 \,r^{k}}{k!}} , \sum_{k=0}^{\infty }{\frac{\left(
 8.176509330039827 \times 10^{-4}\right)^{k}\,r^{k}}{k!}} , \sum_{k=0
 }^{\infty }{\frac{\left(9.704265741758145 \times 10^{-4}\right)^{k}
 \,r^{k}}{k!}} , \sum_{k=0}^{\infty }{\frac{0.001141105023499428^{k}
 \,r^{k}}{k!}} , \sum_{k=0}^{\infty }{\frac{0.001330669204938795^{k}
 \,r^{k}}{k!}} , \sum_{k=0}^{\infty }{\frac{0.001540100153900437^{k}
 \,r^{k}}{k!}} , \sum_{k=0}^{\infty }{\frac{0.001770376919130678^{k}
 \,r^{k}}{k!}} , \sum_{k=0}^{\infty }{\frac{0.002022476464811601^{k}
 \,r^{k}}{k!}} , \sum_{k=0}^{\infty }{\frac{0.002297373572865413^{k}
 \,r^{k}}{k!}} , \sum_{k=0}^{\infty }{\frac{0.002596040745477063^{k}
 \,r^{k}}{k!}} , \sum_{k=0}^{\infty }{\frac{0.002919448107844891^{k}
 \,r^{k}}{k!}} , \sum_{k=0}^{\infty }{\frac{0.003268563311168871^{k}
 \,r^{k}}{k!}} , \sum_{k=0}^{\infty }{\frac{0.003644351435886262^{k}
 \,r^{k}}{k!}} , \sum_{k=0}^{\infty }{\frac{0.004047774895164447^{k}
 \,r^{k}}{k!}} , \sum_{k=0}^{\infty }{\frac{0.004479793338660443^{k}
 \,r^{k}}{k!}} , \sum_{k=0}^{\infty }{\frac{0.0049413635565565^{k}\,r
 ^{k}}{k!}} , \sum_{k=0}^{\infty }{\frac{0.005433439383882244^{k}\,r
 ^{k}}{k!}} , \sum_{k=0}^{\infty }{\frac{0.005956971605131645^{k}\,r
 ^{k}}{k!}} , \sum_{k=0}^{\infty }{\frac{0.006512907859185624^{k}\,r
 ^{k}}{k!}} , \sum_{k=0}^{\infty }{\frac{0.007102192544548636^{k}\,r
 ^{k}}{k!}} , \sum_{k=0}^{\infty }{\frac{0.007725766724910044^{k}\,r
 ^{k}}{k!}} , \sum_{k=0}^{\infty }{\frac{0.00838456803503801^{k}\,r^{
 k}}{k!}} , \sum_{k=0}^{\infty }{\frac{0.009079530587017326^{k}\,r^{k
 }}{k!}} , \sum_{k=0}^{\infty }{\frac{0.009811584876838586^{k}\,r^{k}
 }{k!}} , \sum_{k=0}^{\infty }{\frac{0.0105816576913495^{k}\,r^{k}}{k
 !}} , \sum_{k=0}^{\infty }{\frac{0.01139067201557714^{k}\,r^{k}}{k!}
 } , \sum_{k=0}^{\infty }{\frac{0.01223954694042984^{k}\,r^{k}}{k!}}
  , \sum_{k=0}^{\infty }{\frac{0.01312919757078923^{k}\,r^{k}}{k!}}
  , \sum_{k=0}^{\infty }{\frac{0.01406053493400045^{k}\,r^{k}}{k!}}
  , \sum_{k=0}^{\infty }{\frac{0.01503446588876983^{k}\,r^{k}}{k!}}
  , \sum_{k=0}^{\infty }{\frac{0.01605189303448024^{k}\,r^{k}}{k!}}
  , \sum_{k=0}^{\infty }{\frac{0.01711371462093175^{k}\,r^{k}}{k!}}
  , \sum_{k=0}^{\infty }{\frac{0.01822082445851714^{k}\,r^{k}}{k!}}
  , \sum_{k=0}^{\infty }{\frac{0.01937411182884202^{k}\,r^{k}}{k!}}
  , \sum_{k=0}^{\infty }{\frac{0.02057446139579705^{k}\,r^{k}}{k!}}
  , \sum_{k=0}^{\infty }{\frac{0.02182275311709253^{k}\,r^{k}}{k!}}
  , \sum_{k=0}^{\infty }{\frac{0.02311986215626333^{k}\,r^{k}}{k!}}
  , \sum_{k=0}^{\infty }{\frac{0.02446665879515308^{k}\,r^{k}}{k!}}
  , \sum_{k=0}^{\infty }{\frac{0.02586400834688696^{k}\,r^{k}}{k!}}
  , \sum_{k=0}^{\infty }{\frac{0.02731277106934082^{k}\,r^{k}}{k!}}
  , \sum_{k=0}^{\infty }{\frac{0.02881380207911666^{k}\,r^{k}}{k!}}
  , \sum_{k=0}^{\infty }{\frac{0.03036795126603076^{k}\,r^{k}}{k!}}
  , \sum_{k=0}^{\infty }{\frac{0.03197606320812652^{k}\,r^{k}}{k!}}
  , \sum_{k=0}^{\infty }{\frac{0.0336389770872163^{k}\,r^{k}}{k!}} , 
 \sum_{k=0}^{\infty }{\frac{0.03535752660496472^{k}\,r^{k}}{k!}} , 
 \sum_{k=0}^{\infty }{\frac{0.03713253989951881^{k}\,r^{k}}{k!}} , 
 \sum_{k=0}^{\infty }{\frac{0.03896483946269502^{k}\,r^{k}}{k!}} , 
 \sum_{k=0}^{\infty }{\frac{0.0408552420577305^{k}\,r^{k}}{k!}} , 
 \sum_{k=0}^{\infty }{\frac{0.04280455863760801^{k}\,r^{k}}{k!}} , 
 \sum_{k=0}^{\infty }{\frac{0.04481359426396048^{k}\,r^{k}}{k!}} , 
 \sum_{k=0}^{\infty }{\frac{0.04688314802656623^{k}\,r^{k}}{k!}} , 
 \sum_{k=0}^{\infty }{\frac{0.04901401296344043^{k}\,r^{k}}{k!}} , 
 \sum_{k=0}^{\infty }{\frac{0.05120697598153157^{k}\,r^{k}}{k!}} , 
 \sum_{k=0}^{\infty }{\frac{0.05346281777803219^{k}\,r^{k}}{k!}} , 
 \sum_{k=0}^{\infty }{\frac{0.05578231276230905^{k}\,r^{k}}{k!}} , 
 \sum_{k=0}^{\infty }{\frac{0.05816622897846346^{k}\,r^{k}}{k!}} , 
 \sum_{k=0}^{\infty }{\frac{0.06061532802852698^{k}\,r^{k}}{k!}} , 
 \sum_{k=0}^{\infty }{\frac{0.0631303649963022^{k}\,r^{k}}{k!}} , 
 \sum_{k=0}^{\infty }{\frac{0.06571208837185505^{k}\,r^{k}}{k!}} , 
 \sum_{k=0}^{\infty }{\frac{0.06836123997666599^{k}\,r^{k}}{k!}} , 
 \sum_{k=0}^{\infty }{\frac{0.07107855488944881^{k}\,r^{k}}{k!}} , 
 \sum_{k=0}^{\infty }{\frac{0.07386476137264342^{k}\,r^{k}}{k!}} , 
 \sum_{k=0}^{\infty }{\frac{0.07672058079958999^{k}\,r^{k}}{k!}} , 
 \sum_{k=0}^{\infty }{\frac{0.07964672758239233^{k}\,r^{k}}{k!}} , 
 \sum_{k=0}^{\infty }{\frac{0.08264390910047736^{k}\,r^{k}}{k!}} , 
 \sum_{k=0}^{\infty }{\frac{0.0857128256298576^{k}\,r^{k}}{k!}} , 
 \sum_{k=0}^{\infty }{\frac{0.08885417027310427^{k}\,r^{k}}{k!}} , 
 \sum_{k=0}^{\infty }{\frac{0.09206862889003742^{k}\,r^{k}}{k!}} , 
 \sum_{k=0}^{\infty }{\frac{0.09535688002914089^{k}\,r^{k}}{k!}} , 
 \sum_{k=0}^{\infty }{\frac{0.0987195948597075^{k}\,r^{k}}{k!}} , 
 \sum_{k=0}^{\infty }{\frac{0.1021574371047232^{k}\,r^{k}}{k!}} , 
 \sum_{k=0}^{\infty }{\frac{0.1056710629744951^{k}\,r^{k}}{k!}} , 
 \sum_{k=0}^{\infty }{\frac{0.1092611211010309^{k}\,r^{k}}{k!}} , 
 \sum_{k=0}^{\infty }{\frac{0.1129282524731764^{k}\,r^{k}}{k!}} , 
 \sum_{k=0}^{\infty }{\frac{0.1166730903725168^{k}\,r^{k}}{k!}} , 
 \sum_{k=0}^{\infty }{\frac{0.1204962603100498^{k}\,r^{k}}{k!}} , 
 \sum_{k=0}^{\infty }{\frac{0.1243983799636342^{k}\,r^{k}}{k!}} , 
 \sum_{k=0}^{\infty }{\frac{0.1283800591162231^{k}\,r^{k}}{k!}} , 
 \sum_{k=0}^{\infty }{\frac{0.1324418995948859^{k}\,r^{k}}{k!}} , 
 \sum_{k=0}^{\infty }{\frac{0.1365844952106265^{k}\,r^{k}}{k!}} , 
 \sum_{k=0}^{\infty }{\frac{0.140808431699002^{k}\,r^{k}}{k!}} , 
 \sum_{k=0}^{\infty }{\frac{0.1451142866615502^{k}\,r^{k}}{k!}} , 
 \sum_{k=0}^{\infty }{\frac{0.1495026295080298^{k}\,r^{k}}{k!}} , 
 \sum_{k=0}^{\infty }{\frac{0.1539740213994798^{k}\,r^{k}}{k!}}
  \right] 
\]
\end{eulerformula}
\begin{eulerprompt}
>$limit(sum(x^k/k!,k,0,n),n,inf)
\end{eulerprompt}
\begin{eulerformula}
\[
\left[ 0 , \lim_{n\rightarrow \infty }{\sum_{k=0}^{n}{\frac{\left(
 1.66665833335744 \times 10^{-7}\right)^{k}\,r^{k}}{k!}}} , \lim_{n
 \rightarrow \infty }{\sum_{k=0}^{n}{\frac{\left(
 1.33330666692022 \times 10^{-6}\right)^{k}\,r^{k}}{k!}}} , \lim_{n
 \rightarrow \infty }{\sum_{k=0}^{n}{\frac{\left(
 4.499797504338432 \times 10^{-6}\right)^{k}\,r^{k}}{k!}}} , \lim_{n
 \rightarrow \infty }{\sum_{k=0}^{n}{\frac{\left(
 1.066581336583994 \times 10^{-5}\right)^{k}\,r^{k}}{k!}}} , \lim_{n
 \rightarrow \infty }{\sum_{k=0}^{n}{\frac{\left(
 2.083072932167196 \times 10^{-5}\right)^{k}\,r^{k}}{k!}}} , \lim_{n
 \rightarrow \infty }{\sum_{k=0}^{n}{\frac{\left(
 3.599352055540239 \times 10^{-5}\right)^{k}\,r^{k}}{k!}}} , \lim_{n
 \rightarrow \infty }{\sum_{k=0}^{n}{\frac{\left(
 5.71526624672386 \times 10^{-5}\right)^{k}\,r^{k}}{k!}}} , \lim_{n
 \rightarrow \infty }{\sum_{k=0}^{n}{\frac{\left(
 8.530603082730626 \times 10^{-5}\right)^{k}\,r^{k}}{k!}}} , \lim_{n
 \rightarrow \infty }{\sum_{k=0}^{n}{\frac{\left(
 1.214508019889565 \times 10^{-4}\right)^{k}\,r^{k}}{k!}}} , \lim_{n
 \rightarrow \infty }{\sum_{k=0}^{n}{\frac{\left(
 1.665833531718508 \times 10^{-4}\right)^{k}\,r^{k}}{k!}}} , \lim_{n
 \rightarrow \infty }{\sum_{k=0}^{n}{\frac{\left(
 2.216991628251896 \times 10^{-4}\right)^{k}\,r^{k}}{k!}}} , \lim_{n
 \rightarrow \infty }{\sum_{k=0}^{n}{\frac{\left(
 2.877927110806339 \times 10^{-4}\right)^{k}\,r^{k}}{k!}}} , \lim_{n
 \rightarrow \infty }{\sum_{k=0}^{n}{\frac{\left(
 3.658573803051457 \times 10^{-4}\right)^{k}\,r^{k}}{k!}}} , \lim_{n
 \rightarrow \infty }{\sum_{k=0}^{n}{\frac{\left(
 4.568853557635201 \times 10^{-4}\right)^{k}\,r^{k}}{k!}}} , \lim_{n
 \rightarrow \infty }{\sum_{k=0}^{n}{\frac{\left(
 5.618675264007778 \times 10^{-4}\right)^{k}\,r^{k}}{k!}}} , \lim_{n
 \rightarrow \infty }{\sum_{k=0}^{n}{\frac{\left(
 6.817933857540259 \times 10^{-4}\right)^{k}\,r^{k}}{k!}}} , \lim_{n
 \rightarrow \infty }{\sum_{k=0}^{n}{\frac{\left(
 8.176509330039827 \times 10^{-4}\right)^{k}\,r^{k}}{k!}}} , \lim_{n
 \rightarrow \infty }{\sum_{k=0}^{n}{\frac{\left(
 9.704265741758145 \times 10^{-4}\right)^{k}\,r^{k}}{k!}}} , \lim_{n
 \rightarrow \infty }{\sum_{k=0}^{n}{\frac{0.001141105023499428^{k}\,
 r^{k}}{k!}}} , \lim_{n\rightarrow \infty }{\sum_{k=0}^{n}{\frac{
 0.001330669204938795^{k}\,r^{k}}{k!}}} , \lim_{n\rightarrow \infty 
 }{\sum_{k=0}^{n}{\frac{0.001540100153900437^{k}\,r^{k}}{k!}}} , 
 \lim_{n\rightarrow \infty }{\sum_{k=0}^{n}{\frac{
 0.001770376919130678^{k}\,r^{k}}{k!}}} , \lim_{n\rightarrow \infty 
 }{\sum_{k=0}^{n}{\frac{0.002022476464811601^{k}\,r^{k}}{k!}}} , 
 \lim_{n\rightarrow \infty }{\sum_{k=0}^{n}{\frac{
 0.002297373572865413^{k}\,r^{k}}{k!}}} , \lim_{n\rightarrow \infty 
 }{\sum_{k=0}^{n}{\frac{0.002596040745477063^{k}\,r^{k}}{k!}}} , 
 \lim_{n\rightarrow \infty }{\sum_{k=0}^{n}{\frac{
 0.002919448107844891^{k}\,r^{k}}{k!}}} , \lim_{n\rightarrow \infty 
 }{\sum_{k=0}^{n}{\frac{0.003268563311168871^{k}\,r^{k}}{k!}}} , 
 \lim_{n\rightarrow \infty }{\sum_{k=0}^{n}{\frac{
 0.003644351435886262^{k}\,r^{k}}{k!}}} , \lim_{n\rightarrow \infty 
 }{\sum_{k=0}^{n}{\frac{0.004047774895164447^{k}\,r^{k}}{k!}}} , 
 \lim_{n\rightarrow \infty }{\sum_{k=0}^{n}{\frac{
 0.004479793338660443^{k}\,r^{k}}{k!}}} , \lim_{n\rightarrow \infty 
 }{\sum_{k=0}^{n}{\frac{0.0049413635565565^{k}\,r^{k}}{k!}}} , \lim_{
 n\rightarrow \infty }{\sum_{k=0}^{n}{\frac{0.005433439383882244^{k}
 \,r^{k}}{k!}}} , \lim_{n\rightarrow \infty }{\sum_{k=0}^{n}{\frac{
 0.005956971605131645^{k}\,r^{k}}{k!}}} , \lim_{n\rightarrow \infty 
 }{\sum_{k=0}^{n}{\frac{0.006512907859185624^{k}\,r^{k}}{k!}}} , 
 \lim_{n\rightarrow \infty }{\sum_{k=0}^{n}{\frac{
 0.007102192544548636^{k}\,r^{k}}{k!}}} , \lim_{n\rightarrow \infty 
 }{\sum_{k=0}^{n}{\frac{0.007725766724910044^{k}\,r^{k}}{k!}}} , 
 \lim_{n\rightarrow \infty }{\sum_{k=0}^{n}{\frac{0.00838456803503801
 ^{k}\,r^{k}}{k!}}} , \lim_{n\rightarrow \infty }{\sum_{k=0}^{n}{
 \frac{0.009079530587017326^{k}\,r^{k}}{k!}}} , \lim_{n\rightarrow 
 \infty }{\sum_{k=0}^{n}{\frac{0.009811584876838586^{k}\,r^{k}}{k!}}}
  , \lim_{n\rightarrow \infty }{\sum_{k=0}^{n}{\frac{
 0.0105816576913495^{k}\,r^{k}}{k!}}} , \lim_{n\rightarrow \infty }{
 \sum_{k=0}^{n}{\frac{0.01139067201557714^{k}\,r^{k}}{k!}}} , \lim_{n
 \rightarrow \infty }{\sum_{k=0}^{n}{\frac{0.01223954694042984^{k}\,r
 ^{k}}{k!}}} , \lim_{n\rightarrow \infty }{\sum_{k=0}^{n}{\frac{
 0.01312919757078923^{k}\,r^{k}}{k!}}} , \lim_{n\rightarrow \infty }{
 \sum_{k=0}^{n}{\frac{0.01406053493400045^{k}\,r^{k}}{k!}}} , \lim_{n
 \rightarrow \infty }{\sum_{k=0}^{n}{\frac{0.01503446588876983^{k}\,r
 ^{k}}{k!}}} , \lim_{n\rightarrow \infty }{\sum_{k=0}^{n}{\frac{
 0.01605189303448024^{k}\,r^{k}}{k!}}} , \lim_{n\rightarrow \infty }{
 \sum_{k=0}^{n}{\frac{0.01711371462093175^{k}\,r^{k}}{k!}}} , \lim_{n
 \rightarrow \infty }{\sum_{k=0}^{n}{\frac{0.01822082445851714^{k}\,r
 ^{k}}{k!}}} , \lim_{n\rightarrow \infty }{\sum_{k=0}^{n}{\frac{
 0.01937411182884202^{k}\,r^{k}}{k!}}} , \lim_{n\rightarrow \infty }{
 \sum_{k=0}^{n}{\frac{0.02057446139579705^{k}\,r^{k}}{k!}}} , \lim_{n
 \rightarrow \infty }{\sum_{k=0}^{n}{\frac{0.02182275311709253^{k}\,r
 ^{k}}{k!}}} , \lim_{n\rightarrow \infty }{\sum_{k=0}^{n}{\frac{
 0.02311986215626333^{k}\,r^{k}}{k!}}} , \lim_{n\rightarrow \infty }{
 \sum_{k=0}^{n}{\frac{0.02446665879515308^{k}\,r^{k}}{k!}}} , \lim_{n
 \rightarrow \infty }{\sum_{k=0}^{n}{\frac{0.02586400834688696^{k}\,r
 ^{k}}{k!}}} , \lim_{n\rightarrow \infty }{\sum_{k=0}^{n}{\frac{
 0.02731277106934082^{k}\,r^{k}}{k!}}} , \lim_{n\rightarrow \infty }{
 \sum_{k=0}^{n}{\frac{0.02881380207911666^{k}\,r^{k}}{k!}}} , \lim_{n
 \rightarrow \infty }{\sum_{k=0}^{n}{\frac{0.03036795126603076^{k}\,r
 ^{k}}{k!}}} , \lim_{n\rightarrow \infty }{\sum_{k=0}^{n}{\frac{
 0.03197606320812652^{k}\,r^{k}}{k!}}} , \lim_{n\rightarrow \infty }{
 \sum_{k=0}^{n}{\frac{0.0336389770872163^{k}\,r^{k}}{k!}}} , \lim_{n
 \rightarrow \infty }{\sum_{k=0}^{n}{\frac{0.03535752660496472^{k}\,r
 ^{k}}{k!}}} , \lim_{n\rightarrow \infty }{\sum_{k=0}^{n}{\frac{
 0.03713253989951881^{k}\,r^{k}}{k!}}} , \lim_{n\rightarrow \infty }{
 \sum_{k=0}^{n}{\frac{0.03896483946269502^{k}\,r^{k}}{k!}}} , \lim_{n
 \rightarrow \infty }{\sum_{k=0}^{n}{\frac{0.0408552420577305^{k}\,r
 ^{k}}{k!}}} , \lim_{n\rightarrow \infty }{\sum_{k=0}^{n}{\frac{
 0.04280455863760801^{k}\,r^{k}}{k!}}} , \lim_{n\rightarrow \infty }{
 \sum_{k=0}^{n}{\frac{0.04481359426396048^{k}\,r^{k}}{k!}}} , \lim_{n
 \rightarrow \infty }{\sum_{k=0}^{n}{\frac{0.04688314802656623^{k}\,r
 ^{k}}{k!}}} , \lim_{n\rightarrow \infty }{\sum_{k=0}^{n}{\frac{
 0.04901401296344043^{k}\,r^{k}}{k!}}} , \lim_{n\rightarrow \infty }{
 \sum_{k=0}^{n}{\frac{0.05120697598153157^{k}\,r^{k}}{k!}}} , \lim_{n
 \rightarrow \infty }{\sum_{k=0}^{n}{\frac{0.05346281777803219^{k}\,r
 ^{k}}{k!}}} , \lim_{n\rightarrow \infty }{\sum_{k=0}^{n}{\frac{
 0.05578231276230905^{k}\,r^{k}}{k!}}} , \lim_{n\rightarrow \infty }{
 \sum_{k=0}^{n}{\frac{0.05816622897846346^{k}\,r^{k}}{k!}}} , \lim_{n
 \rightarrow \infty }{\sum_{k=0}^{n}{\frac{0.06061532802852698^{k}\,r
 ^{k}}{k!}}} , \lim_{n\rightarrow \infty }{\sum_{k=0}^{n}{\frac{
 0.0631303649963022^{k}\,r^{k}}{k!}}} , \lim_{n\rightarrow \infty }{
 \sum_{k=0}^{n}{\frac{0.06571208837185505^{k}\,r^{k}}{k!}}} , \lim_{n
 \rightarrow \infty }{\sum_{k=0}^{n}{\frac{0.06836123997666599^{k}\,r
 ^{k}}{k!}}} , \lim_{n\rightarrow \infty }{\sum_{k=0}^{n}{\frac{
 0.07107855488944881^{k}\,r^{k}}{k!}}} , \lim_{n\rightarrow \infty }{
 \sum_{k=0}^{n}{\frac{0.07386476137264342^{k}\,r^{k}}{k!}}} , \lim_{n
 \rightarrow \infty }{\sum_{k=0}^{n}{\frac{0.07672058079958999^{k}\,r
 ^{k}}{k!}}} , \lim_{n\rightarrow \infty }{\sum_{k=0}^{n}{\frac{
 0.07964672758239233^{k}\,r^{k}}{k!}}} , \lim_{n\rightarrow \infty }{
 \sum_{k=0}^{n}{\frac{0.08264390910047736^{k}\,r^{k}}{k!}}} , \lim_{n
 \rightarrow \infty }{\sum_{k=0}^{n}{\frac{0.0857128256298576^{k}\,r
 ^{k}}{k!}}} , \lim_{n\rightarrow \infty }{\sum_{k=0}^{n}{\frac{
 0.08885417027310427^{k}\,r^{k}}{k!}}} , \lim_{n\rightarrow \infty }{
 \sum_{k=0}^{n}{\frac{0.09206862889003742^{k}\,r^{k}}{k!}}} , \lim_{n
 \rightarrow \infty }{\sum_{k=0}^{n}{\frac{0.09535688002914089^{k}\,r
 ^{k}}{k!}}} , \lim_{n\rightarrow \infty }{\sum_{k=0}^{n}{\frac{
 0.0987195948597075^{k}\,r^{k}}{k!}}} , \lim_{n\rightarrow \infty }{
 \sum_{k=0}^{n}{\frac{0.1021574371047232^{k}\,r^{k}}{k!}}} , \lim_{n
 \rightarrow \infty }{\sum_{k=0}^{n}{\frac{0.1056710629744951^{k}\,r
 ^{k}}{k!}}} , \lim_{n\rightarrow \infty }{\sum_{k=0}^{n}{\frac{
 0.1092611211010309^{k}\,r^{k}}{k!}}} , \lim_{n\rightarrow \infty }{
 \sum_{k=0}^{n}{\frac{0.1129282524731764^{k}\,r^{k}}{k!}}} , \lim_{n
 \rightarrow \infty }{\sum_{k=0}^{n}{\frac{0.1166730903725168^{k}\,r
 ^{k}}{k!}}} , \lim_{n\rightarrow \infty }{\sum_{k=0}^{n}{\frac{
 0.1204962603100498^{k}\,r^{k}}{k!}}} , \lim_{n\rightarrow \infty }{
 \sum_{k=0}^{n}{\frac{0.1243983799636342^{k}\,r^{k}}{k!}}} , \lim_{n
 \rightarrow \infty }{\sum_{k=0}^{n}{\frac{0.1283800591162231^{k}\,r
 ^{k}}{k!}}} , \lim_{n\rightarrow \infty }{\sum_{k=0}^{n}{\frac{
 0.1324418995948859^{k}\,r^{k}}{k!}}} , \lim_{n\rightarrow \infty }{
 \sum_{k=0}^{n}{\frac{0.1365844952106265^{k}\,r^{k}}{k!}}} , \lim_{n
 \rightarrow \infty }{\sum_{k=0}^{n}{\frac{0.140808431699002^{k}\,r^{
 k}}{k!}}} , \lim_{n\rightarrow \infty }{\sum_{k=0}^{n}{\frac{
 0.1451142866615502^{k}\,r^{k}}{k!}}} , \lim_{n\rightarrow \infty }{
 \sum_{k=0}^{n}{\frac{0.1495026295080298^{k}\,r^{k}}{k!}}} , \lim_{n
 \rightarrow \infty }{\sum_{k=0}^{n}{\frac{0.1539740213994798^{k}\,r
 ^{k}}{k!}}} \right] 
\]
\end{eulerformula}
\begin{eulerprompt}
>function d(n) &= sum(1/(k^2-k),k,2,n); $'d(n)=d(n)
\end{eulerprompt}
\begin{eulerformula}
\[
d\left(n\right)=\sum_{k=2}^{n}{\frac{1}{-k+k^2}}
\]
\end{eulerformula}
\begin{eulerprompt}
>$d(10)=ev(d(10),simpsum=true)
\end{eulerprompt}
\begin{eulerformula}
\[
\sum_{k=2}^{10}{\frac{1}{-k+k^2}}=\frac{9}{10}
\]
\end{eulerformula}
\begin{eulerprompt}
>$d(100)=ev(d(100),simpsum=true)
\end{eulerprompt}
\begin{eulerformula}
\[
\sum_{k=2}^{100}{\frac{1}{-k+k^2}}=\frac{99}{100}
\]
\end{eulerformula}
\eulerheading{Deret Taylor}
\begin{eulercomment}
Deret Taylor suatu fungsi f yang diferensiabel sampai tak hingga di sekitar x=a adalah:

\end{eulercomment}
\begin{eulerformula}
\[
f(x) = \sum_{k=0}^\infty \frac{(x-a)^k f^{(k)}(a)}{k!}.
\]
\end{eulerformula}
\begin{eulerprompt}
>$'e^x =taylor(exp(x),x,0,10) // deret Taylor e^x di sekitar x=0, sampai suku ke-11
\end{eulerprompt}
\begin{euleroutput}
  Maxima said:
  taylor: 0.1539740213994798*r cannot be a variable.
   -- an error. To debug this try: debugmode(true);
  
  Error in:
  $'e^x =taylor(exp(x),x,0,10) // deret Taylor e^x di sekitar x= ...
                               ^
\end{euleroutput}
\begin{eulerprompt}
>$'log(x)=taylor(log(x),x,1,10) // deret log(x) di sekitar x=1
\end{eulerprompt}
\begin{euleroutput}
  Maxima said:
  log: encountered log(0).
   -- an error. To debug this try: debugmode(true);
  
  Error in:
   $'log(x)=taylor(log(x),x,1,10) // deret log(x) di sekitar x=1 ...
                                 ^
\end{euleroutput}

\chapter{EMT untuk geometri}

\eulerheading{Visualisasi dan Perhitungan Geometri dengan EMT}
\begin{eulercomment}
Euler menyediakan beberapa fungsi untuk melakukan visualisasi dan
perhitungan geometri, baik secara numerik maupun analitik (seperti
biasanya tentunya, menggunakan Maxima). Fungsi-fungsi untuk
visualisasi dan perhitungan geometeri tersebut disimpan di dalam file
program "geometry.e", sehingga file tersebut harus dipanggil sebelum
menggunakan fungsi-fungsi atau perintah-perintah untuk geometri.
\end{eulercomment}
\begin{eulerprompt}
>load geometry
\end{eulerprompt}
\begin{euleroutput}
  Numerical and symbolic geometry.
\end{euleroutput}
\eulersubheading{Fungsi-fungsi Geometri}
\begin{eulercomment}
Fungsi-fungsi untuk Menggambar Objek Geometri:

\end{eulercomment}
\begin{eulerttcomment}
  defaultd:=textheight()*1.5: nilai asli untuk parameter d
  setPlotrange(x1,x2,y1,y2): menentukan rentang x dan y pada bidang
\end{eulerttcomment}
\begin{eulercomment}
koordinat\\
\end{eulercomment}
\begin{eulerttcomment}
  setPlotRange(r): pusat bidang koordinat (0,0) dan batas-batas
\end{eulerttcomment}
\begin{eulercomment}
sumbu-x dan y adalah -r sd r\\
\end{eulercomment}
\begin{eulerttcomment}
  plotPoint (P, "P"): menggambar titik P dan diberi label "P"
  plotSegment (A,B, "AB", d): menggambar ruas garis AB, diberi label
\end{eulerttcomment}
\begin{eulercomment}
"AB" sejauh d\\
\end{eulercomment}
\begin{eulerttcomment}
  plotLine (g, "g", d): menggambar garis g diberi label "g" sejauh d
  plotCircle (c,"c",v,d): Menggambar lingkaran c dan diberi label "c"
  plotLabel (label, P, V, d): menuliskan label pada posisi P
\end{eulerttcomment}
\begin{eulercomment}

Fungsi-fungsi Geometri Analitik (numerik maupun simbolik):

\end{eulercomment}
\begin{eulerttcomment}
  turn(v, phi): memutar vektor v sejauh phi
  turnLeft(v):   memutar vektor v ke kiri
  turnRight(v):  memutar vektor v ke kanan
  normalize(v): normal vektor v
  crossProduct(v, w): hasil kali silang vektorv dan w.
  lineThrough(A, B): garis melalui A dan B, hasilnya [a,b,c] sdh.
\end{eulerttcomment}
\begin{eulercomment}
ax+by=c.\\
\end{eulercomment}
\begin{eulerttcomment}
  lineWithDirection(A,v): garis melalui A searah vektor v
  getLineDirection(g): vektor arah (gradien) garis g
  getNormal(g): vektor normal (tegak lurus) garis g
  getPointOnLine(g):  titik pada garis g
  perpendicular(A, g):  garis melalui A tegak lurus garis g
  parallel (A, g):  garis melalui A sejajar garis g
  lineIntersection(g, h):  titik potong garis g dan h
  projectToLine(A, g):   proyeksi titik A pada garis g
  distance(A, B):  jarak titik A dan B
  distanceSquared(A, B):  kuadrat jarak A dan B
  quadrance(A, B): kuadrat jarak A dan B
  areaTriangle(A, B, C):  luas segitiga ABC
  computeAngle(A, B, C):   besar sudut <ABC
  angleBisector(A, B, C): garis bagi sudut <ABC
  circleWithCenter (A, r): lingkaran dengan pusat A dan jari-jari r
  getCircleCenter(c):  pusat lingkaran c
  getCircleRadius(c):  jari-jari lingkaran c
  circleThrough(A,B,C):  lingkaran melalui A, B, C
  middlePerpendicular(A, B): titik tengah AB
  lineCircleIntersections(g, c): titik potong garis g dan lingkran c
  circleCircleIntersections (c1, c2):  titik potong lingkaran c1 dan
\end{eulerttcomment}
\begin{eulercomment}
c2\\
\end{eulercomment}
\begin{eulerttcomment}
  planeThrough(A, B, C):  bidang melalui titik A, B, C
\end{eulerttcomment}
\begin{eulercomment}

Fungsi-fungsi Khusus Untuk Geometri Simbolik:

\end{eulercomment}
\begin{eulerttcomment}
  getLineEquation (g,x,y): persamaan garis g dinyatakan dalam x dan y
  getHesseForm (g,x,y,A): bentuk Hesse garis g dinyatakan dalam x dan
\end{eulerttcomment}
\begin{eulercomment}
y dengan titik A pada sisi positif (kanan/atas) garis\\
\end{eulercomment}
\begin{eulerttcomment}
  quad(A,B): kuadrat jarak AB
  spread(a,b,c): Spread segitiga dengan panjang sisi-sisi a,b,c, yakni
\end{eulerttcomment}
\begin{eulercomment}
sin(alpha)\textasciicircum{}2 dengan alpha sudut yang menghadap sisi a.\\
\end{eulercomment}
\begin{eulerttcomment}
  crosslaw(a,b,c,sa): persamaan 3 quads dan 1 spread pada segitiga
\end{eulerttcomment}
\begin{eulercomment}
dengan panjang sisi a, b, c.\\
\end{eulercomment}
\begin{eulerttcomment}
  triplespread(sa,sb,sc): persamaan 3 spread sa,sb,sc yang memebntuk
\end{eulerttcomment}
\begin{eulercomment}
suatu segitiga\\
\end{eulercomment}
\begin{eulerttcomment}
  doublespread(sa): Spread sudut rangkap Spread 2*phi, dengan
\end{eulerttcomment}
\begin{eulercomment}
sa=sin(phi)\textasciicircum{}2 spread a.

\end{eulercomment}
\eulersubheading{Contoh 1: Luas, Lingkaran Luar, Lingkaran Dalam Segitiga}
\begin{eulercomment}
Untuk menggambar objek-objek geometri, langkah pertama adalah
menentukan rentang sumbu-sumbu koordinat. Semua objek geometri akan
digambar pada satu bidang koordinat, sampai didefinisikan bidang
koordinat yang baru.
\end{eulercomment}
\begin{eulerprompt}
>setPlotRange(-0.5,2.5,-0.5,2.5); // mendefinisikan bidang koordinat baru 
\end{eulerprompt}
\begin{eulercomment}
Now set three points and plot them.
\end{eulercomment}
\begin{eulerprompt}
>A=[1,0]; plotPoint(A,"A"); // definisi dan gambar tiga titik
>B=[0,1]; plotPoint(B,"B");
>C=[2,2]; plotPoint(C,"C");
\end{eulerprompt}
\begin{eulercomment}
Then three segments.
\end{eulercomment}
\begin{eulerprompt}
>plotSegment(A,B,"c"); // c=AB
>plotSegment(B,C,"a"); // a=BC
>plotSegment(A,C,"b"); // b=AC
\end{eulerprompt}
\begin{eulercomment}
Fungsi geometri meliputi fungsi untuk membuat garis dan lingkaran.
Format garisnya adalah [a,b,c] yang mewakili garis dengan persamaan
ax+by=c.
\end{eulercomment}
\begin{eulerprompt}
>lineThrough(B,C) // garis yang melalui B dan C
\end{eulerprompt}
\begin{euleroutput}
  [-1,  2,  2]
\end{euleroutput}
\begin{eulercomment}
Compute the perpendicular line through A on BC.
\end{eulercomment}
\begin{eulerprompt}
>h=perpendicular(A,lineThrough(B,C)); // garis h tegak lurus BC melalui A
\end{eulerprompt}
\begin{eulercomment}
And its intersection with BC.
\end{eulercomment}
\begin{eulerprompt}
>D=lineIntersection(h,lineThrough(B,C)); // D adalah titik potong h dan BC
\end{eulerprompt}
\begin{eulercomment}
Plot that.
\end{eulercomment}
\begin{eulerprompt}
>plotPoint(D,value=1); // koordinat D ditampilkan
>aspect(1); plotSegment(A,D): // tampilkan semua gambar hasil plot...()
\end{eulerprompt}
\eulerimg{27}{images/EMT4Geometry-Naela Rizqy Arofah-22305144042-001.png}
\begin{eulercomment}
Hitung luas ABC:

\end{eulercomment}
\begin{eulerformula}
\[
L_{\triangle ABC}= \frac{1}{2}AD.BC.
\]
\end{eulerformula}
\begin{eulerprompt}
>norm(A-D)*norm(B-C)/2 // AD=norm(A-D), BC=norm(B-C)
\end{eulerprompt}
\begin{euleroutput}
  1.5
\end{euleroutput}
\begin{eulercomment}
Bandingkan dengan rumus determinan.
\end{eulercomment}
\begin{eulerprompt}
>areaTriangle(A,B,C) // hitung luas segitiga langusng dengan fungsi
\end{eulerprompt}
\begin{euleroutput}
  1.5
\end{euleroutput}
\begin{eulercomment}
Cara lain menghitung luas segitigas ABC:
\end{eulercomment}
\begin{eulerprompt}
>distance(A,D)*distance(B,C)/2
\end{eulerprompt}
\begin{euleroutput}
  1.5
\end{euleroutput}
\begin{eulercomment}
The angle at C.
\end{eulercomment}
\begin{eulerprompt}
>degprint(computeAngle(B,C,A))
\end{eulerprompt}
\begin{euleroutput}
  36°52'11.63''
\end{euleroutput}
\begin{eulercomment}
Now the circumcircle of the triangle.
\end{eulercomment}
\begin{eulerprompt}
>c=circleThrough(A,B,C); // lingkaran luar segitiga ABC
>R=getCircleRadius(c); // jari2 lingkaran luar 
>O=getCircleCenter(c); // titik pusat lingkaran c 
>plotPoint(O,"O"); // gambar titik "O"
>plotCircle(c,"Lingkaran luar segitiga ABC"):
\end{eulerprompt}
\eulerimg{27}{images/EMT4Geometry-Naela Rizqy Arofah-22305144042-002.png}
\begin{eulercomment}
Tampilkan koordinat titik pusat dan jari-jari lingkaran luar.
\end{eulercomment}
\begin{eulerprompt}
>O, R
\end{eulerprompt}
\begin{euleroutput}
  [1.16667,  1.16667]
  1.17851130198
\end{euleroutput}
\begin{eulercomment}
Sekarang akan digambar lingkaran dalam segitiga ABC. Titik pusat
lingkaran dalam adalah titik potong garis-garis bagi sudut.
\end{eulercomment}
\begin{eulerprompt}
>l=angleBisector(A,C,B); // garis bagi <ACB
>g=angleBisector(C,A,B); // garis bagi <CAB
>P=lineIntersection(l,g) // titik potong kedua garis bagi sudut
\end{eulerprompt}
\begin{euleroutput}
  [0.86038,  0.86038]
\end{euleroutput}
\begin{eulercomment}
Add everything to the plot.
\end{eulercomment}
\begin{eulerprompt}
>color(5); plotLine(l); plotLine(g); color(1); // gambar kedua garis bagi sudut
>plotPoint(P,"P"); // gambar titik potongnya
>r=norm(P-projectToLine(P,lineThrough(A,B))) // jari-jari lingkaran dalam
\end{eulerprompt}
\begin{euleroutput}
  0.509653732104
\end{euleroutput}
\begin{eulerprompt}
>plotCircle(circleWithCenter(P,r),"Lingkaran dalam segitiga ABC"): // gambar lingkaran dalam
\end{eulerprompt}
\eulerimg{27}{images/EMT4Geometry-Naela Rizqy Arofah-22305144042-003.png}
\eulersubheading{Latihan}
\begin{eulercomment}
1. Tentukan ketiga titik singgung lingkaran dalam dengan sisi-sisi
segitiga ABC.
\end{eulercomment}
\begin{eulerprompt}
>reset
\end{eulerprompt}
\begin{euleroutput}
  0
\end{euleroutput}
\begin{eulerprompt}
>setPlotRange(0,3,0,3);
>A=[1.5,0]; plotPoint(A,"A");
>B=[0,1.5]; plotPoint(B,"B");
>C=[2.5,2]; plotPoint(C,"C");
\end{eulerprompt}
\begin{eulerttcomment}
 
\end{eulerttcomment}
\begin{eulercomment}
2. Gambar segitiga dengan titik-titik sudut ketiga titik singgung
tersebut
\end{eulercomment}
\begin{eulerprompt}
>plotSegment(A,B,"c");
>plotSegment(B,C,"a");
>plotSegment(A,C,"b");
\end{eulerprompt}
\begin{eulercomment}
3. Tunjukkan bahwa garis bagi sudut yang ke tiga juga melalui titik
pusat lingkaran dalam.
\end{eulercomment}
\begin{eulerprompt}
>g=angleBisector(C,A,B);
>l=angleBisector(A,C,B);
>P=lineIntersection(l,g)
\end{eulerprompt}
\begin{euleroutput}
  [1.32151,  1.09988]
\end{euleroutput}
\begin{eulerprompt}
>color(5); plotLine(l); plotLine(g); color(1);
>plotPoint(P,"P");
>plotCircle(circleWithCenter(P,r),"Lingkaran dalam segitiga ABC"):
\end{eulerprompt}
\eulerimg{27}{images/EMT4Geometry-Naela Rizqy Arofah-22305144042-004.png}
\begin{eulerprompt}
>j=angleBisector(A,B,C);
>color(5); plotLine(j);
>plotCircle(circleWithCenter(P,r),"Lingkaran dalam segitiga ABC"):
\end{eulerprompt}
\eulerimg{27}{images/EMT4Geometry-Naela Rizqy Arofah-22305144042-005.png}
\begin{eulercomment}
Jadi, terbukti bahwa garis bagi sudut ketiga juga melalui titik pusat
lingkaran

4. Gambar jari-jari lingkaran dalam.
\end{eulercomment}
\begin{eulerprompt}
>r=norm(P-projectToLine(P,lineThrough(A,B)))
\end{eulerprompt}
\begin{euleroutput}
  0.651522571967
\end{euleroutput}
\begin{eulerprompt}
>plotCircle(circleWithCenter(P,r),"Lingkaran dalam segitiga ABC"):
\end{eulerprompt}
\eulerimg{27}{images/EMT4Geometry-Naela Rizqy Arofah-22305144042-006.png}
\eulersubheading{Contoh 2 : Geometri Simbolik}
\begin{eulerprompt}
>A &= [1,0]; B &= [0,1]; C &= [2,2]; // menentukan tiga titik A, B, C
\end{eulerprompt}
\begin{eulercomment}
Fungsi garis dan lingkaran berfungsi sama seperti fungsi Euler, namun
menyediakan komputasi simbolik.
\end{eulercomment}
\begin{eulerprompt}
>c &= lineThrough(B,C) // c=BC
\end{eulerprompt}
\begin{euleroutput}
  
                               [- 1, 2, 2]
  
\end{euleroutput}
\begin{eulercomment}
Kita bisa mendapatkan persamaan garis dengan mudah.
\end{eulercomment}
\begin{eulerprompt}
>$getLineEquation(c,x,y), $solve(%,y) | expand // persamaan garis c
\end{eulerprompt}
\begin{eulerformula}
\[
2\,y-x=2
\]
\end{eulerformula}
\begin{eulerformula}
\[
\left[ y=\frac{x}{2}+1 \right] 
\]
\end{eulerformula}
\begin{eulerprompt}
>$getLineEquation(lineThrough(A,[x1,y1]),x,y) // persamaan garis melalui A dan (x1, y1)
\end{eulerprompt}
\begin{eulerformula}
\[
\left({\it x_1}-1\right)\,y-x\,{\it y_1}=-{\it y_1}
\]
\end{eulerformula}
\begin{eulerprompt}
>h &= perpendicular(A,lineThrough(B,C)) // h melalui A tegak lurus BC
\end{eulerprompt}
\begin{euleroutput}
  
                                [2, 1, 2]
  
\end{euleroutput}
\begin{eulerprompt}
>Q &= lineIntersection(c,h) // Q titik potong garis c=BC dan h
\end{eulerprompt}
\begin{euleroutput}
  
                                   2  6
                                  [-, -]
                                   5  5
  
\end{euleroutput}
\begin{eulerprompt}
>$projectToLine(A,lineThrough(B,C)) // proyeksi A pada BC
\end{eulerprompt}
\begin{eulerformula}
\[
\left[ \frac{2}{5} , \frac{6}{5} \right] 
\]
\end{eulerformula}
\begin{eulerprompt}
>$distance(A,Q) // jarak AQ
\end{eulerprompt}
\begin{eulerformula}
\[
\frac{3}{\sqrt{5}}
\]
\end{eulerformula}
\begin{eulerprompt}
>cc &= circleThrough(A,B,C); $cc // (titik pusat dan jari-jari) lingkaran melalui A, B, C
\end{eulerprompt}
\begin{eulerformula}
\[
\left[ \frac{7}{6} , \frac{7}{6} , \frac{5}{3\,\sqrt{2}} \right] 
\]
\end{eulerformula}
\begin{eulerprompt}
>r&=getCircleRadius(cc); $r , $float(r) // tampilkan nilai jari-jari
\end{eulerprompt}
\begin{eulerformula}
\[
\frac{5}{3\,\sqrt{2}}
\]
\end{eulerformula}
\begin{eulerformula}
\[
1.178511301977579
\]
\end{eulerformula}
\begin{eulerprompt}
>$computeAngle(A,C,B) // nilai <ACB
\end{eulerprompt}
\begin{eulerformula}
\[
\arccos \left(\frac{4}{5}\right)
\]
\end{eulerformula}
\begin{eulerprompt}
>$solve(getLineEquation(angleBisector(A,C,B),x,y),y)[1] // persamaan garis bagi <ACB
\end{eulerprompt}
\begin{eulerformula}
\[
y=x
\]
\end{eulerformula}
\begin{eulerprompt}
>P &= lineIntersection(angleBisector(A,C,B),angleBisector(C,B,A)); $P // titik potong 2 garis bagi sudut
\end{eulerprompt}
\begin{eulerformula}
\[
\left[ \frac{\sqrt{2}\,\sqrt{5}+2}{6} , \frac{\sqrt{2}\,\sqrt{5}+2
 }{6} \right] 
\]
\end{eulerformula}
\begin{eulerprompt}
>P() // hasilnya sama dengan perhitungan sebelumnya
\end{eulerprompt}
\begin{euleroutput}
  [0.86038,  0.86038]
\end{euleroutput}
\eulersubheading{Perpotongan Garis dan Lingkaran}
\begin{eulercomment}
Tentu saja, kita juga bisa memotong garis dengan lingkaran, dan
lingkaran dengan lingkaran.
\end{eulercomment}
\begin{eulerprompt}
>A &:= [1,0]; c=circleWithCenter(A,4);
>B &:= [1,2]; C &:= [2,1]; l=lineThrough(B,C);
>setPlotRange(5); plotCircle(c); plotLine(l);
\end{eulerprompt}
\begin{eulercomment}
Perpotongan garis dengan lingkaran menghasilkan dua titik dan jumlah
titik perpotongan.
\end{eulercomment}
\begin{eulerprompt}
>\{P1,P2,f\}=lineCircleIntersections(l,c);
>P1, P2,
\end{eulerprompt}
\begin{euleroutput}
  [4.64575,  -1.64575]
  [-0.645751,  3.64575]
\end{euleroutput}
\begin{eulerprompt}
>plotPoint(P1); plotPoint(P2):
\end{eulerprompt}
\eulerimg{27}{images/EMT4Geometry-Naela Rizqy Arofah-22305144042-018.png}
\begin{eulercomment}
The same in Maxima.
\end{eulercomment}
\begin{eulerprompt}
>c &= circleWithCenter(A,4) // lingkaran dengan pusat A jari-jari 4
\end{eulerprompt}
\begin{euleroutput}
  
                                [1, 0, 4]
  
\end{euleroutput}
\begin{eulerprompt}
>l &= lineThrough(B,C) // garis l melalui B dan C
\end{eulerprompt}
\begin{euleroutput}
  
                                [1, 1, 3]
  
\end{euleroutput}
\begin{eulerprompt}
>$lineCircleIntersections(l,c) | radcan, // titik potong lingkaran c dan garis l
\end{eulerprompt}
\begin{eulerformula}
\[
\left[ \left[ \sqrt{7}+2 , 1-\sqrt{7} \right]  , \left[ 2-\sqrt{7}
  , \sqrt{7}+1 \right]  \right] 
\]
\end{eulerformula}
\begin{eulercomment}
Akan ditunjukkan bahwa sudut-sudut yang menghadap bsuusr yang sama
adalah sama besar.
\end{eulercomment}
\begin{eulerprompt}
>C=A+normalize([-2,-3])*4; plotPoint(C); plotSegment(P1,C); plotSegment(P2,C);
>degprint(computeAngle(P1,C,P2))
\end{eulerprompt}
\begin{euleroutput}
  69°17'42.68''
\end{euleroutput}
\begin{eulerprompt}
>C=A+normalize([-4,-3])*4; plotPoint(C); plotSegment(P1,C); plotSegment(P2,C);
>degprint(computeAngle(P1,C,P2))
\end{eulerprompt}
\begin{euleroutput}
  69°17'42.68''
\end{euleroutput}
\begin{eulerprompt}
>insimg;
\end{eulerprompt}
\eulerimg{27}{images/EMT4Geometry-Naela Rizqy Arofah-22305144042-020.png}
\eulersubheading{Garis Sumbu}
\begin{eulercomment}
Berikut adalah langkah-langkah menggambar garis sumbu ruas garis AB:

1. Gambar lingkaran dengan pusat A melalui B.\\
2. Gambar lingkaran dengan pusat B melalui A.\\
3. Tarik garis melallui kedua titik potong kedua lingkaran tersebut.
Garis ini merupakan garis sumbu (melalui titik tengah dan tegak lurus)
AB.
\end{eulercomment}
\begin{eulerprompt}
>A=[2,2]; B=[-1,-2];
>c1=circleWithCenter(A,distance(A,B));
>c2=circleWithCenter(B,distance(A,B));
>\{P1,P2,f\}=circleCircleIntersections(c1,c2);
>l=lineThrough(P1,P2);
>setPlotRange(5); plotCircle(c1); plotCircle(c2);
>plotPoint(A); plotPoint(B); plotSegment(A,B); plotLine(l):
\end{eulerprompt}
\eulerimg{27}{images/EMT4Geometry-Naela Rizqy Arofah-22305144042-021.png}
\begin{eulercomment}
Selanjutnya kita melakukan hal yang sama di Maxima dengan koordinat
umum.
\end{eulercomment}
\begin{eulerprompt}
>A &= [a1,a2]; B &= [b1,b2];
>c1 &= circleWithCenter(A,distance(A,B));
>c2 &= circleWithCenter(B,distance(A,B));
>P &= circleCircleIntersections(c1,c2); P1 &= P[1]; P2 &= P[2];
\end{eulerprompt}
\begin{eulercomment}
Persamaan untuk persimpangan cukup rumit. Tapi kita bisa
menyederhanakannya jika kita mencari y.
\end{eulercomment}
\begin{eulerprompt}
>g &= getLineEquation(lineThrough(P1,P2),x,y);
>$solve(g,y)
\end{eulerprompt}
\begin{eulerformula}
\[
\left[ y=\frac{-\left(2\,{\it b_1}-2\,{\it a_1}\right)\,x+{\it b_2}
 ^2+{\it b_1}^2-{\it a_2}^2-{\it a_1}^2}{2\,{\it b_2}-2\,{\it a_2}}
  \right] 
\]
\end{eulerformula}
\begin{eulercomment}
Ini memang sama dengan garis tengah tegak lurus, yang dihitung dengan
cara yang sangat berbeda.
\end{eulercomment}
\begin{eulerprompt}
>$solve(getLineEquation(middlePerpendicular(A,B),x,y),y)
\end{eulerprompt}
\begin{eulerformula}
\[
\left[ y=\frac{-\left(2\,{\it b_1}-2\,{\it a_1}\right)\,x+{\it b_2}
 ^2+{\it b_1}^2-{\it a_2}^2-{\it a_1}^2}{2\,{\it b_2}-2\,{\it a_2}}
  \right] 
\]
\end{eulerformula}
\begin{eulerprompt}
>h &=getLineEquation(lineThrough(A,B),x,y);
>$solve(h,y)
\end{eulerprompt}
\begin{eulerformula}
\[
\left[ y=\frac{\left({\it b_2}-{\it a_2}\right)\,x-{\it a_1}\,
 {\it b_2}+{\it a_2}\,{\it b_1}}{{\it b_1}-{\it a_1}} \right] 
\]
\end{eulerformula}
\begin{eulercomment}
Perhatikan hasil kali gradien garis g dan h adalah:

\end{eulercomment}
\begin{eulerformula}
\[
\frac{-(b_1-a_1)}{(b_2-a_2)}\times \frac{(b_2-a_2)}{(b_1-a_1)} = -1.
\]
\end{eulerformula}
\begin{eulercomment}
Artinya kedua garis tegak lurus.
\end{eulercomment}
\eulerheading{Contoh 3: Rumus Heron}
\begin{eulercomment}
Rumus Heron menyatakan bahwa luas segitiga dengan panjang sisi-sisi a,
b dan c adalah:

\end{eulercomment}
\begin{eulerformula}
\[
L = \sqrt{s(s-a)(s-b)(s-c)}\quad \text{ dengan } s=(a+b+c)/2.
\]
\end{eulerformula}
\begin{eulercomment}
Untuk membuktikan hal ini kita misalkan C(0,0), B(a,0) dan A(x,y),
b=AC, c=AB. Luas segitiga ABC adalah

\end{eulercomment}
\begin{eulerformula}
\[
L_{\triangle ABC}=\frac{1}{2}a\times y.
\]
\end{eulerformula}
\begin{eulercomment}
Nilai y didapat dengan menyelesaikan sistem persamaan:

\end{eulercomment}
\begin{eulerformula}
\[
x^2+y^2=b^2, \quad (x-a)^2+y^2=c^2.
\]
\end{eulerformula}
\begin{eulerprompt}
>sol &= solve([x^2+y^2=b^2,(x-a)^2+y^2=c^2],[x,y])
\end{eulerprompt}
\begin{euleroutput}
  
                                    []
  
\end{euleroutput}
\begin{eulercomment}
Extract the solution y.
\end{eulercomment}
\begin{eulerprompt}
>ysol &= y with sol[2][2]; $ysol
\end{eulerprompt}
\begin{euleroutput}
  Maxima said:
  part: invalid index of list or matrix.
   -- an error. To debug this try: debugmode(true);
  
  Error in:
  ysol &= y with sol[2][2]; $ysol ...
                          ^
\end{euleroutput}
\begin{eulercomment}
We get the Heron formula.
\end{eulercomment}
\begin{eulerprompt}
>function H(a,b,c) &= sqrt(factor((ysol*a/2)^2)); $'H(a,b,c)=H(a,b,c)
\end{eulerprompt}
\begin{eulerformula}
\[
H\left(a , b , \left[ 1 , 0 , 4 \right] \right)=\frac{\left| a
 \right| \,\left| {\it ysol}\right| }{2}
\]
\end{eulerformula}
\begin{eulercomment}
Tentu saja, setiap segitiga siku-siku adalah kasus yang terkenal.
\end{eulercomment}
\begin{eulerprompt}
>H(3,4,5) //luas segitiga siku-siku dengan panjang sisi 3, 4, 5
\end{eulerprompt}
\begin{euleroutput}
  Variable or function ysol not found.
  Try "trace errors" to inspect local variables after errors.
  H:
      useglobal; return abs(a)*abs(ysol)/2 
  Error in:
  H(3,4,5) //luas segitiga siku-siku dengan panjang sisi 3, 4, 5 ...
          ^
\end{euleroutput}
\begin{eulercomment}
Dan jelas juga bahwa ini adalah segitiga dengan luas maksimal dan
kedua sisinya 3 dan 4.
\end{eulercomment}
\begin{eulerprompt}
>aspect (1.5); plot2d(&H(3,4,x),1,7): // Kurva luas segitiga sengan panjang sisi 3, 4, x (1<= x <=7)
\end{eulerprompt}
\begin{euleroutput}
  Variable or function ysol not found.
  Error in expression: 3*abs(ysol)/2
   %ploteval:
      y0=f$(x[1],args());
  adaptiveevalone:
      s=%ploteval(g$,t;args());
  Try "trace errors" to inspect local variables after errors.
  plot2d:
      dw/n,dw/n^2,dw/n,auto;args());
\end{euleroutput}
\begin{eulercomment}
The general case works too.
\end{eulercomment}
\begin{eulerprompt}
>$solve(diff(H(a,b,c)^2,c)=0,c)
\end{eulerprompt}
\begin{euleroutput}
  Maxima said:
  diff: second argument must be a variable; found [1,0,4]
   -- an error. To debug this try: debugmode(true);
  
  Error in:
   $solve(diff(H(a,b,c)^2,c)=0,c) ...
                                ^
\end{euleroutput}
\begin{eulercomment}
Sekarang mari kita cari himpunan semua titik di mana b+c=d untuk suatu
konstanta d. Diketahui bahwa ini adalah elips.
\end{eulercomment}
\begin{eulerprompt}
>s1 &= subst(d-c,b,sol[2]); $s1
\end{eulerprompt}
\begin{euleroutput}
  Maxima said:
  part: invalid index of list or matrix.
   -- an error. To debug this try: debugmode(true);
  
  Error in:
  s1 &= subst(d-c,b,sol[2]); $s1 ...
                           ^
\end{euleroutput}
\begin{eulercomment}
And make functions of this.
\end{eulercomment}
\begin{eulerprompt}
>function fx(a,c,d) &= rhs(s1[1]); $fx(a,c,d), function fy(a,c,d) &= rhs(s1[2]); $fy(a,c,d)
\end{eulerprompt}
\begin{eulerformula}
\[
0
\]
\end{eulerformula}
\begin{eulerformula}
\[
0
\]
\end{eulerformula}
\begin{eulercomment}
Sekarang kita bisa menggambar setnya. Sisi b bervariasi dari 1 sampai
4. Diketahui bahwa kita memperoleh elips.
\end{eulercomment}
\begin{eulerprompt}
>aspect(1); plot2d(&fx(3,x,5),&fy(3,x,5),xmin=1,xmax=4,square=1):
\end{eulerprompt}
\eulerimg{27}{images/EMT4Geometry-Naela Rizqy Arofah-22305144042-028.png}
\begin{eulercomment}
Kita dapat memeriksa persamaan umum elips ini, yaitu :

\end{eulercomment}
\begin{eulerformula}
\[
\frac{(x-x_m)^2}{u^2}+\frac{(y-y_m)}{v^2}=1,
\]
\end{eulerformula}
\begin{eulercomment}
dimana (xm,ym) adalah pusat, dan u dan v adalah setengah sumbu.
\end{eulercomment}
\begin{eulerprompt}
>$ratsimp((fx(a,c,d)-a/2)^2/u^2+fy(a,c,d)^2/v^2 with [u=d/2,v=sqrt(d^2-a^2)/2])
\end{eulerprompt}
\begin{eulerformula}
\[
\frac{a^2}{d^2}
\]
\end{eulerformula}
\begin{eulercomment}
Kita melihat bahwa tinggi dan luas segitiga adalah maksimal untuk x=0.
Jadi luas segitiga dengan a+b+c=d adalah maksimal jika segitiga
tersebut sama sisi. Kami ingin memperolehnya secara analitis.
\end{eulercomment}
\begin{eulerprompt}
>eqns &= [diff(H(a,b,d-(a+b))^2,a)=0,diff(H(a,b,d-(a+b))^2,b)=0]; $eqns
\end{eulerprompt}
\begin{eulerformula}
\[
\left[ \frac{a\,{\it ysol}^2}{2}=0 , 0=0 \right] 
\]
\end{eulerformula}
\begin{eulercomment}
Kita mendapatkan nilai minimum yang dimiliki oleh segitiga dengan
salah satu sisinya 0, dan solusinya a=b=c=d/3.
\end{eulercomment}
\begin{eulerprompt}
>$solve(eqns,[a,b])
\end{eulerprompt}
\begin{eulerformula}
\[
\left[ \left[ a=0 , b={\it \%r_1} \right]  \right] 
\]
\end{eulerformula}
\begin{eulercomment}
Ada juga metode Lagrange, yang memaksimalkan H(a,b,c)\textasciicircum{}2 terhadap
a+b+d=d.
\end{eulercomment}
\begin{eulerprompt}
>&solve([diff(H(a,b,c)^2,a)=la,diff(H(a,b,c)^2,b)=la, ...
>   diff(H(a,b,c)^2,c)=la,a+b+c=d],[a,b,c,la])
\end{eulerprompt}
\begin{euleroutput}
  Maxima said:
  diff: second argument must be a variable; found [1,0,4]
   -- an error. To debug this try: debugmode(true);
  
  Error in:
  ... la,    diff(H(a,b,c)^2,c)=la,a+b+c=d],[a,b,c,la]) ...
                                                       ^
\end{euleroutput}
\begin{eulercomment}
We can make a plot of the situation
\end{eulercomment}
\begin{eulercomment}
First set the points in Maxima.
\end{eulercomment}
\begin{eulerprompt}
>A &= at([x,y],sol[2]); $A
\end{eulerprompt}
\begin{euleroutput}
  Maxima said:
  part: invalid index of list or matrix.
   -- an error. To debug this try: debugmode(true);
  
  Error in:
  A &= at([x,y],sol[2]); $A ...
                       ^
\end{euleroutput}
\begin{eulerprompt}
>B &= [0,0]; $B, C &= [a,0]; $C
\end{eulerprompt}
\begin{eulerformula}
\[
\left[ 0 , 0 \right] 
\]
\end{eulerformula}
\begin{eulerformula}
\[
\left[ a , 0 \right] 
\]
\end{eulerformula}
\begin{eulercomment}
Then set the plot range, and plot the points.
\end{eulercomment}
\begin{eulerprompt}
>setPlotRange(0,5,-2,3); ...
>a=4; b=3; c=2; ...
>plotPoint(mxmeval("B"),"B"); plotPoint(mxmeval("C"),"C"); ...
>plotPoint(mxmeval("A"),"A"):
\end{eulerprompt}
\begin{euleroutput}
  Variable a1 not found!
  Use global variables or parameters for string evaluation.
  Error in Evaluate, superfluous characters found.
  Try "trace errors" to inspect local variables after errors.
  mxmeval:
      return evaluate(mxm(s));
  Error in:
  ... otPoint(mxmeval("C"),"C"); plotPoint(mxmeval("A"),"A"): ...
                                                       ^
\end{euleroutput}
\begin{eulercomment}
Plot the segments.
\end{eulercomment}
\begin{eulerprompt}
>plotSegment(mxmeval("A"),mxmeval("C")); ...
>plotSegment(mxmeval("B"),mxmeval("C")); ...
>plotSegment(mxmeval("B"),mxmeval("A")):
\end{eulerprompt}
\begin{euleroutput}
  Variable a1 not found!
  Use global variables or parameters for string evaluation.
  Error in Evaluate, superfluous characters found.
  Try "trace errors" to inspect local variables after errors.
  mxmeval:
      return evaluate(mxm(s));
  Error in:
  plotSegment(mxmeval("A"),mxmeval("C")); plotSegment(mxmeval("B ...
                          ^
\end{euleroutput}
\begin{eulercomment}
Hitung garis tengah tegak lurus di Maxima.
\end{eulercomment}
\begin{eulerprompt}
>h &= middlePerpendicular(A,B); g &= middlePerpendicular(B,C);
\end{eulerprompt}
\begin{eulercomment}
Dan pusat lingkarannya.
\end{eulercomment}
\begin{eulerprompt}
>U &= lineIntersection(h,g);
\end{eulerprompt}
\begin{eulercomment}
Kita mendapatkan rumus jari-jari lingkaran luar.
\end{eulercomment}
\begin{eulerprompt}
>&assume(a>0,b>0,c>0); $distance(U,B) | radcan
\end{eulerprompt}
\begin{eulerformula}
\[
\frac{\sqrt{{\it a_2}^2+{\it a_1}^2}\,\sqrt{{\it a_2}^2+{\it a_1}^2
 -2\,a\,{\it a_1}+a^2}}{2\,\left| {\it a_2}\right| }
\]
\end{eulerformula}
\begin{eulercomment}
Let us add this to the plot.
\end{eulercomment}
\begin{eulerprompt}
>plotPoint(U()); ...
>plotCircle(circleWithCenter(mxmeval("U"),mxmeval("distance(U,C)"))):
\end{eulerprompt}
\begin{euleroutput}
  Variable a2 not found!
  Use global variables or parameters for string evaluation.
  Error in ^
  Error in expression: [a/2,(a2^2+a1^2-a*a1)/(2*a2)]
  Error in:
  plotPoint(U()); plotCircle(circleWithCenter(mxmeval("U"),mxmev ...
               ^
\end{euleroutput}
\begin{eulercomment}
Dengan menggunakan geometri, kita memperoleh rumus sederhana

\end{eulercomment}
\begin{eulerformula}
\[
\frac{a}{\sin(\alpha)}=2r
\]
\end{eulerformula}
\begin{eulercomment}
untuk radius. Kita bisa cek, apakah ini benar adanya pada Maxima.
Maxima akan memfaktorkan ini hanya jika kita mengkuadratkannya.
\end{eulercomment}
\begin{eulerprompt}
>$c^2/sin(computeAngle(A,B,C))^2  | factor
\end{eulerprompt}
\begin{eulerformula}
\[
\left[ \frac{{\it a_2}^2+{\it a_1}^2}{{\it a_2}^2} , 0 , \frac{16\,
 \left({\it a_2}^2+{\it a_1}^2\right)}{{\it a_2}^2} \right] 
\]
\end{eulerformula}
\eulerheading{Contoh 4: Garis Euler dan Parabola}
\begin{eulercomment}
Garis Euler adalah garis yang ditentukan dari sembarang segitiga yang
tidak sama sisi. Merupakan garis tengah segitiga, dan melewati
beberapa titik penting yang ditentukan dari segitiga, antara lain
ortocenter, sirkumcenter, centroid, titik Exeter dan pusat lingkaran
sembilan titik segitiga.

Untuk demonstrasinya, kita menghitung dan memplot garis Euler dalam
sebuah segitiga.

Pertama, kita mendefinisikan sudut-sudut segitiga di Euler. Kami
menggunakan definisi, yang terlihat dalam ekspresi simbolik.
\end{eulercomment}
\begin{eulerprompt}
>A::=[-1,-1]; B::=[2,0]; C::=[1,2];
\end{eulerprompt}
\begin{eulercomment}
Untuk memplot objek geometris, kita menyiapkan area plot, dan
menambahkan titik ke dalamnya. Semua plot objek geometris ditambahkan
ke plot saat ini.
\end{eulercomment}
\begin{eulerprompt}
>setPlotRange(3); plotPoint(A,"A"); plotPoint(B,"B"); plotPoint(C,"C");
\end{eulerprompt}
\begin{eulercomment}
Kita juga bisa menjumlahkan sisi-sisi segitiga.
\end{eulercomment}
\begin{eulerprompt}
>plotSegment(A,B,""); plotSegment(B,C,""); plotSegment(C,A,""):
\end{eulerprompt}
\eulerimg{27}{images/EMT4Geometry-Naela Rizqy Arofah-22305144042-036.png}
\begin{eulercomment}
Berikut luas segitiga menggunakan rumus determinan. Tentu saja kami
harus mengambil nilai absolut dari hasil ini.
\end{eulercomment}
\begin{eulerprompt}
>$areaTriangle(A,B,C)
\end{eulerprompt}
\begin{eulerformula}
\[
-\frac{7}{2}
\]
\end{eulerformula}
\begin{eulercomment}
Kita dapat menghitung koefisien sisi c.
\end{eulercomment}
\begin{eulerprompt}
>c &= lineThrough(A,B)
\end{eulerprompt}
\begin{euleroutput}
  
                              [- 1, 3, - 2]
  
\end{euleroutput}
\begin{eulercomment}
Dan dapatkan juga rumus untuk baris ini.
\end{eulercomment}
\begin{eulerprompt}
>$getLineEquation(c,x,y)
\end{eulerprompt}
\begin{eulerformula}
\[
3\,y-x=-2
\]
\end{eulerformula}
\begin{eulercomment}
Untuk bentuk Hesse, kita perlu menentukan sebuah titik, sehingga titik
tersebut berada di sisi positif dari Hesseform. Memasukkan titik akan
menghasilkan jarak positif ke garis.
\end{eulercomment}
\begin{eulerprompt}
>$getHesseForm(c,x,y,C), $at(%,[x=C[1],y=C[2]])
\end{eulerprompt}
\begin{eulerformula}
\[
\frac{3\,y-x+2}{\sqrt{10}}
\]
\end{eulerformula}
\begin{eulerformula}
\[
\frac{7}{\sqrt{10}}
\]
\end{eulerformula}
\begin{eulercomment}
Sekarang kita menghitung lingkaran luar ABC.
\end{eulercomment}
\begin{eulerprompt}
>LL &= circleThrough(A,B,C); $getCircleEquation(LL,x,y)
\end{eulerprompt}
\begin{eulerformula}
\[
\left(y-\frac{5}{14}\right)^2+\left(x-\frac{3}{14}\right)^2=\frac{
 325}{98}
\]
\end{eulerformula}
\begin{eulerprompt}
>O &= getCircleCenter(LL); $O
\end{eulerprompt}
\begin{eulerformula}
\[
\left[ \frac{3}{14} , \frac{5}{14} \right] 
\]
\end{eulerformula}
\begin{eulercomment}
Plot lingkaran dan pusatnya. Cu dan U bersifat simbolis. Kami
mengevaluasi ekspresi ini untuk Euler.
\end{eulercomment}
\begin{eulerprompt}
>plotCircle(LL()); plotPoint(O(),"O"):
\end{eulerprompt}
\eulerimg{27}{images/EMT4Geometry-Naela Rizqy Arofah-22305144042-043.png}
\begin{eulercomment}
Kita dapat menghitung perpotongan ketinggian di ABC (ortocenter)
secara numerik dengan perintah berikut.
\end{eulercomment}
\begin{eulerprompt}
>H &= lineIntersection(perpendicular(A,lineThrough(C,B)),...
>  perpendicular(B,lineThrough(A,C))); $H
\end{eulerprompt}
\begin{eulerformula}
\[
\left[ \frac{11}{7} , \frac{2}{7} \right] 
\]
\end{eulerformula}
\begin{eulercomment}
Sekarang kita dapat menghitung garis segitiga Euler.
\end{eulercomment}
\begin{eulerprompt}
>el &= lineThrough(H,O); $getLineEquation(el,x,y)
\end{eulerprompt}
\begin{eulerformula}
\[
-\frac{19\,y}{14}-\frac{x}{14}=-\frac{1}{2}
\]
\end{eulerformula}
\begin{eulercomment}
Add it to our plot.
\end{eulercomment}
\begin{eulerprompt}
>plotPoint(H(),"H"); plotLine(el(),"Garis Euler"):
\end{eulerprompt}
\eulerimg{27}{images/EMT4Geometry-Naela Rizqy Arofah-22305144042-046.png}
\begin{eulercomment}
Pusat gravitasi seharusnya berada di garis ini.
\end{eulercomment}
\begin{eulerprompt}
>M &= (A+B+C)/3; $getLineEquation(el,x,y) with [x=M[1],y=M[2]]
\end{eulerprompt}
\begin{eulerformula}
\[
-\frac{1}{2}=-\frac{1}{2}
\]
\end{eulerformula}
\begin{eulerprompt}
>plotPoint(M(),"M"): // titik berat
\end{eulerprompt}
\eulerimg{27}{images/EMT4Geometry-Naela Rizqy Arofah-22305144042-048.png}
\begin{eulercomment}
Teorinya memberitahu kita MH=2*MO. Kita perlu menyederhanakan dengan
radcan untuk mencapai hal ini.
\end{eulercomment}
\begin{eulerprompt}
>$distance(M,H)/distance(M,O)|radcan
\end{eulerprompt}
\begin{eulerformula}
\[
2
\]
\end{eulerformula}
\begin{eulercomment}
Fungsinya mencakup fungsi untuk sudut juga.
\end{eulercomment}
\begin{eulerprompt}
>$computeAngle(A,C,B), degprint(%())
\end{eulerprompt}
\begin{eulerformula}
\[
\arccos \left(\frac{4}{\sqrt{5}\,\sqrt{13}}\right)
\]
\end{eulerformula}
\begin{euleroutput}
  60°15'18.43''
\end{euleroutput}
\begin{eulercomment}
Persamaan pusat lingkaran tidak terlalu bagus.
\end{eulercomment}
\begin{eulerprompt}
>Q &= lineIntersection(angleBisector(A,C,B),angleBisector(C,B,A))|radcan; $Q
\end{eulerprompt}
\begin{eulerformula}
\[
\left[ \frac{\left(2^{\frac{3}{2}}+1\right)\,\sqrt{5}\,\sqrt{13}-15
 \,\sqrt{2}+3}{14} , \frac{\left(\sqrt{2}-3\right)\,\sqrt{5}\,\sqrt{
 13}+5\,2^{\frac{3}{2}}+5}{14} \right] 
\]
\end{eulerformula}
\begin{eulercomment}
Mari kita hitung juga ekspresi jari-jari lingkaran yang tertulis.
\end{eulercomment}
\begin{eulerprompt}
>r &= distance(Q,projectToLine(Q,lineThrough(A,B)))|ratsimp; $r
\end{eulerprompt}
\begin{eulerformula}
\[
\frac{\sqrt{\left(-41\,\sqrt{2}-31\right)\,\sqrt{5}\,\sqrt{13}+115
 \,\sqrt{2}+614}}{7\,\sqrt{2}}
\]
\end{eulerformula}
\begin{eulerprompt}
>LD &=  circleWithCenter(Q,r); // Lingkaran dalam
\end{eulerprompt}
\begin{eulercomment}
Let us add this to the plot.
\end{eulercomment}
\begin{eulerprompt}
>color(5); plotCircle(LD()):
\end{eulerprompt}
\eulerimg{27}{images/EMT4Geometry-Naela Rizqy Arofah-22305144042-053.png}
\eulersubheading{Parabola}
\begin{eulercomment}
Selanjutnya akan dicari persamaan tempat kedudukan titik-titik yang
berjarak sama ke titik C dan ke garis AB.
\end{eulercomment}
\begin{eulerprompt}
>p &= getHesseForm(lineThrough(A,B),x,y,C)-distance([x,y],C); $p='0
\end{eulerprompt}
\begin{eulerformula}
\[
\frac{3\,y-x+2}{\sqrt{10}}-\sqrt{\left(2-y\right)^2+\left(1-x
 \right)^2}=0
\]
\end{eulerformula}
\begin{eulercomment}
Persamaan tersebut dapat digambar menjadi satu dengan gambar
sebelumnya.
\end{eulercomment}
\begin{eulerprompt}
>plot2d(p,level=0,add=1,contourcolor=6):
\end{eulerprompt}
\eulerimg{27}{images/EMT4Geometry-Naela Rizqy Arofah-22305144042-055.png}
\begin{eulercomment}
Ini seharusnya merupakan suatu fungsi, tetapi pemecah default Maxima
hanya dapat menemukan solusinya, jika kita mengkuadratkan
persamaannya. Akibatnya, kami mendapatkan solusi palsu.
\end{eulercomment}
\begin{eulerprompt}
>akar &= solve(getHesseForm(lineThrough(A,B),x,y,C)^2-distance([x,y],C)^2,y)
\end{eulerprompt}
\begin{euleroutput}
  
          [y = - 3 x - sqrt(70) sqrt(9 - 2 x) + 26, 
                                y = - 3 x + sqrt(70) sqrt(9 - 2 x) + 26]
  
\end{euleroutput}
\begin{eulercomment}
The first solution is

maxima: akar[1]

Menambahkan solusi pertama pada plot menunjukkan, bahwa itu memang
jalan yang kita cari. Teorinya memberitahu kita bahwa itu adalah
parabola yang diputar.
\end{eulercomment}
\begin{eulerprompt}
>plot2d(&rhs(akar[1]),add=1):
\end{eulerprompt}
\eulerimg{27}{images/EMT4Geometry-Naela Rizqy Arofah-22305144042-056.png}
\begin{eulerprompt}
>function g(x) &= rhs(akar[1]); $'g(x)= g(x)// fungsi yang mendefinisikan kurva di atas
\end{eulerprompt}
\begin{eulerformula}
\[
g\left(x\right)=-3\,x-\sqrt{70}\,\sqrt{9-2\,x}+26
\]
\end{eulerformula}
\begin{eulerprompt}
>T &=[-1, g(-1)]; // ambil sebarang titik pada kurva tersebut
>dTC &= distance(T,C); $fullratsimp(dTC), $float(%) // jarak T ke C
\end{eulerprompt}
\begin{eulerformula}
\[
\sqrt{1503-54\,\sqrt{11}\,\sqrt{70}}
\]
\end{eulerformula}
\begin{eulerformula}
\[
2.135605779339061
\]
\end{eulerformula}
\begin{eulerprompt}
>U &= projectToLine(T,lineThrough(A,B)); $U // proyeksi T pada garis AB 
\end{eulerprompt}
\begin{eulerformula}
\[
\left[ \frac{80-3\,\sqrt{11}\,\sqrt{70}}{10} , \frac{20-\sqrt{11}\,
 \sqrt{70}}{10} \right] 
\]
\end{eulerformula}
\begin{eulerprompt}
>dU2AB &= distance(T,U); $fullratsimp(dU2AB), $float(%) // jatak T ke AB
\end{eulerprompt}
\begin{eulerformula}
\[
\sqrt{1503-54\,\sqrt{11}\,\sqrt{70}}
\]
\end{eulerformula}
\begin{eulerformula}
\[
2.135605779339061
\]
\end{eulerformula}
\begin{eulercomment}
Ternyata jarak T ke C sama dengan jarak T ke AB. Coba Anda pilih titik
T yang lain dan ulangi perhitungan-perhitungan di atas untuk
menunjukkan bahwa hasilnya juga sama.
\end{eulercomment}
\begin{eulercomment}

\begin{eulercomment}
\eulerheading{Contoh 5: Trigonometri Rasional}
\begin{eulercomment}
Hal ini terinspirasi dari ceramah N.J.Wildberger. Dalam bukunya
"Divine Proportions", Wildberger mengusulkan untuk mengganti gagasan
klasik tentang jarak dan sudut dengan kuadran dan penyebaran. Dengan
menggunakan hal ini, memang mungkin untuk menghindari fungsi
trigonometri dalam banyak contoh, dan tetap "rasional".

Berikut ini, saya memperkenalkan konsep, dan memecahkan beberapa
masalah. Saya menggunakan perhitungan simbolik Maxima di sini, yang
menyembunyikan keunggulan utama trigonometri rasional yaitu
perhitungan hanya dapat dilakukan dengan kertas dan pensil. Anda
diundang untuk memeriksa hasilnya tanpa komputer.

Intinya adalah perhitungan rasional simbolik seringkali memberikan
hasil yang sederhana. Sebaliknya, trigonometri klasik menghasilkan
hasil trigonometri yang rumit, yang hanya mengevaluasi perkiraan
numerik saja.
\end{eulercomment}
\begin{eulerprompt}
>load geometry;
\end{eulerprompt}
\begin{eulercomment}
Untuk pengenalan pertama, kami menggunakan segitiga siku-siku dengan
proporsi Mesir yang terkenal 3, 4 dan 5. Perintah berikut adalah
perintah Euler untuk memplot geometri bidang yang terdapat dalam file
Euler "geometry.e".
\end{eulercomment}
\begin{eulerprompt}
>C&:=[0,0]; A&:=[4,0]; B&:=[0,3]; ...
>setPlotRange(-1,5,-1,5); ...
>plotPoint(A,"A"); plotPoint(B,"B"); plotPoint(C,"C"); ...
>plotSegment(B,A,"c"); plotSegment(A,C,"b"); plotSegment(C,B,"a"); ...
>insimg(30);
\end{eulerprompt}
\eulerimg{27}{images/EMT4Geometry-Naela Rizqy Arofah-22305144042-063.png}
\begin{eulercomment}
Of course,

\end{eulercomment}
\begin{eulerformula}
\[
\sin(w_a)=\frac{a}{c},
\]
\end{eulerformula}
\begin{eulercomment}
dimana wa adalah sudut di A. Cara umum untuk menghitung sudut ini
adalah dengan mengambil invers dari fungsi sinus. Hasilnya adalah
sudut yang tidak dapat dicerna, yang hanya dapat dicetak secara kasar.
\end{eulercomment}
\begin{eulerprompt}
>wa := arcsin(3/5); degprint(wa)
\end{eulerprompt}
\begin{euleroutput}
  36°52'11.63''
\end{euleroutput}
\begin{eulercomment}
Trigonometri rasional mencoba menghindari hal ini.

Gagasan pertama tentang trigonometri rasional adalah kuadran, yang
menggantikan jarak. Faktanya, itu hanyalah jarak yang dikuadratkan. Di
bawah ini, a, b, dan c menyatakan kuadran sisi-sisinya.

Teorema Pythogoras menjadi a+b=c.
\end{eulercomment}
\begin{eulerprompt}
>a &= 3^2; b &= 4^2; c &= 5^2; &a+b=c
\end{eulerprompt}
\begin{euleroutput}
  
                                 25 = 25
  
\end{euleroutput}
\begin{eulercomment}
Pengertian trigonometri rasional yang kedua adalah penyebaran.
Penyebaran mengukur pembukaan antar garis. Nilainya 0 jika garisnya
sejajar, dan 1 jika garisnya persegi panjang. Ini adalah kuadrat sinus
sudut antara dua garis.

Luas garis AB dan AC pada gambar di atas didefinisikan sebagai

\end{eulercomment}
\begin{eulerformula}
\[
s_a = \sin(\alpha)^2 = \frac{a}{c},
\]
\end{eulerformula}
\begin{eulercomment}
dimana a dan c adalah kuadran suatu segitiga siku-siku yang salah satu
sudutnya berada di A.
\end{eulercomment}
\begin{eulerprompt}
>sa &= a/c; $sa
\end{eulerprompt}
\begin{eulerformula}
\[
\frac{9}{25}
\]
\end{eulerformula}
\begin{eulercomment}
Tentu saja ini lebih mudah dihitung daripada sudutnya. Namun Anda
kehilangan properti bahwa sudut dapat ditambahkan dengan mudah.

Tentu saja, kita dapat mengonversi nilai perkiraan sudut wa menjadi
sprad, dan mencetaknya sebagai pecahan.
\end{eulercomment}
\begin{eulerprompt}
>fracprint(sin(wa)^2)
\end{eulerprompt}
\begin{euleroutput}
  9/25
\end{euleroutput}
\begin{eulercomment}
Hukum kosinus trgonometri klasik diterjemahkan menjadi "hukum silang"
berikut.

\end{eulercomment}
\begin{eulerformula}
\[
(c+b-a)^2 = 4 b c \, (1-s_a)
\]
\end{eulerformula}
\begin{eulercomment}
Di sini a, b, dan c adalah kuadran sisi-sisi segitiga, dan sa adalah
jarak di sudut A. Sisi a, seperti biasa, berhadapan dengan sudut A.

Hukum-hukum ini diterapkan dalam file geometri.e yang kami muat ke
Euler.
\end{eulercomment}
\begin{eulerprompt}
>$crosslaw(aa,bb,cc,saa)
\end{eulerprompt}
\begin{eulerformula}
\[
\left[ \left({\it bb}-{\it aa}+\frac{7}{6}\right)^2 , \left(
 {\it bb}-{\it aa}+\frac{7}{6}\right)^2 , \left({\it bb}-{\it aa}+
 \frac{5}{3\,\sqrt{2}}\right)^2 \right] =\left[ \frac{14\,{\it bb}\,
 \left(1-{\it saa}\right)}{3} , \frac{14\,{\it bb}\,\left(1-{\it saa}
 \right)}{3} , \frac{5\,2^{\frac{3}{2}}\,{\it bb}\,\left(1-{\it saa}
 \right)}{3} \right] 
\]
\end{eulerformula}
\begin{eulercomment}
In our case we get
\end{eulercomment}
\begin{eulerprompt}
>$crosslaw(a,b,c,sa)
\end{eulerprompt}
\begin{eulerformula}
\[
1024=1024
\]
\end{eulerformula}
\begin{eulercomment}
Mari kita gunakan hukum silang ini untuk mencari penyebaran di A.
Untuk melakukannya, kita membuat hukum silang untuk kuadran a, b, dan
c, dan menyelesaikannya untuk penyebaran sa yang tidak diketahui.

Anda bisa melakukannya dengan tangan dengan mudah, tapi saya
menggunakan Maxima. Tentu saja, kami mendapatkan hasilnya, kami sudah
mendapatkannya.
\end{eulercomment}
\begin{eulerprompt}
>$crosslaw(a,b,c,x), $solve(%,x)
\end{eulerprompt}
\begin{eulerformula}
\[
1024=1600\,\left(1-x\right)
\]
\end{eulerformula}
\begin{eulerformula}
\[
\left[ x=\frac{9}{25} \right] 
\]
\end{eulerformula}
\begin{eulercomment}
Kami sudah mengetahui hal ini. Pengertian penyebaran merupakan kasus
khusus dari hukum silang.

Kita juga dapat menyelesaikannya untuk persamaan umum a,b,c. Hasilnya
adalah rumus yang menghitung penyebaran sudut suatu segitiga dengan
mengetahui kuadran ketiga sisinya.
\end{eulercomment}
\begin{eulerprompt}
>$solve(crosslaw(aa,bb,cc,x),x)
\end{eulerprompt}
\begin{eulerformula}
\[
\left[ \left[ \frac{168\,{\it bb}\,x+36\,{\it bb}^2+\left(-72\,
 {\it aa}-84\right)\,{\it bb}+36\,{\it aa}^2-84\,{\it aa}+49}{36} , 
 \frac{168\,{\it bb}\,x+36\,{\it bb}^2+\left(-72\,{\it aa}-84\right)
 \,{\it bb}+36\,{\it aa}^2-84\,{\it aa}+49}{36} , \frac{15\,2^{\frac{
 5}{2}}\,{\it bb}\,x+18\,{\it bb}^2+\left(-36\,{\it aa}-15\,2^{\frac{
 3}{2}}\right)\,{\it bb}+18\,{\it aa}^2-15\,2^{\frac{3}{2}}\,{\it aa}
 +25}{18} \right] =0 \right] 
\]
\end{eulerformula}
\begin{eulercomment}
Kita bisa membuat fungsi dari hasilnya. Fungsi seperti itu sudah
didefinisikan dalam file geometri.e Euler.
\end{eulercomment}
\begin{eulerprompt}
>$spread(a,b,c)
\end{eulerprompt}
\begin{eulerformula}
\[
\frac{9}{25}
\]
\end{eulerformula}
\begin{eulercomment}
Sebagai contoh, kita dapat menggunakannya untuk menghitung sudut
segitiga dengan sisi-sisinya

\end{eulercomment}
\begin{eulerformula}
\[
a, \quad a, \quad \frac{4a}{7}
\]
\end{eulerformula}
\begin{eulercomment}
Hasilnya rasional, yang tidak mudah didapat jika kita menggunakan
trigonometri klasik.
\end{eulercomment}
\begin{eulerprompt}
>$spread(a,a,4*a/7)
\end{eulerprompt}
\begin{eulerformula}
\[
\frac{6}{7}
\]
\end{eulerformula}
\begin{eulercomment}
This is the angle in degrees.
\end{eulercomment}
\begin{eulerprompt}
>degprint(arcsin(sqrt(6/7)))
\end{eulerprompt}
\begin{euleroutput}
  67°47'32.44''
\end{euleroutput}
\eulersubheading{Contoh lain}
\begin{eulercomment}
Sekarang, mari kita coba contoh lebih lanjut.

Kita tentukan tiga sudut segitiga sebagai berikut.
\end{eulercomment}
\begin{eulerprompt}
>A&:=[1,2]; B&:=[4,3]; C&:=[0,4]; ...
>setPlotRange(-1,5,1,7); ...
>plotPoint(A,"A"); plotPoint(B,"B"); plotPoint(C,"C"); ...
>plotSegment(B,A,"c"); plotSegment(A,C,"b"); plotSegment(C,B,"a"); ...
>insimg;
\end{eulerprompt}
\eulerimg{27}{images/EMT4Geometry-Naela Rizqy Arofah-22305144042-072.png}
\begin{eulercomment}
Dengan menggunakan Pythogoras, mudah untuk menghitung jarak antara dua
titik. Saya pertama kali menggunakan fungsi jarak file Euler untuk
geometri. Fungsi jarak menggunakan geometri klasik.
\end{eulercomment}
\begin{eulerprompt}
>$distance(A,B)
\end{eulerprompt}
\begin{eulerformula}
\[
\sqrt{10}
\]
\end{eulerformula}
\begin{eulercomment}
Euler juga memuat fungsi kuadran antara dua titik.

Pada contoh berikut, karena c+b bukan a, maka segitiga tersebut bukan
persegi panjang.
\end{eulercomment}
\begin{eulerprompt}
>c &= quad(A,B); $c, b &= quad(A,C); $b, a &= quad(B,C); $a,
\end{eulerprompt}
\begin{eulerformula}
\[
10
\]
\end{eulerformula}
\begin{eulerformula}
\[
5
\]
\end{eulerformula}
\begin{eulerformula}
\[
17
\]
\end{eulerformula}
\begin{eulercomment}
Pertama, mari kita hitung sudut tradisional. Fungsi computeAngle
menggunakan metode biasa berdasarkan perkalian titik dua vektor.
Hasilnya adalah beberapa perkiraan floating point.
\end{eulercomment}
\begin{eulerprompt}
>wb &= computeAngle(A,B,C); $wb, $(wb/pi*180)()
\end{eulerprompt}
\begin{eulerformula}
\[
\arccos \left(\frac{11}{\sqrt{10}\,\sqrt{17}}\right)
\]
\end{eulerformula}
\begin{euleroutput}
  32.4711922908
\end{euleroutput}
\begin{eulercomment}
Dengan menggunakan pensil dan kertas, kita dapat melakukan hal yang
sama dengan hukum silang. Kita masukkan kuadran a, b, dan c ke dalam
hukum silang dan selesaikan x.
\end{eulercomment}
\begin{eulerprompt}
>$crosslaw(a,b,c,x), $solve(%,x),
\end{eulerprompt}
\begin{eulerformula}
\[
4=200\,\left(1-x\right)
\]
\end{eulerformula}
\begin{eulerformula}
\[
\left[ x=\frac{49}{50} \right] 
\]
\end{eulerformula}
\begin{eulercomment}
Yaitu, fungsi penyebaran yang didefinisikan dalam "geometri.e".
\end{eulercomment}
\begin{eulerprompt}
>sb &= spread(b,a,c); $sb
\end{eulerprompt}
\begin{eulerformula}
\[
\frac{49}{170}
\]
\end{eulerformula}
\begin{eulercomment}
Maxima mendapatkan hasil yang sama dengan menggunakan trigonometri
biasa, jika kita memaksakannya. Itu menyelesaikan suku
sin(arccos(...)) menjadi hasil pecahan. Kebanyakan siswa tidak dapat
melakukan hal ini.
\end{eulercomment}
\begin{eulerprompt}
>$sin(computeAngle(A,B,C))^2
\end{eulerprompt}
\begin{eulerformula}
\[
\frac{49}{170}
\]
\end{eulerformula}
\begin{eulercomment}
Setelah kita mendapatkan sebaran di B, kita dapat menghitung tinggi ha
pada sisi a. Ingat itu!

\end{eulercomment}
\begin{eulerformula}
\[
s_b=\frac{h_a}{c}
\]
\end{eulerformula}
\begin{eulercomment}
by definition.
\end{eulercomment}
\begin{eulerprompt}
>ha &= c*sb; $ha
\end{eulerprompt}
\begin{eulerformula}
\[
\frac{49}{17}
\]
\end{eulerformula}
\begin{eulercomment}
Gambar berikut dihasilkan dengan program geometri C.a.R., yang dapat
menggambar kuadran dan sebaran.

image: (20) Rational\_Geometry\_CaR.png

Menurut definisi, panjang ha adalah akar kuadrat dari kuadrannya.
\end{eulercomment}
\begin{eulerprompt}
>$sqrt(ha)
\end{eulerprompt}
\begin{eulerformula}
\[
\frac{7}{\sqrt{17}}
\]
\end{eulerformula}
\begin{eulercomment}
Sekarang kita dapat menghitung luas segitiga tersebut. Jangan lupa,
bahwa kita sedang berhadapan dengan kuadran!
\end{eulercomment}
\begin{eulerprompt}
>$sqrt(ha)*sqrt(a)/2
\end{eulerprompt}
\begin{eulerformula}
\[
\frac{7}{2}
\]
\end{eulerformula}
\begin{eulercomment}
The usual determinant formula yields the same result.
\end{eulercomment}
\begin{eulerprompt}
>$areaTriangle(B,A,C)
\end{eulerprompt}
\begin{eulerformula}
\[
\frac{7}{2}
\]
\end{eulerformula}
\eulersubheading{The Heron Formula}
\begin{eulercomment}
Sekarang, mari kita selesaikan masalah ini secara umum!
\end{eulercomment}
\begin{eulerprompt}
>&remvalue(a,b,c,sb,ha);
\end{eulerprompt}
\begin{eulercomment}
Pertama-tama kita menghitung penyebaran di B untuk sebuah segitiga
dengan sisi a, b, dan c. Kemudian kita menghitung luas kuadrat
("quadrea"?), memfaktorkannya dengan Maxima, dan kita mendapatkan
rumus Heron yang terkenal.

Memang benar, hal ini sulit dilakukan dengan pensil dan kertas.
\end{eulercomment}
\begin{eulerprompt}
>$spread(b^2,c^2,a^2), $factor(%*c^2*a^2/4)
\end{eulerprompt}
\begin{eulerformula}
\[
\frac{-c^4-\left(-2\,b^2-2\,a^2\right)\,c^2-b^4+2\,a^2\,b^2-a^4}{4
 \,a^2\,c^2}
\]
\end{eulerformula}
\begin{eulerformula}
\[
\frac{\left(-c+b+a\right)\,\left(c-b+a\right)\,\left(c+b-a\right)\,
 \left(c+b+a\right)}{16}
\]
\end{eulerformula}
\eulersubheading{Aturan Penyebaran Tiga Kali Lipat}
\begin{eulercomment}
Kerugian dari spread adalah bahwa mereka tidak lagi hanya menambahkan
sudut yang sama.

Namun, tiga spread segitiga memenuhi aturan "triple spread" berikut.
\end{eulercomment}
\begin{eulerprompt}
>&remvalue(sa,sb,sc); $triplespread(sa,sb,sc)
\end{eulerprompt}
\begin{eulerformula}
\[
\left({\it sc}+{\it sb}+{\it sa}\right)^2=2\,\left({\it sc}^2+
 {\it sb}^2+{\it sa}^2\right)+4\,{\it sa}\,{\it sb}\,{\it sc}
\]
\end{eulerformula}
\begin{eulercomment}
Aturan ini berlaku untuk tiga sudut mana pun yang besarnya 180°.

\end{eulercomment}
\begin{eulerformula}
\[
\alpha+\beta+\gamma=\pi
\]
\end{eulerformula}
\begin{eulercomment}
Sejak menyebarnya :

\end{eulercomment}
\begin{eulerformula}
\[
\alpha, \pi-\alpha
\]
\end{eulerformula}
\begin{eulercomment}
sama, aturan penyebaran tiga kali lipat juga benar, jika\\
\end{eulercomment}
\begin{eulerformula}
\[
\alpha+\beta=\gamma
\]
\end{eulerformula}
\begin{eulercomment}
Karena penyebaran sudut negatifnya sama, maka aturan penyebaran tiga
kali lipat juga berlaku, jika

\end{eulercomment}
\begin{eulerformula}
\[
\alpha+\beta+\gamma=0
\]
\end{eulerformula}
\begin{eulercomment}
Misalnya, kita dapat menghitung penyebaran sudut 60°. Ini 3/4. Namun
persamaan tersebut memiliki solusi kedua, dimana semua spread adalah
0.
\end{eulercomment}
\begin{eulerprompt}
>$solve(triplespread(x,x,x),x)
\end{eulerprompt}
\begin{eulerformula}
\[
\left[ x=\frac{3}{4} , x=0 \right] 
\]
\end{eulerformula}
\begin{eulercomment}
Penyebaran 90° jelas sama dengan 1. Jika dua sudut dijumlahkan menjadi
90°, penyebarannya menyelesaikan persamaan penyebaran rangkap tiga
dengan a,b,1. Dengan perhitungan berikut kita mendapatkan a+b=1.
\end{eulercomment}
\begin{eulerprompt}
>$triplespread(x,y,1), $solve(%,x)
\end{eulerprompt}
\begin{eulerformula}
\[
\left(y+x+1\right)^2=2\,\left(y^2+x^2+1\right)+4\,x\,y
\]
\end{eulerformula}
\begin{eulerformula}
\[
\left[ x=1-y \right] 
\]
\end{eulerformula}
\begin{eulercomment}
Karena penyebaran 180°-t sama dengan penyebaran t, rumus penyebaran
tiga kali lipat juga berlaku, jika salah satu sudut adalah jumlah atau
selisih dua sudut lainnya.

Sehingga kita dapat mencari penyebaran sudut dua kali lipat tersebut.
Perhatikan bahwa ada dua solusi lagi. Kami menjadikan ini sebuah
fungsi.
\end{eulercomment}
\begin{eulerprompt}
>$solve(triplespread(a,a,x),x), function doublespread(a) &= factor(rhs(%[1]))
\end{eulerprompt}
\begin{eulerformula}
\[
\left[ x=4\,a-4\,a^2 , x=0 \right] 
\]
\end{eulerformula}
\begin{euleroutput}
  
                              - 4 (a - 1) a
  
\end{euleroutput}
\eulersubheading{Pembagi Sudut}
\begin{eulercomment}
Inilah situasinya, kita sudah tahu.
\end{eulercomment}
\begin{eulerprompt}
>C&:=[0,0]; A&:=[4,0]; B&:=[0,3]; ...
>setPlotRange(-1,5,-1,5); ...
>plotPoint(A,"A"); plotPoint(B,"B"); plotPoint(C,"C"); ...
>plotSegment(B,A,"c"); plotSegment(A,C,"b"); plotSegment(C,B,"a"); ...
>insimg;
\end{eulerprompt}
\eulerimg{27}{images/EMT4Geometry-Naela Rizqy Arofah-22305144042-093.png}
\begin{eulercomment}
Mari kita hitung panjang garis bagi sudut di A. Namun kita ingin
menyelesaikannya secara umum a,b,c.
\end{eulercomment}
\begin{eulerprompt}
>&remvalue(a,b,c);
\end{eulerprompt}
\begin{eulercomment}
Jadi pertama-tama kita menghitung penyebaran sudut yang dibagi dua di
A, menggunakan rumus penyebaran tiga kali lipat.

Masalah dengan rumus ini muncul lagi. Ini memiliki dua solusi. Kita
harus memilih yang benar. Solusi lainnya mengacu pada sudut membagi
dua 180°-wa.
\end{eulercomment}
\begin{eulerprompt}
>$triplespread(x,x,a/(a+b)), $solve(%,x), sa2 &= rhs(%[1]); $sa2
\end{eulerprompt}
\begin{eulerformula}
\[
\left(2\,x+\frac{a}{b+a}\right)^2=2\,\left(2\,x^2+\frac{a^2}{\left(
 b+a\right)^2}\right)+\frac{4\,a\,x^2}{b+a}
\]
\end{eulerformula}
\begin{eulerformula}
\[
\left[ x=\frac{-\sqrt{b}\,\sqrt{b+a}+b+a}{2\,b+2\,a} , x=\frac{
 \sqrt{b}\,\sqrt{b+a}+b+a}{2\,b+2\,a} \right] 
\]
\end{eulerformula}
\begin{eulerformula}
\[
\frac{-\sqrt{b}\,\sqrt{b+a}+b+a}{2\,b+2\,a}
\]
\end{eulerformula}
\begin{eulercomment}
Let us check for the Egyptian rectangle.
\end{eulercomment}
\begin{eulerprompt}
>$sa2 with [a=3^2,b=4^2]
\end{eulerprompt}
\begin{eulerformula}
\[
\frac{1}{10}
\]
\end{eulerformula}
\begin{eulercomment}
Kita dapat mencetak sudut dalam Euler, setelah mentransfer
penyebarannya ke radian.
\end{eulercomment}
\begin{eulerprompt}
>wa2 := arcsin(sqrt(1/10)); degprint(wa2)
\end{eulerprompt}
\begin{euleroutput}
  18°26'5.82''
\end{euleroutput}
\begin{eulercomment}
Titik P merupakan perpotongan garis bagi sudut dengan sumbu y.
\end{eulercomment}
\begin{eulerprompt}
>P := [0,tan(wa2)*4]
\end{eulerprompt}
\begin{euleroutput}
  [0,  1.33333]
\end{euleroutput}
\begin{eulerprompt}
>plotPoint(P,"P"); plotSegment(A,P):
\end{eulerprompt}
\eulerimg{27}{images/EMT4Geometry-Naela Rizqy Arofah-22305144042-098.png}
\begin{eulercomment}
Let us check the angles in our specific example.
\end{eulercomment}
\begin{eulerprompt}
>computeAngle(C,A,P), computeAngle(P,A,B)
\end{eulerprompt}
\begin{euleroutput}
  2.66770578995
  0.667705789951
\end{euleroutput}
\begin{eulercomment}
Sekarang kita menghitung panjang garis bagi AP.

Kita menggunakan teorema sinus pada segitiga APC. Teorema ini
menyatakan bahwa

\end{eulercomment}
\begin{eulerformula}
\[
\frac{BC}{\sin(w_a)} = \frac{AC}{\sin(w_b)} = \frac{AB}{\sin(w_c)}
\]
\end{eulerformula}
\begin{eulercomment}
berlaku di segitiga mana pun. Jika digabungkan, maka hal ini akan
diterjemahkan ke dalam apa yang disebut dengan “hukum penyebaran”

\end{eulercomment}
\begin{eulerformula}
\[
\frac{a}{s_a} = \frac{b}{s_b} = \frac{c}{s_b}
\]
\end{eulerformula}
\begin{eulercomment}
dimana a,b,c menunjukkan qudrance.

Karena spread CPA adalah 1-sa2, kita memperolehnya bisa/1=b/(1-sa2)
dan dapat menghitung bisa (kuadran dari garis bagi sudut).
\end{eulercomment}
\begin{eulerprompt}
>&factor(ratsimp(b/(1-sa2))); bisa &= %; $bisa
\end{eulerprompt}
\begin{eulerformula}
\[
\frac{2\,b\,\left(b+a\right)}{\sqrt{b}\,\sqrt{b+a}+b+a}
\]
\end{eulerformula}
\begin{eulercomment}
Let us check this formula for our Egyptian values.
\end{eulercomment}
\begin{eulerprompt}
>sqrt(mxmeval("at(bisa,[a=3^2,b=4^2])")), distance(A,P)
\end{eulerprompt}
\begin{euleroutput}
  4.21637021356
  0.730653920556
\end{euleroutput}
\begin{eulercomment}
We can also compute P using the spread formula.
\end{eulercomment}
\begin{eulerprompt}
>py&=factor(ratsimp(sa2*bisa)); $py
\end{eulerprompt}
\begin{eulerformula}
\[
-\frac{b\,\left(\sqrt{b}\,\sqrt{b+a}-b-a\right)}{\sqrt{b}\,\sqrt{b+
 a}+b+a}
\]
\end{eulerformula}
\begin{eulercomment}
Nilainya sama dengan yang kita peroleh dengan rumus trigonometri.
\end{eulercomment}
\begin{eulerprompt}
>sqrt(mxmeval("at(py,[a=3^2,b=4^2])"))
\end{eulerprompt}
\begin{euleroutput}
  1.33333333333
\end{euleroutput}
\eulersubheading{Sudut Akord}
\begin{eulercomment}
Lihatlah situasi berikut :
\end{eulercomment}
\begin{eulerprompt}
>setPlotRange(1.2); ...
>color(1); plotCircle(circleWithCenter([0,0],1)); ...
>A:=[cos(1),sin(1)]; B:=[cos(2),sin(2)]; C:=[cos(6),sin(6)]; ...
>plotPoint(A,"A"); plotPoint(B,"B"); plotPoint(C,"C"); ...
>color(3); plotSegment(A,B,"c"); plotSegment(A,C,"b"); plotSegment(C,B,"a"); ...
>color(1); O:=[0,0];  plotPoint(O,"0"); ...
>plotSegment(A,O); plotSegment(B,O); plotSegment(C,O,"r"); ...
>insimg;
\end{eulerprompt}
\eulerimg{27}{images/EMT4Geometry-Naela Rizqy Arofah-22305144042-101.png}
\begin{eulercomment}
Kita dapat menggunakan Maxima untuk menyelesaikan rumus penyebaran
rangkap tiga untuk sudut di pusat O untuk r. Jadi kita mendapatkan
rumus jari-jari kuadrat dari perilingkaran dalam kuadran sisi-sisinya.

Kali ini, Maxima menghasilkan beberapa angka nol kompleks, yang kita
abaikan.
\end{eulercomment}
\begin{eulerprompt}
>&remvalue(a,b,c,r); // hapus nilai-nilai sebelumnya untuk perhitungan baru
>rabc &= rhs(solve(triplespread(spread(b,r,r),spread(a,r,r),spread(c,r,r)),r)[4]); $rabc
\end{eulerprompt}
\begin{eulerformula}
\[
-\frac{a\,b\,c}{c^2-2\,b\,c+a\,\left(-2\,c-2\,b\right)+b^2+a^2}
\]
\end{eulerformula}
\begin{eulercomment}
Kita dapat menjadikannya fungsi Euler.
\end{eulercomment}
\begin{eulerprompt}
>function periradius(a,b,c) &= rabc;
\end{eulerprompt}
\begin{eulercomment}
Let us check the result for our points A,B,C.
\end{eulercomment}
\begin{eulerprompt}
>a:=quadrance(B,C); b:=quadrance(A,C); c:=quadrance(A,B);
\end{eulerprompt}
\begin{eulercomment}
The radius is indeed 1.
\end{eulercomment}
\begin{eulerprompt}
>periradius(a,b,c)
\end{eulerprompt}
\begin{euleroutput}
  1
\end{euleroutput}
\begin{eulercomment}
Faktanya, penyebaran CBA hanya bergantung pada b dan c. Ini adalah
teorema sudut tali busur.
\end{eulercomment}
\begin{eulerprompt}
>$spread(b,a,c)*rabc | ratsimp
\end{eulerprompt}
\begin{eulerformula}
\[
\frac{b}{4}
\]
\end{eulerformula}
\begin{eulercomment}
Faktanya, penyebarannya adalah b/(4r), dan kita melihat bahwa sudut
tali busur b adalah setengah sudut pusatnya.
\end{eulercomment}
\begin{eulerprompt}
>$doublespread(b/(4*r))-spread(b,r,r) | ratsimp
\end{eulerprompt}
\begin{eulerformula}
\[
0
\]
\end{eulerformula}
\begin{eulercomment}
\begin{eulercomment}
\eulerheading{Contoh 6: Jarak Minimal pada Bidang}
\begin{eulercomment}
\end{eulercomment}
\eulersubheading{Catatan awal}
\begin{eulercomment}
Fungsi yang, ke titik M pada bidang, menetapkan jarak AM antara titik
tetap A dan M, mempunyai garis datar yang cukup sederhana: lingkaran
berpusat di A.
\end{eulercomment}
\begin{eulerprompt}
>&remvalue();
>A=[-1,-1];
>function d1(x,y):=sqrt((x-A[1])^2+(y-A[2])^2)
>fcontour("d1",xmin=-2,xmax=0,ymin=-2,ymax=0,hue=1, ...
>title="If you see ellipses, please set your window square"):
\end{eulerprompt}
\eulerimg{27}{images/EMT4Geometry-Naela Rizqy Arofah-22305144042-105.png}
\begin{eulercomment}
dan grafiknya juga cukup sederhana: bagian atas kerucut:
\end{eulercomment}
\begin{eulerprompt}
>plot3d("d1",xmin=-2,xmax=0,ymin=-2,ymax=0):
\end{eulerprompt}
\eulerimg{27}{images/EMT4Geometry-Naela Rizqy Arofah-22305144042-106.png}
\begin{eulercomment}
Tentu saja minimum 0 dicapai di A.

\end{eulercomment}
\eulersubheading{Dua poin}
\begin{eulercomment}
Sekarang kita lihat fungsi MA+MB dimana A dan B adalah dua titik
(tetap). Merupakan "fakta yang diketahui" bahwa kurva tingkat
berbentuk elips, titik fokusnya adalah A dan B; kecuali AB minimum
yang konstan pada ruas [AB]:
\end{eulercomment}
\begin{eulerprompt}
>B=[1,-1];
>function d2(x,y):=d1(x,y)+sqrt((x-B[1])^2+(y-B[2])^2)
>fcontour("d2",xmin=-2,xmax=2,ymin=-3,ymax=1,hue=1):
\end{eulerprompt}
\eulerimg{27}{images/EMT4Geometry-Naela Rizqy Arofah-22305144042-107.png}
\begin{eulercomment}
Grafiknya lebih menarik:
\end{eulercomment}
\begin{eulerprompt}
>plot3d("d2",xmin=-2,xmax=2,ymin=-3,ymax=1):
\end{eulerprompt}
\eulerimg{27}{images/EMT4Geometry-Naela Rizqy Arofah-22305144042-108.png}
\begin{eulercomment}
Pembatasan pada garis (AB) lebih terkenal:
\end{eulercomment}
\begin{eulerprompt}
>plot2d("abs(x+1)+abs(x-1)",xmin=-3,xmax=3):
\end{eulerprompt}
\eulerimg{27}{images/EMT4Geometry-Naela Rizqy Arofah-22305144042-109.png}
\begin{eulercomment}
\end{eulercomment}
\eulersubheading{Tiga poin}
\begin{eulercomment}
Kini segalanya menjadi lebih sederhana: Tidak diketahui secara luas
bahwa MA+MB+MC mencapai nilai minimumnya pada satu titik pada bidang
tersebut, namun untuk menentukannya tidaklah mudah:

1) Jika salah satu sudut segitiga ABC lebih dari 120° (katakanlah di
A), maka sudut minimum dicapai pada titik tersebut (katakanlah AB+AC).

Contoh:
\end{eulercomment}
\begin{eulerprompt}
>C=[-4,1];
>function d3(x,y):=d2(x,y)+sqrt((x-C[1])^2+(y-C[2])^2)
>plot3d("d3",xmin=-5,xmax=3,ymin=-4,ymax=4);
>insimg;
\end{eulerprompt}
\eulerimg{27}{images/EMT4Geometry-Naela Rizqy Arofah-22305144042-110.png}
\begin{eulerprompt}
>fcontour("d3",xmin=-4,xmax=1,ymin=-2,ymax=2,hue=1,title="The minimum is on A");
>P=(A_B_C_A)'; plot2d(P[1],P[2],add=1,color=12);
>insimg;
\end{eulerprompt}
\eulerimg{27}{images/EMT4Geometry-Naela Rizqy Arofah-22305144042-111.png}
\begin{eulercomment}
2) Tetapi jika semua sudut segitiga ABC kurang dari 120°, maka titik
minimum ada di titik F di bagian dalam segitiga, yaitu satu-satunya
titik yang melihat sisi-sisi ABC dengan sudut yang sama (maka
masing-masing sudutnya 120° ):
\end{eulercomment}
\begin{eulerprompt}
>C=[-0.5,1];
>plot3d("d3",xmin=-2,xmax=2,ymin=-2,ymax=2):
\end{eulerprompt}
\eulerimg{27}{images/EMT4Geometry-Naela Rizqy Arofah-22305144042-112.png}
\begin{eulerprompt}
>fcontour("d3",xmin=-2,xmax=2,ymin=-2,ymax=2,hue=1,title="The Fermat point");
>P=(A_B_C_A)'; plot2d(P[1],P[2],add=1,color=12);
>insimg;
\end{eulerprompt}
\eulerimg{27}{images/EMT4Geometry-Naela Rizqy Arofah-22305144042-113.png}
\begin{eulercomment}
Merupakan kegiatan yang menarik untuk merealisasikan gambar di atas
dengan perangkat lunak geometri; misalnya, saya tahu soft tertulis di
Java yang memiliki instruksi "garis kontur"...

Semua hal di atas ditemukan oleh seorang hakim Perancis bernama Pierre
de Fermat; dia menulis surat kepada para penggila lainnya seperti
pendeta Marin Mersenne dan Blaise Pascal yang bekerja di bagian pajak
penghasilan. Jadi titik unik F sehingga FA+FB+FC minimal disebut titik
Fermat segitiga. Namun nampaknya beberapa tahun sebelumnya,
Torriccelli dari Italia telah menemukan titik ini sebelum Fermat
menemukannya! Pokoknya tradisinya adalah memperhatikan hal ini F...

\end{eulercomment}
\eulersubheading{Empat poin}
\begin{eulercomment}
Langkah selanjutnya adalah menambahkan poin ke-4 D dan mencoba
meminimalkan MA+MB+MC+MD; katakanlah Anda seorang operator TV kabel
dan ingin mencari di bidang mana Anda harus memasang antena sehingga
Anda dapat memberi makan empat desa dan menggunakan kabel sesedikit
mungkin!
\end{eulercomment}
\begin{eulerprompt}
>D=[1,1];
>function d4(x,y):=d3(x,y)+sqrt((x-D[1])^2+(y-D[2])^2)
>plot3d("d4",xmin=-1.5,xmax=1.5,ymin=-1.5,ymax=1.5):
\end{eulerprompt}
\eulerimg{27}{images/EMT4Geometry-Naela Rizqy Arofah-22305144042-114.png}
\begin{eulerprompt}
>fcontour("d4",xmin=-1.5,xmax=1.5,ymin=-1.5,ymax=1.5,hue=1);
>P=(A_B_C_D)'; plot2d(P[1],P[2],points=1,add=1,color=12);
>insimg;
\end{eulerprompt}
\eulerimg{27}{images/EMT4Geometry-Naela Rizqy Arofah-22305144042-115.png}
\begin{eulercomment}
Masih ada nilai minimum dan tidak tercapai di simpul A, B, C, atau D:
\end{eulercomment}
\begin{eulerprompt}
>function f(x):=d4(x[1],x[2])
>neldermin("f",[0.2,0.2])
\end{eulerprompt}
\begin{euleroutput}
  [0.142858,  0.142857]
\end{euleroutput}
\begin{eulercomment}
Nampaknya dalam hal ini koordinat titik optimal bersifat rasional atau
mendekati rasional...

Sekarang ABCD adalah persegi, kita berharap titik optimalnya adalah
pusat ABCD:
\end{eulercomment}
\begin{eulerprompt}
>C=[-1,1];
>plot3d("d4",xmin=-1,xmax=1,ymin=-1,ymax=1):
\end{eulerprompt}
\eulerimg{27}{images/EMT4Geometry-Naela Rizqy Arofah-22305144042-116.png}
\begin{eulerprompt}
>fcontour("d4",xmin=-1.5,xmax=1.5,ymin=-1.5,ymax=1.5,hue=1);
>P=(A_B_C_D)'; plot2d(P[1],P[2],add=1,color=12,points=1);
>insimg;
\end{eulerprompt}
\eulerimg{27}{images/EMT4Geometry-Naela Rizqy Arofah-22305144042-117.png}
\begin{eulercomment}
*Contoh 7: Bola Dandelin dengan Povray

Anda dapat menjalankan demonstrasi ini, jika Anda telah menginstal
Povray, dan pvengine.exe di jalur program.

Pertama kita hitung jari-jari bola.

Jika diperhatikan gambar di bawah, terlihat bahwa kita membutuhkan dua
lingkaran yang menyentuh dua garis yang membentuk kerucut, dan satu
garis yang membentuk bidang yang memotong kerucut.

Kami menggunakan file geometri.e Euler untuk ini.
\end{eulercomment}
\begin{eulerprompt}
>load geometry;
\end{eulerprompt}
\begin{eulercomment}
First the two lines forming the cone.
\end{eulercomment}
\begin{eulerprompt}
>g1 &= lineThrough([0,0],[1,a])
\end{eulerprompt}
\begin{euleroutput}
  
                               [- a, 1, 0]
  
\end{euleroutput}
\begin{eulerprompt}
>g2 &= lineThrough([0,0],[-1,a])
\end{eulerprompt}
\begin{euleroutput}
  
                              [- a, - 1, 0]
  
\end{euleroutput}
\begin{eulercomment}
Thenm a third line.
\end{eulercomment}
\begin{eulerprompt}
>g &= lineThrough([-1,0],[1,1])
\end{eulerprompt}
\begin{euleroutput}
  
                               [- 1, 2, 1]
  
\end{euleroutput}
\begin{eulercomment}
We plot everything so far.
\end{eulercomment}
\begin{eulerprompt}
>setPlotRange(-1,1,0,2);
>color(black); plotLine(g(),"")
>a:=2; color(blue); plotLine(g1(),""), plotLine(g2(),""):
\end{eulerprompt}
\eulerimg{27}{images/EMT4Geometry-Naela Rizqy Arofah-22305144042-118.png}
\begin{eulercomment}
Now we take a general point on the y-axis.
\end{eulercomment}
\begin{eulerprompt}
>P &= [0,u]
\end{eulerprompt}
\begin{euleroutput}
  
                                  [0, u]
  
\end{euleroutput}
\begin{eulercomment}
Compute the distance to g1.
\end{eulercomment}
\begin{eulerprompt}
>d1 &= distance(P,projectToLine(P,g1)); $d1
\end{eulerprompt}
\begin{eulerformula}
\[
\sqrt{\left(\frac{a^2\,u}{a^2+1}-u\right)^2+\frac{a^2\,u^2}{\left(a
 ^2+1\right)^2}}
\]
\end{eulerformula}
\begin{eulercomment}
Compute the distance to g.
\end{eulercomment}
\begin{eulerprompt}
>d &= distance(P,projectToLine(P,g)); $d
\end{eulerprompt}
\begin{eulerformula}
\[
\sqrt{\left(\frac{u+2}{5}-u\right)^2+\frac{\left(2\,u-1\right)^2}{
 25}}
\]
\end{eulerformula}
\begin{eulercomment}
Dan tentukan pusat kedua lingkaran yang jaraknya sama.
\end{eulercomment}
\begin{eulerprompt}
>sol &= solve(d1^2=d^2,u); $sol
\end{eulerprompt}
\begin{eulerformula}
\[
\left[ u=\frac{-\sqrt{5}\,\sqrt{a^2+1}+2\,a^2+2}{4\,a^2-1} , u=
 \frac{\sqrt{5}\,\sqrt{a^2+1}+2\,a^2+2}{4\,a^2-1} \right] 
\]
\end{eulerformula}
\begin{eulercomment}
Ada dua solusi.

Kami mengevaluasi solusi simbolis, dan menemukan kedua pusat, dan
kedua jarak.
\end{eulercomment}
\begin{eulerprompt}
>u := sol()
\end{eulerprompt}
\begin{euleroutput}
  [0.333333,  1]
\end{euleroutput}
\begin{eulerprompt}
>dd := d()
\end{eulerprompt}
\begin{euleroutput}
  [0.149071,  0.447214]
\end{euleroutput}
\begin{eulercomment}
Plot lingkaran ke dalam gambar.
\end{eulercomment}
\begin{eulerprompt}
>color(red);
>plotCircle(circleWithCenter([0,u[1]],dd[1]),"");
>plotCircle(circleWithCenter([0,u[2]],dd[2]),"");
>insimg;
\end{eulerprompt}
\eulerimg{27}{images/EMT4Geometry-Naela Rizqy Arofah-22305144042-122.png}
\eulersubheading{Plot dengan Povray}
\begin{eulercomment}
Selanjutnya kita plot semuanya dengan Povray. Perhatikan bahwa Anda
mengubah perintah apa pun dalam urutan perintah Povray berikut, dan
menjalankan kembali semua perintah dengan Shift-Return.

Pertama kita memuat fungsi povray.
\end{eulercomment}
\begin{eulerprompt}
>load povray;
>defaultpovray="C:\(\backslash\)Program Files\(\backslash\)POV-Ray\(\backslash\)v3.7\(\backslash\)bin\(\backslash\)pvengine.exe"
\end{eulerprompt}
\begin{euleroutput}
  C:\(\backslash\)Program Files\(\backslash\)POV-Ray\(\backslash\)v3.7\(\backslash\)bin\(\backslash\)pvengine.exe
\end{euleroutput}
\begin{eulercomment}
Kami mengatur adegan dengan tepat.
\end{eulercomment}
\begin{eulerprompt}
>povstart(zoom=11,center=[0,0,0.5],height=10°,angle=140°);
\end{eulerprompt}
\begin{eulercomment}
Selanjutnya kita menulis kedua bola tersebut ke file Povray.
\end{eulercomment}
\begin{eulerprompt}
>writeln(povsphere([0,0,u[1]],dd[1],povlook(red)));
>writeln(povsphere([0,0,u[2]],dd[2],povlook(red)));
\end{eulerprompt}
\begin{eulercomment}
Dan kerucutnya, transparan.
\end{eulercomment}
\begin{eulerprompt}
>writeln(povcone([0,0,0],0,[0,0,a],1,povlook(lightgray,1)));
\end{eulerprompt}
\begin{eulercomment}
We generate a plane restricted to the cone.
\end{eulercomment}
\begin{eulerprompt}
>gp=g();
>pc=povcone([0,0,0],0,[0,0,a],1,"");
>vp=[gp[1],0,gp[2]]; dp=gp[3];
>writeln(povplane(vp,dp,povlook(blue,0.5),pc));
\end{eulerprompt}
\begin{eulercomment}
Sekarang kita buat dua titik pada lingkaran, dimana bola menyentuh
kerucut.
\end{eulercomment}
\begin{eulerprompt}
>function turnz(v) := return [-v[2],v[1],v[3]]
>P1=projectToLine([0,u[1]],g1()); P1=turnz([P1[1],0,P1[2]]);
>writeln(povpoint(P1,povlook(yellow)));
>P2=projectToLine([0,u[2]],g1()); P2=turnz([P2[1],0,P2[2]]);
>writeln(povpoint(P2,povlook(yellow)));
\end{eulerprompt}
\begin{eulercomment}
Lalu kita buat dua titik di mana bola menyentuh bidang. Ini adalah
fokus elips.
\end{eulercomment}
\begin{eulerprompt}
>P3=projectToLine([0,u[1]],g()); P3=[P3[1],0,P3[2]];
>writeln(povpoint(P3,povlook(yellow)));
>P4=projectToLine([0,u[2]],g()); P4=[P4[1],0,P4[2]];
>writeln(povpoint(P4,povlook(yellow)));
\end{eulerprompt}
\begin{eulercomment}
Selanjutnya kita hitung perpotongan P1P2 dengan bidang.
\end{eulercomment}
\begin{eulerprompt}
>t1=scalp(vp,P1)-dp; t2=scalp(vp,P2)-dp; P5=P1+t1/(t1-t2)*(P2-P1);
>writeln(povpoint(P5,povlook(yellow)));
\end{eulerprompt}
\begin{eulercomment}
Kami menghubungkan titik-titik dengan segmen garis.
\end{eulercomment}
\begin{eulerprompt}
>writeln(povsegment(P1,P2,povlook(yellow)));
>writeln(povsegment(P5,P3,povlook(yellow)));
>writeln(povsegment(P5,P4,povlook(yellow)));
\end{eulerprompt}
\begin{eulercomment}
Sekarang kita menghasilkan pita abu-abu, dimana bola menyentuh
kerucut.
\end{eulercomment}
\begin{eulerprompt}
>pcw=povcone([0,0,0],0,[0,0,a],1.01);
>pc1=povcylinder([0,0,P1[3]-defaultpointsize/2],[0,0,P1[3]+defaultpointsize/2],1);
>writeln(povintersection([pcw,pc1],povlook(gray)));
>pc2=povcylinder([0,0,P2[3]-defaultpointsize/2],[0,0,P2[3]+defaultpointsize/2],1);
>writeln(povintersection([pcw,pc2],povlook(gray)));
\end{eulerprompt}
\begin{eulercomment}
Sekarang kita menghasilkan pita abu-abu, dimana bola menyentuh
kerucut.
\end{eulercomment}
\begin{eulerprompt}
>povend();
\end{eulerprompt}
\begin{euleroutput}
  exec:
      return _exec(program,param,dir,print,hidden,wait);
  povray:
      exec(program,params,defaulthome);
  Try "trace errors" to inspect local variables after errors.
  povend:
      povray(file,w,h,aspect,exit); 
\end{euleroutput}
\begin{eulercomment}
Untuk mendapatkan Anaglyph ini kita perlu memasukkan semuanya ke dalam
fungsi scene. Fungsi ini akan digunakan dua kali kemudian.
\end{eulercomment}
\begin{eulerprompt}
>function scene () ...
\end{eulerprompt}
\begin{eulerudf}
  global a,u,dd,g,g1,defaultpointsize;
  writeln(povsphere([0,0,u[1]],dd[1],povlook(red)));
  writeln(povsphere([0,0,u[2]],dd[2],povlook(red)));
  writeln(povcone([0,0,0],0,[0,0,a],1,povlook(lightgray,1)));
  gp=g();
  pc=povcone([0,0,0],0,[0,0,a],1,"");
  vp=[gp[1],0,gp[2]]; dp=gp[3];
  writeln(povplane(vp,dp,povlook(blue,0.5),pc));
  P1=projectToLine([0,u[1]],g1()); P1=turnz([P1[1],0,P1[2]]);
  writeln(povpoint(P1,povlook(yellow)));
  P2=projectToLine([0,u[2]],g1()); P2=turnz([P2[1],0,P2[2]]);
  writeln(povpoint(P2,povlook(yellow)));
  P3=projectToLine([0,u[1]],g()); P3=[P3[1],0,P3[2]];
  writeln(povpoint(P3,povlook(yellow)));
  P4=projectToLine([0,u[2]],g()); P4=[P4[1],0,P4[2]];
  writeln(povpoint(P4,povlook(yellow)));
  t1=scalp(vp,P1)-dp; t2=scalp(vp,P2)-dp; P5=P1+t1/(t1-t2)*(P2-P1);
  writeln(povpoint(P5,povlook(yellow)));
  writeln(povsegment(P1,P2,povlook(yellow)));
  writeln(povsegment(P5,P3,povlook(yellow)));
  writeln(povsegment(P5,P4,povlook(yellow)));
  pcw=povcone([0,0,0],0,[0,0,a],1.01);
  pc1=povcylinder([0,0,P1[3]-defaultpointsize/2],[0,0,P1[3]+defaultpointsize/2],1);
  writeln(povintersection([pcw,pc1],povlook(gray)));
  pc2=povcylinder([0,0,P2[3]-defaultpointsize/2],[0,0,P2[3]+defaultpointsize/2],1);
  writeln(povintersection([pcw,pc2],povlook(gray)));
  endfunction
\end{eulerudf}
\begin{eulercomment}
Anda memerlukan kacamata merah/cyan untuk melihat efek berikut.
\end{eulercomment}
\begin{eulerprompt}
>povanaglyph("scene",zoom=11,center=[0,0,0.5],height=10°,angle=140°);
\end{eulerprompt}
\eulerheading{Contoh 8: Geometri Bumi}
\begin{eulercomment}
Di buku catatan ini, kami ingin melakukan beberapa perhitungan bola.
Fungsi-fungsi tersebut terdapat dalam file "spherical.e" di folder
contoh. Kita perlu memuat file itu terlebih dahulu.
\end{eulercomment}
\begin{eulerprompt}
>load "spherical.e";
\end{eulerprompt}
\begin{eulercomment}
Untuk memasukkan posisi geografis, kita menggunakan vektor dengan dua
koordinat dalam radian (utara dan timur, nilai negatif untuk selatan
dan barat). Berikut koordinat Kampus FMIPA UNY.
\end{eulercomment}
\begin{eulerprompt}
>FMIPA=[rad(-7,-46.467),rad(110,23.05)]
\end{eulerprompt}
\begin{euleroutput}
  [-0.13569,  1.92657]
\end{euleroutput}
\begin{eulercomment}
Anda dapat mencetak posisi ini dengan sposprint (cetak posisi bulat).
\end{eulercomment}
\begin{eulerprompt}
>sposprint(FMIPA) // posisi garis lintang dan garis bujur FMIPA UNY
\end{eulerprompt}
\begin{euleroutput}
  S 7°46.467' E 110°23.050'
\end{euleroutput}
\begin{eulercomment}
Mari kita tambahkan dua kota lagi, Solo dan Semarang.
\end{eulercomment}
\begin{eulerprompt}
>Solo=[rad(-7,-34.333),rad(110,49.683)]; Semarang=[rad(-6,-59.05),rad(110,24.533)];
>sposprint(Solo), sposprint(Semarang),
\end{eulerprompt}
\begin{euleroutput}
  S 7°34.333' E 110°49.683'
  S 6°59.050' E 110°24.533'
\end{euleroutput}
\begin{eulercomment}
Pertama kita menghitung vektor dari satu bola ke bola ideal lainnya.
Vektor ini adalah [pos, jarak] dalam radian. Untuk menghitung jarak di
bumi, kita kalikan dengan jari-jari bumi pada garis lintang 7°.
\end{eulercomment}
\begin{eulerprompt}
>br=svector(FMIPA,Solo); degprint(br[1]), br[2]*rearth(7°)->km // perkiraan jarak FMIPA-Solo
\end{eulerprompt}
\begin{euleroutput}
  65°20'26.60''
  53.8945384608
\end{euleroutput}
\begin{eulercomment}
Ini adalah perkiraan yang bagus. Rutinitas berikut menggunakan
perkiraan yang lebih baik. Pada jarak sedekat itu, hasilnya hampir
sama.
\end{eulercomment}
\begin{eulerprompt}
>esdist(FMIPA,Semarang)->" km" // perkiraan jarak FMIPA-Semarang
\end{eulerprompt}
\begin{euleroutput}
  Commands must be separated by semicolon or comma!
  Found:  // perkiraan jarak FMIPA-Semarang (character 32)
  You can disable this in the Options menu.
  Error in:
  esdist(FMIPA,Semarang)->" km" // perkiraan jarak FMIPA-Semaran ...
                               ^
\end{euleroutput}
\begin{eulercomment}
Judulnya ada fungsinya, dengan mempertimbangkan bentuk bumi yang
elips. Sekali lagi, kami mencetak dengan cara yang canggih.
\end{eulercomment}
\begin{eulerprompt}
>sdegprint(esdir(FMIPA,Solo))
\end{eulerprompt}
\begin{euleroutput}
       65.34°
\end{euleroutput}
\begin{eulercomment}
Sudut suatu segitiga melebihi 180° pada bola.
\end{eulercomment}
\begin{eulerprompt}
>asum=sangle(Solo,FMIPA,Semarang)+sangle(FMIPA,Solo,Semarang)+sangle(FMIPA,Semarang,Solo); degprint(asum)
\end{eulerprompt}
\begin{euleroutput}
  180°0'10.77''
\end{euleroutput}
\begin{eulercomment}
Ini dapat digunakan untuk menghitung luas segitiga. Catatan: Untuk
segitiga kecil, ini tidak akurat karena kesalahan pengurangan pada
asum-pi.
\end{eulercomment}
\begin{eulerprompt}
>(asum-pi)*rearth(48°)^2->" km^2" // perkiraan luas segitiga FMIPA-Solo-Semarang
\end{eulerprompt}
\begin{euleroutput}
  Commands must be separated by semicolon or comma!
  Found:  // perkiraan luas segitiga FMIPA-Solo-Semarang (character 32)
  You can disable this in the Options menu.
  Error in:
  (asum-pi)*rearth(48°)^2->" km^2" // perkiraan luas segitiga FM ...
                                  ^
\end{euleroutput}
\begin{eulercomment}
Ada fungsi untuk ini, yang menggunakan garis lintang rata-rata dari
segitiga untuk menghitung jari-jari bumi, dan menangani kesalahan
pembulatan untuk segitiga yang sangat kecil.
\end{eulercomment}
\begin{eulerprompt}
>esarea(Solo,FMIPA,Semarang)->" km^2", //perkiraan yang sama dengan fungsi esarea()
\end{eulerprompt}
\begin{euleroutput}
  2123.64310526 km^2
\end{euleroutput}
\begin{eulercomment}
Kita juga dapat menambahkan vektor ke posisi. Vektor berisi arah dan
jarak, keduanya dalam radian. Untuk mendapatkan vektor, kita
menggunakan svector. Untuk menambahkan vektor ke suatu posisi, kita
menggunakan saddvector.
\end{eulercomment}
\begin{eulerprompt}
>v=svector(FMIPA,Solo); sposprint(saddvector(FMIPA,v)), sposprint(Solo),
\end{eulerprompt}
\begin{euleroutput}
  S 7°34.333' E 110°49.683'
  S 7°34.333' E 110°49.683'
\end{euleroutput}
\begin{eulercomment}
Fungsi-fungsi ini mengasumsikan bola ideal. Hal yang sama terjadi di
bumi.
\end{eulercomment}
\begin{eulerprompt}
>sposprint(esadd(FMIPA,esdir(FMIPA,Solo),esdist(FMIPA,Solo))), sposprint(Solo),
\end{eulerprompt}
\begin{euleroutput}
  S 7°34.333' E 110°49.683'
  S 7°34.333' E 110°49.683'
\end{euleroutput}
\begin{eulercomment}
Mari kita lihat contoh yang lebih besar, Tugu Jogja dan Monas Jakarta
(menggunakan Google Earth untuk mencari koordinatnya).
\end{eulercomment}
\begin{eulerprompt}
>Tugu=[-7.7833°,110.3661°]; Monas=[-6.175°,106.811944°];
>sposprint(Tugu), sposprint(Monas)
\end{eulerprompt}
\begin{euleroutput}
  S 7°46.998' E 110°21.966'
  S 6°10.500' E 106°48.717'
\end{euleroutput}
\begin{eulercomment}
Menurut Google Earth, jaraknya 429,66km. Kami mendapatkan perkiraan
yang bagus.
\end{eulercomment}
\begin{eulerprompt}
>esdist(Tugu,Monas)->" km" // perkiraan jarak Tugu Jogja - Monas Jakarta
\end{eulerprompt}
\begin{euleroutput}
  Commands must be separated by semicolon or comma!
  Found:  // perkiraan jarak Tugu Jogja - Monas Jakarta (character 32)
  You can disable this in the Options menu.
  Error in:
  esdist(Tugu,Monas)->" km" // perkiraan jarak Tugu Jogja - Mona ...
                           ^
\end{euleroutput}
\begin{eulercomment}
Judulnya sama dengan yang dihitung di Google Earth.
\end{eulercomment}
\begin{eulerprompt}
>degprint(esdir(Tugu,Monas))
\end{eulerprompt}
\begin{euleroutput}
  294°17'2.85''
\end{euleroutput}
\begin{eulercomment}
Namun kita tidak lagi mendapatkan posisi sasaran yang tepat, jika kita
menambahkan heading dan jarak ke posisi semula. Hal ini terjadi karena
kita tidak menghitung fungsi invers secara tepat, namun melakukan
perkiraan jari-jari bumi di sepanjang lintasan.
\end{eulercomment}
\begin{eulerprompt}
>sposprint(esadd(Tugu,esdir(Tugu,Monas),esdist(Tugu,Monas)))
\end{eulerprompt}
\begin{euleroutput}
  S 6°10.500' E 106°48.717'
\end{euleroutput}
\begin{eulercomment}
The error is not large, however.
\end{eulercomment}
\begin{eulerprompt}
>sposprint(Monas),
\end{eulerprompt}
\begin{euleroutput}
  S 6°10.500' E 106°48.717'
\end{euleroutput}
\begin{eulercomment}
Tentu kita tidak bisa berlayar dengan tujuan yang sama dari satu
tujuan ke tujuan lainnya, jika ingin mengambil jalur terpendek.
Bayangkan, Anda terbang NE mulai dari titik mana saja di bumi.
Kemudian Anda akan berputar ke kutub utara. Lingkaran besar tidak
mengikuti arah yang konstan!

Perhitungan berikut menunjukkan bahwa kita jauh dari tujuan yang
benar, jika kita menggunakan tujuan yang sama selama perjalanan.
\end{eulercomment}
\begin{eulerprompt}
>dist=esdist(Tugu,Monas); hd=esdir(Tugu,Monas);
\end{eulerprompt}
\begin{eulercomment}
Sekarang kita tambah 10 kali sepersepuluh jarak, pakai jurusan Monas,
kita sampai di Tugu.
\end{eulercomment}
\begin{eulerprompt}
>p=Tugu; loop 1 to 10; p=esadd(p,hd,dist/10); end;
\end{eulerprompt}
\begin{eulercomment}
The result is far off.
\end{eulercomment}
\begin{eulerprompt}
>sposprint(p), skmprint(esdist(p,Monas))
\end{eulerprompt}
\begin{euleroutput}
  S 6°11.250' E 106°48.372'
       1.529km
\end{euleroutput}
\begin{eulercomment}
Sebagai contoh lain, mari kita ambil dua titik di bumi yang mempunyai
garis lintang yang sama.
\end{eulercomment}
\begin{eulerprompt}
>P1=[30°,10°]; P2=[30°,50°];
\end{eulerprompt}
\begin{eulercomment}
Jalur terpendek dari P1 ke P2 bukanlah lingkaran dengan garis lintang
30°, melainkan jalur yang lebih pendek yang dimulai 10° lebih jauh ke
utara di P1.
\end{eulercomment}
\begin{eulerprompt}
>sdegprint(esdir(P1,P2))
\end{eulerprompt}
\begin{euleroutput}
       79.69°
\end{euleroutput}
\begin{eulercomment}
Namun, jika kita mengikuti pembacaan kompas ini, kita akan berputar ke
kutub utara! Jadi kita harus menyesuaikan arah perjalanan kita. Untuk
tujuan kasarnya, kita sesuaikan pada 1/10 dari total jarak.
\end{eulercomment}
\begin{eulerprompt}
>p=P1;  dist=esdist(P1,P2); ...
>  loop 1 to 10; dir=esdir(p,P2); sdegprint(dir), p=esadd(p,dir,dist/10); end;
\end{eulerprompt}
\begin{euleroutput}
       79.69°
       81.67°
       83.71°
       85.78°
       87.89°
       90.00°
       92.12°
       94.22°
       96.29°
       98.33°
\end{euleroutput}
\begin{eulercomment}
Jaraknya tidak tepat, karena kita akan menambahkan sedikit kesalahan
jika kita mengikuti arah yang sama terlalu lama.
\end{eulercomment}
\begin{eulerprompt}
>skmprint(esdist(p,P2))
\end{eulerprompt}
\begin{euleroutput}
       0.203km
\end{euleroutput}
\begin{eulercomment}
Kita mendapatkan perkiraan yang baik, jika kita menyesuaikan arah
setiap 1/100 dari total jarak dari Tugu ke Monas.
\end{eulercomment}
\begin{eulerprompt}
>p=Tugu; dist=esdist(Tugu,Monas); ...
>  loop 1 to 100; p=esadd(p,esdir(p,Monas),dist/100); end;
>skmprint(esdist(p,Monas))
\end{eulerprompt}
\begin{euleroutput}
       0.000km
\end{euleroutput}
\begin{eulercomment}
Untuk keperluan navigasi, kita bisa mendapatkan urutan posisi GPS
sepanjang lingkaran besar menuju Monas dengan fungsi navigasi.
\end{eulercomment}
\begin{eulerprompt}
>load spherical; v=navigate(Tugu,Monas,10); ...
>  loop 1 to rows(v); sposprint(v[#]), end;
\end{eulerprompt}
\begin{euleroutput}
  S 7°46.998' E 110°21.966'
  S 7°37.422' E 110°0.573'
  S 7°27.829' E 109°39.196'
  S 7°18.219' E 109°17.834'
  S 7°8.592' E 108°56.488'
  S 6°58.948' E 108°35.157'
  S 6°49.289' E 108°13.841'
  S 6°39.614' E 107°52.539'
  S 6°29.924' E 107°31.251'
  S 6°20.219' E 107°9.977'
  S 6°10.500' E 106°48.717'
\end{euleroutput}
\begin{eulercomment}
Kita menulis sebuah fungsi yang memplot bumi, dua posisi, dan posisi
di antaranya.
\end{eulercomment}
\begin{eulerprompt}
>function testplot ...
\end{eulerprompt}
\begin{eulerudf}
  useglobal;
  plotearth;
  plotpos(Tugu,"Tugu Jogja"); plotpos(Monas,"Tugu Monas");
  plotposline(v);
  endfunction
\end{eulerudf}
\begin{eulercomment}
Now plot everything.
\end{eulercomment}
\begin{eulerprompt}
>plot3d("testplot",angle=25, height=6,>own,>user,zoom=4):
\end{eulerprompt}
\begin{eulercomment}
Atau gunakan plot3d untuk mendapatkan tampilan anaglyph. Ini terlihat
sangat bagus dengan kacamata merah/cyan.
\end{eulercomment}
\begin{eulerprompt}
>plot3d("testplot",angle=25,height=6,distance=5,own=1,anaglyph=1,zoom=4):
\end{eulerprompt}
\begin{euleroutput}
  Variable testplot not found!
  Use global variables or parameters for string evaluation.
  Error in expression: testplot
  Try "trace errors" to inspect local variables after errors.
  plot3d:
      f(args());
\end{euleroutput}
\eulerheading{Latihan}
\begin{eulercomment}
1. Gambarlah segi-n beraturan jika diketahui titik pusat O, n, dan
jarak titik pusat ke titik-titik sudut segi-n tersebut (jari-jari
lingkaran luar segi-n), r.

Petunjuk:

- Besar sudut pusat yang menghadap masing-masing sisi segi-n adalah
(360/n).\\
- Titik-titik sudut segi-n merupakan perpotongan lingkaran luar segi-n
dan garis-garis yang melalui pusat dan saling membentuk sudut sebesar
kelipatan (360/n).\\
- Untuk n ganjil, pilih salah satu titik sudut adalah di atas.\\
- Untuk n genap, pilih 2 titik di kanan dan kiri lurus dengan titik
pusat.\\
- Anda dapat menggambar segi-3, 4, 5, 6, 7, dst beraturan.

\end{eulercomment}
\begin{eulerprompt}
>load geometry
\end{eulerprompt}
\begin{euleroutput}
  Numerical and symbolic geometry.
\end{euleroutput}
\begin{eulerprompt}
>setPlotRange(-3.5,3.5,-3.5,3.5);
>O=[0,0]; plotPoint(O,"O");
>A=[-2,-1]; plotPoint(A,"A");
>B=[2,-1]; plotPoint(B,"B");
>C=[0,2*3^0.5-1]; plotPoint(A,"A");
>plotSegment(A,B,"c");
>plotSegment(B,C,"a");
>plotSegment(A,C,"b");
>aspect(1):
\end{eulerprompt}
\eulerimg{27}{images/EMT4Geometry-Naela Rizqy Arofah-22305144042-123.png}
\begin{eulerprompt}
>c=circleThrough(A,B,C):
\end{eulerprompt}
\eulerimg{27}{images/EMT4Geometry-Naela Rizqy Arofah-22305144042-124.png}
\begin{eulerprompt}
>R=getCircleRadius(c);
>O=getCircleCenter(c)
\end{eulerprompt}
\begin{euleroutput}
  [0,  0.154701]
\end{euleroutput}
\begin{eulerprompt}
>plotCircle(c,"Lingkaran luar segitiga ABC"):
\end{eulerprompt}
\eulerimg{27}{images/EMT4Geometry-Naela Rizqy Arofah-22305144042-125.png}
\begin{eulercomment}
2. Gambarlah suatu parabola yang melalui 3 titik yang diketahui.

Petunjuk:\\
- Misalkan persamaan parabolanya y= ax\textasciicircum{}2+bx+c.\\
- Substitusikan koordinat titik-titik yang diketahui ke persamaan
tersebut.\\
- Selesaikan SPL yang terbentuk untuk mendapatkan nilai-nilai a, b, c.

\end{eulercomment}
\begin{eulerprompt}
>setPlotRange(5); 
>K=[-4,0];  L=[4,0] ; M=[0,2];
>plotPoint(K,"K"); plotPoint(L,"L"); plotPoint(M,"M"): 
\end{eulerprompt}
\eulerimg{27}{images/EMT4Geometry-Naela Rizqy Arofah-22305144042-126.png}
\begin{eulerprompt}
>sol &= solve([16*a+8*b=-c,16*a-8*b=-c,c=2],[a,b,c])
\end{eulerprompt}
\begin{euleroutput}
  
                                1
                        [[a = - -, b = 0, c = 2]]
                                8
  
\end{euleroutput}
\begin{eulerprompt}
>function y&=-1/8*(x^2)-0*x+2
\end{eulerprompt}
\begin{euleroutput}
  
                                       2
                                      x
                                  2 - --
                                      8
  
\end{euleroutput}
\begin{eulerprompt}
>plot2d("-1/8*(x^2)-0*x+2",-5,5,-5,5); ...
>plotPoint(K,"K"); plotPoint(L,"L"); plotPoint(M,"M"): 
\end{eulerprompt}
\eulerimg{27}{images/EMT4Geometry-Naela Rizqy Arofah-22305144042-127.png}
\begin{eulercomment}
3. Gambarlah suatu segi-4 yang diketahui keempat titik sudutnya,
misalnya A, B, C, D.\\
\end{eulercomment}
\begin{eulerttcomment}
   - Tentukan apakah segi-4 tersebut merupakan segi-4 garis singgung
\end{eulerttcomment}
\begin{eulercomment}
(sisinya-sisintya merupakan garis singgung lingkaran yang sama yakni
lingkaran dalam segi-4 tersebut).\\
\end{eulercomment}
\begin{eulerttcomment}
   - Suatu segi-4 merupakan segi-4 garis singgung apabila keempat
\end{eulerttcomment}
\begin{eulercomment}
garis bagi sudutnya bertemu di satu titik.\\
\end{eulercomment}
\begin{eulerttcomment}
   - Jika segi-4 tersebut merupakan segi-4 garis singgung, gambar
\end{eulerttcomment}
\begin{eulercomment}
lingkaran dalamnya.\\
\end{eulercomment}
\begin{eulerttcomment}
   - Tunjukkan bahwa syarat suatu segi-4 merupakan segi-4 garis
\end{eulerttcomment}
\begin{eulercomment}
singgung apabila hasil kali panjang sisi-sisi yang berhadapan sama.

\end{eulercomment}
\begin{eulerprompt}
>setPlotRange(5);
>A=[-3,-3];  B=[3,-3] ; C=[3,3]; D=[-3,3];
>plotPoint(A,"A"); plotPoint(B,"B"); plotPoint(C,"C"); plotPoint(D,"D");
>plotSegment(A,B,""); plotSegment(B,C,""); plotSegment(C,D,""); plotSegment(D,A,""):
\end{eulerprompt}
\eulerimg{27}{images/EMT4Geometry-Naela Rizqy Arofah-22305144042-128.png}
\begin{eulerprompt}
>l=angleBisector(A,B,C);
>g=angleBisector(B,C,D);
>P=lineIntersection(l,g)
\end{eulerprompt}
\begin{euleroutput}
  [0,  0]
\end{euleroutput}
\begin{eulerprompt}
>color(5); plotLine(l); plotLine(g); color(1);
>plotPoint(P,"P"):
\end{eulerprompt}
\eulerimg{27}{images/EMT4Geometry-Naela Rizqy Arofah-22305144042-129.png}
\begin{eulerprompt}
>r=norm(P-projectToLine(P,lineThrough(A,B)))
\end{eulerprompt}
\begin{euleroutput}
  3
\end{euleroutput}
\begin{eulerprompt}
>plotCircle(circleWithCenter(P,r),"Lingkaran dalam segiempat ABC"):
\end{eulerprompt}
\eulerimg{27}{images/EMT4Geometry-Naela Rizqy Arofah-22305144042-130.png}
\begin{eulercomment}
Darai gambar di atas, terlihat bahwa sisi-sisinya merupakan garis
singgung lingkaran yang sama yaitu lingkaran dalam segiempat.

Kemudian akan ditunjukkan bahwa hasil kali panjang sisi-sisi yang
berhadapan sama.
\end{eulercomment}
\begin{eulerprompt}
>AB=norm(A-B) // panjang sisi AB
\end{eulerprompt}
\begin{euleroutput}
  6
\end{euleroutput}
\begin{eulerprompt}
>BC=norm(B-C) // panjang sisi ABC
\end{eulerprompt}
\begin{euleroutput}
  6
\end{euleroutput}
\begin{eulerprompt}
>CD=norm(C-D) // panjang sisi CD
\end{eulerprompt}
\begin{euleroutput}
  6
\end{euleroutput}
\begin{eulerprompt}
>DA=norm(D-A) // panjang sisi DA
\end{eulerprompt}
\begin{euleroutput}
  6
\end{euleroutput}
\begin{eulerprompt}
>AB.CD
\end{eulerprompt}
\begin{euleroutput}
  36
\end{euleroutput}
\begin{eulerprompt}
>DA.BC
\end{eulerprompt}
\begin{euleroutput}
  36
\end{euleroutput}
\begin{eulercomment}
5. Gambarlah suatu hiperbola jika diketahui kedua titik fokusnya,
misalnya P dan Q. Ingat ellips dengan fokus P dan Q adalah tempat
kedudukan titik-titik yang selisih jarak ke P dan ke Q selalu sama
(konstan).
\end{eulercomment}
\begin{eulerprompt}
>P=[-1,2]; Q=[1,2];
>function d1(x,y):=sqrt((x-P[1])^2+(y-P[2])^2)
>function d2(x,y):=d1(x,y)+sqrt((x-Q[1])^2+(y-Q[2])^2)
>fcontour("d2",xmin=-2,xmax=2,ymin=0,ymax=4,hue=1):
\end{eulerprompt}
\eulerimg{27}{images/EMT4Geometry-Naela Rizqy Arofah-22305144042-131.png}
\begin{eulercomment}
Grafik yang lebih menarik
\end{eulercomment}
\begin{eulerprompt}
>plot3d("d2",xmin=-2,xmax=2,ymin=0,ymax=4,hue=1):
\end{eulerprompt}
\eulerimg{27}{images/EMT4Geometry-Naela Rizqy Arofah-22305144042-132.png}
\begin{eulercomment}
Batasan garis PQ
\end{eulercomment}
\begin{eulerprompt}
>plot2d("abs(x+1)+abs(x-1)",xmin=-3,xmax=3):
\end{eulerprompt}
\eulerimg{27}{images/EMT4Geometry-Naela Rizqy Arofah-22305144042-133.png}
\begin{eulercomment}
5. Gambarlah suatu hiperbola jika diketahui kedua titik fokusnya,
misalnya P dan Q. Ingat ellips dengan fokus P dan Q adalah tempat
kedudukan titik-titik yang selisih jarak ke P dan ke Q selalu sama
(konstan).
\end{eulercomment}
\begin{eulerprompt}
>P=[-1,2]; Q=[1,2];
>function d3(x,y):=sqrt((x-P[1])^2+(y-P[2])^2)
>function d4(x,y):=d3(x,y)+sqrt((x-Q[1])^2+(y-Q[2])^2
>fcontour("d3",xmin=-4,xmax=2,ymin=0,ymax=4,hue=1):
\end{eulerprompt}
\eulerimg{27}{images/EMT4Geometry-Naela Rizqy Arofah-22305144042-134.png}
\begin{eulercomment}
Grafik yang lebih menarik
\end{eulercomment}
\begin{eulerprompt}
>plot3d("d3",xmin=-2,xmax=2,ymin=0,ymax=4,hue=1):
\end{eulerprompt}
\eulerimg{27}{images/EMT4Geometry-Naela Rizqy Arofah-22305144042-135.png}
\begin{eulerprompt}
>plot2d("abs(x+1)+abs(x-1)",xmin=-3,xmax=3):
\end{eulerprompt}
\eulerimg{27}{images/EMT4Geometry-Naela Rizqy Arofah-22305144042-136.png}

\chapter{EMT untuk Statistika}

\eulersubheading{}
\begin{eulercomment}
Dalam buku catatan ini, kami mendemonstrasikan plot statistik utama,
pengujian, dan distribusi di Euler.

Mari kita mulai dengan beberapa statistik deskriptif. Ini bukan
pengantar statistik. Jadi, Anda mungkin memerlukan beberapa latar
belakang untuk memahami detailnya.

Asumsikan pengukuran berikut. Kami ingin menghitung nilai rata-rata
dan standar deviasi yang diukur.
\end{eulercomment}
\begin{eulerprompt}
>M=[1000,1004,998,997,1002,1001,998,1004,998,997]; ...
>mean(M), dev(M),
\end{eulerprompt}
\begin{euleroutput}
  999.9
  2.72641400622
\end{euleroutput}
\begin{eulercomment}
Kita dapat memplot plot kotak-dan-kumis untuk data. Dalam kasus kami
tidak ada outlier.
\end{eulercomment}
\begin{eulerprompt}
>boxplot(M):
\end{eulerprompt}
\eulerimg{27}{images/EMT4Statistika-Naela Rizqy Arofah-22305144042-001.png}
\begin{eulercomment}
Kami menghitung probabilitas bahwa suatu nilai lebih besar dari 1005,
dengan asumsi nilai terukur dan distribusi normal.

Semua fungsi untuk distribusi di Euler diakhiri dengan ...dis dan
menghitung distribusi probabilitas kumulatif (CPF).

\end{eulercomment}
\begin{eulerformula}
\[
\text{normaldis(x,m,d)}=\int_{-\infty}^x \frac{1}{d\sqrt{2\pi}}e^{-\frac{1}{2}(\frac{t-m}{d})^2}\ dt.
\]
\end{eulerformula}
\begin{eulercomment}
Kami mencetak hasilnya dalam \% dengan akurasi 2 digit menggunakan
fungsi cetak.
\end{eulercomment}
\begin{eulerprompt}
>print((1-normaldis(1005,mean(M),dev(M)))*100,2,unit=" %")
\end{eulerprompt}
\begin{euleroutput}
        3.07 %
\end{euleroutput}
\begin{eulercomment}
Untuk contoh berikutnya, kami mengasumsikan jumlah pria berikut dalam
rentang ukuran yang diberikan.
\end{eulercomment}
\begin{eulerprompt}
>r=155.5:4:187.5; v=[22,71,136,169,139,71,32,8];
\end{eulerprompt}
\begin{eulercomment}
Berikut adalah plot distribusinya.
\end{eulercomment}
\begin{eulerprompt}
>plot2d(r,v,a=150,b=200,c=0,d=190,bar=1,style="\(\backslash\)/"):
\end{eulerprompt}
\eulerimg{27}{images/EMT4Statistika-Naela Rizqy Arofah-22305144042-002.png}
\begin{eulercomment}
Kita bisa memasukkan data mentah tersebut ke dalam sebuah tabel.

Tabel adalah metode untuk menyimpan data statistik. Tabel kita harus
berisi tiga kolom: Awal jangkauan, akhir jangkauan, jumlah orang dalam
jangkauan.

Tabel dapat dicetak dengan header. Kami menggunakan vektor string
untuk mengatur header.
\end{eulercomment}
\begin{eulerprompt}
>T:=r[1:8]' | r[2:9]' | v'; writetable(T,labc=["from","to","count"])
\end{eulerprompt}
\begin{euleroutput}
        from        to     count
       155.5     159.5        22
       159.5     163.5        71
       163.5     167.5       136
       167.5     171.5       169
       171.5     175.5       139
       175.5     179.5        71
       179.5     183.5        32
       183.5     187.5         8
\end{euleroutput}
\begin{eulercomment}
Jika kita membutuhkan nilai rata-rata dan statistik lain dari ukuran,
kita perlu menghitung titik tengah rentang. Kita dapat menggunakan dua
kolom pertama dari tabel kita untuk ini.

Sumbul "\textbar{}" digunakan untuk memisahkan kolom, fungsi "writetable"
digunakan untuk menulis tabel, dengan opsion "labc" adalah untuk
menentukan header kolom.
\end{eulercomment}
\begin{eulerprompt}
>(T[,1]+T[,2])/2 // the midpoint of each interval
\end{eulerprompt}
\begin{euleroutput}
          157.5 
          161.5 
          165.5 
          169.5 
          173.5 
          177.5 
          181.5 
          185.5 
\end{euleroutput}
\begin{eulercomment}
Tetapi lebih mudah, untuk melipat rentang dengan vektor [1/2.1/2].
\end{eulercomment}
\begin{eulerprompt}
>M=fold(r,[0.5,0.5])
\end{eulerprompt}
\begin{euleroutput}
  [157.5,  161.5,  165.5,  169.5,  173.5,  177.5,  181.5,  185.5]
\end{euleroutput}
\begin{eulercomment}
Sekarang kita dapat menghitung rata-rata dan deviasi sampel dengan
frekuensi yang diberikan.
\end{eulercomment}
\begin{eulerprompt}
>\{m,d\}=meandev(M,v); m, d,
\end{eulerprompt}
\begin{euleroutput}
  169.901234568
  5.98912964449
\end{euleroutput}
\begin{eulercomment}
Mari kita tambahkan distribusi normal dari nilai-nilai ke plot batang
di atas. Rumus untuk distribusi normal dengan mean m dan standar
deviasi d adalah:

\end{eulercomment}
\begin{eulerformula}
\[
y=\frac{1}{d\sqrt{2\pi}}e^{\frac{-(x-m)^2}{2d^2}}.
\]
\end{eulerformula}
\begin{eulercomment}
Karena nilainya antara 0 dan 1, untuk memplotnya pada bar plot harus
dikalikan dengan 4 kali jumlah total data.
\end{eulercomment}
\begin{eulerprompt}
>plot2d("qnormal(x,m,d)*sum(v)*4", ...
>  xmin=min(r),xmax=max(r),thickness=3,add=1):
\end{eulerprompt}
\eulerimg{27}{images/EMT4Statistika-Naela Rizqy Arofah-22305144042-003.png}
\eulerheading{Tabel}
\begin{eulercomment}
Di direktori notebook ini Anda menemukan file dengan tabel. Data
tersebut mewakili hasil survei. Berikut adalah empat baris pertama
dari file tersebut. Data berasal dari buku online Jerman "Einführung
in die Statistik mit R" oleh A. Handl.
\end{eulercomment}
\begin{eulerprompt}
>printfile("table.dat",4);
\end{eulerprompt}
\begin{euleroutput}
  Person Sex Age Titanic Evaluation Tip Problem
  1 m 30 n . 1.80 n
  2 f 23 y g 1.80 n
  3 f 26 y g 1.80 y
\end{euleroutput}
\begin{eulercomment}
Tabel berisi 7 kolom angka atau token (string). Kami ingin membaca
tabel dari file. Pertama, kami menggunakan terjemahan kami sendiri
untuk token.

Untuk ini, kami mendefinisikan set token. Fungsi strtokens()
mendapatkan vektor string token dari string yang diberikan.
\end{eulercomment}
\begin{eulerprompt}
>mf:=["m","f"]; yn:=["y","n"]; ev:=strtokens("g vg m b vb");
\end{eulerprompt}
\begin{eulercomment}
Sekarang kita membaca tabel dengan terjemahan ini.

Argumen tok2, tok4 dll. adalah terjemahan dari kolom tabel. Argumen
ini tidak ada dalam daftar parameter readtable(), jadi Anda harus
menyediakannya dengan ":=".
\end{eulercomment}
\begin{eulerprompt}
>\{MT,hd\}=readtable("table.dat",tok2:=mf,tok4:=yn,tok5:=ev,tok7:=yn);
>load over statistics;
\end{eulerprompt}
\begin{eulercomment}
Untuk mencetak, kita perlu menentukan set token yang sama. Kami
mencetak empat baris pertama saja.
\end{eulercomment}
\begin{eulerprompt}
>writetable(MT[1:4],labc=hd,wc=5,tok2:=mf,tok4:=yn,tok5:=ev,tok7:=yn);
\end{eulerprompt}
\begin{euleroutput}
   Person  Sex  Age Titanic Evaluation  Tip Problem
        1    m   30       n          .  1.8       n
        2    f   23       y          g  1.8       n
        3    f   26       y          g  1.8       y
        4    m   33       n          .  2.8       n
\end{euleroutput}
\begin{eulercomment}
Titik "." mewakili nilai-nilai, yang tidak tersedia.

Jika kita tidak ingin menentukan token untuk terjemahan terlebih
dahulu, kita hanya perlu menentukan, kolom mana yang berisi token dan
bukan angka.
\end{eulercomment}
\begin{eulerprompt}
>ctok=[2,4,5,7]; \{MT,hd,tok\}=readtable("table.dat",ctok=ctok);
\end{eulerprompt}
\begin{eulercomment}
Fungsi readtable() sekarang mengembalikan satu set token.
\end{eulercomment}
\begin{eulerprompt}
>tok
\end{eulerprompt}
\begin{euleroutput}
  m
  n
  f
  y
  g
  vg
\end{euleroutput}
\begin{eulercomment}
Tabel berisi entri dari file dengan token yang diterjemahkan ke angka.

String khusus NA="." ditafsirkan sebagai "Tidak Tersedia", dan
mendapatkan NAN (bukan angka) dalam tabel. Terjemahan ini dapat diubah
dengan parameter NA, dan NAval.
\end{eulercomment}
\begin{eulerprompt}
>MT[1]
\end{eulerprompt}
\begin{euleroutput}
  [1,  1,  30,  2,  NAN,  1.8,  2]
\end{euleroutput}
\begin{eulercomment}
Berikut isi tabel dengan angka yang belum diterjemahkan.
\end{eulercomment}
\begin{eulerprompt}
>writetable(MT,wc=5)
\end{eulerprompt}
\begin{euleroutput}
      1    1   30    2    .  1.8    2
      2    3   23    4    5  1.8    2
      3    3   26    4    5  1.8    4
      4    1   33    2    .  2.8    2
      5    1   37    2    .  1.8    2
      6    1   28    4    5  2.8    4
      7    3   31    4    6  2.8    2
      8    1   23    2    .  0.8    2
      9    3   24    4    6  1.8    4
     10    1   26    2    .  1.8    2
     11    3   23    4    6  1.8    4
     12    1   32    4    5  1.8    2
     13    1   29    4    6  1.8    4
     14    3   25    4    5  1.8    4
     15    3   31    4    5  0.8    2
     16    1   26    4    5  2.8    2
     17    1   37    2    .  3.8    2
     18    1   38    4    5    .    2
     19    3   29    2    .  3.8    2
     20    3   28    4    6  1.8    2
     21    3   28    4    1  2.8    4
     22    3   28    4    6  1.8    4
     23    3   38    4    5  2.8    2
     24    3   27    4    1  1.8    4
     25    1   27    2    .  2.8    4
\end{euleroutput}
\begin{eulercomment}
Untuk kenyamanan, Anda dapat memasukkan output readtable() ke dalam
daftar.
\end{eulercomment}
\begin{eulerprompt}
>Table=\{\{readtable("table.dat",ctok=ctok)\}\};
\end{eulerprompt}
\begin{eulercomment}
Menggunakan kolom token yang sama dan token yang dibaca dari file,
kita dapat mencetak tabel. Kita dapat menentukan ctok, tok, dll. Atau
menggunakan daftar Tabel.
\end{eulercomment}
\begin{eulerprompt}
>writetable(Table,ctok=ctok,wc=5);
\end{eulerprompt}
\begin{euleroutput}
   Person  Sex  Age Titanic Evaluation  Tip Problem
        1    m   30       n          .  1.8       n
        2    f   23       y          g  1.8       n
        3    f   26       y          g  1.8       y
        4    m   33       n          .  2.8       n
        5    m   37       n          .  1.8       n
        6    m   28       y          g  2.8       y
        7    f   31       y         vg  2.8       n
        8    m   23       n          .  0.8       n
        9    f   24       y         vg  1.8       y
       10    m   26       n          .  1.8       n
       11    f   23       y         vg  1.8       y
       12    m   32       y          g  1.8       n
       13    m   29       y         vg  1.8       y
       14    f   25       y          g  1.8       y
       15    f   31       y          g  0.8       n
       16    m   26       y          g  2.8       n
       17    m   37       n          .  3.8       n
       18    m   38       y          g    .       n
       19    f   29       n          .  3.8       n
       20    f   28       y         vg  1.8       n
       21    f   28       y          m  2.8       y
       22    f   28       y         vg  1.8       y
       23    f   38       y          g  2.8       n
       24    f   27       y          m  1.8       y
       25    m   27       n          .  2.8       y
\end{euleroutput}
\begin{eulercomment}
Fungsi tablecol() mengembalikan nilai kolom tabel, melewatkan setiap
baris dengan nilai NAN("." dalam file), dan indeks kolom, yang berisi
nilai-nilai ini.
\end{eulercomment}
\begin{eulerprompt}
>\{c,i\}=tablecol(MT,[5,6]);
\end{eulerprompt}
\begin{eulercomment}
Kita dapat menggunakan ini untuk mengekstrak kolom dari tabel untuk
tabel baru.
\end{eulercomment}
\begin{eulerprompt}
>j=[1,5,6]; writetable(MT[i,j],labc=hd[j],ctok=[2],tok=tok)
\end{eulerprompt}
\begin{euleroutput}
      Person Evaluation       Tip
           2          g       1.8
           3          g       1.8
           6          g       2.8
           7         vg       2.8
           9         vg       1.8
          11         vg       1.8
          12          g       1.8
          13         vg       1.8
          14          g       1.8
          15          g       0.8
          16          g       2.8
          20         vg       1.8
          21          m       2.8
          22         vg       1.8
          23          g       2.8
          24          m       1.8
\end{euleroutput}
\begin{eulercomment}
Tentu saja, kita perlu mengekstrak tabel itu sendiri dari daftar Tabel
dalam kasus ini.
\end{eulercomment}
\begin{eulerprompt}
>MT=Table[1];
\end{eulerprompt}
\begin{eulercomment}
Tentu saja, kita juga dapat menggunakannya untuk menentukan nilai
rata-rata kolom atau nilai statistik lainnya.
\end{eulercomment}
\begin{eulerprompt}
>mean(tablecol(MT,6))
\end{eulerprompt}
\begin{euleroutput}
  2.175
\end{euleroutput}
\begin{eulercomment}
Fungsi getstatistics() mengembalikan elemen dalam vektor, dan
jumlahnya. Kami menerapkannya pada nilai "m" dan "f" di kolom kedua
tabel kami.
\end{eulercomment}
\begin{eulerprompt}
>\{xu,count\}=getstatistics(tablecol(MT,2)); xu, count,
\end{eulerprompt}
\begin{euleroutput}
  [1,  3]
  [12,  13]
\end{euleroutput}
\begin{eulercomment}
Kami dapat mencetak hasilnya dalam tabel baru.
\end{eulercomment}
\begin{eulerprompt}
>writetable(count',labr=tok[xu])
\end{eulerprompt}
\begin{euleroutput}
           m        12
           f        13
\end{euleroutput}
\begin{eulercomment}
Fungsi selecttable() mengembalikan tabel baru dengan nilai dalam satu
kolom yang dipilih dari vektor indeks. Pertama kita mencari indeks
dari dua nilai kita di tabel token.
\end{eulercomment}
\begin{eulerprompt}
>v:=indexof(tok,["g","vg"])
\end{eulerprompt}
\begin{euleroutput}
  [5,  6]
\end{euleroutput}
\begin{eulercomment}
Sekarang kita dapat memilih baris tabel, yang memiliki salah satu
nilai dalam v di baris ke-5.
\end{eulercomment}
\begin{eulerprompt}
>MT1:=MT[selectrows(MT,5,v)]; i:=sortedrows(MT1,5);
\end{eulerprompt}
\begin{eulercomment}
Sekarang kita dapat mencetak tabel, dengan nilai yang diekstrak dan
diurutkan di kolom ke-5.
\end{eulercomment}
\begin{eulerprompt}
>writetable(MT1[i],labc=hd,ctok=ctok,tok=tok,wc=7);
\end{eulerprompt}
\begin{euleroutput}
   Person    Sex    Age Titanic Evaluation    Tip Problem
        2      f     23       y          g    1.8       n
        3      f     26       y          g    1.8       y
        6      m     28       y          g    2.8       y
       18      m     38       y          g      .       n
       16      m     26       y          g    2.8       n
       15      f     31       y          g    0.8       n
       12      m     32       y          g    1.8       n
       23      f     38       y          g    2.8       n
       14      f     25       y          g    1.8       y
        9      f     24       y         vg    1.8       y
        7      f     31       y         vg    2.8       n
       20      f     28       y         vg    1.8       n
       22      f     28       y         vg    1.8       y
       13      m     29       y         vg    1.8       y
       11      f     23       y         vg    1.8       y
\end{euleroutput}
\begin{eulercomment}
Untuk statistik berikutnya, kami ingin menghubungkan dua kolom tabel.
Jadi kami mengekstrak kolom 2 dan 4 dan mengurutkan tabel.
\end{eulercomment}
\begin{eulerprompt}
>i=sortedrows(MT,[2,4]);  ...
>  writetable(tablecol(MT[i],[2,4])',ctok=[1,2],tok=tok)
\end{eulerprompt}
\begin{euleroutput}
           m         n
           m         n
           m         n
           m         n
           m         n
           m         n
           m         n
           m         y
           m         y
           m         y
           m         y
           m         y
           f         n
           f         y
           f         y
           f         y
           f         y
           f         y
           f         y
           f         y
           f         y
           f         y
           f         y
           f         y
           f         y
\end{euleroutput}
\begin{eulercomment}
Dengan getstatistics(), kita juga dapat menghubungkan hitungan dalam
dua kolom tabel satu sama lain.
\end{eulercomment}
\begin{eulerprompt}
>MT24=tablecol(MT,[2,4]); ...
>\{xu1,xu2,count\}=getstatistics(MT24[1],MT24[2]); ...
>writetable(count,labr=tok[xu1],labc=tok[xu2])
\end{eulerprompt}
\begin{euleroutput}
                     n         y
           m         7         5
           f         1        12
\end{euleroutput}
\begin{eulercomment}
Sebuah tabel dapat ditulis ke file.
\end{eulercomment}
\begin{eulerprompt}
>filename="test.dat"; ...
>writetable(count,labr=tok[xu1],labc=tok[xu2],file=filename);
\end{eulerprompt}
\begin{eulercomment}
Kemudian kita bisa membaca tabel dari file.
\end{eulercomment}
\begin{eulerprompt}
>\{MT2,hd,tok2,hdr\}=readtable(filename,>clabs,>rlabs); ...
>writetable(MT2,labr=hdr,labc=hd)
\end{eulerprompt}
\begin{euleroutput}
                     n         y
           m         7         5
           f         1        12
\end{euleroutput}
\begin{eulercomment}
Dan hapus filenya.
\end{eulercomment}
\begin{eulerprompt}
>fileremove(filename);
\end{eulerprompt}
\eulerheading{Distribusi}
\begin{eulercomment}
Dengan plot2d, ada metode yang sangat mudah untuk memplot distribusi
data eksperimen.
\end{eulercomment}
\begin{eulerprompt}
>p=normal(1,1000); //1000 random normal-distributed sample p
>plot2d(p,distribution=20,style="\(\backslash\)/"); // plot the random sample p
>plot2d("qnormal(x,0,1)",add=1): // add the standard normal distribution plot
\end{eulerprompt}
\eulerimg{27}{images/EMT4Statistika-Naela Rizqy Arofah-22305144042-004.png}
\begin{eulercomment}
Harap dicatat perbedaan antara plot batang (sampel) dan kurva normal
(distribusi nyata). Masukkan kembali tiga perintah untuk melihat hasil
pengambilan sampel lainnya.
\end{eulercomment}
\begin{eulercomment}
Berikut adalah perbandingan 10 simulasi 1000 nilai terdistribusi
normal menggunakan apa yang disebut plot kotak. Plot ini menunjukkan
median, kuartil 25\% dan 75\%, nilai minimal dan maksimal, dan outlier.
\end{eulercomment}
\begin{eulerprompt}
>p=normal(10,1000); boxplot(p):
\end{eulerprompt}
\eulerimg{27}{images/EMT4Statistika-Naela Rizqy Arofah-22305144042-005.png}
\begin{eulercomment}
Untuk menghasilkan bilangan bulat acak, Euler memiliki intrarandom.
Mari kita simulasikan lemparan dadu dan plot distribusinya.

Kami menggunakan fungsi getmultiplicities(v,x), yang menghitung
seberapa sering elemen v muncul di x. Kemudian kita plot hasilnya
menggunakan columnplot().
\end{eulercomment}
\begin{eulerprompt}
>k=intrandom(1,6000,6);  ...
>columnsplot(getmultiplicities(1:6,k));  ...
>ygrid(1000,color=red):
\end{eulerprompt}
\eulerimg{27}{images/EMT4Statistika-Naela Rizqy Arofah-22305144042-006.png}
\begin{eulercomment}
Sementara intrandom(n,m,k) mengembalikan bilangan bulat terdistribusi
seragam dari 1 ke k, dimungkinkan untuk menggunakan distribusi
bilangan bulat lainnya dengan randpint().

Dalam contoh berikut, probabilitas untuk 1,2,3 berturut-turut adalah
0,4,0,1,0,5.
\end{eulercomment}
\begin{eulerprompt}
>randpint(1,1000,[0.4,0.1,0.5]); getmultiplicities(1:3,%)
\end{eulerprompt}
\begin{euleroutput}
  [378,  102,  520]
\end{euleroutput}
\begin{eulercomment}
Euler dapat menghasilkan nilai acak dari lebih banyak distribusi. Coba
lihat referensinya.

Misalnya, kami mencoba distribusi eksponensial. Sebuah variabel acak
kontinu X dikatakan memiliki distribusi eksponensial, jika PDF-nya
diberikan oleh\\
\end{eulercomment}
\begin{eulerformula}
\[
f_X(x)=\lambda e^{-\lambda x},\quad x>0,\quad \lambda>0,
\]
\end{eulerformula}
\begin{eulercomment}
with parameter\\
\end{eulercomment}
\begin{eulerformula}
\[
\lambda=\frac{1}{\mu},\quad \mu \text{ is the mean, and denoted by } X \sim \text{Exponential}(\lambda).
\]
\end{eulerformula}
\begin{eulerprompt}
>plot2d(randexponential(1,1000,2),>distribution):
\end{eulerprompt}
\eulerimg{27}{images/EMT4Statistika-Naela Rizqy Arofah-22305144042-007.png}
\begin{eulercomment}
Untuk banyak distribusi, Euler dapat menghitung fungsi distribusi dan
kebalikannya.
\end{eulercomment}
\begin{eulerprompt}
>plot2d("normaldis",-4,4): 
\end{eulerprompt}
\eulerimg{27}{images/EMT4Statistika-Naela Rizqy Arofah-22305144042-008.png}
\begin{eulercomment}
Berikut ini adalah salah satu cara untuk memplot kuantil.
\end{eulercomment}
\begin{eulerprompt}
>plot2d("qnormal(x,1,1.5)",-4,6);  ...
>plot2d("qnormal(x,1,1.5)",a=2,b=5,>add,>filled):
\end{eulerprompt}
\eulerimg{27}{images/EMT4Statistika-Naela Rizqy Arofah-22305144042-009.png}
\begin{eulerformula}
\[
\text{normaldis(x,m,d)}=\int_{-\infty}^x \frac{1}{d\sqrt{2\pi}}e^{-\frac{1}{2}(\frac{t-m}{d})^2}\ dt.
\]
\end{eulerformula}
\begin{eulercomment}
Peluang berada di area hijau adalah sebagai berikut.
\end{eulercomment}
\begin{eulerprompt}
>normaldis(5,1,1.5)-normaldis(2,1,1.5)
\end{eulerprompt}
\begin{euleroutput}
  0.248662156979
\end{euleroutput}
\begin{eulercomment}
Ini dapat dihitung secara numerik dengan integral berikut.\\
\end{eulercomment}
\begin{eulerformula}
\[
\int_2^5 \frac{1}{1.5\sqrt{2\pi}}e^{-\frac{1}{2}(\frac{x-1}{1.5})^2}\ dx.
\]
\end{eulerformula}
\begin{eulerprompt}
>gauss("qnormal(x,1,1.5)",2,5)
\end{eulerprompt}
\begin{euleroutput}
  0.248662156979
\end{euleroutput}
\begin{eulercomment}
Mari kita bandingkan distribusi binomial dengan distribusi normal mean
dan deviasi yang sama. Fungsi invbindis() memecahkan interpolasi
linier antara nilai integer.
\end{eulercomment}
\begin{eulerprompt}
>invbindis(0.95,1000,0.5), invnormaldis(0.95,500,0.5*sqrt(1000))
\end{eulerprompt}
\begin{euleroutput}
  525.516721219
  526.007419394
\end{euleroutput}
\begin{eulercomment}
Fungsi qdis() adalah densitas dari distribusi chi-kuadrat. Seperti
biasa, evolusi vektor ke fungsi ini. Dengan demikian kita mendapatkan
plot semua distribusi chi-kuadrat dengan derajat 5 sampai 30 dengan
mudah dengan cara berikut.
\end{eulercomment}
\begin{eulerprompt}
>plot2d("qchidis(x,(5:5:50)')",0,50):
\end{eulerprompt}
\eulerimg{27}{images/EMT4Statistika-Naela Rizqy Arofah-22305144042-010.png}
\begin{eulercomment}
Euler memiliki fungsi yang akurat untuk mengevaluasi distribusi. Mari
kita periksa chidis() dengan integral.

Penamaan mencoba untuk konsisten. Misalnya.,

- distribusi chi-kuadrat adalah chidis(),\\
- fungsi kebalikannya adalah invchidis(),\\
- kepadatannya adalah qchidis().

Komplemen dari distribusi (ekor atas) adalah chicdis().
\end{eulercomment}
\begin{eulerprompt}
>chidis(1.5,2), integrate("qchidis(x,2)",0,1.5)
\end{eulerprompt}
\begin{euleroutput}
  0.527633447259
  0.527633447259
\end{euleroutput}
\eulerheading{Distribusi Diskrit}
\begin{eulercomment}
Untuk menentukan distribusi diskrit Anda sendiri, Anda dapat
menggunakan metode berikut.

Pertama kita atur fungsi distribusinya.
\end{eulercomment}
\begin{eulerprompt}
>wd = 0|((1:6)+[-0.01,0.01,0,0,0,0])/6
\end{eulerprompt}
\begin{euleroutput}
  [0,  0.165,  0.335,  0.5,  0.666667,  0.833333,  1]
\end{euleroutput}
\begin{eulercomment}
Artinya dengan probabilitas wd[i+1]-wd[i] kita menghasilkan nilai acak
i.

Ini adalah distribusi yang hampir seragam. Mari kita mendefinisikan
generator nomor acak untuk ini. Fungsi find(v,x) menemukan nilai x
dalam vektor v. Fungsi ini juga berfungsi untuk vektor x.
\end{eulercomment}
\begin{eulerprompt}
>function wrongdice (n,m) := find(wd,random(n,m))
\end{eulerprompt}
\begin{eulercomment}
Kesalahannya sangat halus sehingga kita hanya melihatnya dengan sangat
banyak iterasi.
\end{eulercomment}
\begin{eulerprompt}
>columnsplot(getmultiplicities(1:6,wrongdice(1,1000000))):
\end{eulerprompt}
\eulerimg{27}{images/EMT4Statistika-Naela Rizqy Arofah-22305144042-011.png}
\begin{eulercomment}
Berikut adalah fungsi sederhana untuk memeriksa distribusi seragam
dari nilai 1...K dalam v. Kami menerima hasilnya, jika untuk semua
frekuensi

\end{eulercomment}
\begin{eulerformula}
\[
\left|f_i-\frac{1}{K}\right| < \frac{\delta}{\sqrt{n}}.
\]
\end{eulerformula}
\begin{eulerprompt}
>function checkrandom (v, delta=1) ...
\end{eulerprompt}
\begin{eulerudf}
    K=max(v); n=cols(v);
    fr=getfrequencies(v,1:K);
    return max(fr/n-1/K)<delta/sqrt(n);
    endfunction
\end{eulerudf}
\begin{eulercomment}
Memang fungsi menolak distribusi seragam.
\end{eulercomment}
\begin{eulerprompt}
>checkrandom(wrongdice(1,1000000))
\end{eulerprompt}
\begin{euleroutput}
  0
\end{euleroutput}
\begin{eulercomment}
Dan itu menerima generator acak bawaan.
\end{eulercomment}
\begin{eulerprompt}
>checkrandom(intrandom(1,1000000,6))
\end{eulerprompt}
\begin{euleroutput}
  1
\end{euleroutput}
\begin{eulercomment}
Kita dapat menghitung distribusi binomial. Pertama ada binomialsum(),
yang mengembalikan probabilitas i atau kurang hit dari n percobaan.
\end{eulercomment}
\begin{eulerprompt}
>bindis(410,1000,0.4)
\end{eulerprompt}
\begin{euleroutput}
  0.751401349654
\end{euleroutput}
\begin{eulercomment}
Fungsi Beta terbalik digunakan untuk menghitung interval kepercayaan
Clopper-Pearson untuk parameter p. Tingkat default adalah alfa.

Arti interval ini adalah jika p berada di luar interval, hasil
pengamatan 410 dalam 1000 jarang terjadi.
\end{eulercomment}
\begin{eulerprompt}
>clopperpearson(410,1000)
\end{eulerprompt}
\begin{euleroutput}
  [0.37932,  0.441212]
\end{euleroutput}
\begin{eulercomment}
Perintah berikut adalah cara langsung untuk mendapatkan hasil di atas.
Tetapi untuk n besar, penjumlahan langsung tidak akurat dan lambat.
\end{eulercomment}
\begin{eulerprompt}
>p=0.4; i=0:410; n=1000; sum(bin(n,i)*p^i*(1-p)^(n-i))
\end{eulerprompt}
\begin{euleroutput}
  0.751401349655
\end{euleroutput}
\begin{eulercomment}
Omong-omong, invbinsum() menghitung kebalikan dari binomialsum().
\end{eulercomment}
\begin{eulerprompt}
>invbindis(0.75,1000,0.4)
\end{eulerprompt}
\begin{euleroutput}
  409.932733047
\end{euleroutput}
\begin{eulercomment}
Di Bridge, kami mengasumsikan 5 kartu yang beredar (dari 52) di dua
tangan (26 kartu). Mari kita hitung probabilitas distribusi yang lebih
buruk dari 3:2 (misalnya 0:5, 1:4, 4:1 atau 5:0).
\end{eulercomment}
\begin{eulerprompt}
>2*hypergeomsum(1,5,13,26)
\end{eulerprompt}
\begin{euleroutput}
  0.321739130435
\end{euleroutput}
\begin{eulercomment}
Ada juga simulasi distribusi multinomial.
\end{eulercomment}
\begin{eulerprompt}
>randmultinomial(10,1000,[0.4,0.1,0.5])
\end{eulerprompt}
\begin{euleroutput}
            381           100           519 
            376            91           533 
            417            80           503 
            440            94           466 
            406           112           482 
            408            94           498 
            395           107           498 
            399            96           505 
            428            87           485 
            400            99           501 
\end{euleroutput}
\eulerheading{Merencanakan Data}
\begin{eulercomment}
Untuk plot data, kami mencoba hasil pemilu Jerman sejak tahun 1990,
diukur dalam kursi.
\end{eulercomment}
\begin{eulerprompt}
>BW := [ ...
>1990,662,319,239,79,8,17; ...
>1994,672,294,252,47,49,30; ...
>1998,669,245,298,43,47,36; ...
>2002,603,248,251,47,55,2; ...
>2005,614,226,222,61,51,54; ...
>2009,622,239,146,93,68,76; ...
>2013,631,311,193,0,63,64];
\end{eulerprompt}
\begin{eulercomment}
Untuk para pihak, kami menggunakan serangkaian nama.
\end{eulercomment}
\begin{eulerprompt}
>P:=["CDU/CSU","SPD","FDP","Gr","Li"];
\end{eulerprompt}
\begin{eulercomment}
Mari kita mencetak persentase dengan baik.

Pertama kita ekstrak kolom yang diperlukan. Kolom 3 sampai 7 adalah
kursi masing-masing partai, dan kolom 2 adalah jumlah kursi. kolom 1
adalah tahun pemilihan.
\end{eulercomment}
\begin{eulerprompt}
>BT:=BW[,3:7]; BT:=BT/sum(BT); YT:=BW[,1]';
\end{eulerprompt}
\begin{eulercomment}
Kemudian kami mencetak statistik dalam bentuk tabel. Kami menggunakan
nama sebagai tajuk kolom, dan tahun sebagai tajuk untuk baris. Lebar
default untuk kolom adalah wc=10, tetapi kami lebih memilih output
yang lebih padat. Kolom akan diperluas untuk label kolom, jika perlu.
\end{eulercomment}
\begin{eulerprompt}
>writetable(BT*100,wc=6,dc=0,>fixed,labc=P,labr=YT)
\end{eulerprompt}
\begin{euleroutput}
         CDU/CSU   SPD   FDP    Gr    Li
    1990      48    36    12     1     3
    1994      44    38     7     7     4
    1998      37    45     6     7     5
    2002      41    42     8     9     0
    2005      37    36    10     8     9
    2009      38    23    15    11    12
    2013      49    31     0    10    10
\end{euleroutput}
\begin{eulercomment}
Perkalian matriks berikut mengekstrak jumlah persentase dua partai
besar yang menunjukkan bahwa partai-partai kecil telah memperoleh
rekaman di parlemen hingga 2009.
\end{eulercomment}
\begin{eulerprompt}
>BT1:=(BT.[1;1;0;0;0])'*100
\end{eulerprompt}
\begin{euleroutput}
  [84.29,  81.25,  81.1659,  82.7529,  72.9642,  61.8971,  79.8732]
\end{euleroutput}
\begin{eulercomment}
Ada juga plot statistik sederhana. Kami menggunakannya untuk
menampilkan garis dan titik secara bersamaan. Alternatifnya adalah
memanggil plot2d dua kali dengan \textgreater{}add.
\end{eulercomment}
\begin{eulerprompt}
>statplot(YT,BT1,"b"):
\end{eulerprompt}
\eulerimg{27}{images/EMT4Statistika-Naela Rizqy Arofah-22305144042-012.png}
\begin{eulercomment}
Tentukan beberapa warna untuk masing-masing pihak.
\end{eulercomment}
\begin{eulerprompt}
>CP:=[rgb(0.5,0.5,0.5),red,yellow,green,rgb(0.8,0,0)];
\end{eulerprompt}
\begin{eulercomment}
Sekarang kita dapat memplot hasil pemilu 2009 dan perubahannya menjadi
satu plot menggunakan gambar. Kita dapat menambahkan vektor kolom ke
setiap plot.
\end{eulercomment}
\begin{eulerprompt}
>figure(2,1);  ...
>figure(1); columnsplot(BW[6,3:7],P,color=CP); ...
>figure(2); columnsplot(BW[6,3:7]-BW[5,3:7],P,color=CP);  ...
>figure(0):
\end{eulerprompt}
\eulerimg{27}{images/EMT4Statistika-Naela Rizqy Arofah-22305144042-013.png}
\begin{eulercomment}
Plot data menggabungkan deretan data statistik dalam satu plot.
\end{eulercomment}
\begin{eulerprompt}
>J:=BW[,1]'; DP:=BW[,3:7]'; ...
>dataplot(YT,BT',color=CP);  ...
>labelbox(P,colors=CP,styles="[]",>points,w=0.2,x=0.3,y=0.4):
\end{eulerprompt}
\eulerimg{27}{images/EMT4Statistika-Naela Rizqy Arofah-22305144042-014.png}
\begin{eulercomment}
Sebuah kolom plot 3D menunjukkan baris data statistik dalam bentuk
kolom. Kami menyediakan label untuk baris dan kolom. angle adalah
sudut pandang.
\end{eulercomment}
\begin{eulerprompt}
>columnsplot3d(BT,scols=P,srows=YT, ...
>  angle=30°,ccols=CP):
\end{eulerprompt}
\eulerimg{27}{images/EMT4Statistika-Naela Rizqy Arofah-22305144042-015.png}
\begin{eulercomment}
Representasi lain adalah plot mosaik. Perhatikan bahwa kolom plot
mewakili kolom matriks di sini. Karena panjangnya label CDU/CSU, kami
mengambil jendela yang lebih kecil dari biasanya.
\end{eulercomment}
\begin{eulerprompt}
>shrinkwindow(>smaller);  ...
>mosaicplot(BT',srows=YT,scols=P,color=CP,style="#"); ...
>shrinkwindow():
\end{eulerprompt}
\eulerimg{27}{images/EMT4Statistika-Naela Rizqy Arofah-22305144042-016.png}
\begin{eulercomment}
Kita juga bisa membuat diagram lingkaran. Karena hitam dan kuning
membentuk koalisi, kami menyusun ulang elemen-elemennya.
\end{eulercomment}
\begin{eulerprompt}
>i=[1,3,5,4,2]; piechart(BW[6,3:7][i],color=CP[i],lab=P[i]):
\end{eulerprompt}
\eulerimg{27}{images/EMT4Statistika-Naela Rizqy Arofah-22305144042-017.png}
\begin{eulercomment}
Berikut adalah jenis plot lainnya.
\end{eulercomment}
\begin{eulerprompt}
>starplot(normal(1,10)+4,lab=1:10,>rays):
\end{eulerprompt}
\eulerimg{27}{images/EMT4Statistika-Naela Rizqy Arofah-22305144042-018.png}
\begin{eulercomment}
Beberapa plot di plot2d bagus untuk statika. Berikut adalah plot
impuls dari data acak, terdistribusi secara merata di [0,1].
\end{eulercomment}
\begin{eulerprompt}
>plot2d(makeimpulse(1:10,random(1,10)),>bar):
\end{eulerprompt}
\eulerimg{27}{images/EMT4Statistika-Naela Rizqy Arofah-22305144042-019.png}
\begin{eulercomment}
Tetapi untuk data yang terdistribusi secara eksponensial, kita mungkin
memerlukan plot logaritmik.
\end{eulercomment}
\begin{eulerprompt}
>logimpulseplot(1:10,-log(random(1,10))*10):
\end{eulerprompt}
\eulerimg{27}{images/EMT4Statistika-Naela Rizqy Arofah-22305144042-020.png}
\begin{eulercomment}
Fungsi columnplot() lebih mudah digunakan, karena hanya membutuhkan
vektor nilai. Selain itu, ia dapat mengatur labelnya ke apa pun yang
kita inginkan, kita sudah mendemonstrasikannya dalam tutorial ini.

Ini adalah aplikasi lain, di mana kita menghitung karakter dalam
sebuah kalimat dan menyusun statistik.
\end{eulercomment}
\begin{eulerprompt}
>v=strtochar("the quick brown fox jumps over the lazy dog"); ...
>w=ascii("a"):ascii("z"); x=getmultiplicities(w,v); ...
>cw=[]; for k=w; cw=cw|char(k); end; ...
>columnsplot(x,lab=cw,width=0.05):
\end{eulerprompt}
\eulerimg{27}{images/EMT4Statistika-Naela Rizqy Arofah-22305144042-021.png}
\begin{eulercomment}
Dimungkinkan juga untuk mengatur sumbu secara manual.
\end{eulercomment}
\begin{eulerprompt}
>n=10; p=0.4; i=0:n; x=bin(n,i)*p^i*(1-p)^(n-i); ...
>columnsplot(x,lab=i,width=0.05,<frame,<grid); ...
>yaxis(0,0:0.1:1,style="->",>left); xaxis(0,style="."); ...
>label("p",0,0.25), label("i",11,0); ...
>textbox(["Binomial distribution","with p=0.4"]):
\end{eulerprompt}
\eulerimg{27}{images/EMT4Statistika-Naela Rizqy Arofah-22305144042-022.png}
\begin{eulercomment}
Berikut ini adalah cara untuk memplot frekuensi bilangan dalam sebuah
vektor.

Kami membuat vektor bilangan bulat bilangan acak 1 hingga 6.
\end{eulercomment}
\begin{eulerprompt}
>v:=intrandom(1,10,10)
\end{eulerprompt}
\begin{euleroutput}
  [8,  5,  8,  8,  6,  8,  8,  3,  5,  5]
\end{euleroutput}
\begin{eulercomment}
Kemudian ekstrak nomor unik di v.
\end{eulercomment}
\begin{eulerprompt}
>vu:=unique(v)
\end{eulerprompt}
\begin{euleroutput}
  [3,  5,  6,  8]
\end{euleroutput}
\begin{eulercomment}
Dan plot frekuensi dalam plot kolom.
\end{eulercomment}
\begin{eulerprompt}
>columnsplot(getmultiplicities(vu,v),lab=vu,style="/"):
\end{eulerprompt}
\eulerimg{27}{images/EMT4Statistika-Naela Rizqy Arofah-22305144042-023.png}
\begin{eulercomment}
Kami ingin menunjukkan fungsi untuk distribusi nilai empiris.
\end{eulercomment}
\begin{eulerprompt}
>x=normal(1,20);
\end{eulerprompt}
\begin{eulercomment}
Fungsi empdist(x,vs) membutuhkan array nilai yang diurutkan. Jadi kita
harus mengurutkan x sebelum kita dapat menggunakannya.
\end{eulercomment}
\begin{eulerprompt}
>xs=sort(x);
\end{eulerprompt}
\begin{eulercomment}
Kemudian kami memplot distribusi empiris dan beberapa batang kepadatan
menjadi satu plot. Alih-alih plot bar untuk distribusi, kami
menggunakan plot gigi gergaji kali ini.
\end{eulercomment}
\begin{eulerprompt}
>figure(2,1); ...
>figure(1); plot2d("empdist",-4,4;xs); ...
>figure(2); plot2d(histo(x,v=-4:0.2:4,<bar));  ...
>figure(0):
\end{eulerprompt}
\eulerimg{27}{images/EMT4Statistika-Naela Rizqy Arofah-22305144042-024.png}
\begin{eulercomment}
Plot pencar mudah dilakukan di Euler dengan plot titik biasa. Grafik
berikut menunjukkan bahwa X dan X+Y jelas berkorelasi positif.
\end{eulercomment}
\begin{eulerprompt}
>x=normal(1,100); plot2d(x,x+rotright(x),>points,style=".."):
\end{eulerprompt}
\eulerimg{27}{images/EMT4Statistika-Naela Rizqy Arofah-22305144042-025.png}
\begin{eulercomment}
Seringkali, kita ingin membandingkan dua sampel dari distribusi yang
berbeda. Ini dapat dilakukan dengan plot kuantil-kuantil.

Untuk pengujian, kami mencoba distribusi student-t dan distribusi
eksponensial.
\end{eulercomment}
\begin{eulerprompt}
>x=randt(1,1000,5); y=randnormal(1,1000,mean(x),dev(x)); ...
>plot2d("x",r=6,style="--",yl="normal",xl="student-t",>vertical); ...
>plot2d(sort(x),sort(y),>points,color=red,style="x",>add):
\end{eulerprompt}
\eulerimg{27}{images/EMT4Statistika-Naela Rizqy Arofah-22305144042-026.png}
\begin{eulercomment}
Plot dengan jelas menunjukkan bahwa nilai terdistribusi normal
cenderung lebih kecil di ujung ekstrim.

Jika kita memiliki dua distribusi dengan ukuran yang berbeda, kita
dapat memperluas yang lebih kecil atau mengecilkan yang lebih besar.
Fungsi berikut baik untuk keduanya. Dibutuhkan nilai median dengan
persentase antara 0 dan 1.
\end{eulercomment}
\begin{eulerprompt}
>function medianexpand (x,n) := median(x,p=linspace(0,1,n-1));
\end{eulerprompt}
\begin{eulercomment}
Mari kita bandingkan dua distribusi yang sama.
\end{eulercomment}
\begin{eulerprompt}
>x=random(1000); y=random(400); ...
>plot2d("x",0,1,style="--"); ...
>plot2d(sort(medianexpand(x,400)),sort(y),>points,color=red,style="x",>add):
\end{eulerprompt}
\eulerimg{27}{images/EMT4Statistika-Naela Rizqy Arofah-22305144042-027.png}
\eulerheading{Regresi dan Korelasi}
\begin{eulercomment}
Regresi linier dapat dilakukan dengan fungsi polyfit() atau berbagai
fungsi fit.

Sebagai permulaan, kami menemukan garis regresi untuk data univariat
dengan polifit(x,y,1).
\end{eulercomment}
\begin{eulerprompt}
>x=1:10; y=[2,3,1,5,6,3,7,8,9,8]; writetable(x'|y',labc=["x","y"])
\end{eulerprompt}
\begin{euleroutput}
           x         y
           1         2
           2         3
           3         1
           4         5
           5         6
           6         3
           7         7
           8         8
           9         9
          10         8
\end{euleroutput}
\begin{eulercomment}
Kami ingin membandingkan non-weighted dan weighted fit. Pertama,
koefisien kecocokan linier.
\end{eulercomment}
\begin{eulerprompt}
>p=polyfit(x,y,1)
\end{eulerprompt}
\begin{euleroutput}
  [0.733333,  0.812121]
\end{euleroutput}
\begin{eulercomment}
Sekarang koefisien dengan bobot yang menekankan nilai terakhir.
\end{eulercomment}
\begin{eulerprompt}
>w &= "exp(-(x-10)^2/10)"; pw=polyfit(x,y,1,w=w(x))
\end{eulerprompt}
\begin{euleroutput}
  [4.71566,  0.38319]
\end{euleroutput}
\begin{eulercomment}
Kami memasukkan semuanya ke dalam satu plot untuk titik dan garis
regresi, dan untuk bobot yang digunakan.
\end{eulercomment}
\begin{eulerprompt}
>figure(2,1);  ...
>figure(1); statplot(x,y,"b",xl="Regression"); ...
>  plot2d("evalpoly(x,p)",>add,color=blue,style="--"); ...
>  plot2d("evalpoly(x,pw)",5,10,>add,color=red,style="--"); ...
>figure(2); plot2d(w,1,10,>filled,style="/",fillcolor=red,xl=w); ...
>figure(0):
\end{eulerprompt}
\eulerimg{27}{images/EMT4Statistika-Naela Rizqy Arofah-22305144042-028.png}
\begin{eulercomment}
Sebagai contoh lain kita membaca survei siswa, usia mereka, usia orang
tua mereka dan jumlah saudara kandung dari sebuah file.

Tabel ini berisi "m" dan "f" di kolom kedua. Kami menggunakan variabel
tok2 untuk mengatur terjemahan yang tepat daripada membiarkan
readtable() mengumpulkan terjemahan.
\end{eulercomment}
\begin{eulerprompt}
>\{MS,hd\}:=readtable("table1.dat",tok2:=["m","f"]);  ...
>writetable(MS,labc=hd,tok2:=["m","f"]);
\end{eulerprompt}
\begin{euleroutput}
      Person       Sex       Age    Mother    Father  Siblings
           1         m        29        58        61         1
           2         f        26        53        54         2
           3         m        24        49        55         1
           4         f        25        56        63         3
           5         f        25        49        53         0
           6         f        23        55        55         2
           7         m        23        48        54         2
           8         m        27        56        58         1
           9         m        25        57        59         1
          10         m        24        50        54         1
          11         f        26        61        65         1
          12         m        24        50        52         1
          13         m        29        54        56         1
          14         m        28        48        51         2
          15         f        23        52        52         1
          16         m        24        45        57         1
          17         f        24        59        63         0
          18         f        23        52        55         1
          19         m        24        54        61         2
          20         f        23        54        55         1
\end{euleroutput}
\begin{eulercomment}
Bagaimana usia bergantung satu sama lain? Kesan pertama datang dari
scatterplot berpasangan.
\end{eulercomment}
\begin{eulerprompt}
>scatterplots(tablecol(MS,3:5),hd[3:5]):
\end{eulerprompt}
\eulerimg{27}{images/EMT4Statistika-Naela Rizqy Arofah-22305144042-029.png}
\begin{eulercomment}
Jelas bahwa usia ayah dan ibu bergantung satu sama lain. Mari kita
tentukan dan plot garis regresinya.
\end{eulercomment}
\begin{eulerprompt}
>cs:=MS[,4:5]'; ps:=polyfit(cs[1],cs[2],1)
\end{eulerprompt}
\begin{euleroutput}
  [17.3789,  0.740964]
\end{euleroutput}
\begin{eulercomment}
Ini jelas model yang salah. Garis regresinya adalah s=17+0,74t, di
mana t adalah usia ibu dan s usia ayah. Perbedaan usia mungkin sedikit
bergantung pada usia, tetapi tidak terlalu banyak.

Sebaliknya, kami menduga fungsi seperti s=a+t. Maka a adalah mean dari
s-t. Ini adalah perbedaan usia rata-rata antara ayah dan ibu.
\end{eulercomment}
\begin{eulerprompt}
>da:=mean(cs[2]-cs[1])
\end{eulerprompt}
\begin{euleroutput}
  3.65
\end{euleroutput}
\begin{eulercomment}
Mari kita plot ini menjadi satu plot pencar.
\end{eulercomment}
\begin{eulerprompt}
>plot2d(cs[1],cs[2],>points);  ...
>plot2d("evalpoly(x,ps)",color=red,style=".",>add);  ...
>plot2d("x+da",color=blue,>add):
\end{eulerprompt}
\eulerimg{27}{images/EMT4Statistika-Naela Rizqy Arofah-22305144042-030.png}
\begin{eulercomment}
Berikut adalah plot kotak dari dua zaman. Ini hanya menunjukkan, bahwa
usianya berbeda.
\end{eulercomment}
\begin{eulerprompt}
>boxplot(cs,["mothers","fathers"]):
\end{eulerprompt}
\eulerimg{27}{images/EMT4Statistika-Naela Rizqy Arofah-22305144042-031.png}
\begin{eulercomment}
Sangat menarik bahwa perbedaan median tidak sebesar perbedaan
rata-rata.
\end{eulercomment}
\begin{eulerprompt}
>median(cs[2])-median(cs[1])
\end{eulerprompt}
\begin{euleroutput}
  1.5
\end{euleroutput}
\begin{eulercomment}
Koefisien korelasi menunjukkan korelasi positif.
\end{eulercomment}
\begin{eulerprompt}
>correl(cs[1],cs[2])
\end{eulerprompt}
\begin{euleroutput}
  0.7588307236
\end{euleroutput}
\begin{eulercomment}
Korelasi peringkat adalah ukuran untuk urutan yang sama di kedua
vektor. Ini juga cukup positif.
\end{eulercomment}
\begin{eulerprompt}
>rankcorrel(cs[1],cs[2])
\end{eulerprompt}
\begin{euleroutput}
  0.758925292358
\end{euleroutput}
\eulerheading{Membuat Fungsi baru}
\begin{eulercomment}
Tentu saja, bahasa EMT dapat digunakan untuk memprogram fungsi-fungsi
baru. Misalnya, kita mendefinisikan fungsi skewness.

\end{eulercomment}
\begin{eulerformula}
\[
\text{sk}(x) = \dfrac{\sqrt{n} \sum_i (x_i-m)^3}{\left(\sum_i (x_i-m)^2\right)^{3/2}}
\]
\end{eulerformula}
\begin{eulercomment}
dimana m adalah mean dari x.
\end{eulercomment}
\begin{eulerprompt}
>function skew (x:vector) ...
\end{eulerprompt}
\begin{eulerudf}
  m=mean(x);
  return sqrt(cols(x))*sum((x-m)^3)/(sum((x-m)^2))^(3/2);
  endfunction
\end{eulerudf}
\begin{eulercomment}
Seperti yang Anda lihat, kita dapat dengan mudah menggunakan bahasa
matriks untuk mendapatkan implementasi yang sangat singkat dan
efisien. Mari kita coba fungsi ini.
\end{eulercomment}
\begin{eulerprompt}
>data=normal(20); skew(normal(10))
\end{eulerprompt}
\begin{euleroutput}
  -0.198710316203
\end{euleroutput}
\begin{eulercomment}
Berikut adalah fungsi lain, yang disebut koefisien skewness Pearson.
\end{eulercomment}
\begin{eulerprompt}
>function skew1 (x) := 3*(mean(x)-median(x))/dev(x)
>skew1(data)
\end{eulerprompt}
\begin{euleroutput}
  -0.0801873249135
\end{euleroutput}
\eulerheading{Simulasi Monte Carlo}
\begin{eulercomment}
Euler dapat digunakan untuk mensimulasikan kejadian acak. Kita telah
melihat contoh sederhana di atas. Ini adalah satu lagi, yang
mensimulasikan 1000 kali 3 lemparan dadu, dan meminta distribusi
jumlah.
\end{eulercomment}
\begin{eulerprompt}
>ds:=sum(intrandom(1000,3,6))';  fs=getmultiplicities(3:18,ds)
\end{eulerprompt}
\begin{euleroutput}
  [5,  17,  35,  44,  75,  97,  114,  116,  143,  116,  104,  53,  40,
  22,  13,  6]
\end{euleroutput}
\begin{eulercomment}
kita bisa membuat plot ini sekarang
\end{eulercomment}
\begin{eulerprompt}
>columnsplot(fs,lab=3:18):
\end{eulerprompt}
\eulerimg{27}{images/EMT4Statistika-Naela Rizqy Arofah-22305144042-032.png}
\begin{eulercomment}
Untuk menentukan distribusi yang diharapkan tidak begitu mudah. Kami
menggunakan rekursi lanjutan untuk ini.

Fungsi berikut menghitung banyaknya cara bilangan k dapat
direpresentasikan sebagai jumlah n bilangan dalam rentang 1 sampai m.
Ia bekerja secara rekursif dengan cara yang jelas.
\end{eulercomment}
\begin{eulerprompt}
>function map countways (k; n, m) ...
\end{eulerprompt}
\begin{eulerudf}
    if n==1 then return k>=1 && k<=m
    else
      sum=0; 
      loop 1 to m; sum=sum+countways(k-#,n-1,m); end;
      return sum;
    end;
  endfunction
\end{eulerudf}
\begin{eulercomment}
Berikut adalah hasil dari tiga lemparan dadu.
\end{eulercomment}
\begin{eulerprompt}
>cw=countways(3:18,3,6)
\end{eulerprompt}
\begin{euleroutput}
  [1,  3,  6,  10,  15,  21,  25,  27,  27,  25,  21,  15,  10,  6,  3,
  1]
\end{euleroutput}
\begin{eulercomment}
Kami menambahkan nilai yang diharapkan ke plot.
\end{eulercomment}
\begin{eulerprompt}
>plot2d(cw/6^3*1000,>add); plot2d(cw/6^3*1000,>points,>add):
\end{eulerprompt}
\eulerimg{27}{images/EMT4Statistika-Naela Rizqy Arofah-22305144042-033.png}
\begin{eulercomment}
Untuk simulasi lain, simpangan nilai rata-rata dari n 0-1-variabel
acak terdistribusi normal adalah 1/sqrt(n).
\end{eulercomment}
\begin{eulerprompt}
>longformat; 1/sqrt(10)
\end{eulerprompt}
\begin{euleroutput}
  0.316227766017
\end{euleroutput}
\begin{eulercomment}
Mari kita periksa ini dengan simulasi. Kami memproduksi 10000 kali 10
vektor acak.
\end{eulercomment}
\begin{eulerprompt}
>M=normal(10000,10); dev(mean(M)')
\end{eulerprompt}
\begin{euleroutput}
  0.319493614817
\end{euleroutput}
\begin{eulerprompt}
>plot2d(mean(M)',>distribution):
\end{eulerprompt}
\eulerimg{27}{images/EMT4Statistika-Naela Rizqy Arofah-22305144042-034.png}
\begin{eulercomment}
Median 10 0-1-bilangan acak terdistribusi normal memiliki simpangan
yang lebih besar.
\end{eulercomment}
\begin{eulerprompt}
>dev(median(M)')
\end{eulerprompt}
\begin{euleroutput}
  0.374460271535
\end{euleroutput}
\begin{eulercomment}
Karena kita dapat dengan mudah menghasilkan jalan acak, kita dapat
mensimulasikan proses Wiener. Kami mengambil 1000 langkah dari 1000
proses. Kami kemudian memplot deviasi standar dan rata-rata dari
langkah ke-n dari proses ini bersama dengan nilai yang diharapkan
dalam warna merah.
\end{eulercomment}
\begin{eulerprompt}
>n=1000; m=1000; M=cumsum(normal(n,m)/sqrt(m)); ...
>t=(1:n)/n; figure(2,1); ...
>figure(1); plot2d(t,mean(M')'); plot2d(t,0,color=red,>add); ...
>figure(2); plot2d(t,dev(M')'); plot2d(t,sqrt(t),color=red,>add); ...
>figure(0):
\end{eulerprompt}
\eulerimg{27}{images/EMT4Statistika-Naela Rizqy Arofah-22305144042-035.png}
\eulerheading{Uji}
\begin{eulercomment}
Uji adalah alat penting dalam statistik. Di Euler, banyak tes
diimplementasikan. Semua tes ini mengembalikan kesalahan yang kami
terima jika kami menolak hipotesis nol.

Sebagai contoh, kami menguji lemparan dadu untuk distribusi seragam.
Pada 600 lemparan, kami mendapatkan nilai berikut, yang kami masukkan
ke dalam uji chi-kuadrat.
\end{eulercomment}
\begin{eulerprompt}
>chitest([90,103,114,101,103,89],dup(100,6)')
\end{eulerprompt}
\begin{euleroutput}
  0.498830517952
\end{euleroutput}
\begin{eulercomment}
Tes chi-kuadrat juga memiliki mode, yang menggunakan simulasi Monte
Carlo untuk menguji statistik. Hasilnya harus hampir sama. Parameter
\textgreater{}p menginterpretasikan vektor-y sebagai vektor probabilitas.
\end{eulercomment}
\begin{eulerprompt}
>chitest([90,103,114,101,103,89],dup(1/6,6)',>p,>montecarlo)
\end{eulerprompt}
\begin{euleroutput}
  0.526
\end{euleroutput}
\begin{eulercomment}
Kesalahan ini terlalu besar. Jadi kita tidak bisa menolak distribusi
seragam. Ini tidak membuktikan bahwa dadu kami adil. Tapi kita tidak
bisa menolak hipotesis kita.

Selanjutnya kita menghasilkan 1000 lemparan dadu menggunakan generator
angka acak, dan melakukan tes yang sama.
\end{eulercomment}
\begin{eulerprompt}
>n=1000; t=random([1,n*6]); chitest(count(t*6,6),dup(n,6)')
\end{eulerprompt}
\begin{euleroutput}
  0.528028118442
\end{euleroutput}
\begin{eulercomment}
Mari kita uji nilai rata-rata 100 dengan uji-t.
\end{eulercomment}
\begin{eulerprompt}
>s=200+normal([1,100])*10; ...
>ttest(mean(s),dev(s),100,200)
\end{eulerprompt}
\begin{euleroutput}
  0.0218365848476
\end{euleroutput}
\begin{eulercomment}
Fungsi ttest() membutuhkan nilai rata-rata, simpangan, jumlah data,
dan nilai rata-rata yang akan diuji.

Sekarang mari kita periksa dua pengukuran untuk mean yang sama. Kami
menolak hipotesis bahwa mereka memiliki rata-rata yang sama, jika
hasilnya \textless{}0,05.
\end{eulercomment}
\begin{eulerprompt}
>tcomparedata(normal(1,10),normal(1,10))
\end{eulerprompt}
\begin{euleroutput}
  0.38722000942
\end{euleroutput}
\begin{eulercomment}
Jika kita menambahkan bias ke satu distribusi, kita mendapatkan lebih
banyak penolakan. Ulangi simulasi ini beberapa kali untuk melihat
efeknya.
\end{eulercomment}
\begin{eulerprompt}
>tcomparedata(normal(1,10),normal(1,10)+2)
\end{eulerprompt}
\begin{euleroutput}
  5.60009101758e-07
\end{euleroutput}
\begin{eulercomment}
Pada contoh berikutnya, kita menghasilkan 20 lemparan dadu acak
sebanyak 100 kali dan menghitung yang ada di dalamnya. Harus ada
20/6=3,3 yang rata-rata.
\end{eulercomment}
\begin{eulerprompt}
>R=random(100,20); R=sum(R*6<=1)'; mean(R)
\end{eulerprompt}
\begin{euleroutput}
  3.28
\end{euleroutput}
\begin{eulercomment}
Kami sekarang membandingkan jumlah satu dengan distribusi binomial.
Pertama kita plot distribusi yang.
\end{eulercomment}
\begin{eulerprompt}
>plot2d(R,distribution=max(R)+1,even=1,style="\(\backslash\)/"):
\end{eulerprompt}
\eulerimg{27}{images/EMT4Statistika-Naela Rizqy Arofah-22305144042-036.png}
\begin{eulerprompt}
>t=count(R,21);
\end{eulerprompt}
\begin{eulercomment}
Kemudian kami menghitung nilai yang diharapkan.
\end{eulercomment}
\begin{eulerprompt}
>n=0:20; b=bin(20,n)*(1/6)^n*(5/6)^(20-n)*100;
\end{eulerprompt}
\begin{eulercomment}
Kita harus mengumpulkan beberapa angka untuk mendapatkan kategori yang
cukup besar.
\end{eulercomment}
\begin{eulerprompt}
>t1=sum(t[1:2])|t[3:7]|sum(t[8:21]); ...
>b1=sum(b[1:2])|b[3:7]|sum(b[8:21]);
\end{eulerprompt}
\begin{eulercomment}
Uji chi-kuadrat menolak hipotesis bahwa distribusi kami adalah
distribusi binomial, jika hasilnya \textless{}0,05.
\end{eulercomment}
\begin{eulerprompt}
>chitest(t1,b1)
\end{eulerprompt}
\begin{euleroutput}
  0.53921579764
\end{euleroutput}
\begin{eulercomment}
Contoh berikut berisi hasil dua kelompok orang (laki-laki dan
perempuan, katakanlah) memberikan suara untuk satu dari enam partai.
\end{eulercomment}
\begin{eulerprompt}
>A=[23,37,43,52,64,74;27,39,41,49,63,76];  ...
>  writetable(A,wc=6,labr=["m","f"],labc=1:6)
\end{eulerprompt}
\begin{euleroutput}
             1     2     3     4     5     6
       m    23    37    43    52    64    74
       f    27    39    41    49    63    76
\end{euleroutput}
\begin{eulercomment}
Kami ingin menguji independensi suara dari jenis kelamin. Tes tabel
chi\textasciicircum{}2 melakukan ini. Akibatnya terlalu besar untuk menolak
kemerdekaan. Jadi kita tidak bisa mengatakan, jika voting tergantung
pada jenis kelamin dari data ini.
\end{eulercomment}
\begin{eulerprompt}
>tabletest(A)
\end{eulerprompt}
\begin{euleroutput}
  0.990701632326
\end{euleroutput}
\begin{eulercomment}
Berikut ini adalah tabel yang diharapkan, jika kita mengasumsikan
frekuensi pemungutan suara yang diamati.
\end{eulercomment}
\begin{eulerprompt}
>writetable(expectedtable(A),wc=6,dc=1,labr=["m","f"],labc=1:6)
\end{eulerprompt}
\begin{euleroutput}
             1     2     3     4     5     6
       m  24.9  37.9  41.9  50.3  63.3  74.7
       f  25.1  38.1  42.1  50.7  63.7  75.3
\end{euleroutput}
\begin{eulercomment}
Kita dapat menghitung koefisien kontingensi yang dikoreksi. Karena
sangat dekat dengan 0, kami menyimpulkan bahwa pemungutan suara tidak
tergantung pada jenis kelamin.
\end{eulercomment}
\begin{eulerprompt}
>contingency(A)
\end{eulerprompt}
\begin{euleroutput}
  0.0427225484717
\end{euleroutput}
\eulerheading{Uji Lainnya}
\begin{eulercomment}
Selanjutnya kami menggunakan analisis varians (Uji-F) untuk menguji
tiga sampel data yang terdistribusi normal untuk nilai rata-rata yang
sama. Metode tersebut disebut ANOVA (analisis varians). Di Euler,
fungsi varanalysis() digunakan.
\end{eulercomment}
\begin{eulerprompt}
>x1=[109,111,98,119,91,118,109,99,115,109,94]; mean(x1),
\end{eulerprompt}
\begin{euleroutput}
  106.545454545
\end{euleroutput}
\begin{eulerprompt}
>x2=[120,124,115,139,114,110,113,120,117]; mean(x2),
\end{eulerprompt}
\begin{euleroutput}
  119.111111111
\end{euleroutput}
\begin{eulerprompt}
>x3=[120,112,115,110,105,134,105,130,121,111]; mean(x3)
\end{eulerprompt}
\begin{euleroutput}
  116.3
\end{euleroutput}
\begin{eulerprompt}
>varanalysis(x1,x2,x3)
\end{eulerprompt}
\begin{euleroutput}
  0.0138048221371
\end{euleroutput}
\begin{eulercomment}
Ini berarti, kami menolak hipotesis nilai rata-rata yang sama. Kami
melakukan ini dengan probabilitas kesalahan 1,3\%.

Ada juga uji median, yang menolak sampel data dengan distribusi
rata-rata berbeda menguji median sampel bersatu.
\end{eulercomment}
\begin{eulerprompt}
>a=[56,66,68,49,61,53,45,58,54];
>b=[72,81,51,73,69,78,59,67,65,71,68,71];
>mediantest(a,b)
\end{eulerprompt}
\begin{euleroutput}
  0.0241724220052
\end{euleroutput}
\begin{eulercomment}
Tes lain tentang kesetaraan adalah tes peringkat. Ini jauh lebih tajam
daripada tes median.
\end{eulercomment}
\begin{eulerprompt}
>ranktest(a,b)
\end{eulerprompt}
\begin{euleroutput}
  0.00199969612469
\end{euleroutput}
\begin{eulercomment}
Dalam contoh berikut, kedua distribusi memiliki mean yang sama.
\end{eulercomment}
\begin{eulerprompt}
>ranktest(random(1,100),random(1,50)*3-1)
\end{eulerprompt}
\begin{euleroutput}
  0.129608141484
\end{euleroutput}
\begin{eulercomment}
Sekarang mari kita coba mensimulasikan dua perlakuan a dan b yang
diterapkan pada orang yang berbeda.
\end{eulercomment}
\begin{eulerprompt}
>a=[8.0,7.4,5.9,9.4,8.6,8.2,7.6,8.1,6.2,8.9];
>b=[6.8,7.1,6.8,8.3,7.9,7.2,7.4,6.8,6.8,8.1];
\end{eulerprompt}
\begin{eulercomment}
Tes signum memutuskan, jika a lebih baik dari b.
\end{eulercomment}
\begin{eulerprompt}
>signtest(a,b)
\end{eulerprompt}
\begin{euleroutput}
  0.0546875
\end{euleroutput}
\begin{eulercomment}
Ini terlalu banyak kesalahan. Kita tidak dapat menolak bahwa a sama
baiknya dengan b.

Tes Wilcoxon lebih tajam dari tes ini, tetapi bergantung pada nilai
kuantitatif perbedaan.
\end{eulercomment}
\begin{eulerprompt}
>wilcoxon(a,b)
\end{eulerprompt}
\begin{euleroutput}
  0.0296680599405
\end{euleroutput}
\begin{eulercomment}
Mari kita coba dua tes lagi menggunakan seri yang dihasilkan.
\end{eulercomment}
\begin{eulerprompt}
>wilcoxon(normal(1,20),normal(1,20)-1)
\end{eulerprompt}
\begin{euleroutput}
  0.0068706451766
\end{euleroutput}
\begin{eulerprompt}
>wilcoxon(normal(1,20),normal(1,20))
\end{eulerprompt}
\begin{euleroutput}
  0.275145971064
\end{euleroutput}
\eulerheading{Nomor Acak}
\begin{eulercomment}
Berikut ini adalah pengujian untuk pembangkit bilangan acak. Euler
menggunakan generator yang sangat bagus, jadi kita tidak perlu
mengharapkan masalah.

Pertama kita menghasilkan sepuluh juta angka acak di [0,1].
\end{eulercomment}
\begin{eulerprompt}
>n:=10000000; r:=random(1,n);
\end{eulerprompt}
\begin{eulercomment}
Selanjutnya kita hitung jarak antara dua bilangan kurang dari 0,05.
\end{eulercomment}
\begin{eulerprompt}
>a:=0.05; d:=differences(nonzeros(r<a));
\end{eulerprompt}
\begin{eulercomment}
Akhirnya, kami memplot berapa kali, setiap jarak terjadi, dan
membandingkan dengan nilai yang diharapkan.
\end{eulercomment}
\begin{eulerprompt}
>m=getmultiplicities(1:100,d); plot2d(m); ...
>  plot2d("n*(1-a)^(x-1)*a^2",color=red,>add):
\end{eulerprompt}
\eulerimg{27}{images/EMT4Statistika-Naela Rizqy Arofah-22305144042-037.png}
\begin{eulercomment}
Hapus datanya.
\end{eulercomment}
\begin{eulerprompt}
>remvalue n;
\end{eulerprompt}
\eulerheading{Pengantar untuk Pengguna Proyek R}
\begin{eulercomment}
Jelas, EMT tidak bersaing dengan R sebagai paket statistik. Namun, ada
banyak prosedur dan fungsi statistik yang tersedia di EMT juga. Jadi
EMT dapat memenuhi kebutuhan dasar. Bagaimanapun, EMT hadir dengan
paket numerik dan sistem aljabar komputer.

Notebook ini cocok untuk Anda yang terbiasa dengan R, tetapi perlu
mengetahui perbedaan sintaks EMT dan R. Kami mencoba memberikan
gambaran tentang hal-hal yang jelas dan kurang jelas yang perlu Anda
ketahui.

Selain itu, kami mencari cara untuk bertukar data antara kedua sistem.
\end{eulercomment}
\begin{eulercomment}
Perhatikan bahwa ini adalah pekerjaan yang sedang berjalan.
\end{eulercomment}
\eulerheading{Sintaks Dasar}
\begin{eulercomment}
Hal pertama yang Anda pelajari di R adalah membuat vektor. Di EMT,
perbedaan utama adalah bahwa : operator dapat mengambil ukuran
langkah. Selain itu, ia memiliki daya ikat yang rendah.
\end{eulercomment}
\begin{eulerprompt}
>n=10; 0:n/20:n-1
\end{eulerprompt}
\begin{euleroutput}
  [0,  0.5,  1,  1.5,  2,  2.5,  3,  3.5,  4,  4.5,  5,  5.5,  6,  6.5,
  7,  7.5,  8,  8.5,  9]
\end{euleroutput}
\begin{eulercomment}
Fungsi c() tidak ada. Dimungkinkan untuk menggunakan vektor untuk
menggabungkan sesuatu.

Contoh berikut, seperti banyak contoh lainnya, dari "Interoduction to
R" yang disertakan dengan proyek R. Jika Anda membaca PDF ini, Anda
akan menemukan bahwa saya mengikuti jalannya dalam tutorial ini.
\end{eulercomment}
\begin{eulerprompt}
>x=[10.4, 5.6, 3.1, 6.4, 21.7]; [x,0,x]
\end{eulerprompt}
\begin{euleroutput}
  [10.4,  5.6,  3.1,  6.4,  21.7,  0,  10.4,  5.6,  3.1,  6.4,  21.7]
\end{euleroutput}
\begin{eulercomment}
Operator titik dua dengan ukuran langkah EMT diganti dengan fungsi
seq() di R. Kita bisa menulis fungsi ini di EMT.
\end{eulercomment}
\begin{eulerprompt}
>function seq(a,b,c) := a:b:c; ...
>seq(0,-0.1,-1)
\end{eulerprompt}
\begin{euleroutput}
  [0,  -0.1,  -0.2,  -0.3,  -0.4,  -0.5,  -0.6,  -0.7,  -0.8,  -0.9,  -1]
\end{euleroutput}
\begin{eulercomment}
Fungsi rep() dari R tidak ada di EMT. Untuk input vektor, dapat
ditulis sebagai berikut.
\end{eulercomment}
\begin{eulerprompt}
>function rep(x:vector,n:index) := flatten(dup(x,n)); ...
>rep(x,2)
\end{eulerprompt}
\begin{euleroutput}
  [10.4,  5.6,  3.1,  6.4,  21.7,  10.4,  5.6,  3.1,  6.4,  21.7]
\end{euleroutput}
\begin{eulercomment}
Perhatikan bahwa "=" atau ":=" digunakan untuk tugas. Operator "-\textgreater{}"
digunakan untuk unit di EMT.
\end{eulercomment}
\begin{eulerprompt}
>125km -> " miles"
\end{eulerprompt}
\begin{euleroutput}
  77.6713990297 miles
\end{euleroutput}
\begin{eulercomment}
Operator "\textless{}-" untuk penugasan tetap menyesatkan, dan bukan ide yang
baik untuk R. Berikut ini akan membandingkan a dan -4 di EMT.
\end{eulercomment}
\begin{eulerprompt}
>a=2; a<-4
\end{eulerprompt}
\begin{euleroutput}
  0
\end{euleroutput}
\begin{eulercomment}
Di R, "a\textless{}-4\textless{}3" berfungsi, tetapi "a\textless{}-4\textless{}-3" tidak. Saya juga memiliki
ambiguitas serupa di EMT, tetapi mencoba menghilangkannya
perlahan-lahan.

EMT dan R memiliki vektor bertipe boolean. Namun di EMT, angka 0 dan 1
digunakan untuk mewakili salah dan benar. Di R, nilai true dan false
dapat digunakan dalam aritmatika biasa seperti di EMT.
\end{eulercomment}
\begin{eulerprompt}
>x<5, %*x
\end{eulerprompt}
\begin{euleroutput}
  [0,  0,  1,  0,  0]
  [0,  0,  3.1,  0,  0]
\end{euleroutput}
\begin{eulercomment}
EMT melempar kesalahan atau menghasilkan NAN tergantung pada tanda
"kesalahan".
\end{eulercomment}
\begin{eulerprompt}
>errors off; 0/0, isNAN(sqrt(-1)), errors on;
\end{eulerprompt}
\begin{euleroutput}
  NAN
  1
\end{euleroutput}
\begin{eulercomment}
String sama di R dan EMT. Keduanya berada di lokal saat ini, bukan di
Unicode.

Di R ada paket untuk Unicode. Di EMT, sebuah string dapat berupa
string Unicode. String unicode dapat diterjemahkan ke pengkodean lokal
dan sebaliknya. Selain itu, u"..." dapat berisi entitas HTML.
\end{eulercomment}
\begin{eulerprompt}
>u"&#169; Ren&eacut; Grothmann"
\end{eulerprompt}
\begin{euleroutput}
  © René Grothmann
\end{euleroutput}
\begin{eulercomment}
Berikut ini mungkin atau mungkin tidak ditampilkan dengan benar di
sistem Anda sebagai A dengan titik dan garis di atasnya. Itu
tergantung pada font yang Anda gunakan.
\end{eulercomment}
\begin{eulerprompt}
>chartoutf([480])
\end{eulerprompt}
\begin{euleroutput}
  Ǡ
\end{euleroutput}
\begin{eulercomment}
Penggabungan string dilakukan dengan "+" atau "\textbar{}". Ini dapat mencakup
angka, yang akan dicetak dalam format saat ini.
\end{eulercomment}
\begin{eulerprompt}
>"pi = "+pi
\end{eulerprompt}
\begin{euleroutput}
  pi = 3.14159265359
\end{euleroutput}
\eulerheading{Pengindeksan}
\begin{eulercomment}
Sebagian besar waktu, ini akan berfungsi seperti pada R.

Tetapi EMT akan menginterpretasikan indeks negatif dari belakang
vektor, sedangkan R menginterpretasikan x[n] sebagai x tanpa elemen
ke-n.
\end{eulercomment}
\begin{eulerprompt}
>x, x[1:3], x[-2]
\end{eulerprompt}
\begin{euleroutput}
  [10.4,  5.6,  3.1,  6.4,  21.7]
  [10.4,  5.6,  3.1]
  6.4
\end{euleroutput}
\begin{eulercomment}
Perilaku R dapat dicapai dalam EMT dengan drop().
\end{eulercomment}
\begin{eulerprompt}
>drop(x,2)
\end{eulerprompt}
\begin{euleroutput}
  [10.4,  3.1,  6.4,  21.7]
\end{euleroutput}
\begin{eulercomment}
Vektor logis tidak diperlakukan secara berbeda sebagai indeks di EMT,
berbeda dengan R. Anda perlu mengekstrak elemen bukan nol terlebih
dahulu di EMT.
\end{eulercomment}
\begin{eulerprompt}
>x, x>5, x[nonzeros(x>5)]
\end{eulerprompt}
\begin{euleroutput}
  [10.4,  5.6,  3.1,  6.4,  21.7]
  [1,  1,  0,  1,  1]
  [10.4,  5.6,  6.4,  21.7]
\end{euleroutput}
\begin{eulercomment}
Sama seperti di R, vektor indeks dapat berisi pengulangan.
\end{eulercomment}
\begin{eulerprompt}
>x[[1,2,2,1]]
\end{eulerprompt}
\begin{euleroutput}
  [10.4,  5.6,  5.6,  10.4]
\end{euleroutput}
\begin{eulercomment}
Tetapi nama untuk indeks tidak dimungkinkan di EMT. Untuk paket
statistik, ini mungkin sering diperlukan untuk memudahkan akses ke
elemen vektor.

Untuk meniru perilaku ini, kita dapat mendefinisikan fungsi sebagai
berikut.
\end{eulercomment}
\begin{eulerprompt}
>function sel (v,i,s) := v[indexof(s,i)]; ...
>s=["first","second","third","fourth"]; sel(x,["first","third"],s)
\end{eulerprompt}
\begin{euleroutput}
  
  Trying to overwrite protected function sel!
  Error in:
  function sel (v,i,s) := v[indexof(s,i)]; ... ...
               ^
  
  Trying to overwrite protected function sel!
  Error in:
  function sel (v,i,s) := v[indexof(s,i)]; ... ...
               ^
  [10.4,  3.1]
  
\end{euleroutput}
\eulerheading{Tipe Data}
\begin{eulercomment}
EMT memiliki lebih banyak tipe data tetap daripada R. Jelas, di R ada
vektor yang tumbuh. Anda dapat mengatur vektor numerik kosong v dan
menetapkan nilai ke elemen v[17]. Ini tidak mungkin di EMT.

Berikut ini agak tidak efisien.
\end{eulercomment}
\begin{eulerprompt}
>v=[]; for i=1 to 10000; v=v|i; end;
\end{eulerprompt}
\begin{eulercomment}
EMT sekarang akan membuat vektor dengan v dan i ditambahkan pada
tumpukan dan menyalin vektor itu kembali ke variabel global v.

Semakin efisien pra-mendefinisikan vektor.
\end{eulercomment}
\begin{eulerprompt}
>v=zeros(10000); for i=1 to 10000; v[i]=i; end;
\end{eulerprompt}
\begin{eulercomment}
Untuk mengubah jenis tanggal di EMT, Anda dapat menggunakan fungsi
seperti complex().
\end{eulercomment}
\begin{eulerprompt}
>complex(1:4)
\end{eulerprompt}
\begin{euleroutput}
  [ 1+0i ,  2+0i ,  3+0i ,  4+0i  ]
\end{euleroutput}
\begin{eulercomment}
Konversi ke string hanya dimungkinkan untuk tipe data dasar. Format
saat ini digunakan untuk rangkaian string sederhana. Tetapi ada fungsi
seperti print() atau frac().

Untuk vektor, Anda dapat dengan mudah menulis fungsi Anda sendiri.
\end{eulercomment}
\begin{eulerprompt}
>function tostr (v) ...
\end{eulerprompt}
\begin{eulerudf}
  s="[";
  loop 1 to length(v);
     s=s+print(v[#],2,0);
     if #<length(v) then s=s+","; endif;
  end;
  return s+"]";
  endfunction
\end{eulerudf}
\begin{eulerprompt}
>tostr(linspace(0,1,10))
\end{eulerprompt}
\begin{euleroutput}
  [0.00,0.10,0.20,0.30,0.40,0.50,0.60,0.70,0.80,0.90,1.00]
\end{euleroutput}
\begin{eulercomment}
Untuk komunikasi dengan Maxima, terdapat fungsi convertmxm(), yang
juga dapat digunakan untuk memformat vektor untuk output.
\end{eulercomment}
\begin{eulerprompt}
>convertmxm(1:10)
\end{eulerprompt}
\begin{euleroutput}
  [1,2,3,4,5,6,7,8,9,10]
\end{euleroutput}
\begin{eulercomment}
Untuk Latex perintah tex dapat digunakan untuk mendapatkan perintah
Latex.
\end{eulercomment}
\begin{eulerprompt}
>tex(&[1,2,3])
\end{eulerprompt}
\begin{euleroutput}
  \(\backslash\)left[ 1 , 2 , 3 \(\backslash\)right] 
\end{euleroutput}
\eulerheading{Faktor dan Tabel}
\begin{eulercomment}
Dalam pengantar R ada contoh dengan apa yang disebut faktor.

Berikut ini adalah daftar wilayah dari 30 negara bagian.
\end{eulercomment}
\begin{eulerprompt}
>austates = ["tas", "sa", "qld", "nsw", "nsw", "nt", "wa", "wa", ...
>"qld", "vic", "nsw", "vic", "qld", "qld", "sa", "tas", ...
>"sa", "nt", "wa", "vic", "qld", "nsw", "nsw", "wa", ...
>"sa", "act", "nsw", "vic", "vic", "act"];
\end{eulerprompt}
\begin{eulercomment}
Asumsikan, kita memiliki pendapatan yang sesuai di setiap negara
bagian.
\end{eulercomment}
\begin{eulerprompt}
>incomes = [60, 49, 40, 61, 64, 60, 59, 54, 62, 69, 70, 42, 56, ...
>61, 61, 61, 58, 51, 48, 65, 49, 49, 41, 48, 52, 46, ...
>59, 46, 58, 43];
\end{eulerprompt}
\begin{eulercomment}
Sekarang, kami ingin menghitung rata-rata pendapatan di wilayah
tersebut. Menjadi program statistik, R memiliki factor() dan tappy()
untuk ini.

EMT dapat melakukannya dengan menemukan indeks wilayah dalam daftar
wilayah yang unik.
\end{eulercomment}
\begin{eulerprompt}
>auterr=sort(unique(austates)); f=indexofsorted(auterr,austates)
\end{eulerprompt}
\begin{euleroutput}
  [6,  5,  4,  2,  2,  3,  8,  8,  4,  7,  2,  7,  4,  4,  5,  6,  5,  3,
  8,  7,  4,  2,  2,  8,  5,  1,  2,  7,  7,  1]
\end{euleroutput}
\begin{eulercomment}
Pada titik itu, kita dapat menulis fungsi loop kita sendiri untuk
melakukan sesuatu hanya untuk satu faktor.

Atau kita bisa meniru fungsi tapply() dengan cara berikut.
\end{eulercomment}
\begin{eulerprompt}
>function map tappl (i; f$:call, cat, x) ...
\end{eulerprompt}
\begin{eulerudf}
  u=sort(unique(cat));
  f=indexof(u,cat);
  return f$(x[nonzeros(f==indexof(u,i))]);
  endfunction
\end{eulerudf}
\begin{eulercomment}
Ini agak tidak efisien, karena menghitung wilayah unik untuk setiap i,
tetapi berhasil.
\end{eulercomment}
\begin{eulerprompt}
>tappl(auterr,"mean",austates,incomes)
\end{eulerprompt}
\begin{euleroutput}
  [44.5,  57.3333333333,  55.5,  53.6,  55,  60.5,  56,  52.25]
\end{euleroutput}
\begin{eulercomment}
Perhatikan bahwa ini berfungsi untuk setiap vektor wilayah.
\end{eulercomment}
\begin{eulerprompt}
>tappl(["act","nsw"],"mean",austates,incomes)
\end{eulerprompt}
\begin{euleroutput}
  [44.5,  57.3333333333]
\end{euleroutput}
\begin{eulercomment}
Sekarang, paket statistik EMT mendefinisikan tabel seperti di R.
Fungsi readtable() dan writetable() dapat digunakan untuk input dan
output.

Jadi kita bisa mencetak rata-rata pendapatan negara di wilayah dengan
cara yang bersahabat.
\end{eulercomment}
\begin{eulerprompt}
>writetable(tappl(auterr,"mean",austates,incomes),labc=auterr,wc=7)
\end{eulerprompt}
\begin{euleroutput}
      act    nsw     nt    qld     sa    tas    vic     wa
     44.5  57.33   55.5   53.6     55   60.5     56  52.25
\end{euleroutput}
\begin{eulercomment}
Kita juga dapat mencoba meniru perilaku R sepenuhnya.

Faktor-faktor tersebut harus dengan jelas disimpan dalam kumpulan
dengan jenis dan kategori (negara bagian dan teritori dalam contoh
kami). Untuk EMT, kami menambahkan indeks yang telah dihitung
sebelumnya.
\end{eulercomment}
\begin{eulerprompt}
>function makef (t) ...
\end{eulerprompt}
\begin{eulerudf}
  ## Factor data
  ## Returns a collection with data t, unique data, indices.
  ## See: tapply
  u=sort(unique(t));
  return \{\{t,u,indexofsorted(u,t)\}\};
  endfunction
\end{eulerudf}
\begin{eulerprompt}
>statef=makef(austates);
\end{eulerprompt}
\begin{eulercomment}
Sekarang elemen ketiga dari koleksi akan berisi indeks.
\end{eulercomment}
\begin{eulerprompt}
>statef[3]
\end{eulerprompt}
\begin{euleroutput}
  [6,  5,  4,  2,  2,  3,  8,  8,  4,  7,  2,  7,  4,  4,  5,  6,  5,  3,
  8,  7,  4,  2,  2,  8,  5,  1,  2,  7,  7,  1]
\end{euleroutput}
\begin{eulercomment}
Sekarang kita bisa meniru tapply() dengan cara berikut. Ini akan
mengembalikan tabel sebagai kumpulan data tabel dan judul kolom.
\end{eulercomment}
\begin{eulerprompt}
>function tapply (t:vector,tf,f$:call) ...
\end{eulerprompt}
\begin{eulerudf}
  ## Makes a table of data and factors
  ## tf : output of makef()
  ## See: makef
  uf=tf[2]; f=tf[3]; x=zeros(length(uf));
  for i=1 to length(uf);
     ind=nonzeros(f==i);
     if length(ind)==0 then x[i]=NAN;
     else x[i]=f$(t[ind]);
     endif;
  end;
  return \{\{x,uf\}\};
  endfunction
\end{eulerudf}
\begin{eulercomment}
Kami tidak menambahkan banyak jenis pengecekan di sini. Satu-satunya
tindakan pencegahan menyangkut kategori (faktor) tanpa data. Tetapi
orang harus memeriksa panjang t yang benar dan kebenaran koleksi tf.

Tabel ini dapat dicetak sebagai tabel dengan writetable().
\end{eulercomment}
\begin{eulerprompt}
>writetable(tapply(incomes,statef,"mean"),wc=7)
\end{eulerprompt}
\begin{euleroutput}
      act    nsw     nt    qld     sa    tas    vic     wa
     44.5  57.33   55.5   53.6     55   60.5     56  52.25
\end{euleroutput}
\eulerheading{Array}
\begin{eulercomment}
EMT hanya memiliki dua dimensi untuk array. Tipe datanya disebut
matriks. Akan mudah untuk menulis fungsi untuk dimensi yang lebih
tinggi atau pustaka C untuk ini.

R memiliki lebih dari dua dimensi. Dalam R array adalah vektor dengan
bidang dimensi.

Dalam EMT, vektor adalah matriks dengan satu baris. Itu dapat dibuat
menjadi matriks dengan redim().
\end{eulercomment}
\begin{eulerprompt}
>shortformat; X=redim(1:20,4,5)
\end{eulerprompt}
\begin{euleroutput}
          1         2         3         4         5 
          6         7         8         9        10 
         11        12        13        14        15 
         16        17        18        19        20 
\end{euleroutput}
\begin{eulercomment}
Ekstraksi baris dan kolom, atau sub-matriks, sangat mirip dengan R.
\end{eulercomment}
\begin{eulerprompt}
>X[,2:3]
\end{eulerprompt}
\begin{euleroutput}
          2         3 
          7         8 
         12        13 
         17        18 
\end{euleroutput}
\begin{eulercomment}
Namun, dalam R dimungkinkan untuk menetapkan daftar indeks spesifik
dari vektor ke suatu nilai. Hal yang sama dimungkinkan di EMT hanya
dengan loop.
\end{eulercomment}
\begin{eulerprompt}
>function setmatrixvalue (M, i, j, v) ...
\end{eulerprompt}
\begin{eulerudf}
  loop 1 to max(length(i),length(j),length(v))
     M[i\{#\},j\{#\}] = v\{#\};
  end;
  endfunction
\end{eulerudf}
\begin{eulercomment}
Kami mendemonstrasikan ini untuk menunjukkan bahwa matriks dilewatkan
dengan referensi di EMT. Jika Anda tidak ingin mengubah matriks asli
M, Anda perlu menyalinnya ke dalam fungsi.
\end{eulercomment}
\begin{eulerprompt}
>setmatrixvalue(X,1:3,3:-1:1,0); X,
\end{eulerprompt}
\begin{euleroutput}
          1         2         0         4         5 
          6         0         8         9        10 
          0        12        13        14        15 
         16        17        18        19        20 
\end{euleroutput}
\begin{eulercomment}
Perkalian luar dalam EMT hanya dapat dilakukan antar vektor. Ini
otomatis karena bahasa matriks. Satu vektor harus menjadi vektor kolom
dan yang lainnya vektor baris.
\end{eulercomment}
\begin{eulerprompt}
>(1:5)*(1:5)'
\end{eulerprompt}
\begin{euleroutput}
          1         2         3         4         5 
          2         4         6         8        10 
          3         6         9        12        15 
          4         8        12        16        20 
          5        10        15        20        25 
\end{euleroutput}
\begin{eulercomment}
Dalam pengantar PDF untuk R ada sebuah contoh, yang menghitung
distribusi ab-cd untuk a,b,c,d yang dipilih dari 0 hingga n secara
acak. Solusi dalam R adalah membentuk matriks 4 dimensi dan
menjalankan table() di atasnya.

Tentu saja, ini dapat dicapai dengan loop. Tapi loop tidak efektif di
EMT atau R. Di EMT, kita bisa menulis loop di C dan itu akan menjadi
solusi tercepat.

Tapi kita ingin meniru perilaku R. Untuk ini, kita perlu meratakan
perkalian ab dan membuat matriks ab-cd.
\end{eulercomment}
\begin{eulerprompt}
>a=0:6; b=a'; p=flatten(a*b); q=flatten(p-p'); ...
>u=sort(unique(q)); f=getmultiplicities(u,q); ...
>statplot(u,f,"h"):
\end{eulerprompt}
\eulerimg{27}{images/EMT4Statistika-Naela Rizqy Arofah-22305144042-038.png}
\begin{eulercomment}
Selain multiplisitas yang tepat, EMT dapat menghitung frekuensi dalam
vektor.
\end{eulercomment}
\begin{eulerprompt}
>getfrequencies(q,-50:10:50)
\end{eulerprompt}
\begin{euleroutput}
  [0,  23,  132,  316,  602,  801,  333,  141,  53,  0]
\end{euleroutput}
\begin{eulercomment}
Cara paling mudah untuk memplot ini sebagai distribusi adalah sebagai
berikut.
\end{eulercomment}
\begin{eulerprompt}
>plot2d(q,distribution=11):
\end{eulerprompt}
\eulerimg{27}{images/EMT4Statistika-Naela Rizqy Arofah-22305144042-039.png}
\begin{eulercomment}
Tetapi juga dimungkinkan untuk menghitung sebelumnya hitungan dalam
interval yang dipilih sebelumnya. Tentu saja, berikut ini menggunakan
getfrequencies() secara internal.

Karena fungsi histo() mengembalikan frekuensi, kita perlu
menskalakannya sehingga integral di bawah grafik batang adalah 1.
\end{eulercomment}
\begin{eulerprompt}
>\{x,y\}=histo(q,v=-55:10:55); y=y/sum(y)/differences(x); ...
>plot2d(x,y,>bar,style="/"):
\end{eulerprompt}
\eulerimg{27}{images/EMT4Statistika-Naela Rizqy Arofah-22305144042-040.png}
\eulerheading{Daftar}
\begin{eulercomment}
EMT memiliki dua macam daftar. Salah satunya adalah daftar global yang
dapat diubah, dan yang lainnya adalah jenis daftar yang tidak dapat
diubah. Kami tidak peduli dengan daftar global di sini.

Jenis daftar yang tidak dapat diubah disebut koleksi di EMT. Itu
berperilaku seperti struktur di C, tetapi elemennya hanya diberi nomor
dan tidak diberi nama.
\end{eulercomment}
\begin{eulerprompt}
>L=\{\{"Fred","Flintstone",40,[1990,1992]\}\}
\end{eulerprompt}
\begin{euleroutput}
  Fred
  Flintstone
  40
  [1990,  1992]
\end{euleroutput}
\begin{eulercomment}
Saat ini elemen tidak memiliki nama, meskipun nama dapat ditetapkan
untuk tujuan khusus. Mereka diakses dengan angka.
\end{eulercomment}
\begin{eulerprompt}
>(L[4])[2]
\end{eulerprompt}
\begin{euleroutput}
  1992
\end{euleroutput}
\eulerheading{File Input dan Output (Membaca dan Menulis Data)}
\begin{eulercomment}
Anda akan sering ingin mengimpor matriks data dari sumber lain ke EMT.
Tutorial ini memberitahu Anda tentang banyak cara untuk mencapai ini.
Fungsi sederhana adalah writematrix() dan readmatrix().

Mari kita tunjukkan cara membaca dan menulis vektor real ke file.
\end{eulercomment}
\begin{eulerprompt}
>a=random(1,100); mean(a), dev(a),
\end{eulerprompt}
\begin{euleroutput}
  0.49815
  0.28037
\end{euleroutput}
\begin{eulercomment}
Untuk menulis data ke file, kita menggunakan fungsi writematrix().

Karena pengenalan ini kemungkinan besar berada di direktori, di mana
pengguna tidak memiliki akses tulis, kami menulis data ke direktori
home pengguna. Untuk notebook sendiri, ini tidak perlu, karena file
data akan ditulis ke dalam direktori yang sama.
\end{eulercomment}
\begin{eulerprompt}
>filename="test.dat";
\end{eulerprompt}
\begin{eulercomment}
Sekarang kita menulis vektor kolom a' ke file. Ini menghasilkan satu
nomor di setiap baris file.
\end{eulercomment}
\begin{eulerprompt}
>writematrix(a',filename);
\end{eulerprompt}
\begin{eulercomment}
Untuk membaca data, kita gunakan readmatrix().
\end{eulercomment}
\begin{eulerprompt}
>a=readmatrix(filename)';
\end{eulerprompt}
\begin{eulercomment}
dan hapus file ini.
\end{eulercomment}
\begin{eulerprompt}
>fileremove(filename);
>mean(a), dev(a),
\end{eulerprompt}
\begin{euleroutput}
  0.49815
  0.28037
\end{euleroutput}
\begin{eulercomment}
Fungsi writematrix() atau writetable() dapat dikonfigurasi untuk
bahasa lain.

Misalnya, jika Anda memiliki sistem Indonesia (titik desimal dengan
koma), Excel Anda memerlukan nilai dengan koma desimal yang dipisahkan
oleh titik koma dalam file csv (defaultnya adalah nilai yang
dipisahkan koma). File "test.csv" berikut akan muncul di folder
cuurent Anda.
\end{eulercomment}
\begin{eulerprompt}
>filename="test.csv"; ...
>writematrix(random(5,3),file=filename,separator=",");
\end{eulerprompt}
\begin{eulercomment}
Anda sekarang dapat membuka file ini dengan Excel Indonesia secara
langsung.
\end{eulercomment}
\begin{eulerprompt}
>fileremove(filename);
\end{eulerprompt}
\begin{eulercomment}
Terkadang kita memiliki string dengan token seperti berikut ini.
\end{eulercomment}
\begin{eulerprompt}
>s1:="f m m f m m m f f f m m f";  ...
>s2:="f f f m m f f";
\end{eulerprompt}
\begin{eulercomment}
Untuk tokenize ini, kita mendefinisikan vektor token.
\end{eulercomment}
\begin{eulerprompt}
>tok:=["f","m"]
\end{eulerprompt}
\begin{euleroutput}
  f
  m
\end{euleroutput}
\begin{eulercomment}
Kemudian kita dapat menghitung berapa kali setiap token muncul dalam
string, dan memasukkan hasilnya ke dalam tabel.
\end{eulercomment}
\begin{eulerprompt}
>M:=getmultiplicities(tok,strtokens(s1))_ ...
>  getmultiplicities(tok,strtokens(s2));
\end{eulerprompt}
\begin{eulercomment}
Tulis tabel dengan header token.
\end{eulercomment}
\begin{eulerprompt}
>writetable(M,labc=tok,labr=1:2,wc=8)
\end{eulerprompt}
\begin{euleroutput}
                 f       m
         1       6       7
         2       5       2
\end{euleroutput}
\begin{eulercomment}
Untuk statika, EMT dapat membaca dan menulis tabel.
\end{eulercomment}
\begin{eulerprompt}
>file="test.dat"; open(file,"w"); ...
>writeln("A,B,C"); writematrix(random(3,3)); ...
>close();
\end{eulerprompt}
\begin{eulercomment}
Filenya terlihat seperti ini.
\end{eulercomment}
\begin{eulerprompt}
>printfile(file)
\end{eulerprompt}
\begin{euleroutput}
  A,B,C
  0.7003664386138074,0.1875530821001213,0.3262339279660414
  0.5926249243193858,0.1522927283984059,0.368140583062521
  0.8065535209872989,0.7265910840408142,0.7332619844597152
  
\end{euleroutput}
\begin{eulercomment}
Fungsi readtable() dalam bentuknya yang paling sederhana dapat membaca
ini dan mengembalikan kumpulan nilai dan baris judul.
\end{eulercomment}
\begin{eulerprompt}
>L=readtable(file,>list);
\end{eulerprompt}
\begin{eulercomment}
Koleksi ini dapat dicetak dengan writetable() ke notebook, atau ke
file.
\end{eulercomment}
\begin{eulerprompt}
>writetable(L,wc=10,dc=5)
\end{eulerprompt}
\begin{euleroutput}
           A         B         C
     0.70037   0.18755   0.32623
     0.59262   0.15229   0.36814
     0.80655   0.72659   0.73326
\end{euleroutput}
\begin{eulercomment}
Matriks nilai adalah elemen pertama dari L. Perhatikan bahwa mean()
dalam EMT menghitung nilai rata-rata dari baris matriks.
\end{eulercomment}
\begin{eulerprompt}
>mean(L[1])
\end{eulerprompt}
\begin{euleroutput}
    0.40472 
    0.37102 
    0.75547 
\end{euleroutput}
\eulerheading{File CSV}
\begin{eulercomment}
Pertama, mari kita menulis matriks ke dalam file. Untuk output, kami
membuat file di direktori kerja saat ini.
\end{eulercomment}
\begin{eulerprompt}
>file="test.csv";  ...
>M=random(3,3); writematrix(M,file);
\end{eulerprompt}
\begin{eulercomment}
Berikut adalah isi dari file ini.
\end{eulercomment}
\begin{eulerprompt}
>printfile(file)
\end{eulerprompt}
\begin{euleroutput}
  0.8221197733097619,0.821531098722547,0.7771240608094004
  0.8482947121863489,0.3237767724883862,0.6501422353377985
  0.1482301827518109,0.3297459716109594,0.6261901074210923
  
\end{euleroutput}
\begin{eulercomment}
CVS ini dapat dibuka pada sistem bahasa Inggris ke Excel dengan klik
dua kali. Jika Anda mendapatkan file seperti itu di sistem Jerman,
Anda perlu mengimpor data ke Excel dengan memperhatikan titik desimal.

Tetapi titik desimal juga merupakan format default untuk EMT. Anda
dapat membaca matriks dari file dengan readmatrix().
\end{eulercomment}
\begin{eulerprompt}
>readmatrix(file)
\end{eulerprompt}
\begin{euleroutput}
    0.82212   0.82153   0.77712 
    0.84829   0.32378   0.65014 
    0.14823   0.32975   0.62619 
\end{euleroutput}
\begin{eulercomment}
Dimungkinkan untuk menulis beberapa matriks ke satu file. Perintah
open() dapat membuka file untuk ditulis dengan parameter "w".
Standarnya adalah "r" untuk membaca.
\end{eulercomment}
\begin{eulerprompt}
>open(file,"w"); writematrix(M); writematrix(M'); close();
\end{eulerprompt}
\begin{eulercomment}
Matriks dipisahkan oleh garis kosong. Untuk membaca matriks, buka file
dan panggil readmatrix() beberapa kali.
\end{eulercomment}
\begin{eulerprompt}
>open(file); A=readmatrix(); B=readmatrix(); A==B, close();
\end{eulerprompt}
\begin{euleroutput}
          1         0         0 
          0         1         0 
          0         0         1 
\end{euleroutput}
\begin{eulercomment}
Di Excel atau spreadsheet serupa, Anda dapat mengekspor matriks
sebagai CSV (nilai yang dipisahkan koma). Di Excel 2007, gunakan
"simpan sebagai" dan "format lain", lalu pilih "CSV". Pastikan, tabel
saat ini hanya berisi data yang ingin Anda ekspor.

Berikut adalah contoh.
\end{eulercomment}
\begin{eulerprompt}
>printfile("excel-data.csv")
\end{eulerprompt}
\begin{euleroutput}
  0;1000;1000
  1;1051,271096;1072,508181
  2;1105,170918;1150,273799
  3;1161,834243;1233,67806
  4;1221,402758;1323,129812
  5;1284,025417;1419,067549
  6;1349,858808;1521,961556
  7;1419,067549;1632,31622
  8;1491,824698;1750,6725
  9;1568,312185;1877,610579
  10;1648,721271;2013,752707
\end{euleroutput}
\begin{eulercomment}
Seperti yang Anda lihat, sistem Jerman saya menggunakan titik koma
sebagai pemisah dan koma desimal. Anda dapat mengubah ini di
pengaturan sistem atau di Excel, tetapi tidak perlu membaca matriks ke
dalam EMT.

Cara termudah untuk membaca ini ke dalam Euler adalah readmatrix().
Semua koma diganti dengan titik dengan parameter \textgreater{}comma. Untuk CSV
bahasa Inggris, cukup abaikan parameter ini.
\end{eulercomment}
\begin{eulerprompt}
>M=readmatrix("excel-data.csv",>comma)
\end{eulerprompt}
\begin{euleroutput}
          0      1000      1000 
          1    1051.3    1072.5 
          2    1105.2    1150.3 
          3    1161.8    1233.7 
          4    1221.4    1323.1 
          5      1284    1419.1 
          6    1349.9      1522 
          7    1419.1    1632.3 
          8    1491.8    1750.7 
          9    1568.3    1877.6 
         10    1648.7    2013.8 
\end{euleroutput}
\begin{eulercomment}
Mari kita plot ini.
\end{eulercomment}
\begin{eulerprompt}
>plot2d(M'[1],M'[2:3],>points,color=[red,green]'):
\end{eulerprompt}
\eulerimg{27}{images/EMT4Statistika-Naela Rizqy Arofah-22305144042-041.png}
\begin{eulercomment}
Ada cara yang lebih mendasar untuk membaca data dari file. Anda dapat
membuka file dan membaca angka baris demi baris. Fungsi
getvectorline() akan membaca angka dari baris data. Secara default, ia
mengharapkan titik desimal. Tapi itu juga bisa menggunakan koma
desimal, jika Anda memanggil setdecimaldot(",") sebelum Anda
menggunakan fungsi ini.

Fungsi berikut adalah contoh untuk ini. Ini akan berhenti di akhir
file atau baris kosong.
\end{eulercomment}
\begin{eulerprompt}
>function myload (file) ...
\end{eulerprompt}
\begin{eulerudf}
  open(file);
  M=[];
  repeat
     until eof();
     v=getvectorline(3);
     if length(v)>0 then M=M_v; else break; endif;
  end;
  return M;
  close(file);
  endfunction
\end{eulerudf}
\begin{eulerprompt}
>myload(file)
\end{eulerprompt}
\begin{euleroutput}
    0.82212   0.82153   0.77712 
    0.84829   0.32378   0.65014 
    0.14823   0.32975   0.62619 
\end{euleroutput}
\begin{eulercomment}
Dimungkinkan juga untuk membaca semua angka dalam file itu dengan
getvector().
\end{eulercomment}
\begin{eulerprompt}
>open(file); v=getvector(10000); close(); redim(v[1:9],3,3)
\end{eulerprompt}
\begin{euleroutput}
    0.82212   0.82153   0.77712 
    0.84829   0.32378   0.65014 
    0.14823   0.32975   0.62619 
\end{euleroutput}
\begin{eulercomment}
Jadi sangat mudah untuk menyimpan vektor nilai, satu nilai di setiap
baris dan membaca kembali vektor ini.
\end{eulercomment}
\begin{eulerprompt}
>v=random(1000); mean(v)
\end{eulerprompt}
\begin{euleroutput}
  0.50303
\end{euleroutput}
\begin{eulerprompt}
>writematrix(v',file); mean(readmatrix(file)')
\end{eulerprompt}
\begin{euleroutput}
  0.50303
\end{euleroutput}
\eulerheading{Menggunakan Tabel}
\begin{eulercomment}
Tabel dapat digunakan untuk membaca atau menulis data numerik. Sebagai
contoh, kami menulis tabel dengan header baris dan kolom ke file.
\end{eulercomment}
\begin{eulerprompt}
>file="test.tab"; M=random(3,3);  ...
>open(file,"w");  ...
>writetable(M,separator=",",labc=["one","two","three"]);  ...
>close(); ...
>printfile(file)
\end{eulerprompt}
\begin{euleroutput}
  one,two,three
        0.09,      0.39,      0.86
        0.39,      0.86,      0.71
         0.2,      0.02,      0.83
\end{euleroutput}
\begin{eulercomment}
Ini dapat diimpor ke Excel.

Untuk membaca file dalam EMT, kami menggunakan readtable().
\end{eulercomment}
\begin{eulerprompt}
>\{M,headings\}=readtable(file,>clabs); ...
>writetable(M,labc=headings)
\end{eulerprompt}
\begin{euleroutput}
         one       two     three
        0.09      0.39      0.86
        0.39      0.86      0.71
         0.2      0.02      0.83
\end{euleroutput}
\eulerheading{Menganalisis Garis}
\begin{eulercomment}
Anda bahkan dapat mengevaluasi setiap baris dengan tangan. Misalkan,
kita memiliki garis dengan format berikut.
\end{eulercomment}
\begin{eulerprompt}
>line="2020-11-03,Tue,1'114.05"
\end{eulerprompt}
\begin{euleroutput}
  2020-11-03,Tue,1'114.05
\end{euleroutput}
\begin{eulercomment}
Pertama kita dapat menandai garis.
\end{eulercomment}
\begin{eulerprompt}
>vt=strtokens(line)
\end{eulerprompt}
\begin{euleroutput}
  2020-11-03
  Tue
  1'114.05
\end{euleroutput}
\begin{eulercomment}
Kemudian kita dapat mengevaluasi setiap elemen garis menggunakan
evaluasi yang sesuai.
\end{eulercomment}
\begin{eulerprompt}
>day(vt[1]),  ...
>indexof(["mon","tue","wed","thu","fri","sat","sun"],tolower(vt[2])),  ...
>strrepl(vt[3],"'","")()
\end{eulerprompt}
\begin{euleroutput}
  7.3816e+05
  2
  1114
\end{euleroutput}
\begin{eulercomment}
Menggunakan ekspresi reguler, dimungkinkan untuk mengekstrak hampir
semua informasi dari baris data.

Asumsikan kita memiliki baris berikut dokumen HTML.
\end{eulercomment}
\begin{eulerprompt}
>line="<tr><td>1145.45</td><td>5.6</td><td>-4.5</td><tr>"
\end{eulerprompt}
\begin{euleroutput}
  <tr><td>1145.45</td><td>5.6</td><td>-4.5</td><tr>
\end{euleroutput}
\begin{eulercomment}
Untuk mengekstrak ini, kami menggunakan ekspresi reguler, yang mencari

\end{eulercomment}
\begin{eulerttcomment}
  - kurung tutup >,
  - string apa pun yang tidak mengandung tanda kurung dengan
\end{eulerttcomment}
\begin{eulercomment}
sub-pertandingan "(...)",\\
\end{eulercomment}
\begin{eulerttcomment}
  - braket pembuka dan penutup menggunakan solusi terpendek,
  - lagi string apa pun yang tidak mengandung tanda kurung,
  - dan kurung buka <.
\end{eulerttcomment}
\begin{eulercomment}

Ekspresi reguler agak sulit dipelajari tetapi sangat kuat.
\end{eulercomment}
\begin{eulerprompt}
>\{pos,s,vt\}=strxfind(line,">([^<>]+)<.+?>([^<>]+)<");
\end{eulerprompt}
\begin{eulercomment}
Hasilnya adalah posisi kecocokan, string yang cocok, dan vektor string
untuk sub-pertandingan.
\end{eulercomment}
\begin{eulerprompt}
>for k=1:length(vt); vt[k](), end;
\end{eulerprompt}
\begin{euleroutput}
  1145.5
  5.6
\end{euleroutput}
\begin{eulercomment}
Berikut adalah fungsi, yang membaca semua item numerik antara \textless{}td\textgreater{} dan
\textless{}/td\textgreater{}.
\end{eulercomment}
\begin{eulerprompt}
>function readtd (line) ...
\end{eulerprompt}
\begin{eulerudf}
  v=[]; cp=0;
  repeat
     \{pos,s,vt\}=strxfind(line,"<td.*?>(.+?)</td>",cp);
     until pos==0;
     if length(vt)>0 then v=v|vt[1]; endif;
     cp=pos+strlen(s);
  end;
  return v;
  endfunction
\end{eulerudf}
\begin{eulerprompt}
>readtd(line+"<td>non-numerical</td>")
\end{eulerprompt}
\begin{euleroutput}
  1145.45
  5.6
  -4.5
  non-numerical
\end{euleroutput}
\eulerheading{Membaca dari Web}
\begin{eulercomment}
Situs web atau file dengan URL dapat dibuka di EMT dan dapat dibaca
baris demi baris.

Dalam contoh, kami membaca versi saat ini dari situs EMT. Kami
menggunakan ekspresi reguler untuk memindai "Versi ..." dalam sebuah
judul.
\end{eulercomment}
\begin{eulerprompt}
>function readversion () ...
\end{eulerprompt}
\begin{eulerudf}
  urlopen("http://www.euler-math-toolbox.de/Programs/Changes.html");
  repeat
    until urleof();
    s=urlgetline();
    k=strfind(s,"Version ",1);
    if k>0 then substring(s,k,strfind(s,"<",k)-1), break; endif;
  end;
  urlclose();
  endfunction
\end{eulerudf}
\begin{eulerprompt}
>readversion
\end{eulerprompt}
\begin{euleroutput}
  Version 2022-05-18
\end{euleroutput}
\eulerheading{Input dan Output Variabel}
\begin{eulercomment}
Anda dapat menulis variabel dalam bentuk definisi Euler ke file atau
ke baris perintah.
\end{eulercomment}
\begin{eulerprompt}
>writevar(pi,"mypi");
\end{eulerprompt}
\begin{euleroutput}
  mypi = 3.141592653589793;
\end{euleroutput}
\begin{eulercomment}
Untuk pengujian, kami membuat file Euler di direktori kerja EMT.
\end{eulercomment}
\begin{eulerprompt}
>file="test.e"; ...
>writevar(random(2,2),"M",file); ...
>printfile(file,3)
\end{eulerprompt}
\begin{euleroutput}
  M = [ ..
  0.5991820585590205, 0.7960280262224293;
  0.5167243983231363, 0.2996684599070898];
\end{euleroutput}
\begin{eulercomment}
Kita sekarang dapat memuat file. Ini akan mendefinisikan matriks M.
\end{eulercomment}
\begin{eulerprompt}
>load(file); show M,
\end{eulerprompt}
\begin{euleroutput}
  M = 
    0.59918   0.79603 
    0.51672   0.29967 
\end{euleroutput}
\begin{eulercomment}
Omong-omong, jika writevar() digunakan pada variabel, itu akan
mencetak definisi variabel dengan nama variabel ini.
\end{eulercomment}
\begin{eulerprompt}
>writevar(M); writevar(inch$)
\end{eulerprompt}
\begin{euleroutput}
  M = [ ..
  0.5991820585590205, 0.7960280262224293;
  0.5167243983231363, 0.2996684599070898];
  inch$ = 0.0254;
\end{euleroutput}
\begin{eulercomment}
Kita juga bisa membuka file baru atau menambahkan file yang sudah ada.
Dalam contoh kami menambahkan ke file yang dihasilkan sebelumnya.
\end{eulercomment}
\begin{eulerprompt}
>open(file,"a"); ...
>writevar(random(2,2),"M1"); ...
>writevar(random(3,1),"M2"); ...
>close();
>load(file); show M1; show M2;
\end{eulerprompt}
\begin{euleroutput}
  M1 = 
    0.30287   0.15372 
     0.7504   0.75401 
  M2 = 
    0.27213 
   0.053211 
    0.70249 
\end{euleroutput}
\begin{eulercomment}
Untuk menghapus file apa pun, gunakan fileremove().
\end{eulercomment}
\begin{eulerprompt}
>fileremove(file);
\end{eulerprompt}
\begin{eulercomment}
Vektor baris dalam file tidak memerlukan koma, jika setiap angka
berada di baris baru. Mari kita buat file seperti itu, menulis setiap
baris satu per satu dengan writeln().
\end{eulercomment}
\begin{eulerprompt}
>open(file,"w"); writeln("M = ["); ...
>for i=1 to 5; writeln(""+random()); end; ...
>writeln("];"); close(); ...
>printfile(file)
\end{eulerprompt}
\begin{euleroutput}
  M = [
  0.344851384551
  0.0807510017715
  0.876519562911
  0.754157709472
  0.688392638934
  ];
\end{euleroutput}
\begin{eulerprompt}
>load(file); M
\end{eulerprompt}
\begin{euleroutput}
  [0.34485,  0.080751,  0.87652,  0.75416,  0.68839]
\end{euleroutput}
\begin{eulercomment}
catatan : ketika mengenter perintah-perintah diatas ternyata hasil
yang didapatkan berbeda-beda

\begin{eulercomment}
\eulerheading{Latihan soal Nomor 1}
\begin{eulercomment}
Carilah rata-rata dan standar deviasi beserta plot dari data berikut\\
X = 1500,1700,2500,3500,4000
\end{eulercomment}
\begin{eulerprompt}
>X=[1500,1700,2500,3500,4000]; ...
>mean(X), dev(X),
\end{eulerprompt}
\begin{euleroutput}
  2640
  1094.5
\end{euleroutput}
\begin{eulerprompt}
>aspect(1.5); boxplot(X):
\end{eulerprompt}
\eulerimg{17}{images/EMT4Statistika-Naela Rizqy Arofah-22305144042-042.png}
\begin{eulercomment}
Nomor 2\\
Misalkan diberikan data skor hasil statistika dari 14 orang mahasiswa
sebagai berikut:\\
50,92,68,72,84,80,96,64,70,48,88,66,56,84\\
Tentukan rata-rata dari data tersebut!
\end{eulercomment}
\begin{eulerprompt}
>X=[62,65,58,90,75,79,82,91,75,75,75,95]
\end{eulerprompt}
\begin{euleroutput}
  [62,  65,  58,  90,  75,  79,  82,  91,  75,  75,  75,  95]
\end{euleroutput}
\begin{eulerprompt}
>mean(X)
\end{eulerprompt}
\begin{euleroutput}
  76.833
\end{euleroutput}

\end{document}
